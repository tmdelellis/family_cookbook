\secpart{Second Edition}{Sides}

%%---------------------------------------------------------------------------%%
\begin{entry}{White Grapes and Sweetened ``Cream''}{Second Edition}
\index{Fruit!white grapes \& sweetened cream}
\index{Vegetarian!white grapes \& sweetened cream}
\index{Lindquist!Julie}

\begin{open}
  Here is a refreshing and simple summery side submitted by Julie Lindquist.
\end{open}
%%
\begin{ingredients}
    grapes\\
    plain yogurt or sour cream\\
    brown sugar
\end{ingredients}
Remove grapes from stems; rinse and dry. Place in a colorful serving dish. Top
with plain yogurt or sour cream (NB: sweet cream doesn't offer the contrast to
the sweet grape taste). Sprinkle some brown sugar on top. Refrigerate for at
least 4 hours; the attractive pattern of the dissolving brown sugar is a
bonus. Lasts a few days, if there’s any left.
\end{entry}

%%---------------------------------------------------------------------------%%
\begin{entry}{Turkish Rice}{Second Edition}
\index{Rice!turkish}
\index{Johnson!Martha}
\index{Johnson!Dodge}

\begin{open}
    From Martha and Dodge, a graduate school special that provides an infinitely
    variable way to prepare a pilaf with your favorite spices and ingredients.
\end{open}
%%
\begin{ingredients}
    chopped onion or shallot\\
    \SI{1}{\cup} long grain rice\\
    2\SI{1/3}{\cup} beef bouillon\\
    raisins, dried cranberries, nuts\\
    cumin, curry powder, garlic
\end{ingredients}
Cook a few tablespoons of chopped onion or shallot in some butter and oil. Add
uncooked rice and stir until light brown. Add the beef bouillon, plus a handful
of raisins, dried cranberries, nuts and seasonings (some mix of cumin, curry
powder, garlic, etc.) and simmer until done (\SI{20}{\minute}).
\end{entry}

%%---------------------------------------------------------------------------%%
\begin{entry}{Boursin Potatoes}{Second Edition}
\index{Potatoes!boursin potatoes}
\index{Vegetarian!boursin potatoes}
\index{Johnson!Martha}

\begin{open}
    From Martha: a favorite side from Barby Buckman at St.~Peter's in the Great
    Valley, PA. A great dish for a crowd (8), or a half recipe also works well
    for 4.
\end{open}
%%
\begin{ingredients}
    \SI{2}{\cup} whipping cream\\
    1 \SI{5}{\ounce} ounce package Boursin cheese (herbs)\\
    \SI{3}{\pound} red new potatoes, or Yukon Gold, unpeeled, scrubbed and
    thinly sliced\\
    salt and pepper\\
    \SIrange{1}{2}{\tblspoon} chopped fresh parsley
\end{ingredients}
Preheat oven to \SI{400}{\degreeF}. Stir whipping cream and Bousing cheese in a
heavy pan over medium heat until cheese melts and no lumps remain. Arrange half
the potatoes in overlapping rows in a buttered \SI{9x13}{\inch} baking dish.
Season with salt and pepper and pour half the cheese mixture over them. Arrange
the rest of the potatoes in a second layer with remaining cheese mix. Bake until
golden brown, about \SI{1}{\hour}. Sprinkle with parsley and serve.
\end{entry}

%%---------------------------------------------------------------------------%%
\begin{entry}{Immune-Boosting Chimichurri from Argentina}{Second Edition}
\index{Sauces!immune-boosting chimichurri}
\index{Horne Rona!Alison}

\begin{open}
    Chimichurri is basically Argentinian pesto.  You can use it on everything
    from meat dishes, vegetables, salads, and breads.  According to Alison Horne Rona, ``This will make any meat taste great!''
\end{open}
%%
\begin{ingredients}
    1 large bunch of flat leafed parsley chopped\\
    1 large bunch of fresh oregano leaves stripped from stems\\
    1 large bunch of fresh cilantro all chopped\\
    1 green jalape\~{n}o pepper chopped\\
    1 small yellow onion (or shallots) chopped\\
    1 lemon (I cut off most of the rind, cut it in half, take out the seeds, and
    throw both halves in the blender)\\
    6 large cloves of garlic chopped finely\\
    \SIrange{2}{3}{\tblspoon} of red wine vinegar to taste\\
    \SIrange{3}{4}{\tblspoon} olive oil\\
    sea salt and black pepper
\end{ingredients}
Throw everything in a blender and mix.
\end{entry}

%%---------------------------------------------------------------------------%%

\begin{entry}{Blanched lettuce with sauce}{Second Edition}
\index{Salads!blanched lettuce with sauce}
\index{Vegan!blanched lettuce with sauce}
\index{Liu!Geyi}

\begin{open}
  Submitted by Geyi, this is a dish often seen in Southern Chinese restaurants. It is often crisp and green in restaurants but soggy and dark at home. The secret is to boil the lettuce leaves very fast (5-10 seconds) in a large amount of water so that they stays green and crisp while keeping in all the nutrition 
\end{open}
%%
\begin{ingredients}

\end{ingredients}
Prepare sauce by mixing what you think is a good amount of oyster sauce, soy sauce, salt, sugar, (corn) starch, water together. Ask Geyi for tips on the right amount; she was busy moving when when we were entering this, but it sounded so good we included it anyway!

\begin{figure}[h]
    \centering
    \includegraphics[width=0.8\textwidth]{figures/lettuce1.png}
    \caption{Sauce Preparation}
    \label{fig:Lettuce1}
\end{figure}

Next cook the lettuce as follows: Take a large pot of water, add a tablespoon of vegetable oil, a teaspoon of salt and bring to a boil. Add lettuce, stir so that all areas get blanched, and take out after 5-10 seconds (refer to figure \ref{fig:lettuce2} to see how the lettuce is still quite green).
 
\begin{figure}
    \centering
    \includegraphics[width=0.8\textwidth]{figures/Lettuce2.png}
    \caption{Blanching the Lettuce}
    \label{fig:lettuce2}
\end{figure}
 
Then, heat oil in a pot, add chopped garlic and stir until it becomes yellow, and add the sauce mixture and stir until it thickens. 

Finally, pour the sauce over the lettuce and serve immediately.

\begin{figure}
    \centering
    \includegraphics[width=0.8\textwidth]{figures/Lettuce3.png}
    \caption{Pour the Sauce over the lettuce}
    \label{fig:lettuce3}
\end{figure}


\end{entry}

\begin{entry}{Roasted Brussell Sprouts}{Second Edition}
\index{Vegetarian!brussel sprouts}
\index{Pross!Katie}

\begin{open}
 This recipe looks absolutely amazing and super easy for our new cooks to try. Katie adds crumbled bacon to it, which sounds delicious!
\end{open}
%%
\begin{ingredients}
    \SI{1}{\pound} fresh brussell sprouts \\
    \SIrange{1}{2}{\tblspoon} olive oil\\
    \SI{1}{\tblspoon} \\
    salt and pepper\\
\end{ingredients}
Preheat oven to \SI{450}{\degreeF}. Trim ends of brussell sprouts and cut into \SI{1}/{2} or \SI{1}/{4}, depending on the size of each sprout. Drizzle sprouts with olive oil, salt and pepper, and toss to coat. 
Roast for \SIrange{10}{20}{\minute}, tossing a few times throughout until sprouts are browned.
\end{entry}
%% From the first edition

\chapter{Sides}

\section{Hash Brown Potatoes\index{sides!hash brown potatoes}}

\begin{open}
  Contributed by Joyce Evans. Try with the Swiss Meatloaf
  (Section~\ref{sec:swiss-meatloaf}), yum! Serves 4.
\end{open}
\begin{ingredients}
  3 large potatoes, boiled \\
  \SI{1/4}{\cup} milk \\
  \SI{3}{\tblspoon} all-purpose flour \\
  \SI{2}{\tblspoon} minced onion \\
  \SI{2}{\tblspoon} minced fresh parsley or chervil \\
  \SI{1/2}{\teaspoon} salt \\
  \SI{1/2}{\teaspoon} pepper \\
  \SI{1/4}{\teaspoon} dried oregano (opt.) \\
  Dash of Tabasco \\
  \SI{3}{\tblspoon} bacon drippings, rendered chicken fat, or vegetable oil
\end{ingredients}
Preheat in electric skillet to \SI{300}{\degree}. Peel and dice the boiled
potatoes and place into a medium bowl. You should have about \SI{3}{\cup}. Add
the rest of the ingredients except the cooking fat and blend.

Add the cooking fat to the skillet and heat. Pack the potato mixture in
firmly, spreading it out in an even layer. Cook \numrange{7}{9} minutes or
until the bottom side is richly brown. Turn the mixture over in segments and
smooth down again into a patty. Continue cooking until the other side is
brown, another \numrange{7}{9} minutes.  Cut into wedges and serve.

\section{Black Rice\index{sides!black rice}}

\begin{open}
  Contributed by Fermina Evans. Serves 4, or 2 healthy eaters. This is Tom and
  Katie's favorite peasant food.
\end{open}
\begin{ingredients}
  \SI{1}{\cup} dry black beans\\
  \SI{5}{\cup} chicken broth\\
  \SI{1/2}{\tblspoon} olive oil \\
  1 small onion, chopped \\
  4 cloves garlic, minced \\
  \SI{1}{\ounce} finely chopped Canadian or regular bacon \\
  \SI{1/2}{\cup} rice \\
  \SI{1/4}{\cup} white wine \\
  1 tomato coarsely chopped \\
  \SI{1/2}{\teaspoon} ground cumin \\
  pinch cayenne \\
  \SI{1/2}{\cup} finely chopped cilantro
\end{ingredients}
You may substitute 2 cans black beans for dry beans if you prefer. If using
dry beans, soak beans overnight in cold water, and simmer beans for
\numrange{120}{150} minutes in \SI{3}{\cup} broth until tender. Drain and
resolve liquid (\SI{1/2}[1]{\cup}). Otherwise, drain them under cold water and
use \SI{1/2}[1]{\cup} chicken broth or chicken bouillon stock for bean broth.

Heat oil in stockpot. Add onion, garlic, and bacon and stir fry for about 5
minutes.  Add rice and stir for 1 minute. Add wine and cook for 2 minutes. Add
tomatoes and cook for 2 more minutes.  Add bean broth \SI{1/2}{\cup} at a
time, stirring until liquid is absorbed before adding more broth. This will
take \numrange{20}{25} minutes to complete. Add the beans and remaining
broth. Season with cumin, cayenne, and cilantro and serve.

\section{Potato Gratin with Mustard and Cheese\index{sides!Potato Gratin}}

\begin{open}
  This is a great entertaining dish because its classy, very smooth and
  flavorful, yet can be prepared before guests arrive. Kate got it from
  \corp{Bon Appetit} magazine.
\end{open}
\begin{ingredients}
  \SI{1}{\tblspoon} butter\\
  \SI{1}{\cup} fresh breadcrumbs\\
  \SI{1}{\tblspoon} dried thyme\\
  \SI{2}{\teaspoon} salt\\
  \SI{1}{\teaspoon} ground pepper\\
  \SI{1}{\pound} sharp white cheddar cheese, grated\\
  \SI{1/4}{\cup} flour\\
  \SI{5}{\pound} russet potatoes, peeled and thinly sliced\\
  \SI{4}{\cup} canned low salt chicken broth\\
  \SI{1}{\cup} whipping cream\\
  \SI{6}{\tblspoon} Dijon mustard
\end{ingredients}
Melt butter in skillet and add breadcrumbs, stirring until golden brown (about
10 min.). Set aside. Preheat oven to \SI{400}{\degree}. Butter a
\SI{15x10x2}{\inch} baking dish. Mix thyme, salt, and pepper in small
bowl. Combine grated cheese and flour, tossing to coat the cheese. Arrange
\num{1/3} potato slices to cover the bottom of the baking dish. Sprinkle
\num{1/3} the thyme mixture, then \num{1/3} the cheese mixture. Repeat
layering 2 more times.  Next whisk chicken broth, cream, and mustard in a
separate bowl, and then pour it over the potato layers. Bake 30
minutes. Sprinkle buttered crumbs over, and bake until potatoes are tender and
top is golden brown, about 1 hour longer.  Enjoy!
