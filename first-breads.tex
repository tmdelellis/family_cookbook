%%---------------------------------------------------------------------------%%
\begin{entry}{Three Kings Bread (and St. Nick)}{First Edition}
\index{breads!three kings bread}
\index{King!Jenn}
\index{King!Rich}

\begin{center}
    \includegraphics[width=.4\textwidth,clip]{figures/kings.pdf}
\end{center}

\begin{open}
  Is this the real name or is it because it comes from three Kings?
  This is a recipe supplied to us from Rich, Jenn and Julian King (Nicholas arrived after the project, began-hence the name). (editor's note from the second edition: and yet more children have kept arriving to the King family)
  This recipe makes one loaf.
\end{open}
\begin{ingredients}
  \SI{1/4}{\cup} plus one \si{\tblspoon} sour cream \\
  \SI{1}{\teaspoon} baking soda \\
  \SI{1/2}{\cup} butter at room temperature \\
  \SI{1}{\cup} sugar \\
  2 eggs lightly beaten \\
  1 ripe mashed banana and 1 medium apple (2 bananas an option) \\
  \SI{1/2}{\teaspoon} baking powder \\
  \SI{1}{\cup} chopped nuts \\
  \SI{1/2}{\teaspoon} cardamom \\
  1 zest of lemon \\
  \SI{1}{\teaspoon} vanilla extract
\end{ingredients}
Preheat the oven to \SI{350}{\degree}.  Grease a \SI{9x5}{\inch} loaf pan.
Combine sour cream and baking soda in small bowl.  Set aside (it will foam).
Cream butter and sugar in a small bowl.  Beat in eggs, fruit, and sour cream
mixture.  Slowly mix in all dry ingredients.  Bake until a toothpick inserted
into center comes out sort of clean and loaf is golden brown.  This should be
about 1 hour.

Cool 10 minutes in pan. Turn loaf out onto rack and cool completely.  Eat
thinking of your most favorite Kings$\ldots$
\end{entry}

%%---------------------------------------------------------------------------%%
\begin{entry}{Kate's Standard Bagels}{First Edition}
\index{breads!bagels}
\index{Evans!Kate}

\begin{open}
  This is a recipe by Kate that she picked up in Washington, D.C. during
  a 91'--92' winter internship at NIST from Dr. Lucatorto.  We enjoy this one
  a lot (especially in the south where getting good bagels is not always
  easy). Note: you need a food processor for this recipe. It makes 16 bagels.
\end{open}
\begin{ingredients}
  2 packets yeast \\
  2 scant \si{\tblspoon} sugar\\
  \SI{1/2}[3]{\cup} warm water with salt \\
  Lots (several pounds) of flour
\end{ingredients}
Combine yeast, sugar, and water.  Add \SI{\sim 2}{\cup} flour to food
processor. Pour in water mixture until a dough is formed (stop pouring when
processor begins to ``growl.'' Listen you'll hear it).  Remove dough to a
casserole dish with lid. Repeat until all mixture is used.  Microwave dish at
\SI{30}{\percent} for 3 minutes.  Let rise for 30~minutes.  Punch down roll
into loaf, cut in 16 pieces and make into bagels. Kate uses a doughnut
stamper. Let rise 30~minutes. In large frying pan, set \SI{2}{\inch} water to
boil.  Boil bagels for 10~seconds each side.  Place on greased cookie sheet
and bake 30~minutes or until lightly brown at \SI{350}{\degree}. If you want
to add extras such as cinnamon or raisins, add when processor starts to
growl. If you would like toppings such as sesame seeds, brush bagel with egg
white and sprinkle on top just before baking.
\end{entry}

%%---------------------------------------------------------------------------%%
\begin{entry}{Kate's Super Stromboli Dough}{First Edition}
\index{Evans!Kate}
\index{breads!stromboli}

\begin{open}
  This is a recipe by Kate.  While this recipe is intended for Stromboli
  or calzone it also makes a fine pizza dough\index{breads!pizza dough}.  The
  recipe serves 6.
\end{open}
\begin{ingredients}
  4 scant \si{\cup} flour \\
  1 package \corp{Quick Rise} yeast \\
  \SI{1}{\teaspoon} salt \\
  \SI{1}{\tblspoon} sugar \\
  \SI{1/3}[1]{\cup} warm water \\
  \SI{1/4}{\cup} oil
\end{ingredients}
In mixing bowl combine flour, yeast, sugar, and salt.  Add water and oil and
form a soft dough. Add flour or water as necessary.  Let rise 30 minutes, then
punch down (you can freeze at this point to thaw later in microwave).  Roll
into \numrange{4}{6} circles (depending on crowd hunger). Add your favorite
toppings, including sauce if desired, and of course cheese. Possible fillings
are broccoli (a Kate favorite), spinach, pepperoni, ham, onions, mushrooms,
and almost anything edible you can think of. Bake at \SI{400}{\degree} for
\numrange{12}{15} minutes until golden brown.
\end{entry}

%%---------------------------------------------------------------------------%%
\begin{entry}{Kuchen}{First Edition}
\index{breads!kuchen}
\index{Evans!Fermina}

\begin{open}
  Provided by Fermina Evans, this German ``bread'' is a Christmas morning
  tradition that she has carried on from Geo's parent's and passed on to the
  Evans's kids. She always makes a double batch and it's still barely enough!
  Don't be fooled by its location in the ``breads'' section; its definitely a
  treat!
\end{open}
\begin{ingredients}
  \SI{1/4}{\cup} shortening\\
  \SI{1}{\cup} sugar \\
  1 egg \\
  \SI{1/2}{\cup} milk \\
  \SI{1/2}[1]{\cup} flour \\
  \SI{2}{\teaspoon} baking powder \\
  salt to taste
\end{ingredients}
The topping:
\begin{ingredients}
  \SI{1/2}{\cup} brown sugar \\
  \SI{1/3}{\cup} flour \\
  \SI{1}{\teaspoon} cinnamon \\
  dash salt\\
  \SI{1/4}{\cup} butter
\end{ingredients}
Preheat oven to \SI{350}{\degreeF}. Cream shortening and sugar. Add egg and mix
well.  Add baking powder, flour and milk. Pour into greased and floured
\SI{9}{\inch} round or \SI{8}{\inch} square baking pan. Combine the rest of
the topping ingredients except butter. Then, cut butter into topping mixture
and sprinkle on top of batter. Bake for \numrange{30}{40} minutes (test with
toothpick for done-ness).
\end{entry}

%%---------------------------------------------------------------------------%%
\begin{entry}{Weedie's Blueberry Muffins}{First Edition}
\index{breakfast!blueberry muffins}
\index{Johnson!Weedie}

\begin{open}
  No joke, Don and Kate (and surely David and Steve) used to beg Weedie
  to make these muffins every time we visited. And she always did. Martha and
  Dodge say, ``Oh Heaven, these and some Lobster salad!'' They are perfection,
  we promise you. Just make the effort to get quality blueberries. We
  recommend going to Maine to get them.
\end{open}
\begin{ingredients}
  2 scant \si{\cup} flour\\
  \SI{3}{\teaspoon} baking powder\\
  \SI{1/2}{\cup} sugar\\
  \SI{1/2}{\teaspoon} salt\\
  \num{1/2} stick butter, melted (\SI{1/4}{\cup})\\
  2 eggs\\
  \SI{1}{\cup} milk\\
  \SI{1}{\cup} blueberries (little wild ones are best)\\
  sugar\\
  cinnamon
\end{ingredients}
Preheat oven to \SI{400}{\degreeF}. Weedie says, ``Usually I wash the berries
awhile before using them, so they can dry off before being added.'' In mixing
bowl combine flour, baking powder, sugar, and salt. Mix butter eggs and milk
in a separate bowl. Pour this over the flour mixture and stir until
smooth. Add blueberries, stir gently, and spoon into buttered muffin
tins. Sprinkle with sugar and cinnamon and bake for \numrange{20}{25} minutes.
\end{entry}