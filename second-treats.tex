\secpart{Second Edition}{Treats}

%%---------------------------------------------------------------------------%%
\begin{entry}{Mint Syrup}
\index{syrup!mint}
\index{Johnson!Weedie}

\begin{open}
  This recipe comes from Louise Johnson of Spruce Head, ME, aka Weedie, the
  sister of the original Donald Dodge Johnson (Dodge and Julie’s father).
\end{open}
%%
\begin{ingredients}
    fresh spearmint or peppermint leaves\\
    granulated sugar\\
    1 or more lemons\\
    1 or more oranges
\end{ingredients}
Using a strainer that fits into a deep bowl, fill with finely cut fresh
spearmint or peppermint leaves. Boil for ten minutes equal measures of water
and granulated sugar, approximately 1\SI{1/2}{\cup} each. Pour directly over
the mint and allow to cool. Remove the strainer and add the juice of one or
more lemons and the juice of one or more oranges. Wonderful over ice cream, in
iced tea, or on a simple cake aching for a hint of mint.
\end{entry}

%%---------------------------------------------------------------------------%%
\begin{entry}{Southern Pecan Pie}
\label{sec:pecanpie}
\index{pie!southern pecan}
\index{Johnson!Martha}

\begin{open}
  Hey y'all, from Martha! This is from an ancient, out-of-print Better Homes
  and Garden Cookbook One can double the sin by brushing the crust with a dark
  chocolate glaze before adding the filling, but it's pretty wonderful as is!
  And so easy! Either a pie crust mix or the real thing work equally well. The
  main thing is to have fresh pecans.
\end{open}
%%
\begin{ingredients}
    3 large eggs\\
    \SI{2/3}{\cup} sugar\\
    \SI{1}{\cup} dark corn syrup\\
    \SI{1/3}{\cup} melted butter\\
    \SI{1}{\cup} pecan halves\\
    1 \SI{9}{\inch} unbaked pastry shell
\end{ingredients}
Beat eggs thoroughly with sugar, a dash of salt, corn syrup, and melted butter.
Add pecans. (Brush shell with melted chocolate if desired). Bake in moderate
oven (\SI{350}{\degreeF}) \SI{50}{\minute} or until a knife inserted near center
comes out clean. Cool.
\end{entry}

%%---------------------------------------------------------------------------%%
\begin{entry}{Grapes \`{a} la Creme}
\index{compote!grapes a la creme}
\index{Horne!Mildred}

\begin{open}
    This is from Paul Horne's mother, Mildred W. Horne, Alexandra, VA, 1959.
\end{open}
%%
\begin{ingredients}
    seedless grapes\\
    sour cream\\
    brown sugar
\end{ingredients}
Wash and de-stem 1\SIrange{1/2}{2}{\pound} of grapes. Drain and place in dessert
dishes, preferably long-stemmed ones.  Spread a tablespoon or so of sour cream
over each mound of grapes and leave in refrigerator for several hours.  An hour
or so before dinner, sprinkle liberally with brown sugar and replace in
refrigerator until dessert time.  Serve with coffee.
\end{entry}

%%---------------------------------------------------------------------------%%
\begin{entry}{Jen's Favorite Sugar Cookies}
\index{cookies!sugar}
\index{Lindquist!Jen}

\begin{open}
  These are Jen Lindquist's favorite sugar cookies.  Judging by the pictures,
  creativity is a plus!
\end{open}
%%
\begin{ingredients}
    2 sticks butter\\
    \SI{8}{\ounce} cream cheese\\
    1\SI{1/2}{\cup} sugar\\
    3\SI{1/2}{\cup} flour\\
    \SI{1}{\teaspoon} vanilla\\
    \SI{1}{\teaspoon} almond\\
    \SI{1}{\teaspoon} baking powder\\
    1 egg
\end{ingredients}
Mix, roll, cut, and bake at \SI{350}{\degreeF} for \SI{8}{\minute}.  For some
amazing decorating ideas check out Fig.~\ref{fig:sugar-cookie-decorating}.
\begin{figure}
    \centering
    \includegraphics[width=0.7\textwidth]{figures/sugar-cookies}
    \caption{Sugar cookie decorating ideas.}
    \label{fig:sugar-cookie-decorating}
\end{figure}
%%
\begin{figure}[b]
    \centering
    \includegraphics[width=0.33\textwidth,clip]{figures/jen-making-cookies}
    \caption{Jen making sugar cookies!}
\end{figure}
\end{entry}

%%---------------------------------------------------------------------------%%
\begin{entry}{Gingerbread, King of Cakes}
\index{cakes!gingerbread}
\index{Lindquist!Dot}

\begin{open}
    This is one of the many amazing recipes you can find at
    \url{myrunawaykitchen.com}, Dot Lindquist's cooking blog.  We agree that
    gingerbread is, if not the King, at least cake royalty.  This recipe makes
    \numrange{14}{16} slices, takes \SI{25}{\minute} of prep time and about
    \SI{1}{\hour} of cooking time.
\end{open}
%%
\begin{ingredients}
    \SI{1/2}{\cup}  granulated sugar\\
    \SI{1/2}{\cup}  unsalted butter\\
    1 large egg\\
    \SI{1}{\cup} molasses\\
    1 large orange, zested and juiced\\
    2\SI{1/2}{\cup}  all purpose flour\\
    1\SI{1/2}{\teaspoon}  baking soda\\
    \SI{2}{\teaspoon} cinnamon\\
    \SI{2}{\teaspoon} ground ginger\\
    \SI{3/4}{\teaspoon}   ground cloves\\
    \SI{1/2}{\teaspoon}  salt\\
    \SI{1/2}{\cup}  hot water
\end{ingredients}
%%
\begin{figure}
    \centering
    \includegraphics[width=0.8\textwidth]{figures/gingerbread}
    \caption{Gingerbread, indeed the King of Cakes!}
\end{figure}
%%
Preheat oven to \SI{350}{\degreeF}. Grease and lightly flour the inside of your
baking pan. I like a bundt pan for this recipe, but you could use a
\SI{9x13}{\inch} pan, \SI{9x9}{\inch}, or even make cupcakes. Whatever your
little heart desires. If you're using cupcake papers or a parchment lining in
your pan or baking dish, no need to grease and flour the pan.

If possible, use room-temperature butter. If your room is very cold, or you
forgot to leave a pound of butter on your countertop for baking, pop your butter
into the microwave for \SI{10}{\second} intervals until it's mushy but not
liquid. Cream butter and sugar together in the stand mixer with the paddle
attachment, or with the handheld mixer.

Add egg and molasses and mix, scraping down the sides and bottom of the bowl as
you go. In a separate bowl, sift together the dry ingredients: flour, baking
soda, cinnamon, ginger, cloves and salt. Add orange zest. Add dry ingredients to
the wet ingredients and mix until well blended. Add the \SI{1/2}{\cup} hot water
and the juice from your orange (should be about \SI{1/2}{\cup}, making the total
amount of liquid added in this step equal to one cup).

Bake in preheated oven for one hour or until a toothpick in the center comes out
clean and the cake springs back against your finger when you press into it.

Remove from pan and cool before frosting.
\end{entry}

%%---------------------------------------------------------------------------%%
\begin{entry}{OMG Avocado Chocolate Mousse}
\index{mousse!OMG avocado chocolate}
\index{Lindquist!Dot}

\begin{open}
  From \url{myrunawaykitchen.com}, Dottie says, ``You’re not going to believe
  how simple this is to make and how eye-poppingly great it tastes… and the
  silky texture: OMG.  All you need is a food processor for this one.  Unless
  you like to slather it with real whipped cream, in which case you’ll also
  want an electric mixer of some kind.  I have been known to whip by hand,
  yes, but only when no motorized option is available!''

  Also, as pointed out by Dot, this recipe only uses healthy fats, so enjoy
  almost guilt free; although frankly, we highly encourage the use of all fats
  in this cookbook!
\end{open}
%%
\begin{ingredients}
    4 ripe avocados, peeled and pitted\\
    \SI{8}{\ounce} semisweet chocolate, melted (baker’s chocolate or chips are fine---you can use a double boiler method if you want, but it's ok to melt this stuff, covered, in the microwave in \SI{30}{\second} increments, stirring until smooth)\\
    \SI{6}{\tblspoon} cocoa powder\\
    \SI{1/2}{\cup} milk of any variety\\
    \SI{2}{\teaspoon} vanilla\\
    \SI{1/4}{\teaspoon} salt\\
    \SI{3/4}{\cup} maple syrup
\end{ingredients}
Just combine all ingredients in a food processor and pulse until it’s as smooth
as a chocolate silk dream (Fig.~\ref{fig:mousse}).  (Add maple syrup to the
mixture last, to your desired sweetness level, pulsing until completely
incorporated.)
%%
\begin{figure}
    \centering
    \includegraphics[width=0.6\textwidth]{figures/avocado-choc-mousse}
    \caption{OMG Avocado Chocolate Mousse!}
    \label{fig:mousse}
\end{figure}
%%

\minisection{Whipped Cream}

\begin{ingredients}
    \SI{1}{\quart} ``heavy'' or ``whipping'' cream\\
    \SI{1/2}{\cup} confectioner's sugar\\
    \SI{2}{\teaspoon} vanilla
\end{ingredients}
Start your mixer on low or medium (so that the cream doesn't splatter all over
creation) and gradually increase the speed as you gradually add sugar and
vanilla.  Beat until soft peaks form.
\end{entry}

%%---------------------------------------------------------------------------%%
\begin{entry}{Rhubarb Raspberry Compote with Mint}
\index{compote!rhubarb raspberry with mint}
\index{Rona!Alison}

\begin{open}
  A nice compote from Alison Rona that can be used with yogurt, on ice cream,
  or even as part of a fruit torte.
\end{open}
%%
\begin{ingredients}
    stalks of rhubarb\\
    1 banana\\
    1 apple\\
    1 blood orange\\
    fresh ginger\\
    cinnamon\\
    2 cloves\\
    orange peel (zested)\\
    nutmeg\\
    honey\\
    molasses\\
    brown sugar\\
    butter\\
    vanilla extract\\
    orange extract\\
    2 boxes of rasberry\\
    fresh mint leaves
\end{ingredients}
Chop the stalks of rhubarb, then simmer in a cup of water.  Cut a banana,
apple, and blood orange into equal sized chunks and add to the rhubarb.  Then
add minced fresh ginger, a spoonful of cinnamon, 2 cloves, a zested orange peel,
grated nutmeg, a spoonful of honey, molasses, brown sugar, butter, vanilla
extract, orange extract, and 2 boxes of raspberries.  You will need to play with
amounts for taste.  Add lots of fresh chopped mint near the end.

Pour it hot over vanilla ice cream or serve warm with plain goat milk yogurt.
Or, use the compote in a pie crust and add a few pieces of fruit on top!
\end{entry}

%%---------------------------------------------------------------------------%%
\begin{entry}{Chocolate cream cheese cupcakes}
\index{cakes!chocolate cream cheese cupcakes}
\index{Evans!Fermina}

\begin{open}
  Kate has fond memories of visiting Tom's house when they were dating and
  Tom's mom, Fermina, would bring out these little harmless looking but
  completely addicting mini-cupcakes.
\end{open}
%%
\begin{ingredients}
    1\SI{1/2}{\cup} flour\\
    \SI{1}{\cup} sugar\\
    \SI{1/4}{\cup} cocoa\\
    \SI{1}{\teaspoon} baking soda\\
    \SI{1}{\cup} water\\
    \SI{1}{\teaspoon} vanilla\\
    \SI{1/3}{\cup} oil\\
    \SI{1}{\tblspoon} cider vinegar\\
    1 dozen cupcake liners\\
\end{ingredients}

Topping:
\begin{ingredients}
    \SI{8}{\ounce} cream cheese, softened\\
    1 egg\\
    \SI{1/3}{\cup} sugar\\
    \SI{1/8}{\teaspoon} salt\\
    \SI{3/4}{package} miniature chocolate chips\\
\end{ingredients}
Preheat oven to (\SI{350}{\degreeF}). In a large bowl, combine the flour,
\SI{1}{\cup} sugar, cocoa, and baking soda. In a separate bowl, mix water,
vanilla, oil, and vinegar and then add to dry ingredients and mix well. Spoon
into liners in muffin tins until \SIrange{1/2}{3/4} full. Blend cream cheese,
egg, \SI{1/3}{\cup} sugar, and salt and mini chips. Place 1 spoonful on top of
chocolate mixture within each liner, and bake \SIrange{20}{23}{\minute}.
\end{entry}

%%---------------------------------------------------------------------------%%
\begin{entry}{Holiday chocolate cream pie}
\index{pie!chocolate cream}
\index{Evans!Kate}
\index{Johnson!Martha}

\begin{open}
  This easy but delicious pie recipe came from a now out of print cookbook by
  Mable Hoffman titled ``Chocolate Cookery'' that Don got her for Christmas in
  1983. Kate wore it out from heavy use (thanks Don, you nailed it!) and
  recently procured a new copy from a secondhand seller. She makes this pie
  for most holiday events, and just this Christmas figured out how to avoid
  having the crust come out soggy! Regarding the pie shell, like Martha in the
  \ref{sec:pecanpie} recipe, the Betty Crocker mix works well here too.
\end{open}
%%
\begin{ingredients}
    \SI{1}{\cup} sugar\\
    \SI{1/4}{\cup} cornstarch\\
    \SI{1/4}{\teaspoon} salt\\
    1\SI{1/2}{\cup} cold water\\
    3 eggs, lightly beaten \\
    \SI{3}{\ounce} semisweet chocolate \\
    \SI{2}{\tblspoon} butter \\
    \SI{1}{\teaspoon} vanilla extract\\
    1 9-inch pie shell, baked \\
    \SI{1/2}{\cup} whipping cream \\
\end{ingredients}
In a medium saucepan, combine the sugar, cornstarch, and salt. Pour in water
and wisk until blended. Add eggs and chocolate and stir/wisk constantly until
thickened and smooth, about 10-20 minutes. Remove from heat and add vanilla
and butter. Cool a bit, then scoop and smooth into pie shell (NB: this bit
prevents the pudding from making the shell soggy). Refrigerate several hours
or until firm. Whip the cream and spread over chocolate and serve.
\end{entry}
