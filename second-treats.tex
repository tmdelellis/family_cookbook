\secpart{Second Edition}{Treats}

%%---------------------------------------------------------------------------%%
\begin{entry}{Mint Syrup}{Second Edition}
\index{syrup!mint}
\index{Johnson!Weedie}

\begin{open}
  This recipe comes from Louise Johnson of Spruce Head, ME, a.k.a. Weedie, the
  sister of the original Donald Dodge Johnson (Dodge and Julie’s father).
\end{open}
%%
\begin{ingredients}
    fresh spearmint or peppermint leaves\\
    granulated sugar\\
    1 or more lemons\\
    1 or more oranges
\end{ingredients}
Using a strainer that fits into a deep bowl, fill with finely cut fresh
spearmint or peppermint leaves. Boil for ten minutes equal measures of water
and granulated sugar, approximately 1\SI{1/2}{\cup} each. Pour directly over
the mint and allow to cool. Remove the strainer and add the juice of one or
more lemons and the juice of one or more oranges. Wonderful over ice cream, in
iced tea, or on a simple cake aching for a hint of mint.
\end{entry}

%%---------------------------------------------------------------------------%%
\begin{entry}{Southern Pecan Pie}{Second Edition}
\label{sec:pecanpie}
\index{pie!southern pecan}
\index{Johnson!Martha}

\begin{open}
  Hey y'all, from Martha! This is from an ancient, out-of-print Better Homes
  and Garden Cookbook One can double the sin by brushing the crust with a dark
  chocolate glaze before adding the filling, but it's pretty wonderful as is!
  And so easy! Either a pie crust mix or the real thing work equally well. The
  main thing is to have fresh pecans.
\end{open}
%%
\begin{ingredients}
    3 large eggs\\
    \SI{2/3}{\cup} sugar\\
    \SI{1}{\cup} dark corn syrup\\
    \SI{1/3}{\cup} melted butter\\
    \SI{1}{\cup} pecan halves\\
    1 \SI{9}{\inch} unbaked pastry shell
\end{ingredients}
Beat eggs thoroughly with sugar, a dash of salt, corn syrup, and melted butter.
Add pecans. (Brush shell with melted chocolate if desired). Bake in moderate
oven (\SI{350}{\degreeF}) \SI{50}{\minute} or until a knife inserted near center
comes out clean. Cool.
\end{entry}

%%---------------------------------------------------------------------------%%
\begin{entry}{Grapes \`{a} la Creme}{Second Edition}
\index{compote!grapes a la creme}
\index{Horne!Mildred}

\begin{open}
    This is from Paul Horne's mother, Mildred W. Horne, Alexandra, VA, 1959.
\end{open}
%%
\begin{ingredients}
    seedless grapes\\
    sour cream\\
    brown sugar
\end{ingredients}
Wash and de-stem 1\SIrange{1/2}{2}{\pound} of grapes. Drain and place in dessert
dishes, preferably long-stemmed ones.  Spread a tablespoon or so of sour cream
over each mound of grapes and leave in refrigerator for several hours.  An hour
or so before dinner, sprinkle liberally with brown sugar and replace in
refrigerator until dessert time.  Serve with coffee.
\end{entry}

%%---------------------------------------------------------------------------%%
\begin{entry}{Jen's Favorite Sugar Cookies}{Second Edition}
\index{cookies!sugar}
\index{Lindquist!Jen}

\begin{open}
  These are Jen Lindquist's favorite sugar cookies.  Judging by the pictures,
  creativity is a plus!
\end{open}
%%
\begin{ingredients}
    2 sticks butter\\
    \SI{8}{\ounce} cream cheese\\
    1\SI{1/2}{\cup} sugar\\
    3\SI{1/2}{\cup} flour\\
    \SI{1}{\teaspoon} vanilla\\
    \SI{1}{\teaspoon} almond\\
    \SI{1}{\teaspoon} baking powder\\
    1 egg
\end{ingredients}
Mix, roll, cut, and bake at \SI{350}{\degreeF} for \SI{8}{\minute}.  For some
amazing decorating ideas check out Fig.~\ref{fig:sugar-cookie-decorating}.
\begin{figure}
    \centering
    \includegraphics[width=0.7\textwidth]{figures/sugar-cookies.pdf}
    \caption{Sugar cookie decorating ideas.}
    \label{fig:sugar-cookie-decorating}
\end{figure}
%%
\begin{figure}[b]
    \centering
    \includegraphics[width=0.33\textwidth,clip]{figures/jen-making-cookies.jpg}
    \caption{Jen making sugar cookies!}
\end{figure}
\end{entry}

%%---------------------------------------------------------------------------%%
\begin{entry}{Gingerbread, King of Cakes}{Second Edition}
\index{cakes!gingerbread}
\index{Lindquist!Dot}

\begin{open}
    This is one of the many amazing recipes you can find at
    \url{myrunawaykitchen.com}, Dot Lindquist's cooking blog.  We agree that
    gingerbread is, if not the King, at least cake royalty.  This recipe makes
    \numrange{14}{16} slices, takes \SI{25}{\minute} of prep time and about
    \SI{1}{\hour} of cooking time.
\end{open}
%%
\begin{ingredients}
    \SI{1/2}{\cup}  granulated sugar\\
    \SI{1/2}{\cup}  unsalted butter\\
    1 large egg\\
    \SI{1}{\cup} molasses\\
    1 large orange, zested and juiced\\
    2\SI{1/2}{\cup}  all purpose flour\\
    1\SI{1/2}{\teaspoon}  baking soda\\
    \SI{2}{\teaspoon} cinnamon\\
    \SI{2}{\teaspoon} ground ginger\\
    \SI{3/4}{\teaspoon}   ground cloves\\
    \SI{1/2}{\teaspoon}  salt\\
    \SI{1/2}{\cup}  hot water
\end{ingredients}
%%
\begin{figure}
    \centering
    \includegraphics[width=0.8\textwidth]{figures/gingerbread.pdf}
    \caption{Gingerbread, indeed the King of Cakes!}
\end{figure}
%%
Preheat oven to \SI{350}{\degreeF}. Grease and lightly flour the inside of your
baking pan. I like a bundt pan for this recipe, but you could use a
\SI{9x13}{\inch} pan, \SI{9x9}{\inch}, or even make cupcakes. Whatever your
little heart desires. If you're using cupcake papers or a parchment lining in
your pan or baking dish, no need to grease and flour the pan.

If possible, use room-temperature butter. If your room is very cold, or you
forgot to leave a pound of butter on your countertop for baking, pop your butter
into the microwave for \SI{10}{\second} intervals until it's mushy but not
liquid. Cream butter and sugar together in the stand mixer with the paddle
attachment, or with the handheld mixer.

Add egg and molasses and mix, scraping down the sides and bottom of the bowl as
you go. In a separate bowl, sift together the dry ingredients: flour, baking
soda, cinnamon, ginger, cloves and salt. Add orange zest. Add dry ingredients to
the wet ingredients and mix until well blended. Add the \SI{1/2}{\cup} hot water
and the juice from your orange (should be about \SI{1/2}{\cup}, making the total
amount of liquid added in this step equal to one cup).

Bake in preheated oven for one hour or until a toothpick in the center comes out
clean and the cake springs back against your finger when you press into it.

Remove from pan and cool before frosting.
\end{entry}

%%---------------------------------------------------------------------------%%
\begin{entry}{OMG Avocado Chocolate Mousse}{Second Edition}
\index{OMG avocado chocolate mousse}
\index{Lindquist!Dot}

\begin{open}
  From \url{myrunawaykitchen.com}, Dottie says, ``You’re not going to believe
  how simple this is to make and how eye-poppingly great it tastes… and the
  silky texture: OMG.  All you need is a food processor for this one.  Unless
  you like to slather it with real whipped cream, in which case you’ll also
  want an electric mixer of some kind.  I have been known to whip by hand,
  yes, but only when no motorized option is available!''

  Also, as pointed out by Dot, this recipe only uses healthy fats, so enjoy
  almost guilt free; although frankly, we highly encourage the use of all fats
  in this cookbook!
\end{open}
%%
\begin{ingredients}
    4 ripe avocados, peeled and pitted\\
    \SI{8}{\ounce} semisweet chocolate, melted (baker’s chocolate or chips are fine---you can use a double boiler method if you want, but it's ok to melt this stuff, covered, in the microwave in \SI{30}{\second} increments, stirring until smooth)\\
    \SI{6}{\tblspoon} cocoa powder\\
    \SI{1/2}{\cup} milk of any variety\\
    \SI{2}{\teaspoon} vanilla\\
    \SI{1/4}{\teaspoon} salt\\
    \SI{3/4}{\cup} maple syrup
\end{ingredients}
Just combine all ingredients in a food processor and pulse until it’s as smooth
as a chocolate silk dream (Fig.~\ref{fig:mousse}).  (Add maple syrup to the
mixture last, to your desired sweetness level, pulsing until completely
incorporated.)
%%
\begin{figure}
    \centering
    \includegraphics[width=0.6\textwidth]{figures/avocado-choc-mousse.pdf}
    \caption{OMG Avocado Chocolate Mousse!}
    \label{fig:mousse}
\end{figure}
%%

\minisection{Whipped Cream}

\begin{ingredients}
    \SI{1}{\quart} ``heavy'' or ``whipping'' cream\\
    \SI{1/2}{\cup} confectioner's sugar\\
    \SI{2}{\teaspoon} vanilla
\end{ingredients}
Start your mixer on low or medium (so that the cream doesn't splatter all over
creation) and gradually increase the speed as you gradually add sugar and
vanilla.  Beat until soft peaks form.
\end{entry}

%%---------------------------------------------------------------------------%%
\begin{entry}{Rhubarb Raspberry Compote with Mint}{Second Edition}
\index{compote!rhubarb raspberry with mint}
\index{Rona!Alison}

\begin{open}
  A nice compote from Alison Rona that can be used with yogurt, on ice cream,
  or even as part of a fruit torte.
\end{open}
%%
\begin{ingredients}
    stalks of rhubarb\\
    1 banana\\
    1 apple\\
    1 blood orange\\
    fresh ginger\\
    cinnamon\\
    2 cloves\\
    orange peel (zested)\\
    nutmeg\\
    honey\\
    molasses\\
    brown sugar\\
    butter\\
    vanilla extract\\
    orange extract\\
    2 boxes of raspberry\\
    fresh mint leaves
\end{ingredients}
Chop the stalks of rhubarb, then simmer in a cup of water.  Cut a banana,
apple, and blood orange into equal sized chunks and add to the rhubarb.  Then
add minced fresh ginger, a spoonful of cinnamon, 2 cloves, a zested orange peel,
grated nutmeg, a spoonful of honey, molasses, brown sugar, butter, vanilla
extract, orange extract, and 2 boxes of raspberries.  You will need to play with
amounts for taste.  Add lots of fresh chopped mint near the end.

Pour it hot over vanilla ice cream or serve warm with plain goat milk yogurt.
Or, use the compote in a pie crust and add a few pieces of fruit on top!
\end{entry}

%%---------------------------------------------------------------------------%%
\begin{entry}{Chocolate cream cheese cupcakes}{Second Edition}
\index{cakes!chocolate cream cheese cupcakes}
\index{Evans!Fermina}

\begin{open}
  Kate has fond memories of visiting Tom's house when they were dating and
  Tom's mom, Fermina, would bring out these little harmless looking but
  completely addicting mini-cupcakes.
\end{open}
%%
\begin{ingredients}
    1\SI{1/2}{\cup} flour\\
    \SI{1}{\cup} sugar\\
    \SI{1/4}{\cup} cocoa\\
    \SI{1}{\teaspoon} baking soda\\
    \SI{1}{\cup} water\\
    \SI{1}{\teaspoon} vanilla\\
    \SI{1/3}{\cup} oil\\
    \SI{1}{\tblspoon} cider vinegar\\
    1 dozen cupcake liners
\end{ingredients}

\minisection{Topping}

\begin{ingredients}
    \SI{8}{\ounce} cream cheese, softened\\
    1 egg\\
    \SI{1/3}{\cup} sugar\\
    \SI{1/8}{\teaspoon} salt\\
    \SI{3/4}{package} miniature chocolate chips
\end{ingredients}
Preheat oven to (\SI{350}{\degreeF}). In a large bowl, combine the flour,
\SI{1}{\cup} sugar, cocoa, and baking soda. In a separate bowl, mix water,
vanilla, oil, and vinegar and then add to dry ingredients and mix well. Spoon
into liners in muffin tins until \SIrange{1/2}{3/4} full. Blend cream cheese,
egg, \SI{1/3}{\cup} sugar, and salt and mini chips. Place 1 spoonful on top of
chocolate mixture within each liner, and bake \SIrange{20}{23}{\minute}.
\end{entry}

%%---------------------------------------------------------------------------%%
\begin{entry}{Holiday chocolate cream pie}{Second Edition}
\index{pie!chocolate cream}
\index{Evans!Kate}
\index{Johnson!Martha}

\begin{open}
  This easy but delicious pie recipe came from a now out of print cookbook by
  Mable Hoffman titled ``Chocolate Cookery'' that Don got her for Christmas in
  1983. Kate wore it out from heavy use (thanks Don, you nailed it!) and
  recently procured a new copy from a secondhand seller. She makes this pie
  for most holiday events, and just this Christmas figured out how to avoid
  having the crust come out soggy! Regarding the pie shell, like Martha in the
  \ref{sec:pecanpie} recipe, the Betty Crocker mix works well here too.
\end{open}
%%
\begin{ingredients}
    \SI{1}{\cup} sugar\\
    \SI{1/4}{\cup} cornstarch\\
    \SI{1/4}{\teaspoon} salt\\
    1\SI{1/2}{\cup} cold water\\
    3 eggs, lightly beaten \\
    \SI{3}{\ounce} semisweet chocolate \\
    \SI{2}{\tblspoon} butter \\
    \SI{1}{\teaspoon} vanilla extract\\
    1 9-inch pie shell, baked \\
    \SI{1/2}{\cup} whipping cream
\end{ingredients}
In a medium saucepan, combine the sugar, cornstarch, and salt. Pour in water
and whisk until blended. Add eggs and chocolate and stir/wisk constantly until
thickened and smooth, about 10-20 minutes. Remove from heat and add vanilla
and butter. Cool a bit, then scoop and smooth into pie shell (NB: this bit
prevents the pudding from making the shell soggy). Refrigerate several hours
or until firm. Whip the cream and spread over chocolate and serve.
\end{entry}

%%---------------------------------------------------------------------------%%
\begin{entry}{Ricotta Cookies}{Second Edition}
\index{Cordova!Betsy}
\index{Cordova!Rich}
\index{cookies!ricotta}

\begin{open}
    From Betsy Cordova who writes, ``These are a favorite in the Cordova household. It was a recipe given to me by Nana McCarron (Richard's Nani) and I make them every holiday (and whenever anyone asks for them). I have two variations: traditional with icing and chocolate chip.''
\end{open}
%%
\begin{ingredients}
    3 eggs\\
    \SI{2}{\cup} sugar\\
    \SI{1/2}{\pound} butter\\
    \SI{1}{\pound} Ricotta\\
    \SI{2}{\teaspoon} vanilla\\
    mini chocolate chips (for chocolate chip version)\\
    Decorator's sugar (for chocolate chip version)
\end{ingredients}
%%
\minisection{Flour Mixture}
\begin{ingredients}
    \SI{4}{\cup} flour\\
    \SI{1}{\teaspoon} baking soda\\
    \SI{1}{\teaspoon} salt
\end{ingredients}
%%
Preheat the oven to \SI{350}{\degreeF}.  Cream butter and sugar together; mix in
vanilla and eggs. Add Ricotta and mix well. Add Flour Mixture one cup at a time
mixing well after each. If making the chocolate chip variation: add one bag
of mini chocolate chips.

Bake tablespoon-sized amounts on an ungreased cookie sheet for \SI{15}{\minute}.
For the chocolate chip variation, put Decorator's sugar on the tops prior to
baking. Cool on a baking rack (Fig.~\ref{fig:ricotta-cookies}).

For the iced version, the icing ingredients are:
\begin{ingredients}
    \SI{1}{\cup} 10$\boldsymbol\times$ sugar\\
    \SI{1/2}{\teaspoon} vanilla\\
    \SI{1/2}{\cup} milk
\end{ingredients}
Mix all ingredients together and ice cooled cookies---Betsy usually decorates with Nonpareils.
%%
\begin{figure}
    \centering
    \includegraphics[width=0.8\textwidth]{figures/ricotta-cookies.png}
    \caption{Ricotta Cookies!}
    \label{fig:ricotta-cookies}
\end{figure}

\end{entry}

%%---------------------------------------------------------------------------%%
\begin{entry}{Italian Cream}{Second Edition}
\index{Cordova!Betsy}
\index{DeLellis!Evelina}
\index{Evans!Fermina}
\index{Italian cream}

\begin{open}
    This is the Evelina DeLellis (Tom, Betsy, and Katie's Nana) Italian Cream
    recipe with modifications by Fermina presented by Betsy.  We know that
    sounds a little complicated, but the final result is great!  It can be used
    in parfait-style deserts (Fig.~\ref{fig:italian-cream}), as a filling for
    pastries, or just all by itself.
\end{open}
%%
\begin{ingredients}
    \SI{3/4}{\cup} cake flour\\
    1\SI{1/2}{\cup} whole milk\\
    \SI{1}{\cup} sugar\\
    \SI{4}{\teaspoon} butter\\
    6 egg yolks\\
    vanilla for flavoring\\
    \SI{1}{\quart} half \& half
\end{ingredients}
%%
Blend sugar and egg yolks together; add flour (I usually use a hand mixer). Put
in a heavy bottomed sauce pan. Add half \& half and milk (I usually use the hand
mixer in the saucepan to get them blended well). Heat on low while constantly
stirring until thickens. Once thick---add butter and vanilla

\protip{Give yourself plenty of time to make this; it usually takes around an
hour or more}
%%
\begin{figure}[b]
    \centering
    \includegraphics[scale=0.65]{figures/italian-creme.png}
    \caption{Italian cream!}
    \label{fig:italian-cream}
\end{figure}
\end{entry}

%%---------------------------------------------------------------------------%%
%% From the first edition

%%---------------------------------------------------------------------------%%
\begin{entry}{Fermina's Ginger Snaps}{First Edition}
\index{cookies!gingersnaps}
\index{Evans!Fermina}

\begin{open}
  This is a super-yummy cookie recipe sent in from Fermina.  She tells us that
  she always doubles this recipe when making a batch. Maybe the increased
  amounts of ingredients help the taste factor.  Either that or we're just
  gluttons.  Here's the recipe.
\end{open}
\begin{ingredients}
  \SI{3/4}{\cup} of shortening \\
  \SI{1}{\cup} sugar \\
  \SI{1/4}{\cup} light molasses \\
  1 slightly beaten egg \\
  \SI{2}{\cup} flour \\
  \SI{1/4}{\teaspoon}  salt \\
  \SI{1}{\teaspoon}  cinnamon \\
  \SI{2}{\teaspoon} soda \\
  \SI{1}{\teaspoon}  clove \\
  \SI{1/2}{\teaspoon} ginger
\end{ingredients}
Cream shortening and sugar, add molasses and egg.  Mix all dry ingredients.
Stir dry ingredients into creamed mixture.  Spoon into balls. Added step for
yumminess: \textit{Drop spoonfuls into sugar before putting on baking sheet}.
One spray of water before baking.  Bake \SI{350}{\degree} for \numrange{8}{10}
minutes.  The cookies should be split in the middle when finished.
\end{entry}

%%---------------------------------------------------------------------------%%
\begin{entry}{Betsy's Chocolate Chip Poundcake}{First Edition}
\index{cakes!chocolate chip pound cake}
\index{Evans!Betsy}

\begin{open}
  This is the famous Betsy Cordova's Chocolate Chip Cake. When she e-mailed this
  to use she sent a request for many treat recipes.  What we tell her we tell
  all.  Send us a recipe and you get a book.  This serves however many you feel
  like depending on your hunger.
\end{open}
\begin{ingredients}
  \SI{3}{\cup} sugar \\
  2 sticks butter \\
  6 eggs \\
  \SI{3}{\cup}s flour  \\
  1 carton heavy whipping cream (small size) \\
  \SI{2}{\tblspoon}  vanilla \\
  \num{1/2} bag mini chocolate chips
\end{ingredients}
Cream butter and sugar, add 2 eggs and \SI{1}{\cup} flour and beat.  Add 2
eggs and \SI{1}{\cup} flour and beat. Add 2 eggs and \SI{1}{\cup} flour and
beat.  Mix in vanilla and whipping cream and add chocolate chips.

Bake in greased and floured bundt pan at \SI{350}{\degree} for 60 to 75
minutes (depending on the temperature of your oven).
\begin{center}
\includegraphics[scale=.5,clip]{figures/pound.pdf}
\end{center}
\end{entry}

%%---------------------------------------------------------------------------%%
\begin{entry}{Amy's Cheesecake}{First Edition}
\index{cakes!cheesecake}
\index{DeLellis, Amelia}

\begin{open}
  This is a holiday favorite at the Evans/DeLellis households by Amelia
  DeLellis.
\end{open}
\begin{ingredients}
  1 box Graham Cracker Crumbs \\
  \SI{1/2}{\cup} sugar \\
  \SI{2}{\tblspoon}  flour \\
  \SI{1/4}{\teaspoon}  salt \\
  \SI{1}{\pound} cream cheese \\
  \SI{1}{\teaspoon}  vanilla extract \\
  4 eggs \\
  \SI{1}{\cup} heavy cream
\end{ingredients}
The topping ingredients are:
\begin{ingredients}
  \SI{2}{\cup}s sour cream \\
  \SI{3}{\tblspoon}  sugar \\
  \SI{1}{\teaspoon} vanilla
\end{ingredients}
Follow the directions on the Graham Cracker Box for the crust.  Use a
\SI{9}{\inch} spring form pan.  Press crumb mixture into the bottom and sides
of the pan.

Let cream cheese soften at room temperature (or use microwave).  Mix sugar,
flour, and salt.  Add dry ingredients to cream cheese.  Cream together with
low speed beater or by hand.  Separate eggs, save the whites in a clean bowl.
Add yolks to cream cheese mixture and beat until smooth.  Add vanilla.  Stir
in cream.  Beat egg whites until stiff.  Fold into cream cheese mixture.  Pour
on top of crumbs.  Bake at \SI{350}{\degree} for 1 hour.  Let cool.  Mix
topping ingredients.  Pour topping onto cheesecake and bake at
\SI{500}{\degree} for 10 minutes.  Serve with cherry, blueberry,
etc. etc. toppings.
\end{entry}

%%---------------------------------------------------------------------------%%
\begin{entry}{Pineapple Upside-down Cake}{First Edition}
\index{cakes!pineapple upside-down cake}
\index{Evans!Fermina}

\begin{open}
  Kate: On his/her birthday most kids I knew asked for chocolate cake, or
  ice-cream cake, or even cheesecake if sophisticated. But not Tom.  Tom always
  begged for this somewhat unusual birthday cake. Luckily, Fermina Evans has a
  great recipe for it!  Tom: I begged for it because it's delicious. Any kid
  would agree.
\end{open}
\begin{ingredients}
  \SI{1/2}{\cup} butter \\
  \SI{1/2}{\cup} packed brown sugar \\
  1 large can pineapple slices in syrup \\
  1 small jar maraschino cherries
  \SI{1/2}[1]{\cup} non packed flour (softasilk flour recommended) \\
  \SI{1}{\cup} sugar \\
  \SI{2}{\teaspoon} baking powder \\
  \SI{1/2}{\teaspoon} lt \\
  \SI{1/3}{\cup} soft shortening \\
  \SI{2/3}{\cup} milk \\
  \SI{1}{\teaspoon} vanilla \\
  1 large egg
\end{ingredients}
Melt butter with brown sugar in \SI{9}{\inch} baking pan (Fermina adds
\SI{1}{\tblspoon} Karo syrup here) Arrange pineapple slices on top of syrup
and place cherries in pineapple centers or wherever they look nice.

In mixing bowl, stir flour, sugar, baking powder and salt. Add shortening,
milk, and vanilla. Beat 2 minutes at medium speed with electric mixer. Add egg
and beat two more minutes. Pour batter over fruit. Bake at \SI{350}{\degree}
for \numrange{40}{50} minutes. Immediately turn upside down on serving dish
(if you don't, sugar will crystallize to pan and you will have a mess).
\end{entry}

%%---------------------------------------------------------------------------%%
\begin{entry}{Chocolate Mint Brownies}{First Edition}
\index{brownies!chocolate mint brownies}
\index{Evans!Fermina}

\begin{open}
  Kate: Blah blah blah chocolate blah. Tom: These are yummy!  My mom makes
  them.
\end{open}
\begin{ingredients}
  \SI{1}{\cup} sugar\\
  \SI{1/2}{\cup} butter or margarine\\
  4 eggs, beaten\\
  \SI{1}{\cup} flour\\
  \SI{1/2}{\teaspoon} salt\\
  1 can \corp{Hershey's Chocolate Syrup} (\SI{16}{\ounce})\\
  \SI{1}{\teaspoon} vanilla
\end{ingredients}
Mix together above ingredients and put in a greased \SI{9x13}{\inch} pan.
Bake at \SI{350}{\degree} for 30 minutes.

The middle layer ingredients are:
\begin{ingredients}
  \SI{2}{\cup} powdered sugar\\
  \SI{1/2}{\cup} butter or margarine\\
  \SI{2}{\tblspoon} \corp{Creme de Menthe} (preferably the green kind)
\end{ingredients}
Mix and spread over cooled cake.

The glaze ingredients are:
\begin{ingredients}
  \SI{1}{\cup} chocolate chips\\
  \SI{6}{\tblspoon} butter
\end{ingredients}
Let the cake cool slightly and spread over brownies.  Chill and cut into
squares.
\end{entry}

%%---------------------------------------------------------------------------%%
\begin{entry}{Gooey Butter Cake}{First Edition}
\index{cakes!gooey butter cake}
\index{Lefkowith, Pam}

\begin{open}
  A recipe from Fermi's friend Pam L.
\end{open}
\begin{ingredients}
  1 pkg. yellow cake mix\\
  \SI{1/2}{\cup} melted butter\\
  1 egg
\end{ingredients}
The topping ingredients are:
\begin{ingredients}
  \SI{8}{\ounce} cream cheese (1 pkg.)\\
  2 eggs\\
  1 box powdered sugar
\end{ingredients}
Preheat oven to \SI{350}{\degree}.  Mix together cake ingredients.  Pat into
\SI{9x13}{\inch} pan.  Beat topping ingredients together for three minutes.
Pour over cake mix.  Bake for 40 minutes.  Do not over-bake. Top should be
set, but not dry.
\end{entry}

%%---------------------------------------------------------------------------%%
\begin{entry}{Chocolate Surprise}{First Edition}
\index{cakes!chocolate surprise}
\index{Evans!Fermina}

\begin{open}
  Fermina's chocolate \& angel food cake.  Geo's birthday favorite.
\end{open}
\begin{ingredients}
  32 large marshmellows\\
  \SI{1/3}{\cup} water\\
  \SI{1/4}{\teaspoon} salt\\
  \SI{6}{\ounce} semi-sweet chocolate chips\\
  heavy whipping cream and sugar\\
  \SI{1/4}{\teaspoon} vanilla
\end{ingredients}
In sauce pan melt marshmallows, water, and salt.  Add chocolate bits.  Stir
until melted.  Let cool.

\begin{wrapfigure}{R}{.25\textwidth}
\centering\includegraphics[width=.22\textwidth,clip]{figures/kiss.pdf}
\end{wrapfigure}

Whip \SI{1}{\cup} whipping cream.  Pour chocolate over cream and fold
together.

Take a store brought or home make angel food cake.  Cut off entire top
\SI{1}{\inch} layer.  With spoon dig a tunnel in remaining cake.  Fill with
chocolate surprise.  Replace top layer.

Sprinkle cake with powdered sugar or frost with whipped cream (\SI{1}{\cup}
heavy cream beaten with \SI{1}{\tblspoon} sugar until stiff).
\end{entry}

%%---------------------------------------------------------------------------%%
\begin{entry}{Tiramisu}{First Edition}
\index{cakes!tiramisu}
\index{Horne!Mimi}

\begin{open}
  Mimi Horne brings us this delicious treat from an Italophile Brazilian Yale
  Art History Professor friend, Ester da Costa Meyer.  Those in less
  gastro-enlightened regions might need to replace the mascarpone with cream
  cheese and cream, and the Marsala with perhaps port or sherry.
\end{open}
\begin{ingredients}
  2-3 pkgs (about 24-30) lady fingers (boudoirs) \\
  \SI{3}{\cup} mascarpone (or part creme fraiche, part carre frais mushed together)\\
  3 egg yolks \\
  \SI{1/3}{\cup} sugar \\
  \SI{2}{\cup} strong coffee \\
  \SI{1/2}{\cup} or more Marsala wine \\
  \SI{1/2}{\cup} cocoa
\end{ingredients}
Prepare coffee and mix with Marsala. One at a time, dip lady fingers in
mixture briefly, then lay them in a row in an approx. \SI{10x18x2.5}{\inch}
deep serving dish.  Cut some to fit the remaining space in dish so that the
bottom is completely covered. Mix sugar, egg yolks and
mascarpone/cream. Spread about half the mixture over the first layer of
cookies to cover completely.  Dip more lady fingers in coffee/Marsala and lay
them over cream to form the next layer; cover remaining cream mixture. Dust
the top thoroughly with cocoa; chill overnight or for several hours before
serving. More cocoa may be added before serving. The texture can be made
lighter by beating the egg whites and folding them into the mascarpone
mixture, which also increases the amount.  Serves \numrange{6}{8}.
\end{entry}

%%---------------------------------------------------------------------------%%
\begin{entry}{Tortelettes}{First Edition}
\index{cookies!tortelettes}
\index{Niepold!Martha}

\begin{open}
  Another Nonnie/Grandpatty and Joy of Cooking original. Niepold kids remember
  Christmas at Lee St. when they eat Tortelettes and California dates stuffed
  with Georgia Pecans and rolled in confectioners' sugar.
\end{open}
\begin{ingredients}
  1 grated lemon rind\\
  \SI{1}{\cup} sugar\\
  \SI{3/4}{\cup} butter\\
  2 egg yolks \\
  \SI{1/2}{\cup} bread flour \\
  \SI{1}{\cup} blanched and shredded almonds or pecan pieces \\
  \SI{1/3}{\cup} sugar \\
  \SI{1}{\teaspoon} cinnamon\\
  \SI{1/4}{\teaspoon} nutmeg\\
  \SI{1/8}{\teaspoon} salt\\
  1 egg white\\
  \SI{1}{\tblspoon} water
\end{ingredients}
Preheat oven to \SI{375}{\degreeF}. Grate lemon into sugar. Cream sugar with
butter and beat in the egg yolks one at a time. Add flour gradually to make a
stiff dough. Pinch off about a teaspoonful of dough at a time. Roll it into a
ball and flatten on cookie sheet until very thin. Prepare nuts and combine
with next 4 ingredients (spices). Beat the egg white and water together
slightly.  Brush the cakes with the egg white mixture, then sprinkle nut/spice
mixture and bake until light brown.
\end{entry}

%%---------------------------------------------------------------------------%%
\begin{entry}{Lime Cream Pie}{First Edition}
\index{pie!lime cream pie}
\index{Evans!Kate}

\begin{open}
  Kate got this recipe from Edie, a receptionist at Bryn Mawr College with
  southern cooking blood.  It's very easy and delicious, especially after a
  rich meal. It's cool, refreshing, and slides right down.
\end{open}
\begin{ingredients}
  1 \corp{Graham Cracker Pie Crust}\\
  3 egg yolks\\
  \SI{2/3}[2]{\cup} sweetened condensed milk (2 cans)\\
  \SI{1}{\cup} plus \SI{2}{\tblspoon} lime juice (about 7 limes if you're
  squeezing)\\
  \SI{2}{\teaspoon} grated lime zest\\
  1 attractive lime
\end{ingredients}
Lightly whisk egg yolks in mixing bowl. Pour in condensed milk and whisk until
completely blended.  Add lime juice and zest and whisk to blend. Gently pour
filling into pie crust shell and smooth over top.  Refrigerate for at least 4
hours (don't skimp or it will be soup!). Slice the attractive lime paper thin
to garnish. I like to slice into half-circles and create a pinwheel pattern
around the center.
\end{entry}