\secpart{Second Edition}{Entrees}

%%---------------------------------------------------------------------------%%
\begin{entry}{Mushroom Cream Pasta
\index{pasta!mushroom cream}}
\index{Johnson!Gege}
\index{Johnson!Don}

\begin{open}
    Lightly adapted, mostly for convenience, from ``Chef John's Creamy Mushroom
    Pasta'' on \url{allrecipes.com}, My family (who are NOT vegetarians) are
    totally satisfied with this quick and easy dish as a main course. From: Don
    and Gege.
\end{open}
\begin{ingredients}
    \SI{1/2}{\pound} linguine, fettuccine or other pasta\\
    \SI{1}{\pound} mushrooms. I like a mix of white and shiitake, but baby
    portobello also work well\\
    Olive oil, salt pepper\\
    Garlic\\
    \SI{1}{\tblspoon} sherry (For some reason, I more often have dry vermouth
    which seems to work fine.) \\
    Chicken stock (optional)\\
    \SI{1}{\cup} heavy whipping cream\\
    \SI{1/2}{\cup} grated parmesan cheese\\
    Fresh chopped thyme, chives, tarragon (n.b. I have never added these. I'm
    sure  they would make it better.)
\end{ingredients}
\Saute sliced mushrooms in olive oil until they are tender and release their
liquid. Add several cloves of diced garlic. Add sherry followed by heavy cream,
and lick residual cream from cup measure. Add salt and pepper and simmer cream
until the mixture thickens a bit and foams---though it does not get very thick;
if it does add some chicken stock. When the mixture reaches reasonable
consistency, stir in the fresh spices, turn off the heat and mix in the parmesan
cheese. Stir the mushroom cream mixture into the pasta and serve.
\end{entry}

%%---------------------------------------------------------------------------%%
\begin{entry}{Broccolini and Chourico Portuguese Sausage}
\index{sausage!broccolini \& Chourico sausage}
\index{Lindquist!Julie}

\begin{open}
    This is one of Julie's Simple/Quick/Tasty entries.  Chourico is similar to
    Spanish Chorizo, and I use only Gaspar’s (and have found the meat department
    variety of chourico isn’t as flavorful and spicy)---see
    Fig.~\ref{fig:chourico}. This is my go-to dinner!

    As a historical side-note, broccolini was developed in Japan and is a hybrid
    between broccoli and Chinese broccoli (Chinese kale).
\end{open}
%%
\begin{figure}
  \centering
  \includegraphics[width=0.6\textwidth]{figures/broccolini-chourizo.pdf}
  \caption{Gaspar's chourico and broccolini.}
  \label{fig:chourico}
\end{figure}
%%
\begin{ingredients}
    Chourico (Gaspar's brand)\\
    broccolini
\end{ingredients}
Cut up raw broccolini (stems especially are so much sweeter than broccoli) and
place in wide soup bowl; microwave Chourico at 8 power for about
\SI{1.5}{\minute} to heat the meat through then cut into bite-size pieces.
Place on top of the broccolini and enjoy with a hearty red (I find a Malbec the
best accompaniment); some warmed sourdough baguette on the side or afterwards
with an excellent creamy French cheese doesn’t hurt.
\end{entry}

%%---------------------------------------------------------------------------%%
\begin{entry}{Portobello Caps Stuffed with Crab
\index{seafood!crab stuffed portobello caps}}
\index{Lindquist!Julie}

\begin{open}
  From Julie Lindquist; this dish harmonizes well with a medium red or a
  hearty semi-sweet white. Tasty as an hors d'oeuvre but also great for a main
  dish.
\end{open}
%%
\begin{ingredients}
    Portobello mushrooms\\
    fresh crabmeat\\
    red pepper\\
    parsley
\end{ingredients}
Using large or medium Portobellos, remove stems. Stuff caps with fresh
crabmeat (preferably not previously frozen; canned just doesn't do it!). Place
under broiler for a few minutes (5 or so) until lightly browned. Top with a
few slices of red pepper and a little curly parsley for color. Season with
salt and pepper at the table as desired.  (One could also add a little
Hellman’s or homemade mayonnaise and/or finely chopped celery to the crab
before stuffing.)
\end{entry}

%%---------------------------------------------------------------------------%%
\begin{entry}{Nothing but Crab Cakes}
\index{seafood!nothing but crab cakes}
\index{Johnson!Martha}

\begin{open}
  From Martha, this recipe is adapted from the Paoli Auxiliary cookbook
  ``Quilted Cuisine.'' Being able to get real Chesapeake Bay crab occasionally
  means I make these probably way too often. But so easy! Serves 4, so often
  just make half a recipe for the two of us. (NB: Kate decided we should do
  this cookbook crab mostly so we could get our hands on this recipe. Thanks,
  Mom!)
\end{open}
%%
\begin{figure}
    \centering
    \includegraphics[width=4in]{figures/crab_cakes.jpg}
    \caption{Nothing but crab cakes!}
    \label{fig:crab-cakes}
\end{figure}
%%
\begin{ingredients}
    \SI{1}{\pound} crab meat (lump works best)\\
    1 egg\\
    \SI{1}{\tblspoon} chopped parsley\\
    \SI{1}{\tblspoon} mayonnaise\\
    \SI{2}{\teaspoon} Worcestershire sauce\\
    \SI{1}{\tblspoon} melted butter\\
    \SI{1}{\teaspoon} dry mustard \\
    pinch salt\\
    ground pepper\\
    big slurp of Tabasco\\
    onion powder
\end{ingredients}
Combine all these ingredients and mix well. Probably be quite moist. Shape
into patties (I like small ones for ease of turning). Coat with Panko or
regular bread crumbs mixed with a little Parmesan. Chill until ready to
cook. \Saute in medium hot olive oil until brown on both sides.  Serve on plate with lemon and parsley leaves (Fig.~\ref{fig:crab-cakes}.)
\end{entry}

%%---------------------------------------------------------------------------%%
\begin{entry}{``Manhattan'' Pancakes}
\index{breakfast!manhattan pancakes}
\index{Johnson!Lizzie}
\index{Johnson!Don}

\begin{open}
  Since this was sent to us from Lizzie Johnson and she didn't give us a
  title, we decided to call them ``Manhattan'' pancakes. She says, ``we use
  this recipe for breakfast almost every Saturday morning, it is always very
  good. Dad (Don Johnson) found it on \url{allrecipes.com}.''
\end{open}
%%
\begin{ingredients}
    \SI{1/2}{\cup} of flour\\
    \SI{1}{\tblspoon} of sugar\\
    1\SI{1/2}{\teaspoon} of baking powder\\
    \SI{1}{\teaspoon} of salt\\
    \SI{1/4}{\cup} of milk\\
    1 egg\\
    \SI{3}{\tblspoon} of butter
\end{ingredients}
Put all the ingredients in a bowl, the dry ones first, and mix until no more
lumps. Heat a skillet to medium. ladle some batter into the pan, and flip it
when the surface bubbles. Keep doing that until you run out of batter, then
eat.
\end{entry}

%%---------------------------------------------------------------------------%%
\begin{entry}{Lettuce-Wrapped Fish}
\index{seafood!lettuce-wrapped fish}
\index{Horne!Mimi}

\begin{open}
  From Mimi Horne, this recipe yields 4 servings and takes about 30 minutes to
  prepare.
\end{open}
%%
\begin{ingredients}
    Salt and freshly ground black pepper\\
    Several big leaves of romaine lettuce, Bibb lettuce or white cabbage\\
    1\SI{1/2}{\pound} thick white fish fillet (rockfish, cod, hake, snapper), in
    pieces about \SIrange[range-phrase={ to }]{3/4}{1}{\inch} thick,
    \SI{1}{\inch} wide, and \SI{2}{\inch} or less across\\
    \SI{1}{\cup} white wine\\
    \SIrange[range-phrase={ to }]{2}{3}{\tblspoon} butter
\end{ingredients}
Bring a large pot of water to a boil and salt it. Take as many big, intact
leaves of lettuce or cabbage as you have pieces of fish. With large outer
leaves, cut out center veins 2 to 3 inches up from bottom of leaves, to the
point where the leaf is more pliable; with inner leaves this may not be
necessary. One or 2 at a time, blanch leaves in boiling water until they are
tender and flexible, 30 seconds to a minute. Remove and drain on paper towels.

Put a piece of fish on each leaf and sprinkle with salt and pepper; fold or roll
fish in leaf so edges overlap. It is not important to make a tight seal, but it
is nice if package covers all the fish. When done, you can cover and refrigerate
packages until ready to serve, or continue.

In a large, broad skillet or casserole with a cover, bring wine to a boil with
butter. Reduce heat to a simmer and add fish packages. Cover and simmer until a
thin-bladed knife easily penetrates fish, \SIrange{5}{10}{\minute}. Remove fish
to a warm platter.

Over high heat, quickly reduce liquid in skillet; it is likely there will be
more than there was when you started. When it is thickened a bit, pour over fish
and serve.
\end{entry}

%%---------------------------------------------------------------------------%%
\begin{entry}{Bachelor Days Stir Fry}
\index{stir-fry!bachelor days stir fry}
\index{Lindquist!David}

\begin{open}
  Here's a recipe from Dave Lindquist, it sounds like it's from days in the
  Trauma Ward. Any vegetables can be omitted or substituted. Leftovers get
  combined in Tupperware or big yogurt container for tomorrow's lunch!
\end{open}
%%
\begin{ingredients}
    Rice (enough for leftovers for lunch)\\
    1 onion\\
    \numrange{2}{5} garlic cloves, depending on preference\\
    ginger root\\
    \numrange{1}{2} carrots\\
    1 handful of snow peas\\
    1 small/medium zucchini or yellow squash\\
    1 red pepper\\
    Protein: shrimp, chicken cutlets, beef strips, or tofu\\
    Spices as desired (salt, pepper, soy sauce, cumin, curry, etc.)\\
    Cooking oil
\end{ingredients}
Heat water for rice. Start low heat to cast iron skillet or wok. Mince garlic
coarsely. Begin chopping vegetables to bite-sized pieces, exact shape is up to
the chef. (If you chop onions last, your crying will interfere less with the
food prep.)

By now water for rice should be boiling. Add rice and simmer as per directions
on container. Usually \SIrange{15}{20}{\minute}. Turn up heat on skillet or
wok.

Finish chopping vegetables and protein. Lightly oil skillet/wok. It should be
smoking gently. \Saute onions, then peppers, then carrots, then squash, then
snow peas. Avoid overloading skillet/wok; it will stay hotter via cooking
smaller batches. Gently season each batch of vegetables as they cook. Combine
each batch of \sauteed vegetables to a large mixing bowl.

\Saute protein. Just before protein is ready, lower the heat, add the garlic,
and grate ginger into the skillet/wok. Add back in all the previously cooked
vegetables. Complete seasoning to taste. A dash of hot pepper will add some
kick.

By now your rice should be ready. Add salt, butter to taste. Fluff.

To serve, I prefer a bed of rice with the veggies and protein on top, drizzled
with soy sauce.
\end{entry}

%%---------------------------------------------------------------------------%%
\begin{entry}{Fish \`{a} la Dave}
\index{seafood!fish \`{a} la Dave}
\index{Lindquist!David}

\begin{open}
    Another quick and easy dish featuring soy sauce from Dave Lindquist.
\end{open}
%%
\begin{ingredients}
    Rice (enough for leftovers)\\
    \SIrange{0.5}{2}{\pound} cod, haddock, or other whitefish, depending on
    number of servings desired\\
    1 red pepper\\
    1 yellow pepper\\
    1 onion\\
    \numrange{3}{6} garlic cloves\\
    ginger root (sensing a theme here?)\\
    1 zucchini and/or yellow squash\\
    soy sauce\\
    olive oil
\end{ingredients}
Heat water for rice. Preheat oven to \SIrange{300}{325}{\degreeF} (depends on
how hot your oven runs). Chop all vegetables into bite-sized pieces. Exact shape
is at chef's discretion. Coarsely chop garlic Lay fish in baking pan---use a
large baking dish with cover (or aluminum foil). Sprinkle garlic liberally on
fish Loosely place vegetables around fish; it's ok if they overflow onto the
fish. Drizzle olive oil all over fish and vegetables. Sprinkle with soy sauce.
Sprinkle with freshly ground pepper. Place fish in oven with cover or foil.

By now water for rice should be boiling. Add rice and simmer as per packaging
directions (usually \SIrange{15}{20}{\minute}). Fish should cook for
approximately \SI{20}{\minute}, or until it looks firm and flaky. Remove cover
to speed cooking, if needed.

Serve fish and vegetables over bed of rice. Add soy sauce if desired.
\end{entry}

%%---------------------------------------------------------------------------%%
\begin{entry}{Pasta with Anchovies}
\index{pasta!pasta w/ anchovies}
\index{seafood!pasta w/ anchovies}
\index{Rona!Alison}

\begin{open}
  From Alison Rona, she likes to use two tins of anchovies with enough short
  fusilli pasta for one lunch and one dinner.
\end{open}
%%
\begin{ingredients}
    anchovy filets in olive oil\\
    \numrange{2}{5} garlic cloves (to taste), chopped\\
    chili pepper flakes\\
    olives, chopped\\
    anchovy paste\\
    flat-leaf parsley, chopped
\end{ingredients}
Finely chop the anchovy filets and briefly \saute with the garlic, chili
pepper, olives, and anchovy paste while cooking the pasta.  Drain the pasta
and toss in the anchovy mixture with lots of parsley.
\end{entry}

%%---------------------------------------------------------------------------%%
\begin{entry}{Swedish insanity meatballs}
\index{turkey!swedish meatballs}
\index{Evans!Kate}

\begin{open}
  Kate normally avoids meatballs to avoid eating too much beef (and onions),
  but turkey meatballs just don't have the same zing (a.k.a. fat). So she
  altered a recipe from fine cooking to arrive at these rich---yet
  non-beef---meatballs. Serve with german egg noodles and a sweet side veggie
  like honey carrots or green beans for a complete meal! As with all of Kate's
  recipes, if you see onion powder in the ingredients, feel free to replace
  with \SI{1/2}cup minced onion.
\end{open}
%%
\begin{ingredients}
    3 slides whole wheat bread (approx.)\\
    \SI{1/4}{\cup} milk (whole or 2\%) \\
    \SI{12}{\ounce} ground turkey, 93 or 99\% fat free\\
    \SI{12}{\ounce} pork sausage\\
    \SI{1}{\teaspoon} onion powder\\
    1 large egg, lightly beaten\\
    \SI{1/4}{\teaspoon} ground allspice\\
    \SI{1/4}{\teaspoon} ground nutmeg\\
    \SI{1/2}{\teaspoon} each of Kosher salt and pepper\\
    \SI{2}{\tblspoon} butter\\
    \SI{2}{\tblspoon} olive oil\\
    \SI{1}{\tblspoon} flour\\
    \SI{1}{\cup} lower-salt chicken broth\\
    \SI{1/4}{\cup} heavy cream
\end{ingredients}
In a stand mixer fitted with a paddle attachment (or in a bowl and use your
hands---works great!), soak the bread in the milk until softened, about 5
minutes. Mix on low speed until uniform, about 30 seconds. Add the turkey,
pork, onion powder, egg, allspice, nutmeg, salt and pepper and mix on low
speed until evenly combined. Using your hands (a bowl of water nearby helps
keep them from getting too gummed up), form about 25 \SI{1.5}{\inch} sized
balls. Heat a large skillet over medium-high heat and add \SI{1}{\tblspoon}
each of the butter and oil. As soon as butter is melted, add half the
meatballs, turning very several minutes on several sides until browned, then
remove to a plate on the side. Repeat with the rest of the meatballs. Then
turn skillet heat to medium and add the other \SI{1}{\tblspoon} each of the
butter and oil. Add the flour and whisk it in with the fat until smooth. Whisk
in the chicken broth, then the cream, in small batches. The roux will first
become firmer as liquid is added but then become a thin sauce. Return all the
meatballs to the skillet and reduce the heat slightly, cover, and cook until
meatballs are cooked through and sauce thickens, about
\SIrange{8}{10}{\minute}. Season the sauce as needed with salt and pepper.
\end{entry}

%%---------------------------------------------------------------------------%%
\begin{entry}{Six Cheese Lasagne}
\index{pasta!six cheese lasagne}
\index{beef!six cheese lasagne}
\index{turkey!six cheese lasagne}
\index{Evans!Kate}

\begin{open}
  This is not a typo. Repeat. This is not a typo. It turns out that adding 6
  different cheeses to a traditional lasagne recipe makes it awesome. This
  recipe was taken from a 1992 Southern living cookbook Kate received as a
  wedding shower gift. I am looking at you, Fermina lunch buddies!
\end{open}

\begin{ingredients}
\numrange{9}{12} lasagne noodles, cooked according to package directions\\
\SI{1/2}{\cup} sharp Cheddar cheese, shredded\\
\SI{1/2}{\cup} Romano cheese, grated\\
\SI{1/2}{\cup} Parmesan cheese, grated \\
\SI{8}{\ounce} Mozzarella cheese, sliced
\end{ingredients}

\minisection{Tomato sauce}
\begin{ingredients}
    \SI{1}{\tblspoon} olive oil\\
    1 clove garlic, minced\\
    \SI{1}{\ounce} ground beef or turkey \\
    \SI{1}{\teaspoon} onion powder\\
    1 \SI{12}{\ounce} can tomato paste\\
    \SI{8}{\ounce} sour cream\\
    1 \SI{16}{\ounce} can tomato sauce\\
    1\SI{1/2}{\cup} water\\
    \SI{1}{\tblspoon} dried basil\\
    \SI{1}{\teaspoon} salt\\
    \SI{1/2}{\teaspoon} dried rosemary\\
    2 bay leaves
\end{ingredients}

\minisection{Cheese layer}
\begin{ingredients}
    2 large eggs, lightly beaten\\
    \SI{2}{\cup} ricotta cheese\\
    \SI{8}{\ounce} sour cream \\
    \SI{1/4}{\cup} chopped parsley, or \SI{2}{\tblspoon} dried\\
    \SI{1/2}{\teaspoon} salt\\
    \SI{1/4}{\teaspoon} pepper
\end{ingredients}
%
\protip{An acceptable weeknight cheat is to use \SI{1}{\cup} of the green
  cylinder grated combo Parmesan and Romano cheese instead of grating your own
  separately.}
%
Heat oil over medium heat in large skillet or saucepan. Add garlic and
stir-fry briefly until browned, then add ground meat and \saute until
cooked. Add tomato paste, tomato sauce, water, basil, salt, rosemary, and bay
leaves and stir gently. Bring to a boil, then reduce heat and cover and cook
until its time to use in the lasagna, anytime more than 15 minutes or so. In a
separate and medium sized bowl, add the cheese layer ingredients and stir to
combine. Preheat oven to \SI{375}{\degreeF}. Next, arrange 3-4 lasagna noodles
(depending if you have 9 or 12) onto bottom of greased \SI{9x13}{\inch} glass
baking pan. Layer with \SI{1/3} of the tomato sauce, \SI{1/3} or the cheese
layer, then \SI{1/3} each of the Cheddar, Romano, and Parmesan cheeses. Repeat
two more times. Place on middle shelf of oven and bake until bubbly, about
\SIrange{30}{35}{minutes}. Then take out and arrange the Mozzarella slices
along the top and return to oven until Mozzarella is melted and starting to
bubble and brown, about \SIrange{5}{10}{minutes}.
\end{entry}

%%---------------------------------------------------------------------------%%
\begin{entry}{Whole beef tenderloin}
\index{beef!whole tenderloin}
\index{Evans!Fermina}

\begin{open}
  Every home cook should have this dish in their arsenal for when you have
  company and want to look like you are a chef. Its deceptively easy (but plan
  ahead time wise!) and makes a great presentation piece.
\end{open}

\begin{ingredients}
    \SI{1/2}{\cup} fresh parsley or mint, chopped\\
    \SI{1}{\cup} red wine\\
    \SI{1/2}{\cup} soy sauce\\
    \SI{1/4}{\cup} Worcestershire sauce\\
    \SI{2}{\tblspoon} fresh rosemary or \SI{2}{\teaspoon} dry\\
    4 cloves garlic, chopped\\
    black pepper
\end{ingredients}
%
\protip{Fermina notes that it's cheaper per pound to get a whole one and trim
  it, but if you are new to working with tenderloin, its perhaps safer to get
  one that is already trimmed.}
%
The day before, combine all ingredients but beef in a small bowl and add with
beef into a plastic bag to marinate the beef overnight in the
refrigerator. Remove from fridge, place beef on roasting dish, discard
marinade, and keep out at room temp at least two hours before cooking. Preheat
oven to \SI{450}{\degreeF}. Roast for 15 minutes at \SI{450}{\degreeF}, then
reduce temperature to 350 and roast for another 15 minutes. Remove from oven
and double wrap in heavy foil before serving.
%
\protip{For a rare center, the rule of thumb is that the meat thermometer in
  the center registers \SI{450}{\degreeF}.}
\end{entry}