\chapter{Entrees}

\section{Mushroom Cream Pasta\index{entrees!mushroom cream pasta}}

\begin{open}
    Lightly adapted, mostly for convenience, from ``Chef John's Creamy Mushroom
    Pasta'' on \url{allrecipes.com}, My family (who are NOT vegetarians) are
    totally satisfied with this quick and easy dish as a main course. From: Don
    and Gege.
\end{open}
\begin{ingredients}
    \SI{1/2}{\pound} linguine, fettuccine or other pasta\\
    \SI{1}{\pound} mushrooms. I like a mix of white and shiitake, but baby
    portobello also work well\\
    Olive oil, salt pepper\\
    Garlic\\
    \SI{1}{\tblspoon} sherry (For some reason, I more often have dry vermouth
    which seems to work fine.) \\
    Chicken stock (optional)\\
    \SI{1}{\cup} heavy whipping cream\\
    \SI{1/2}{\cup} grated parmesan cheese\\
    Fresh chopped thyme, chives, tarragon (n.b. I have never added these. I'm
    sure  they would make it better.)\\
\end{ingredients}
Saute sliced mushrooms in olive oil until they are tender and release their
liquid. Add several cloves of diced garlic. Add sherry followed by heavy cream,
and lick residual cream from cup measure. Add salt and pepper and simmer cream
until the mixture thickens a bit and foams---though it does not get very thick;
if it does add some chicken stock. When the mixture reaches reasonable
consistency, stir in the fresh spices, turn off the heat and mix in the parmesan
cheese. Stir the mushroom cream mixture into the pasta and serve.

\section{Broccolini and Chourico Portuguese Sausage\index{entrees!broccolini Chourico Portuguese sausage}}

\begin{open}
    This is one of Julie's Simple/Quick/Tasty entries.  Chourico is similar to
    Spanish Chorizo, and I use only Gaspar’s (and have found the meat dept.
    variety of chourico isn’t as flavorful and spicy). This is my go-to dinner!

    As a historical side-note, broccolini was developed in Japan and is a hybrid
    between broccoli and Chinese broccoli (Chinese kale).
\end{open}
%%
\begin{figure}[h]
\centering
    \begin{subfigure}[c]{0.25\textwidth}
        \includegraphics[width=\textwidth]{figures/chourico}
        \caption{Chourico}
    \end{subfigure}
    \begin{subfigure}[c]{0.25\textwidth}
        \includegraphics[width=\textwidth]{figures/broccolini}
        \caption{Broccolini}
    \end{subfigure}
\end{figure}
%%
\begin{ingredients}
    Chourico (Gaspar's brand)\\
    broccolini\\
\end{ingredients}
Cut up raw broccolini (stems especially are so much sweeter than broccoli) and
place in wide soup bowl; microwave Chourico at 8 power for about
\SI{1.5}{\minute} to heat the meat through then cut into bite-size pieces.
Place on top of the broccolini and enjoy with a hearty red (I find a Malbec the
best accompaniment); some warmed sourdough baguette on the side or afterwards
with an excellent creamy French cheese doesn’t hurt.

\section{Portobello Caps Stuffed with Crab\index{entrees!portobello caps stuffed
crab}}

\begin{open}
    From Julie Lindquist; this dish harmonizes well with a medium red or a hearty semi-sweet white. Tasty as an hors d'oeuvre but great for a main dish.
\end{open}
%%
\begin{ingredients}
    Portobello mushrooms\\
    fresh crabmeat\\
    red pepper\\
    parsley\\
\end{ingredients}
Using large or medium Portobellos, remove stems. Stuff caps with fresh crabmeat (preferably not previously frozen; canned just doesn't do it!). Place under broiler for a few minutes (5 or so) until lightly browned. Top with a few slices of red pepper and a little curly parsley for color. Season w/salt and pepper at the table as desired.  [One could also add a little Hellman’s or homemade mayonnaise and/or finely chopped celery to the crab before stuffing.]

\section{Nothing but Crab Cakes\index{entrees!nothing but crab cakes}}

\begin{open}
    From Martha, this recipe is adapted from the Paoli Auxiliary cookbook
    ``Quilted Cuisine.'' Being able to get real Chesapeake Bay crab
    occasionally means I make these probably way too often. But so easy! Serves
    4, so often just make half a recipe for the two of us.
\end{open}
%%
\begin{figure}[h]
    \centering
    \includegraphics[width=4in]{figures/crab_cakes}
    \caption*{Nothing but crab cakes!}
\end{figure}
%%
\begin{ingredients}
    \SI{1}{\pound} crab meat (lump works best)\\
    1 egg\\
    \SI{1}{\tblspoon} chopped parsley\\
    \SI{1}{\tblspoon} mayonnaise\\
    \SI{2}{\teaspoon} Worcestershire sauce\\
    \SI{1}{\tblspoon} melted butter\\
    \SI{1}{\teaspoon} dry mustard
    pinch salt\\
    ground pepper\\
    big slurp of Tabasco\\
    onion powder\\
\end{ingredients}
Combine all these ingredients and mix well. Probably be quite moist. Shape into
patties (I like small ones for ease of turning). Coat with Panko or regular
bread crumbs mixed with a little Parmesan. Chill until ready to cook. Sauté in
medium hot olive oil until brown on both sides.