\secpart{Second Edition}{Entrees}

%%---------------------------------------------------------------------------%%
\begin{entry}{Mushroom Cream Pasta}{Second Edition}
\index{Pasta!mushroom cream}
\index{Liu!Geyi}
\index{Johnson!Don}

\begin{open}
    Lightly adapted, mostly for convenience, from ``Chef John's Creamy Mushroom
    Pasta'' on \url{allrecipes.com}, My family (who are NOT vegetarians) are
    totally satisfied with this quick and easy dish as a main course. From: Don
    and Geyi.
\end{open}
\begin{ingredients}
    \SI{1/2}{\pound} linguine, fettuccine or other pasta\\
    \SI{1}{\pound} mushrooms. I like a mix of white and shiitake, but baby
    portobello also work well\\
    Olive oil, salt pepper\\
    Garlic\\
    \SI{1}{\tblspoon} sherry (For some reason, I more often have dry vermouth
    which seems to work fine.) \\
    Chicken stock (optional)\\
    \SI{1}{\cup} heavy whipping cream\\
    \SI{1/2}{\cup} grated parmesan cheese\\
    Fresh chopped thyme, chives, tarragon (n.b. I have never added these. I'm
    sure  they would make it better.)
\end{ingredients}
\Saute sliced mushrooms in olive oil until they are tender and release their
liquid. Add several cloves of diced garlic. Add sherry followed by heavy cream,
and lick residual cream from cup measure. Add salt and pepper and simmer cream
until the mixture thickens a bit and foams---though it does not get very thick;
if it does add some chicken stock. When the mixture reaches reasonable
consistency, stir in the fresh spices, turn off the heat and mix in the parmesan
cheese. Stir the mushroom cream mixture into the pasta and serve.
\end{entry}

%%---------------------------------------------------------------------------%%
\begin{entry}{Broccolini and Chourico Portuguese Sausage}{Second Edition}
\index{Sausage!broccolini \& Chourico sausage}
\index{Lindquist!Julie}

\begin{open}
    This is one of Julie's Simple/Quick/Tasty entries.  Chourico is similar to
    Spanish Chorizo, and I use only Gaspar’s (and have found the meat department
    variety of chourico isn’t as flavorful and spicy)---see
    Fig.~\ref{fig:chourico}. This is my go-to dinner!

    As a historical side-note, broccolini was developed in Japan and is a hybrid
    between broccoli and Chinese broccoli (Chinese kale).
\end{open}
%%
\begin{figure}
  \centering
  \includegraphics[width=0.6\textwidth]{figures/broccolini-chourizo.pdf}
  \caption{Gaspar's chourico and broccolini.}
  \label{fig:chourico}
\end{figure}
%%
\begin{ingredients}
    Chourico (Gaspar's brand)\\
    broccolini
\end{ingredients}
Cut up raw broccolini (stems especially are so much sweeter than broccoli) and
place in wide soup bowl; microwave Chourico at 8 power for about
\SI{1.5}{\minute} to heat the meat through then cut into bite-size pieces.
Place on top of the broccolini and enjoy with a hearty red (I find a Malbec the
best accompaniment); some warmed sourdough baguette on the side or afterwards
with an excellent creamy French cheese doesn’t hurt.
\end{entry}

%%---------------------------------------------------------------------------%%
\begin{entry}{Portobello Caps Stuffed with Crab}{Second Edition}
\index{Seafood!crab stuffed portobello caps}
\index{Lindquist!Julie}

\begin{open}
  From Julie Lindquist; this dish harmonizes well with a medium red or a
  hearty semi-sweet white. Tasty as an hors d'oeuvre but also great for a main
  dish.
\end{open}
%%
\begin{ingredients}
    Portobello mushrooms\\
    fresh crabmeat\\
    red pepper\\
    parsley
\end{ingredients}
Using large or medium Portobellos, remove stems. Stuff caps with fresh
crabmeat (preferably not previously frozen; canned just doesn't do it!). Place
under broiler for a few minutes (5 or so) until lightly browned. Top with a
few slices of red pepper and a little curly parsley for color. Season with
salt and pepper at the table as desired.  (One could also add a little
Hellman’s or homemade mayonnaise and/or finely chopped celery to the crab
before stuffing.)
\end{entry}

%%---------------------------------------------------------------------------%%
\begin{entry}{Nothing but Crab Cakes}{Second Edition}
\index{Seafood!nothing but crab cakes}
\index{Johnson!Martha}

\begin{open}
  From Martha, this recipe is adapted from the Paoli Auxiliary cookbook
  ``Quilted Cuisine.'' Being able to get real Chesapeake Bay crab occasionally
  means I make these probably way too often. But so easy! Serves 4, so often
  just make half a recipe for the two of us. (NB: Kate decided we should do
  this cookbook crab mostly so we could get our hands on this recipe. Thanks,
  Mom!)
\end{open}
%%
\begin{figure}
    \centering
    \includegraphics[width=4in]{figures/crab_cakes.jpg}
    \caption{Nothing but crab cakes!}
    \label{fig:crab-cakes}
\end{figure}
%%
\begin{ingredients}
    \SI{1}{\pound} crab meat (lump works best)\\
    1 egg\\
    \SI{1}{\tblspoon} chopped parsley\\
    \SI{1}{\tblspoon} mayonnaise\\
    \SI{2}{\teaspoon} Worcestershire sauce\\
    \SI{1}{\tblspoon} melted butter\\
    \SI{1}{\teaspoon} dry mustard \\
    pinch salt\\
    ground pepper\\
    big slurp of Tabasco\\
    onion powder
\end{ingredients}
Combine all these ingredients and mix well. Probably be quite moist. Shape
into patties (I like small ones for ease of turning). Coat with Panko or
regular bread crumbs mixed with a little Parmesan. Chill until ready to
cook. \Saute in medium hot olive oil until brown on both sides.  Serve on plate with lemon and parsley leaves (Fig.~\ref{fig:crab-cakes}.)
\end{entry}

%%---------------------------------------------------------------------------%%
\begin{entry}{``Manhattan'' Pancakes}{Second Edition}
\index{Breakfast!manhattan pancakes}
\index{Johnson!Liz}
\index{Johnson!Don}

\begin{open}
  Since this was sent to us from Liz Johnson and she didn't give us a
  title, we decided to call them ``Manhattan'' pancakes. She says, ``we use
  this recipe for breakfast almost every Saturday morning, it is always very
  good. Dad (Don Johnson) found it on \url{allrecipes.com}.''
\end{open}
%%
\begin{ingredients}
    \SI{1/2}{\cup} of flour\\
    \SI{1}{\tblspoon} of sugar\\
    1\SI{1/2}{\teaspoon} of baking powder\\
    \SI{1}{\teaspoon} of salt\\
    \SI{1/4}{\cup} of milk\\
    1 egg\\
    \SI{3}{\tblspoon} of butter
\end{ingredients}
Put all the ingredients in a bowl, the dry ones first, and mix until no more
lumps. Heat a skillet to medium. ladle some batter into the pan, and flip it
when the surface bubbles. Keep doing that until you run out of batter, then
eat.
\end{entry}

%%---------------------------------------------------------------------------%%
\begin{entry}{Lettuce-Wrapped Fish}{Second Edition}
\index{Seafood!lettuce-wrapped fish}
\index{Horne!Mimi}

\begin{open}
  From Mimi Horne, this recipe yields 4 servings and takes about 30 minutes to prepare. It looks like a good recipes for the youngins to try. as there are not too many ingredients!
\end{open}
%%
\begin{ingredients}
    Salt and freshly ground black pepper\\
    Several big leaves of romaine lettuce, Bibb lettuce or white cabbage\\
    1\SI{1/2}{\pound} thick white fish fillet (rockfish, cod, hake, snapper), in
    pieces about \SIrange[range-phrase={ to }]{3/4}{1}{\inch} thick,
    \SI{1}{\inch} wide, and \SI{2}{\inch} or less across\\
    \SI{1}{\cup} white wine\\
    \SIrange[range-phrase={ to }]{2}{3}{\tblspoon} butter
\end{ingredients}
Bring a large pot of water to a boil and salt it. Take as many big, intact
leaves of lettuce or cabbage as you have pieces of fish. With large outer
leaves, cut out center veins 2 to 3 inches up from bottom of leaves, to the
point where the leaf is more pliable; with inner leaves this may not be
necessary. One or 2 at a time, blanch leaves in boiling water until they are
tender and flexible, 30 seconds to a minute. Remove and drain on paper towels.

Put a piece of fish on each leaf and sprinkle with salt and pepper; fold or roll
fish in leaf so edges overlap. It is not important to make a tight seal, but it
is nice if package covers all the fish. When done, you can cover and refrigerate
packages until ready to serve, or continue.

In a large, broad skillet or casserole with a cover, bring wine to a boil with
butter. Reduce heat to a simmer and add fish packages. Cover and simmer until a
thin-bladed knife easily penetrates fish, \SIrange{5}{10}{\minute}. Remove fish
to a warm platter.

Over high heat, quickly reduce liquid in skillet; it is likely there will be
more than there was when you started. When it is thickened a bit, pour over fish
and serve.
\end{entry}

%%---------------------------------------------------------------------------%%
\begin{entry}{Bachelor Days Stir Fry}{Second Edition}
\index{Stir-fry!bachelor days stir fry}
\index{Beef!bachelor days stir fry}
\index{Seafood!bachelor days stir fry}
\index{Chicken!bachelor days stir fry}
\index{Pork!bachelor days stir fry}
\index{Vegetarian!bachelor days stir fry}
\index{Lindquist!David}

\begin{open}
  Here's a recipe from Dave Lindquist, it sounds like it's from days in the
  Trauma Ward. Any vegetables can be omitted or substituted. Leftovers get
  combined in Tupperware or big yogurt container for tomorrow's lunch!
\end{open}
%%
\begin{ingredients}
    Rice (enough for leftovers for lunch)\\
    1 onion\\
    \numrange{2}{5} garlic cloves, depending on preference\\
    ginger root\\
    \numrange{1}{2} carrots\\
    1 handful of snow peas\\
    1 small/medium zucchini or yellow squash\\
    1 red pepper\\
    Protein: shrimp, chicken cutlets, beef strips, or tofu\\
    Spices as desired (salt, pepper, soy sauce, cumin, curry, etc.)\\
    Cooking oil
\end{ingredients}
Heat water for rice. Start low heat to cast iron skillet or wok. Mince garlic
coarsely. Begin chopping vegetables to bite-sized pieces, exact shape is up to
the chef. (If you chop onions last, your crying will interfere less with the
food prep.)

By now water for rice should be boiling. Add rice and simmer as per directions
on container. Usually \SIrange{15}{20}{\minute}. Turn up heat on skillet or
wok.

Finish chopping vegetables and protein. Lightly oil skillet/wok. It should be
smoking gently. \Saute onions, then peppers, then carrots, then squash, then
snow peas. Avoid overloading skillet/wok; it will stay hotter via cooking
smaller batches. Gently season each batch of vegetables as they cook. Combine
each batch of \sauteed vegetables to a large mixing bowl.

\Saute protein. Just before protein is ready, lower the heat, add the garlic,
and grate ginger into the skillet/wok. Add back in all the previously cooked
vegetables. Complete seasoning to taste. A dash of hot pepper will add some
kick.

By now your rice should be ready. Add salt, butter to taste. Fluff.

To serve, I prefer a bed of rice with the veggies and protein on top, drizzled
with soy sauce.
\end{entry}

%%---------------------------------------------------------------------------%%
\begin{entry}{Fish \`{a} la Dave}{Second Edition}
\index{Seafood!fish \`{a} la Dave}
\index{Lindquist!David}

\begin{open}
    Another quick and easy dish featuring soy sauce from Dave Lindquist.
\end{open}
%%
\begin{ingredients}
    Rice (enough for leftovers)\\
    \SIrange{0.5}{2}{\pound} cod, haddock, or other whitefish, depending on
    number of servings desired\\
    1 red pepper\\
    1 yellow pepper\\
    1 onion\\
    \numrange{3}{6} garlic cloves\\
    ginger root (sensing a theme here?)\\
    1 zucchini and/or yellow squash\\
    soy sauce\\
    olive oil
\end{ingredients}
Heat water for rice. Preheat oven to \SIrange{300}{325}{\degreeF} (depends on
how hot your oven runs). Chop all vegetables into bite-sized pieces. Exact shape
is at chef's discretion. Coarsely chop garlic Lay fish in baking pan---use a
large baking dish with cover (or aluminum foil). Sprinkle garlic liberally on
fish Loosely place vegetables around fish; it's ok if they overflow onto the
fish. Drizzle olive oil all over fish and vegetables. Sprinkle with soy sauce.
Sprinkle with freshly ground pepper. Place fish in oven with cover or foil.

By now water for rice should be boiling. Add rice and simmer as per packaging
directions (usually \SIrange{15}{20}{\minute}). Fish should cook for
approximately \SI{20}{\minute}, or until it looks firm and flaky. Remove cover
to speed cooking, if needed.

Serve fish and vegetables over bed of rice. Add soy sauce if desired.
\end{entry}

%%---------------------------------------------------------------------------%%
\begin{entry}{Pasta with Anchovies}{Second Edition}
\index{Pasta!pasta w/ anchovies}
\index{Seafood!pasta w/ anchovies}
\index{Horne Rona!Alison}

\begin{open}
  From Alison Horne Rona, she likes to use two tins of anchovies with enough short
  fusilli pasta for one lunch and one dinner.
\end{open}
%%
\begin{ingredients}
    anchovy filets in olive oil\\
    \numrange{2}{5} garlic cloves (to taste), chopped\\
    chili pepper flakes\\
    olives, chopped\\
    anchovy paste\\
    flat-leaf parsley, chopped
\end{ingredients}
Finely chop the anchovy filets and briefly \saute with the garlic, chili
pepper, olives, and anchovy paste while cooking the pasta.  Drain the pasta
and toss in the anchovy mixture with lots of parsley.
\end{entry}

%%---------------------------------------------------------------------------%%
\begin{entry}{Swedish insanity meatballs}{Second Edition}
\index{turkey!swedish meatballs}
\index{Evans!Kate}

\begin{open}
  Kate normally avoids meatballs to avoid eating too much beef (and onions),
  but turkey meatballs just don't have the same zing (a.k.a. fat). So she
  altered a recipe from fine cooking to arrive at these rich---yet
  non-beef---meatballs. Serve with german egg noodles and a sweet side veggie
  like honey carrots or green beans for a complete meal! As with all of Kate's
  recipes, if you see onion powder in the ingredients, feel free to replace
  with \SI{1/2}cup minced onion.
\end{open}
%%
\begin{ingredients}
    3 slides whole wheat bread (approx.)\\
    \SI{1/4}{\cup} milk (whole or 2\%) \\
    \SI{12}{\ounce} ground turkey, 93 or 99\% fat free\\
    \SI{12}{\ounce} pork sausage\\
    \SI{1}{\teaspoon} onion powder\\
    1 large egg, lightly beaten\\
    \SI{1/4}{\teaspoon} ground allspice\\
    \SI{1/4}{\teaspoon} ground nutmeg\\
    \SI{1/2}{\teaspoon} each of Kosher salt and pepper\\
    \SI{2}{\tblspoon} butter\\
    \SI{2}{\tblspoon} olive oil\\
    \SI{1}{\tblspoon} flour\\
    \SI{1}{\cup} lower-salt chicken broth\\
    \SI{1/4}{\cup} heavy cream
\end{ingredients}
In a stand mixer fitted with a paddle attachment (or in a bowl and use your
hands---works great!), soak the bread in the milk until softened, about 5
minutes. Mix on low speed until uniform, about 30 seconds. Add the turkey,
pork, onion powder, egg, allspice, nutmeg, salt and pepper and mix on low
speed until evenly combined. Using your hands (a bowl of water nearby helps
keep them from getting too gummed up), form about 25 \SI{1.5}{\inch} sized
balls. Heat a large skillet over medium-high heat and add \SI{1}{\tblspoon}
each of the butter and oil. As soon as butter is melted, add half the
meatballs, turning very several minutes on several sides until browned, then
remove to a plate on the side. Repeat with the rest of the meatballs. Then
turn skillet heat to medium and add the other \SI{1}{\tblspoon} each of the
butter and oil. Add the flour and whisk it in with the fat until smooth. Whisk
in the chicken broth, then the cream, in small batches. The roux will first
become firmer as liquid is added but then become a thin sauce. Return all the
meatballs to the skillet and reduce the heat slightly, cover, and cook until
meatballs are cooked through and sauce thickens, about
\SIrange{8}{10}{\minute}. Season the sauce as needed with salt and pepper.
\end{entry}

%%---------------------------------------------------------------------------%%
\begin{entry}{Six Cheese Lasagne}{Second Edition}
\index{Pasta!six cheese lasagne}
\index{Beef!six cheese lasagne}
\index{Turkey!six cheese lasagne}
\index{Evans!Kate}

\begin{open}
  This is not a typo. Repeat. This is not a typo. It turns out that adding 6
  different cheeses to a traditional lasagne recipe makes it awesome. This
  recipe was taken from a 1992 Southern living cookbook Kate received as a
  wedding shower gift. I am looking at you, Fermina lunch buddies!
\end{open}

\begin{ingredients}
\numrange{9}{12} lasagne noodles, cooked according to package directions\\
\SI{1/2}{\cup} sharp Cheddar cheese, shredded\\
\SI{1/2}{\cup} Romano cheese, grated\\
\SI{1/2}{\cup} Parmesan cheese, grated \\
\SI{8}{\ounce} Mozzarella cheese, sliced
\end{ingredients}

\minisection{Tomato sauce}
\begin{ingredients}
    \SI{1}{\tblspoon} olive oil\\
    1 clove garlic, minced\\
    \SI{1}{\ounce} ground beef or turkey \\
    \SI{1}{\teaspoon} onion powder\\
    1 \SI{12}{\ounce} can tomato paste\\
    \SI{8}{\ounce} sour cream\\
    1 \SI{16}{\ounce} can tomato sauce\\
    1\SI{1/2}{\cup} water\\
    \SI{1}{\tblspoon} dried basil\\
    \SI{1}{\teaspoon} salt\\
    \SI{1/2}{\teaspoon} dried rosemary\\
    2 bay leaves
\end{ingredients}

\minisection{Cheese layer}
\begin{ingredients}
    2 large eggs, lightly beaten\\
    \SI{2}{\cup} ricotta cheese\\
    \SI{8}{\ounce} sour cream \\
    \SI{1/4}{\cup} chopped parsley, or \SI{2}{\tblspoon} dried\\
    \SI{1/2}{\teaspoon} salt\\
    \SI{1/4}{\teaspoon} pepper
\end{ingredients}
%
\protip{An acceptable weeknight cheat is to use \SI{1}{\cup} of the green
  cylinder grated combo Parmesan and Romano cheese instead of grating your own
  separately.}
%
Heat oil over medium heat in large skillet or saucepan. Add garlic and
stir-fry briefly until browned, then add ground meat and \saute until
cooked. Add tomato paste, tomato sauce, water, basil, salt, rosemary, and bay
leaves and stir gently. Bring to a boil, then reduce heat and cover and cook
until its time to use in the lasagna, anytime more than 15 minutes or so. In a
separate and medium sized bowl, add the cheese layer ingredients and stir to
combine. Preheat oven to \SI{375}{\degreeF}. Next, arrange 3-4 lasagna noodles
(depending if you have 9 or 12) onto bottom of greased \SI{9x13}{\inch} glass
baking pan. Layer with \SI{1/3} of the tomato sauce, \SI{1/3} or the cheese
layer, then \SI{1/3} each of the Cheddar, Romano, and Parmesan cheeses. Repeat
two more times. Place on middle shelf of oven and bake until bubbly, about
\SIrange{30}{35}{minutes}. Then take out and arrange the Mozzarella slices
along the top and return to oven until Mozzarella is melted and starting to
bubble and brown, about \SIrange{5}{10}{minutes}.

\begin{figure}
  \centering
  \includegraphics[height=0.9\textheight]{figures/PastaCortaMNHleCookbook.jpg}
  \caption{Although these pastas are not used in the lasagne they inspire us to cook italian. Credit: Mimi Horne, and first appeared in ``Le Cookbook.''}
  \label{fig:mimi_pasta}
\end{figure}

\end{entry}

%%---------------------------------------------------------------------------%%
\begin{entry}{Whole beef tenderloin}{Second Edition}
\index{Beef!whole tenderloin}
\index{Evans!Fermina}

\begin{open}
  Every home cook should have this dish in their arsenal for when you have
  company and want to look like you are a chef. Its deceptively easy (but plan
  ahead time wise!) and makes a great presentation piece.
\end{open}

\begin{ingredients}
    \SI{1/2}{\cup} fresh parsley or mint, chopped\\
    \SI{1}{\cup} red wine\\
    \SI{1/2}{\cup} soy sauce\\
    \SI{1/4}{\cup} Worcestershire sauce\\
    \SI{2}{\tblspoon} fresh rosemary or \SI{2}{\teaspoon} dry\\
    4 cloves garlic, chopped\\
    black pepper
\end{ingredients}
%
\protip{Fermina notes that it's cheaper per pound to get a whole one and trim
  it, but if you are new to working with tenderloin, its perhaps safer to get
  one that is already trimmed.}
%
The day before, combine all ingredients but beef in a small bowl and add with
beef into a plastic bag to marinate the beef overnight in the
refrigerator. Remove from fridge, place beef on roasting dish, discard
marinade, and keep out at room temp at least two hours before cooking. Preheat
oven to \SI{450}{\degreeF}. Roast for 15 minutes at \SI{450}{\degreeF}, then
reduce temperature to 350 and roast for another 15 minutes. Remove from oven
and double wrap in heavy foil before serving.
%
\protip{For a rare center, the rule of thumb is that the meat thermometer in
  the center registers \SI{450}{\degreeF}.}
\end{entry}

%%---------------------------------------------------------------------------%%
\begin{entry}{Braised Pork in Soy Sauce, ``Pork Candy''}{Second Edition}
\index{Liu!Geyi}
\index{Pork!braised pork in soy sauce}

\begin{open}
  From Geyi, who writes:

  This is a common household dish in China, it is in every child's memory as one of their favorites of their mom's. Every family has its own way of making it. In general, there are regional differences in the ingredients: families in the south like to use carameled soy sauces for a dark color, Northern families do
  not think much of the color, and only use rock crystal sugar for the caramel effect.

  I remember my mom’s, she poaches the pork first and does not put other things
  with her meat. My aunt’s family in Beijing put tofu in the pot, which soaked
  up the meat juice and flavors and tended to be even more sought-after than the
  meat. Our ayi (nanny) in Shanghai added boiled eggs to it to the same effect.

  I cook it sometimes this way, and sometimes that way. They all come through well. Some say it tastes better the day after, but we never had the chance to test the theory---it is always gone the first day. My current standard is a variation from ``Miss Vegetable's gourmet cooking diary'' a WeChat channel that uses black tea to cut through the grease. ``Miss Vegetable'' started cooking for herself and friends, then got so popular that she started her own company with staff and everything.

  \begin{CJK*}{UTF8}{gbsn}红烧肉\end{CJK*} is the name of the dish. Literally translated it means ``red roast pork;'' a more common translation is Braised Pork in Soy Sauce---neither captures the image, experience or culture. I cooked this dish for Martha's birthday 2019, and it is she who was able to give this dish a proper name---Pork Candy!
\end{open}
%%
\begin{ingredients}
  \SI{2}{\pound} pork belly with skin (Fig.~\ref{fig:pork-candy-ingredients})\\
  Rock Crystal sugar\\
  \numrange{3}{5} pieces of sliced ginger\\
  \SI{3}{\tblspoon} Chinese cooking wine (or sherry)\\
  \SI{3}{\tblspoon} soy sauce (\begin{CJK*}{UTF8}{gbsn}生抽\end{CJK*})\\
  \SI{2}{\tblspoon} dark soy sauce (regular soy sauce with an additional
  caramelization step. Used often for cooking meat to give it a darker color and
  maybe more complex flavor.)\\
  black tea
\end{ingredients}
%%
Before starting cut the pork belly into \SI{1}{\inch} cubes and make a large pot
of hot tea.  The cooking instructions are:
\begin{enumerate}
    \item  Heat a pan with lid on low. Add the meat without adding oil. Cook the
    meat slowly on low until the oil seeps out, then turn the pieces one by one
    to cook the other side (Fig.~\ref{fig:pork-candy-prep}a).
    \item When the surfaces are all slightly browned and the oil has leaked out,
    add rock sugar and stir fry until it is completely dissolved
    (Fig.~\ref{fig:pork-candy-prep}b).
    \item Add ginger slices, bay leaves, dried chili, then pour rice wine, dark
    soy sauce and light soy sauce, and stir fry evenly
    (Fig.~\ref{fig:pork-candy-prep}c).
    \item Pour the hot tea into the pot, make sure there is enough to cover the
    meat. Bring the liquid to a boil, then simmer for one hour.
    \item Uncover, turn heat to high, stir till the liquid caramelizes, coats
    the meat cubes and shines (Fig.~\ref{fig:pork-candy-prep}d).
\end{enumerate}
%%
\begin{figure}
  \centering
  \includegraphics[width=0.5\textwidth]{figures/pork-candy-ingredients.png}
  \caption{Pork candy ingredients.}
  \label{fig:pork-candy-ingredients}
\end{figure}
%%
\begin{figure}
  \centering
  \includegraphics[height=0.9\textheight]{figures/pork-candy.pdf}
  \caption{Preparing Pork Candy!}
  \label{fig:pork-candy-prep}
\end{figure}

\end{entry}

%%---------------------------------------------------------------------------%%
\begin{entry}{Herb Roasted Pork Tenderloin}{Second Edition}
\index{Cordova!Betsy}
\index{Cordova!Rich}
\index{Pork!herb roasted pork tenderloin}

\begin{open}
  From Betsy, who says ``Fun Fact:  This is the first meal I made when Rich and I had company as a married couple!!!''
\end{open}
%%
\begin{ingredients}
  1 package pork tenderloin (about 1\SI{1/2}{\pound} per tenderloin)
\end{ingredients}
\minisection{Marinade}
\begin{ingredients}
  \SI{1/2}{\cup} soy sauce\\
  \SI{1/4}{\cup} vegetable oil\\
  \SI{1/4}{\cup} Worchestershire sauce\\
  \SI{1}{\teaspoon} rubbed sage\\
  \SI{1}{\teaspoon} onion powder\\
  \SI{1}{\teaspoon} salt\\
  \SI{1}{\teaspoon} dried marjoram\\
  \SI{1}{\teaspoon} pepper\\
  \SI{1}{\teaspoon} garlic powder\\
  \SI{1}{\teaspoon} dried ginger\\
  \SI{1}{\teaspoon} dried thyme
\end{ingredients}
%%
Mix together all the ingredients (I use a Pampered Chef salad dressing mixer) to
make the marinade. Pierce both pieces of pork with a fork and put in shallow
dish or heavy duty zip lock bag. Add marinade making sure to coat the pork.

Cover dish or close bag and allow pork to stand at room temperature for
\SI{30}{\minute}. Cook on baking dish or rack for \SI{30}{\minute} at
\SI{350}{\degreeF} or until pork registers \SI{160}{\degreeF}.

\end{entry}

\begin{entry}{Oven Tacos}{Second Edition}
\index{Pross!Katie}
\index{Tacos!oven baked}
\index{Beef!oven baked tacos}

\begin{open}
These are finished in the oven so they are less messy and easier to eat than if assembled at the end. Make sure to drain the beef well so it doesn't make the shells soggy. 
\end{open}
%%
\begin{ingredients}
  \SI{2}{\pound} ground beef/turkey/chicken/whatever\\
  \SI{1} small onion, diced\\
  \SI{1} small can diced green chiles\\
  \SI{8}{\ounce} can low sodium tomato sauce \\
  \SI{1}{\pound} can fat free refried beans \\
  \SI{2}{\cup} shredded reduced fat Colby-Jack cheese\\
  \numrange{18}{20} hard taco shells\\
  optional condiments, including shredded lettuce, tomato, salsa, guacamole, sour cream, olives, etc.
\end{ingredients}
\minisection{Taco Seasoning}
\begin{ingredients}
  \SI{1}{\tblspoon} chili powder\\
  \SI{1/4}{\teaspoon} onion powder\\
  \SI{1/4}{\teaspoon} garlic powder\\
  \SI{1/4}{\teaspoon} crushed red pepper flakes\\
  \SI{1/4}{\teaspoon} dried oregano\\
  \SI{1/2}{\teaspoon} paprika\\
  \SIrange{1}{1/2}{\teaspoon} ground cumin\\
  \SI{1}{\teaspoon} salt\\
\SI{1}{\teaspoon} pepper\\
\end{ingredients}

%%
Preheat oven to \SI{400}{\degreeF}. In a large skillet, brown ground beef and onion over medium high heat. Drain off any excess liquid. Return to pan, add chiles, refried beans, and tomato sauce. Combine taco seasoning ingredients and add to beef mixture. Mix well and cook for a few minutes if mixture seems runny.

Spoon the meat mixture into the taco shells and place into a \num{9x13} inch baking dish, standing up. Sprinkle cheese over the top of the meat mixture in each shell. Place into the oven and bake at \SI{400}{\degreeF} for \numrange{10}{12} minutes or until the cheese has melted and the tacos are cooked through. Remove from the oven and top with any optional condiments for serving. 

\end{entry}

\begin{entry}{Ratatouille}{Second Edition}
\index{Horne!Mimi}
\index{Vegetarian!Ratatouille}
\label{sec:lecookbook}

\begin{open}
Given that Ratatouille is a classic French dish, we tell the story here from Paul Horne about Mimi's efforts on ``Le Cookbook,'' from where she first published this recipe. She illustrated the cover, a drawing of her and Paul's Paris apartment kitchen shown in figure \ref{fig:locookbook}, as well as many of the drawings throughout the book. 

\begin{figure}
  \centering
  \includegraphics[height=0.9\textheight]{figures/LeCookbookCover23nov20.jpg}
  \caption{Le Cookbook cover, credit: Mimi Horne}
  \label{fig:pork-candy-prep}
\end{figure}

Paul sent a wonderful summary of Mimi's volunteer work that involved this cookbook, so we include some of his notes here:

Mimi was a volunteer at the American Hospital of Paris from the early 1980s until we moved to London in 1998. The volunteers
provide services in what is one of Europe's best and chic-est hospitals. Founded in 1906, AHP played a crucial role in the world wars and is today a high-tech healthcare center in France. One of the volunteers' most successful projects was "Le Cookbook", a bilingual volume with recipes in English and American measures on the right hand page and in French with French measures on the left page. First published in 1976, Le Cookbook was a great success, contributing financially to AHP. By the early 1990s all copies had been sold so Mimi volunteered to lead a group of volunteers to do a second edition. This proved to be an intensive job of culling, updating and adding recipes and ideas (editor's note: don't we know it!!). But the second edition of "Le Cookbook" was published in 1998, not long after we got to London, and has been a success although sales are now limited to the hospital itself since the French did not want it to "compete" with French cookbooks !

With that, here is the showcase of the recipes we received from ``Le cookbook.'' This dish is best prepared ahead so the flavors have time to blend. 

\end{open}
%%
\begin{ingredients}
  5 tomatoes (or a large can), cut in slices\\
  2 eggplants, sliced, not peeled \\
  2 zucchini, sliced \\
  1 red pepper, cut in strips \\
  1 green pepper, cut in strips \\
  4 small onions, chopped \\
  2 cloves garlic, crushed \\
  Tabasco sauce \\
  \SI{1}{\tblspoon} \emph{herbes de provence} \\
  5 bay leaves \\
  salt and pepper \\
  \SI{4}{\tblspoon} olive oil \\
  
\end{ingredients}

%%
Salt eggplant slices and place a weight on top to make them disgorge their liquid. After an hour, rinse, dry and fry them in olive oil a few at a time, leaving them to dry on paper towels. In a small amount of oil, cook the oinions, garlic, peppers, and zucchini, putting aside each vegetable when it is softened. In a large oven dish make layers of all ingredients, adding to each layer tomato slices, salt, pepper, dash of Tabasco, bay leaf, and \emph{herbes de provence}. Bake uncovered at \SI{350}{\degreeF} for 45 minutes. May be served hot or tepid. 

\begin{figure}
  \centering
  \includegraphics[width=0.6\textwidth]{figures/GrilledEggPlantMNHleCookbook.jpg}
  \caption{}
  \label{fig:mimi_eggplant}
\end{figure}

\begin{figure}
  \centering
  \includegraphics[width=0.6\textwidth]{figures/GarlicPressMNHleCookbook.jpg}
  \caption{Illustration by Mimi Horne of grilling the eggplant for Ratatouille.}
  \label{fig:mimi_garlic}
\end{figure}

\end{entry}

\begin{entry}{Beef with Snow Peas}{Second Edition}
\index{Pross!Katie}
\index{Beef!beef with snow peas}
\index{Stir-Fry!beef with snow peas}

\begin{open}
From Katie Pross, here is a delicious quick-and-easy Asian-inspired dinner ready in 20 minutes. Serves 8.
\end{open}
%%
\begin{ingredients}
  \SIrange{1}{1/2}{\pound} flank steak, trimmed of fat and sliced very thin against the grain\\
  \SI{1/2} cup low sodium soy sauce\\
  \SI{3}{\tblspoon} cooking or drinking sherry\\
  \SI{2}{\tblspoon} brown sugar \\
  \SI{2}{\tblspoon} cornstarch \\
  \SI{2}{\teaspoon} minced fresh ginger\\
  \SI{8}{\ounce} snow peas, ends trimmed \\
   5 whole scallions, cut into half-inch pieces on the diagonal \\ 
  salt as needed \\
    \SI{3}{\tblspoon} peanut or olive oil \\
  crushed red pepper for sprinkling \\
  
\end{ingredients}

%%
In a bowl, mix together the soy sauce, sherry, brown sugar, cornstarch, and ginger. Pour half the liquid over the sliced meat in a bowl and toss. Reserve the other half of the liquid. Set aside.

Heat oil in a heavy skillet (recommend iron) or wok over high heat. Add snow peas and stir for 45 seconds. Remove to a separate plate and set aside.

Allow pan to get very hot again. With tongs, add half the meat mixture, leaving most of the marinade still in the bowl. Add half the scallions. Spread out meat as you add it to the pan but do not stir it for a good minute. Turn the meat to the other side and cook for another 30 seconds and remove to a clean plate. Repeat with the other half of the meat after pan is hot again. Then add first half of meat back to the pan with the second, the rest of the marinade, and the snow peas. Check to see if it needs more salt. Mixture will thicken as it sits. 

Serve immediately with cooked jasmine or other long-grained rice. Sprinkle with crushed flakes as desired. 

\end{entry}
%%---------------------------------------------------------------------------%%
%% From the first edition

%%---------------------------------------------------------------------------%%
\begin{entry}{Scott's Killer Chili}{First Edition}
\index{Beef!chili}
\index{Evans!Scott}

\begin{open}
  I hope you're prepared for this.  This recipe from Scott Evans makes a {\em
  thick} and {\em spicy} chili.  In the word's of the author ``It is pretty
  spicy.''  This is one of those recipes that should include a Disney-style
  warning label, ``\textellipsis those with heart conditions or over the age of
  sixty-five...etc. etc.''  This is this recipe's first time in print so some
  experimentation may be required.  Supposedly this is a campout recipe, however
  I see no way that anyone could possibly carry all these ingredients. The
  recipe makes about 16 servings or less for REALLY big people.  Good luck and
  here it goes.
\end{open}
\begin{ingredients}
  \SI{3}{\pound} of hot Italian sausage (ie. Hot Cincinnati Brand, that homer) \\
  \SI{3}{\pound} bacon \\
  3 large onions \\
  3 bell peppers (2 green, 1 red) \\
  \numrange{4}{5} cloves of garlic \\
  \numrange{4}{5} hot peppers (a cornucopia of jalape\~{n}os, habaneros, and
  others) \\
  3 cans Italian pear tomatoes \\
  \SI{1}{\tblspoon} olive oil \\
  \SI{1}{\tblspoon} mustard powder \\
  \SI{1}{\tblspoon} celery seed \\
  \SI{1}{\tblspoon} chili powder \\
  \SI{1}{\tblspoon} bay leaves \\
  \SI{1}{\tblspoon} Worcester sauce \\
  \SI{1}{\tblspoon} vinegar \\
  red wine \\
  water \\
  salt and pepper
\end{ingredients}
Start with a large Dutch oven and a campfire right after breakfast.  Fry the
sausage and set aside.  Fry the bacon and set aside.  Leave a bit of grease in
the pot and add the minced garlic followed by the roughly chopped onions and
bell peppers (no bell pepper seeds).  Chop the hot peppers and add to the pot,
remember that the seeds make the dish VERY spicy.  Add olive oil as needed.
%%
\begin{wrapfigure}{R}{.2\textwidth}
\centering
\includegraphics[width=.15\textwidth, trim=.5in .25in .5in .25in, clip]{figures/chilli.pdf}
\end{wrapfigure}
%%
Add about one Tbsp full each of: mustard powder, celery seed, Worcester sauce,
vinegar, and chili powder.  This may require some experimentation to alter to
your taste.  Stir and cook until onions become clear and peppers begin to
soften.  Add up to one cup of red wine.  Next add tomatoes and juices.  Stir
and chop tomatoes.  Add sausage, bacon, and two bay leaves.  Season with salt
and pepper.  Now everything should look a bit like chunky soup, but don't
worry.

Let the chili simmer over low heat for a minimum of three hours, but try for
eight (trust Scott on this one). Check periodically and stir.  If mixture
thickens too much add some water.  Taste and adjust to preference.

Serve with shredded cheddar cheese and garlic bread.  This recipe freezes well
in personalized zip-lock bags (In case you're not hungry enough to eat six
pounds of meat in one sitting).
\end{entry}

%%---------------------------------------------------------------------------%%
\begin{entry}{Chicken Breasts with Orange Sauce}{First Edition}
\index{Chicken!chicken with orange sauce}
\index{Johnson!Lil}

\begin{open}
  This is a recipe from Lil and Don Johnson Sr.  Grammie (Lil) passed it to Martha, who passed it to Kate, and so on, and so on, and so on\textellipsis
  It's a good recipe for new cooks.
\end{open}
\begin{ingredients}
  4 halved chicken breasts \\
  1 small can undiluted O.J. concentrate \\
  1 package Lipton's Onion Soup Mix \\
  paprika
\end{ingredients}
In a long baking pan arrange the 8 pieces of chicken.  Pour the O.J.
concentrate (at room temperature) over the chicken.  Sprinkle the soup mix
over the chicken.  Add a little paprika for seasoning.

Cover pan with foil.  Bake at \SI{350}{\degree} for 40 minutes.  Remove foil and
baste chicken.  Back for an additional 20 minutes uncovered.
\end{entry}

%%---------------------------------------------------------------------------%%
\begin{entry}{Porcupine Meatballs}{First Edition}
\index{Beef!meatballs}
\index{Evans!Mickey}
\index{Evans!George Sr.}

\begin{open}
  This is a recipe from Mickey and George, Sr. Evans.
\end{open}
\begin{ingredients}
  \SI{1/2}[1]{\pound} hamburger \\
  \SI{3/4}{\cup} uncooked rice \\
  \SI{1}{\teaspoon} salt \\
  1 egg \\
  \SI{1/2}{\teaspoon} pepper \\
  \SI{1/4}{\cup} chopped onion \\
  \SI{1/2}[2]{\cup} stewed tomatoes \\
  \SI{1}{\teaspoon} chili \\
  \SI{1}{\teaspoon} sugar
\end{ingredients}
Combine hamburger, rice, salt, egg, pepper, and onion.  Shape into
\SI{1/2}[1]{\inch} balls.

Heat sauce and chili to boiling in a kettle.  Drop balls in sauce.  Simmer for
1\num{1/2} hour covered.  Strips of bacon may be wrapped around meatballs
and secured with toothpicks before cooking in sauce.
\end{entry}

%%---------------------------------------------------------------------------%%
\begin{entry}{Lemon-Herb Chicken}{First Edition}
\index{Chicken!lemon-herb chicken}
\index{Evans!Fermina}

\begin{open}
  This is Fermina's favorite.
\end{open}
\begin{ingredients}
  1 chicken (cut) or \SI{1/2}[3]{\pound} of chicken parts      \\
  \SI{1/2}{\cup} olive oil                                     \\
  \SI{1/4}{\cup} lemon juice                                   \\
  2 garlic cloves minced                                       \\
  \SI{3}{\tblspoon} chopped fresh oregano or \SI{1}{\tblspoon} dry \\
  \SI{1/2}{\teaspoon} salt                                     \\
  \SI{1/8}{\teaspoon} pepper                                   \\
  \SI{1}{\tblspoon}chopped fresh rosemary or \SI{1}{\teaspoon} dry \\
  \SI{2}{\tblspoon}chopped fresh parsley
\end{ingredients}
Place chicken meaty side down in a \SI{13x9x2}{\inch} baking pan.  Combine
next 8 ingredients, mix well.  Pour mixture over chicken.  Marinate in
refrigerator for two hours.  Bake uncovered at \SI{350}{\degree} for 40 minutes.
Turn chicken.  Broil 6~inches from heat for \numrange{5}{10} minutes or
until crisp and lightly browned.
\end{entry}

%%---------------------------------------------------------------------------%%
\begin{entry}{Stuffed Flank Steak Teriyaki}{First Edition}
\index{Beef!flank steak teriyaki}
\index{Evans!Tom}
\index{Evans!Fermina}

\begin{open}
  This belongs because it is MY favorite and since I'm (Tom) in charge,
  well there you go.  Makes \numrange{4}{5} servings.
\end{open}
\begin{ingredients}
  1 medium to large beef flank steak (1\SIrange{1/4}{2}{\pound}) \\
  \SI{1/2}{\cup} soy sauce                                    \\
  \SI{1/4}{\cup} cooking oil                                  \\
  \SI{2}{\tblspoon} molasses                                   \\
  \SI{2}{\teaspoon} dry mustard                               \\
  \SI{1}{\teaspoon} ginger root or \SI{1/2}{\teaspoon} dry ginger \\
  1 clove garlic minced                                       \\
  \SI{1}{\cup} water                                          \\
  \SI{1/2}{\cup} long-grain rice                              \\
  \SI{1/2}{\cup} of shredded carrots                          \\
  \SI{1/2}{\cup} sliced water chestnuts (optional)            \\
  \SI{1/4}{\cup} sliced green onions
\end{ingredients}
\begin{wrapfigure}{L}{.3\textwidth}
\centering\includegraphics[width=.25\textwidth,clip]{figures/flank.pdf}
\end{wrapfigure}
Cut a large pocket in flank steak or have your butcher do it.  Combine soy
sauce, oil, molasses, mustard, ginger, and garlic.  Place meat in shallow pan
or plastic bag.  Pour marinade into pocket and over meat. Let stand at room
temp (\SI{300}{\kelvin}) for 30 minutes or in refrigerator for
\numrange{2}{3} hours.

In saucepan combine water, rice, carrots, water chestnuts, and green onion.
Bring to a boil; reduce heat and simmer while covered for 8 minutes.  Remove
from heat and set aside.

Drain meat reserving marinade.  Add \SI{1/4}{\cup} of reserved marinade to the
rice mixture.  Spoon rice stuffing into pocket of meat.  Secure end with
wooden toothpicks.  Place meat in shallow roasting pan and cover with foil.
Bake at \SI{350}{\degree} for 1 hour until meat is done.

Fermina allows an extra \numrange{10}{15} minutes without foil to brown meat.
Brush with marinade while browning and check often.  Slice meat diagonally
across grain to serve.
\end{entry}

%%---------------------------------------------------------------------------%%
\begin{entry}{Better-than-vegetarian Pasta Sauce}{First Edition}
\index{Pasta!better-than-veggie pasta sauce}
\index{Vegetarian!better-than-veggie pasta sauce}
\index{Johnson!Don}

\begin{open}
  This from Kate's brother Don, and he got it from his friend Dawn Ollila. The honey/brown sugar and cinnamon addition is what makes it taste so special.
\end{open}
\begin{ingredients}
  \SI{1}{\tblspoon}olive oil \\
  1 onion \\
  1 green pepper \\
  2 cloves of minced garlic \\
  1 tomato\\
  1 large or two small carrots OR \SI{1/4}{\cup} cup lentil beans \\
  1 or 2 cans of tomato sauce \\
  thyme \\
  basil \\
  oregano \\
  rosemary \\
  \SI{1}{\tblspoon} honey or brown sugar \\
  cinnamon \\
  salt and pepper
\end{ingredients}
\Saute the 4 vegetables in olive oil, adding them as ordered above. When the
onion is clear and the tomato is soft, add the tomato sauce.  Bring to a
simmer. The sauce is now tasty, but to thin to stick to the pasta.  Choose
either the carrots or lentils to give it body. Lentils add a great dark,
almost meaty, flavor but you will need to boil them in 4 times their
measurement in boiling water for \numrange{45}{90} minutes first (no
presoaking required). Do not add them to the sauce until they are bean-like
mush. or, you can grate the carrot as finely as you have the technology to do
and add to the sauce. The taste is minimal, but the texture is great. Add the
rest of the ingredients and adjust to taste. Add just enough cinnamon to make
your guests look at you funny and say, ``What did you put in this?'' The
flavor actually works quite well.
\end{entry}

%%---------------------------------------------------------------------------%%
\begin{entry}{Peachy Chicken}{First Edition}
\index{Chicken!peachy chicken}
\index{Evans!Fermina}
\index{Evans!Betsy}
\index{Evans!Tom}

\begin{open}
  From Fermi, this is Betsy's favorite.  Tom: Actually I really don't
  like this recipe and the only reason she ``likes'' this one is because she
  hates my favorite recipe (stuffed flank steak).  Note: the ``peachy'' in
  ``peachy chicken'' is not a southern thing.  Kate: When Tom and I were first
  married (Tom interjects: forty years ago) I made a chicken dish with fruit.
  He honestly thought I was trying to annoy him (how did he know?).
\end{open}
\begin{ingredients}
  chicken parts for \numrange{4}{6} people\\
  1 large can peach halves (drained, reserve syrup)\\
  \SI{2}{\tblspoon} soy sauce\\
  \SI{2}{\tblspoon} lemon juice\\
  \SI{1/2}{\tblspoon} ginger\\
  2 cloves minced garlic
\end{ingredients}
Preheat oven to \SI{375}{\degree}.  Mix last four ingredients with reserved peach
syrup.  Pour marinade over chicken parts in roasting or baking pan. Bake in
oven for \numrange{45}{60} minutes. Turn once.  Add peach halves last
15 minutes.  Brown under broiler for a few minutes if further browning
is needed.
\begin{center}
    \includegraphics[scale=.5,clip]{figures/meatloaf.pdf}
\end{center}
\end{entry}

%%---------------------------------------------------------------------------%%
\begin{entry}{Swiss Meatloaf}{First Edition}
\index{Beef!swiss meatloaf}
\label{sec:swiss-meatloaf}
\index{Evans!Joyce}

\begin{open}
  This was contributed by Joyce Evans, and is definitely ``comfort
  food''! Serves 6.
\end{open}
\begin{ingredients}
  1 egg \\
  \SI{1/2}{\cup} evaporated milk \\
  \SI{1}{\teaspoon} rubbed sage \\
  \SI{1}{\teaspoon} salt \\
  \SI{1/2}{\teaspoon} black pepper \\
  \SI{1/2}[1]{\pound} lean ground beef \\
  \SI{1}{\cup} cracker crumbs (round buttery type, approx. 24) \\
  \SI{3/4}{\cup} grated Swiss cheese \\
  \SI{1/4}{\cup} finely chopped onion \\
  \numrange{2}{3} strips bacon, cut into \SI{1}{\inch} pieces
\end{ingredients}
Preheat oven to \SI{350}{\degree}. Beat the egg in a large bowl. Add evaporated milk,
sage, salt, and pepper.  Mix together. Add beef, crumbs, \SI{1/2}{\cup} of the
cheese and the onion. Blend. Form into a loaf and place in a \SI{2}{\quart}
rectangular dish.  Arrange bacon pieces on top of the loaf, and bake for
40 minutes. Sprinkle remaining \SI{1/4}{\cup} cheese on top and bake
40 minutes longer.
\end{entry}

%%---------------------------------------------------------------------------%%
\begin{entry}{Devilled Crabs}{First Edition}
\index{Seafood!devilled crabs}
\index{Johnson!Martha}
\index{Johnson!Dodge}
\index{Lamb!Martha}

\begin{open}
  This is from Dodge and Martha Johnson in memory of a great southern cook, Martha Hodgkins Niepold Lamb (Kate's Grandmother)and her mother, Mrs. Henry Bell Hodgkins. Funny story, this was actually submitted for this cookbook as well by Martha's sister Mimi, so you know its a family favorite! Spice amounts can be adjusted to taste.
\end{open}
\begin{ingredients}
  1 stick butter\\
  2 eggs, beaten\\
  \SI{2}{\tblspoon} flour\\
  \SI{1/2}{\teaspoon} Worcestershire sauce\\
  \SI{1}{\teaspoon} dry mustard\\
  \SI{1}{\tblspoon} vinegar\\
  pinch of sugar, salt, pepper, and MSG\\
  \SI{1}{\pound} crab\\
  bread crumbs \\
  paprika (optional) \\ 
\end{ingredients}
Preheat oven to \SI{350}{\degree}.  Melt butter. Mix eggs with flour and a little water and add
to butter. Continue mixing and add everything but paprika. Divide and stuff in scallop shells or
small casseroles and dot with butter and bread crumbs and if desired, sprinkle with paprika. Bake for 1 hour.
\begin{center}
    \includegraphics[width=.3\textwidth,clip]{figures/crab.pdf}
\end{center}
\end{entry}

%%---------------------------------------------------------------------------%%
\begin{entry}{Orange Game Hens}{First Edition}
\index{Chicken!orange game hens}
%\index{Nicholsons}
\index{Johnson!Martha}
\index{Johnson!Dodge}

\begin{open}
  This is a recipe courtesy of Martha and Dodge Johnson's friends, the Nicolsons. It's a great dish for company-especially at their house because they are such nice people and good cooks!
\end{open}
\begin{ingredients}
  2 or more Cornish game hens, whole or halved\\
  Joyce Chen's orange Szechuan sauce (or sub in soy sauce with orange
  concentrate)\\
  \SIrange{2}{3}{\tblspoon} of orange concentrate\\
  several \si{\tblspoon} white wine\\
  garlic powder\\
  ground ginger (fresh or frozen root is best)
\end{ingredients}
Preheat oven to \SI{350}{\degreeF}. Pour sauce, concentrate, and wine over
hens. Sprinkle with garlic powder and ginger. Cover and bake for
1 hour, and uncover the last 10 minutes. Good over a bed of rice.
\end{entry}

%%---------------------------------------------------------------------------%%
\begin{entry}{Chapel Hill Chicken Pie}{First Edition}
\index{Chicken!Chapel Hill chicken pie}
\index{Beef!Chapel Hill chicken pie}
\index{Evans!Kate}
\index{Johnson!Martha}
\index{Johnson!Dodge}

\begin{open}
  This is from Martha and Dodge, and Kate. Kate's addition is only the
  rosemary and measured amounts, for convenience (and my subtractions are
  those nasty onions) No one cares what or how much you put in, as long as you
  are happy. This is one of those recipes that you put some in, then you take
  some out (Nana, does this sound familiar?) This is also the kind of recipe
  where you vary it based on what you like or what's sitting in the fridge!
  It's best when you have leftover gravy along with meat from a past meal.
\end{open}
\begin{ingredients}
  \SI{2}{\cup} chopped meat (roast lamb, beef, or chicken)\\
  \numrange{3}{4} cubed and peeled potatoes\\
  \SIrange{1.5}{2}{\cup} gravy or combination of stock and wine\\
  \SI{2}{\tblspoon} flour\\
  \SI{1}{\teaspoon} salt\\
  \SI{2}{\teaspoon} pepper\\
  \SI{1}{\tblspoon} dried parsley\\
  \SI{1}{\teaspoon} garlic powder or 1 garlic clove\\
  \SI{1}{\teaspoon} thyme (if using chicken or beef)\\
  \SI{1}{\tblspoon} fresh rosemary \\
  \SI{1}{\teaspoon} marjoram (if using lamb)\\
  \SI{1}{\teaspoon} tarragon (if using chicken)\\
  Pie Crust (\corp{Betty Crocker's} mix is good, sorry
  \corp{Duncan Hines}, you don't make one!)
\end{ingredients}
\SIrange{1}{2}{\cup} each of your favorite vegetables, such as
\begin{ingredients}
  chopped carrots\\
  green beans\\
  celery\\
  onion\\
  mushrooms\\
  peas
\end{ingredients}
Preheat oven to \SI{450}{\degreeF}. Boil potatoes until somewhat cooked through, about
15 minutes. In a flat-ish casserole (\SIrange{1.5}{2}{\quart}), layer meat,
vegetables, and potatoes. sprinkle spices over, and then flour. Add gravy
mixture; adjust so the liquid comes up about half the height of the
ingredients.  Top with crust, seal edges, and add fork holes or vents. Brown
for 15 minutes, then then lower temperature to \SI{350}{\degree} and bake for
45 minutes longer. I find that I have to cover it for the last
15 minutes or so to keep the crust from getting too brown.
\end{entry}

%%---------------------------------------------------------------------------%%
\begin{entry}{Pasta with Prosciutto}{First Edition}
\index{Pasta!pasta w/ prosciutto}
\index{Johnson!Martha}
\index{Johnson!Dodge}

\begin{open}
  Martha and Dodge originally got this from the New York Times, but it has
  evolved. It's rather quick and satisfying. They say the order of tasks is a
  little tricky for non-Italian cooks. Luckily, half the family need not worry.
\end{open}
\begin{ingredients}
  \SI{3}{\cup} chopped plum tomatoes\\
  \numrange{2}{3} thinly sliced small zucchini\\
  \SIrange{1/8}{1/4}{\pound} prosciutto, cut into strips\\
  \SI{1}{\teaspoon} salt\\
  \SI{2}{\teaspoon} pepper\\
  \SI{1/2}{\teaspoon} red pepper flakes (optional)\\
  \SI{1}{\cup} whipping cream
  \SI{1/2}{\cup} chopped fresh basil\\
  \SI{1/4}{\cup} grated parmesam (use the real thing not the cylinder, people)\\
  1+ cloves garlic, chopped\\
  \SI{1}{\tblspoon} olive oil\\
  about \SI{3/4}{\pound} pasta
\end{ingredients}
Cook pasta. Save \SI{1/3}{\cup} cooking water. In frying pan, sear garlic, add
zucchini, prosciutto, salt and pepper, red pepper flakes, then tomatoes.  Stir
for \numrange{2}{3} minutes.  Add saved water, cream and simmer briefly. Add
pasta, basil, and Parmesan, and toss. Transfer to serving dish and eat
immediately (not difficult to do!). Serves \numrange{2}{3}.
\end{entry}

%%---------------------------------------------------------------------------%%
\begin{entry}{Pasta al Cavolfiore (with Cauliflower)}{First Edition}
\index{Pasta!pasta al cavolfiore}
\index{Vegetarian!pasta al cavolfiore}
\index{Johnson!Don}

\begin{open}
  Don sends this yummy looking dish from the Moosewood cookbook. Don
  says it's good and adds ``so enough of sending pasta recipes to the
  Italians.''
\end{open}
\begin{ingredients}
  1 onion (optional if you're Kate)\\
  1 cauliflower head, chopped into bite sizes\\
  1 tomato\\
  garlic to taste\\
  \SI{2}{\cup} grated cheese (see below)\\
  \SI{1/4}{\cup} olive oil\\
  1 can tomato sauce\\
  \SI{3/4}{\pound} pasta\\
  basil, dried and some fresh too if possible\\
  \SI{1}{\teaspoon} salt\\
  \SI{1}{\teaspoon} pepper
\end{ingredients}
Chop the onion and garlic and \saute them in \SI{1}{\teaspoon} oil with the
basil. When onion is clear, add cauliflower and cook until tender. (Don tip: add
a handful of water, and cover to speed this along.) Add chopped tomato, tomato
sauce, salt and pepper, and simmer for about twenty minutes. During this time,
cook and drain pasta. Add the remaining olive oil to pasta along with fresh
basil and half the cheese. Don recommends the cheese be a mixture of Parmesan,
Romano, mozzarella, and cheddar. Spread this on a big platter and top with the
cauliflower mixture. Top with remaining cheese. Don recommends a California
Gewurztraminer ``to go with.''  Serves \numrange{2}{3}.
\end{entry}

%%---------------------------------------------------------------------------%%
\begin{entry}{Sweet and Sour Pork}{First Edition}
\index{Pork!sweet and sour pork}
\index{Evans!Kate}
\index{Evans!Tom}

\begin{open}
  Tom and Kate eat this a lot; its a ``regular''. It's word-for-word from a
  Southern Living year-end cookbook I love (1992, if curious). Its quite
  delicious, and its even somewhat healthy.
\end{open}
\begin{ingredients}
  \SI{1}{\tblspoon} sherry\\
  \SI{1}{\tblspoon} soy sauce\\
  \SI{1}{\tblspoon} cornstarch\\
  \SI{1}{\pound} boneless pork, cut into cubes\\
  \SI{1/4}{\cup} vegetable oil, divided\\
  1 clove garlic, minced\\
  1 small onion (optional)\\
  2 green peppers, cut into \SI{1}{\inch} pieces\\
  \SI{1/3}{\cup} sugar\\
  \SI{1/4}{\cup} ketchup\\\
  \SI{1}{\tblspoon} sherry\\
  \SI{2}{\tblspoon} soy sauce\\
  \SI{2}{\tblspoon} white vinegar\\
  \SI{1}{\tblspoon} cornstarch\\
  \SI{1/3}{\cup} water\\
  1 \SI{8}{\ounce} can pineapple slices in juice, each cut into about 8 pieces
\end{ingredients}
%
\protip{Using fresh pineapple makes this dish so much better!}
%
Combine first 3 ingredients, add pork, and let marinate 20 minutes (or however
long it takes to prepare everything else). Heat \SI{2}{\tblspoon} oil in big
frying pan.  Stir fry onion garlic, and green pepper over med-high heat until
crisp tender.  Remove from skillet. Add rest of oil and cook pork until cooked
through.  Stir in cooked vegetables. Combine sugar and next 6 ingredients,
stirring until cornstarch dissolves. Add to pork mixture and cook until it
comes to a boil. Add pineapple and and boil for about 1 minute. Serve over hot
cooked rice. Serves \numrange{2}{3} hungry people.
\end{entry}