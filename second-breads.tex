\secpart{Second Edition}{Breads}

%%---------------------------------------------------------------------------%%
\begin{entry}{Not Very French Crepes}
\index{breads!not very french crepes}
\index{Johnson!Gege}
\index{Johnson!Don}

\begin{open}
    Hey, anything that's not very French I love.  Also, are crepes (or
    cr\^{e}pes as I know some of you will insist, even though both spellings are
    valid), breads, deserts, or main dishes?  Who knows, who cares, we put them
    here.

    As Don and Gege say, ``The crepes are but a vessel for whatever good stuff
    you put into the filling.''  We could not agree more.

    This is a recipe derived from \url{allrecipes.com}; it is absolutely
    foolproof, at least according to Don and Gege
\end{open}
\begin{ingredients}
    \SI{1}{\cup} flour\\
    2 eggs\\
    \SI{1/2}{\cup} milk\\
    \SI{1/2}{\cup} water\\
    dash of salt\\
    \SI{2}{\tblspoon} melted butter
\end{ingredients}
Beat the eggs and mix thoroughly into the flour. Add milk and water and beat as
long as you can or until no lumps. Add salt and beat in melted butter. It's a
very thin batter---don't be alarmed. Drop small ladlefuls into a skillet and
rotate so the batter coats the bottom in a very thin layer; no oil needed after
the first one. Makes \numrange{8}{10} crepes.

\minisection{Filling}

\noindent No measurements here; use the Force and whatever you have handy
\begin{ingredients}
    Half pound to a pound ground meat or sausage\\
    A little soy sauce---dark is better than light\\
    Diced onion\\
    Chopped tomatoes---\numrange{1}{2} large or several small
\end{ingredients}
The above three ingredients are the base, add other things as convenient and
to take the filling in different directions---mushrooms, spinach, chopped
peppers, parmesan cheese are all good.

Brown the ground meat/sausage in oil, add a little soy sauce for flavor and
color. Add onion and cook until tender and transparent. If you have mushrooms,
peppers, etc. add them and \saute. Add tomatoes last and \saute until you
have a thick sauce or paste. If you are using spinach or cheese stir them in
right before serving.

Put a plate of crepes and a big bowl of sauce on the table and let people roll
their own. Crepes are tender and knife and fork are usually required.
\end{entry}

%%---------------------------------------------------------------------------%%
\begin{entry}{Alison's Favorite Maltese Dish from Gozo}
\index{breads!Alison's Maltese Dish}
\index{Rona!Alison}

\begin{open}
  For those of you geographically challenged, or at the least, not familiar
  with the geography of Malta, Gozo is a small island just north of the main
  island of Malta.  This spread comes from Alison Rona.
\end{open}
%%
\begin{ingredients}
    very ripe red tomatoes\\
    capers\\
    black olives with herbs\\
    garlic\\
    basil\\
    mint\\
    olive oil\\
    salt\\
    pepper
\end{ingredients}
Marinate the tomatoes for several days with capers, olives and herbs, lots of
chopped garlic, lots of chopped basil and mint, lots of olive oil, and salt
and pepper.  Once the mixture is tasty, spread thickly on fresh crusty thick
bread.
\end{entry}

%%---------------------------------------------------------------------------%%
\begin{entry}{Cr\`eme Br\^ul\'ee French Toast}
\index{breakfast!cr\`eme br\^ul\'ee french toast}
\index{Evans!Fermina}

\begin{open}

\end{open}
%%
\begin{ingredients}
    \SI{1/2}{\cup} (1 stick) unsalted butter \\
    \SI{1}{\cup} packed brown sugar \\
    \SI{2}{\tblspoon} corn syrup \\
    Bread (see comments below)\\
    5 large eggs\\
    1\SI{1/2}{\cup} half-and-half cream \\
    \SI{1}{\teaspoon} vanilla extract \\
    \SI{1}{\teaspoon} Grand Marnier \\
    \SI{1/4}{\teaspoon} salt
\end{ingredients}
Regarding the bread, the recipe calls for 8 to 9 inch round country loaf with
crusts removed, but challah is Fermina's favorite, with brioche a close
second. Really you can use any type, even baguettes with the crusts still
on. And day-old bread is even better, as it soaks up more of the sauce, so
whatever you have around to use up is the reason to make this dish!

In a small heavy saucepan, melt butter with brown sugar and corn syrup over
medium heat, stirring constantly until smooth and pour into a \SI{9x13}{\inch}
baking pan. Cut 6 1-inch thick slices from center portion of bread, reserving
ends for another use, and trim crusts if you choose. Arrange bread slices in
one layer in baking dish over butter/sugar mixture, squeezing them slightly to
fit. In a bowl, whisk together eggs, cream, vanilla, Grand Marnier, and salt
until well combined and pour evenly over bread. Chill bread mixture, covered,
for at least 8 hours and up to 1 day.

When ready to bake, bring bread to room temperature and preheat oven to
\SI{350}{\degreeF}. Bake bread mixture, uncovered, in the middle of the oven
until puffed and edges are pale golden, 35-40 minutes. Serve immediately.
\end{entry}

%%---------------------------------------------------------------------------%%
\begin{entry}{World's Best Granola Ever}
\index{breakfast!world's best granola}
\index{Evans!Kate}

\begin{open}
  Kate has no strong feelings about granola, but recently read an article
  about how there exists a granola so good, people cannot stop talking about
  it. I mean, the maple sugar and olive oil is divine! So she made it herself
  with several adaptations to suit her tastes and WOW. She now makes it
  monthly so that there is always some on hand. Here is the link to the
  article,
  \url{food52.com/recipes/15831-nekisia-davis-olive-oil-maple-granola} so you
  can go see the original version or check out the comments section for more
  adaptation ideas.
\end{open}
%%
\begin{ingredients}
    \SI{3}{\cup} rolled oats (e.g. Bob's Red Mill whole grain)\\
    \SI{1}{\cup} hulled raw pumpkin seeds\\
    \SI{1}{\cup} dried cranberries\\
    \SI{1}{\cup} unsweetened coconut chips\\
    1\SI{1/4}{\cup} raw pecans, chopped\\
    \SI{3/4}{\cup} pure maple syrup\\
    \SI{3/4}{\cup} extra-virgin olive oil\\
    1 pinch coarse salt, to taste
\end{ingredients}
Preheat oven to \SI{300}{\degreeF}. Place oats, pumpkin seeds, cranberries,
coconut, pecans, syrup, olive oil, sugar, and salt in a large bowl and mix
until well combined. Spread granola mixture in an even layer on a rimmed
baking sheet. Transfer to oven and bake, stirring about every 15 minutes until
granola is toasted, about 45 minutes. Remove granola from oven and let cool
completely before serving. Store in an airtight container for up to 1 month.
\end{entry}
