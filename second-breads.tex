\secpart{Second Edition}{Breads}

%%---------------------------------------------------------------------------%%
\begin{entry}{Not Very French Crepes}{Second Edition}
\index{breads!not very french crepes}
\index{Liu!Geyi}
\index{Johnson!Don}

\begin{open}
    Hey, anything that's not very French I love.  Also, are crepes (or
    cr\^{e}pes as I know some of you will insist, even though both spellings are
    valid), breads, deserts, or main dishes?  Who knows, who cares, we put them
    here.

    As Don and Geyi say, ``The crepes are but a vessel for whatever good stuff
    you put into the filling.''  We could not agree more.

    This is a recipe derived from \url{allrecipes.com}; it is absolutely foolproof, at least according to Don and Geyi
\end{open}
\begin{ingredients}
    \SI{1}{\cup} flour\\
    2 eggs\\
    \SI{1/2}{\cup} milk\\
    \SI{1/2}{\cup} water\\
    dash of salt\\
    \SI{2}{\tblspoon} melted butter
\end{ingredients}
Beat the eggs and mix thoroughly into the flour. Add milk and water and beat as
long as you can or until no lumps. Add salt and beat in melted butter. It's a
very thin batter---don't be alarmed. Drop small ladlefuls into a skillet and
rotate so the batter coats the bottom in a very thin layer; no oil needed after
the first one. Makes \numrange{8}{10} crepes.

\minisection{Filling}

\noindent No measurements here; use the Force and whatever you have handy
\begin{ingredients}
    Half pound to a pound ground meat or sausage\\
    A little soy sauce---dark is better than light\\
    Diced onion\\
    Chopped tomatoes---\numrange{1}{2} large or several small
\end{ingredients}
The above three ingredients are the base, add other things as convenient and
to take the filling in different directions---mushrooms, spinach, chopped
peppers, parmesan cheese are all good.

Brown the ground meat/sausage in oil, add a little soy sauce for flavor and
color. Add onion and cook until tender and transparent. If you have mushrooms,
peppers, etc. add them and \saute. Add tomatoes last and \saute until you
have a thick sauce or paste. If you are using spinach or cheese stir them in
right before serving.

Put a plate of crepes and a big bowl of sauce on the table and let people roll
their own. Crepes are tender and knife and fork are usually required.
\end{entry}

%%---------------------------------------------------------------------------%%
\begin{entry}{Alison's Favorite Maltese Dish from Gozo}{Second Edition}
\index{breads!Alison's Maltese Dish}
\index{Horne Rona!Alison}

\begin{open}
  For those of you geographically challenged, or at the least, not familiar
  with the geography of Malta, Gozo is a small island just north of the main
  island of Malta.  This spread comes from Alison Horne Rona.
\end{open}
%%
\begin{ingredients}
    very ripe red tomatoes\\
    capers\\
    black olives with herbs\\
    garlic\\
    basil\\
    mint\\
    olive oil\\
    salt\\
    pepper
\end{ingredients}
Marinate the tomatoes for several days with capers, olives and herbs, lots of
chopped garlic, lots of chopped basil and mint, lots of olive oil, and salt
and pepper.  Once the mixture is tasty, spread thickly on fresh crusty thick
bread.
\end{entry}

%%---------------------------------------------------------------------------%%
\begin{entry}{Cr\`eme Br\^ul\'ee French Toast}{Second Edition}
\index{breakfast!cr\`eme br\^ul\'ee french toast}
\index{Evans!Fermina}

\begin{open}

\end{open}
%%
\begin{ingredients}
    \SI{1/2}{\cup} (1 stick) unsalted butter \\
    \SI{1}{\cup} packed brown sugar \\
    \SI{2}{\tblspoon} corn syrup \\
    Bread (see comments below)\\
    5 large eggs\\
    1\SI{1/2}{\cup} half-and-half cream \\
    \SI{1}{\teaspoon} vanilla extract \\
    \SI{1}{\teaspoon} Grand Marnier \\
    \SI{1/4}{\teaspoon} salt
\end{ingredients}
Regarding the bread, the recipe calls for 8 to 9 inch round country loaf with
crusts removed, but challah is Fermina's favorite, with brioche a close
second. Really you can use any type, even baguettes with the crusts still
on. And day-old bread is even better, as it soaks up more of the sauce, so
whatever you have around to use up is the reason to make this dish!

In a small heavy saucepan, melt butter with brown sugar and corn syrup over
medium heat, stirring constantly until smooth and pour into a \SI{9x13}{\inch}
baking pan. Cut 6 1-inch thick slices from center portion of bread, reserving
ends for another use, and trim crusts if you choose. Arrange bread slices in
one layer in baking dish over butter/sugar mixture, squeezing them slightly to
fit. In a bowl, whisk together eggs, cream, vanilla, Grand Marnier, and salt
until well combined and pour evenly over bread. Chill bread mixture, covered,
for at least 8 hours and up to 1 day.

When ready to bake, bring bread to room temperature and preheat oven to
\SI{350}{\degreeF}. Bake bread mixture, uncovered, in the middle of the oven
until puffed and edges are pale golden, 35-40 minutes. Serve immediately.
\end{entry}

%%---------------------------------------------------------------------------%%
\begin{entry}{World's Best Granola Ever}{Second Edition}
\index{breakfast!world's best granola}
\index{Evans!Kate}

\begin{open}
 Kate has no strong feelings about granola, but recently read an article
  about how there exists a granola so good, people cannot stop talking about
  it. I mean, the maple sugar and olive oil is divine! So she made it herself
  with several adaptations to suit her tastes and WOW. She now makes it
  almost monthly so that there is always some on hand. Here is the link to the
  article,
  \url{food52.com/recipes/15831-nekisia-davis-olive-oil-maple-granola} so you
  can go see the original version and check out the comments section for more
  adaptation ideas.
\end{open}
%%
\begin{ingredients}
    \SI{3}{\cup} rolled oats (e.g. Bob's Red Mill whole grain)\\
    \SI{1}{\cup} hulled raw pumpkin seeds\\
    \SI{1}{\cup} dried cranberries\\
    \SI{1}{\cup} unsweetened coconut chips\\
    1\SI{1/4}{\cup} raw pecans, chopped\\
    \SI{3/4}{\cup} pure maple syrup\\
    \SI{1/2}{\cup} extra-virgin olive oil\\
    1 pinch coarse salt, to taste
\end{ingredients}
Preheat oven to \SI{300}{\degreeF}. Place oats, pumpkin seeds, cranberries,
coconut, pecans, syrup, olive oil, sugar, and salt in a large bowl and mix
until well combined. Spread granola mixture in an even layer on a rimmed
baking sheet. Transfer to oven and bake, stirring about every 15 minutes until
granola is toasted, about 45 minutes. Remove granola from oven and let cool
completely before serving. Store in an airtight container for up to 1 month.
\end{entry}

%%---------------------------------------------------------------------------%%
\begin{entry}{Katie's Dough}{Second Edition}
\index{!world's best granola}
\index{Pross!Katie}

\begin{open}
 
\end{open}
%%
\begin{ingredients}
    \SI{3}{\cup} rolled oats (e.g. Bob's Red Mill whole grain)\\
    \SI{1}{\cup} hulled raw pumpkin seeds\\
    \SI{1}{\cup} dried cranberries\\
    \SI{1}{\cup} unsweetened coconut chips\\
    1\SI{1/4}{\cup} raw pecans, chopped\\
    \SI{3/4}{\cup} pure maple syrup\\
    \SI{1/2}{\cup} extra-virgin olive oil\\
    1 pinch coarse salt, to taste
\end{ingredients}
Preheat oven to \SI{300}{\degreeF}. Place oats, pumpkin seeds, cranberries,
coconut, pecans, syrup, olive oil, sugar, and salt in a large bowl and mix
until well combined. Spread granola mixture in an even layer on a rimmed
baking sheet. Transfer to oven and bake, stirring about every 15 minutes until
granola is toasted, about 45 minutes. Remove granola from oven and let cool
completely before serving. Store in an airtight container for up to 1 month.
\end{entry}

%%---------------------------------------------------------------------------%%
%% From the first edition

%%---------------------------------------------------------------------------%%
\begin{entry}{Three Kings Bread (and St. Nick)}{First Edition}
\index{Breads!Three Kings Bread}
\index{Vegetarian!Three Kings Bread}
\index{King!Jenn}
\index{King!Rich}
\index{King!Julian}

\begin{center}
    \includegraphics[width=.4\textwidth,clip]{figures/kings.pdf}
\end{center}

\begin{open}
  Is this the real name or is it because it comes from three Kings?
  This is a recipe supplied to us from Rich, Jenn and Julian King (Nicholas arrived after the project, began-hence the name). (editor's note from the second edition: yet more children arrived and have come of age in the King family)
  This recipe makes one loaf.
\end{open}
\begin{ingredients}
  \SI{1/4}{\cup} plus one \si{\tblspoon} sour cream \\
  \SI{1}{\teaspoon} baking soda \\
  \SI{1/2}{\cup} butter at room temperature \\
  \SI{1}{\cup} sugar \\
  2 eggs lightly beaten \\
  1 ripe mashed banana and 1 medium apple (2 bananas an option) \\
  \SI{1/2}{\teaspoon} baking powder \\
  \SI{1}{\cup} chopped nuts \\
  \SI{1/2}{\teaspoon} cardamom \\
  1 zest of lemon \\
  \SI{1}{\teaspoon} vanilla extract
\end{ingredients}
Preheat the oven to \SI{350}{\degree}.  Grease a \SI{9x5}{\inch} loaf pan.
Combine sour cream and baking soda in small bowl.  Set aside (it will foam).
Cream butter and sugar in a small bowl.  Beat in eggs, fruit, and sour cream
mixture.  Slowly mix in all dry ingredients.  Bake until a toothpick inserted
into center comes out sort of clean and loaf is golden brown.  This should be
about 1 hour.

Cool 10 minutes in pan. Turn loaf out onto rack and cool completely.  Eat
thinking of your most favorite Kings$\ldots$
\end{entry}

%%---------------------------------------------------------------------------%%
\begin{entry}{Kate's Standard Bagels}{First Edition}
\index{Breads!bagels}
\index{Vegetarian!bagels}
\index{Evans!Kate}

\begin{open}
  This is a recipe by Kate that she picked up in Washington, D.C. during
  a 91'--92' winter internship at NIST from Dr. Lucatorto.  We enjoy this one
  a lot (especially in the south where getting good bagels is not always
  easy). Note: you need a food processor for this recipe. It makes 16 bagels.
\end{open}
\begin{ingredients}
  2 packets yeast \\
  2 scant \si{\tblspoon} sugar\\
  \SI{1/2}[3]{\cup} warm water with salt \\
  Lots (several pounds) of flour
\end{ingredients}
Combine yeast, sugar, and water.  Add \SI{\sim 2}{\cup} flour to food
processor. Pour in water mixture until a dough is formed (stop pouring when
processor begins to ``growl.'' Listen you'll hear it).  Remove dough to a
casserole dish with lid. Repeat until all mixture is used.  Microwave dish at
\SI{30}{\percent} for 3 minutes.  Let rise for 30~minutes.  Punch down roll
into loaf, cut in 16 pieces and make into bagels. Kate uses a doughnut
stamper. Let rise 30~minutes. In large frying pan, set \SI{2}{\inch} water to
boil.  Boil bagels for 10~seconds each side.  Place on greased cookie sheet
and bake 30~minutes or until lightly brown at \SI{350}{\degree}. If you want
to add extras such as cinnamon or raisins, add when processor starts to
growl. If you would like toppings such as sesame seeds, brush bagel with egg
white and sprinkle on top just before baking.
\end{entry}

%%---------------------------------------------------------------------------%%
\begin{entry}{Kate's Super Stromboli Dough}{First Edition}
\index{Evans!Kate}
\index{Breads!stromboli dough}
\index{Vegetarian!stromboli dough}
\label{sec:stromboli}

\begin{open}
  This is a recipe by Kate.  While this recipe is intended for Stromboli
  or calzone it also makes a fine pizza dough.  The
  recipe serves 6.
\end{open}
\begin{ingredients}
  4 scant \si{\cup} flour \\
  1 package \corp{Quick Rise} yeast \\
  \SI{1}{\teaspoon} salt \\
  \SI{1}{\tblspoon} sugar \\
  \SI{1/3}[1]{\cup} warm water \\
  \SI{1/4}{\cup} oil
\end{ingredients}
In mixing bowl combine flour, yeast, sugar, and salt.  Add water and oil and
form a soft dough. Add flour or water as necessary.  Let rise 30 minutes, then
punch down (you can freeze at this point to thaw later in microwave).  Roll
into \numrange{4}{6} circles (depending on crowd hunger). Add your favorite
toppings, including sauce if desired, and of course cheese. Possible fillings
are broccoli (a Kate favorite), spinach, pepperoni, ham, onions, mushrooms,
and almost anything edible you can think of. Bake at \SI{400}{\degree} for
\numrange{12}{15} minutes until golden brown.
\end{entry}

%%---------------------------------------------------------------------------%%
\begin{entry}{Kuchen}{First Edition}
\index{Breads!kuchen}
\index{Vegetarian!kuchen}
\index{Evans!Fermina}

\begin{open}
  Provided by Fermina Evans, this German ``bread'' is a Christmas morning tradition that she has carried on from Geo's parents and passed on to the Evans's kids. She always makes a double batch and it's still barely enough!
  Don't be fooled by its location in the ``breads'' section; its definitely a
  treat!
\end{open}
\begin{ingredients}
  \SI{1/4}{\cup} shortening (vegetable makes it a vegetarian dish)\\
  \SI{1}{\cup} sugar \\
  1 egg \\
  \SI{1/2}{\cup} milk \\
  \SI{1/2}[1]{\cup} flour \\
  \SI{2}{\teaspoon} baking powder \\
  salt to taste
\end{ingredients}
The topping:
\begin{ingredients}
  \SI{1/2}{\cup} brown sugar \\
  \SI{1/3}{\cup} flour \\
  \SI{1}{\teaspoon} cinnamon \\
  dash salt\\
  \SI{1/4}{\cup} butter
\end{ingredients}
Preheat oven to \SI{350}{\degreeF}. Cream shortening and sugar. Add egg and mix
well.  Add baking powder, flour and milk. Pour into greased and floured
\SI{9}{\inch} round or \SI{8}{\inch} square baking pan. Combine the rest of
the topping ingredients except butter. Then, cut butter into topping mixture
and sprinkle on top of batter. Bake for \numrange{30}{40} minutes (test with
toothpick for done-ness).
\end{entry}

%%---------------------------------------------------------------------------%%
\begin{entry}{Weedie's Blueberry Muffins}{First Edition}
\index{Breakfast!blueberry muffins}
\index{Johnson!Weedie}

\begin{open}
  No joke, Don and Kate (and surely David and Steve) used to beg Weedie to make these muffins every time we visited. And she always did. Martha and Dodge say, ``Oh Heaven, these and some Lobster salad!'' They are perfection,
  we promise you. Just make the effort to get quality blueberries. We
  recommend going to Maine to get them.
\end{open}
\begin{ingredients}
  2 scant \si{\cup} flour\\
  \SI{3}{\teaspoon} baking powder\\
  \SI{1/2}{\cup} sugar\\
  \SI{1/2}{\teaspoon} salt\\
  \num{1/2} stick butter, melted (\SI{1/4}{\cup})\\
  2 eggs\\
  \SI{1}{\cup} milk\\
  \SI{1}{\cup} blueberries (little wild ones are best)\\
  sugar\\
  cinnamon
\end{ingredients}
Preheat oven to \SI{400}{\degreeF}. Weedie says, ``Usually I wash the berries
awhile before using them, so they can dry off before being added.'' In mixing
bowl combine flour, baking powder, sugar, and salt. Mix butter eggs and milk
in a separate bowl. Pour this over the flour mixture and stir until
smooth. Add blueberries, stir gently, and spoon into buttered muffin
tins. Sprinkle with sugar and cinnamon and bake for \numrange{20}{25} minutes.
\end{entry}