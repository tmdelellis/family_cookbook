\secpart{Second Edition}{Soups}

%%---------------------------------------------------------------------------%%
\begin{entry}{Spicy Thai Chicken Soup}
\index{soups!spicy Thai chicken}
\index{Lindquist!Jen}

\begin{open}
    Here's a tempting soup recipe from Jen Lindquist.
\end{open}
%%
\begin{ingredients}
    2 large cloves of garlic, chopped\\
    1 medium onion, chopped\\
    1 red bell pepper, chopped\\
    \SI{2}{\tblspoon} ginger paste or fresh ginger\\
    \SI{1}{\tblspoon} lemongrass paste\\
    \SI{2}{\tblspoon} red curry paste\\
    \SI{1}{\tblspoon} cilantro paste\\
    \SI{1}{\tblspoon} red chili paste (can also use 2 red chilies, chopped)\\
    \SI{2}{\tblspoon} coconut oil\\
    1 lime, zested and juiced\\
    \SI{1}{\tblspoon} fish sauce\\
    \SI{4}{\cup} chicken stock\\
    1 can (\SI{13.5}{\fluidounce}) coconut milk\\
    \SI{2}{\cup} shredded chicken
\end{ingredients}
For garnish:
\begin{ingredients}
    1 bunch of cilantro leaves chopped\\
    \numrange{4}{5} green onions, thinly sliced, both white and green parts\\
    1 green chili, sliced, seeds removed\\
    1 red chili, sliced, seeds removed
\end{ingredients}
To serve:
\begin{ingredients}
    roughly chopped coriander\\
    sliced red chili\\
    sliced green onion
\end{ingredients}
Combine garlic, onion, red pepper, ginger paste, lemongrass paste, red curry
paste, cilantro paste, red chili paste and coconut oil in a food processor, and
process until a paste forms. Add the paste to a medium pot and fry it for a
couple of minutes, just until it's fragrant. Add the chicken stock and coconut
milk and bring to a boil, reduce heat to a simmer and simmer for about
\SI{10}{\minute}. In the last couple of minutes, add your shredded chicken. Add
your fish sauce, lime juice and zest and rice noodles to the soup. Garnish with
chopped cilantro, green onion and green and red chilies---see
Fig.~\ref{fig:spicy-thai-soup}.
%%
\begin{figure}
    \centering
    \includegraphics[width=0.8\textwidth]{figures/spicy-thai-soup}
    \caption{Spicy Thai Chicken Soup!}
    \label{fig:spicy-thai-soup}
\end{figure}
\end{entry}

%%---------------------------------------------------------------------------%%
\begin{entry}{Butternut Squash Soup}
\index{soups!butternut squash}
\index{Lindquist!Dot}

\begin{open}
  More from Dot Lindquist and \url{myrunawaykitchen.com}! According to Dot,
  ``This soup (with bacon and bread) eats like a meal and will warm your belly
  and keep the rest of your bread in your wallet\textellipsis have you noticed
  how super-cheap butternut squash is!?  They are practically giving it away.
  This soup freezes and reheats well and is a treat for lunch, too!'' (Note:
  we make no promises about the future price of butternut squash).

  This makes 15 servings, takes \SI{25}{\minute} of prep time and
  \SI{45}{\minute} of cooking time.
\end{open}
%%
\begin{ingredients}
    4  apples---any variety\\
    4  med sized yellow onions\\
    3  med sized butternut squash\\
    \SIrange{6}{8}{\cup} stock (chicken or vegetable)\\
    drizzle  extra virgin olive oil\\
    \SI{1/2}{\teaspoon}  nutmeg\\
    \SI{1/2}{\teaspoon}  sage\\
    \SI{1.75}{\pound} thick cut bacon (I like applewood smoked)
\end{ingredients}
Set your oven to \SI{375}{\degreeF} and roast your butternut squash. You can
do this simply by
\begin{description}
    \item[Option One] cutting long-ways, scooping out the insides, drizzle with
    olive oil, sprinkle with salt and pepper and place it cut-side down on a
    baking sheet
    \item[Option Two] you can do it the way shown above---halve
    your squash, scoop out the insides, peel and cube it, spread into a pan and
    drizzle with oil, sprinkle with salt and pepper, see Fig.~\ref{fig:butternut-squash-two}.
\end{description}
Once the squash is tender (if you can easily stick a fork into it, it's done),
you either have to scoop out the roasted squash from the outer shell, or you
have to put the work in up front and peel ahead of time. Your choice! Option
one is much quicker\textellipsis if you can afford to wait for the squash to
cool before you scoop it. Option two is quicker if you have lovely teenage
nieces who don't mind peeling for you!!
%%
\begin{figure}
    \centering
    \includegraphics[width=0.7\textwidth]{figures/butternut-squash}
    \caption{Option two for roasting butternut squash.}
    \label{fig:butternut-squash-two}
\end{figure}

Dice your onions, peel and dice your apples. Melt diced apples and onions
together with a drizzle of olive oil in the bottom of your soup pot, over
medium heat, until wilted and aromatic. Add sage, butternut squash, and
nutmeg, and stir. Cover mixture with stock, and bring to a boil. Then, reduce
heat. Blend with immersion blender. Season with salt and pepper to taste.

Don't forget the bacon! I like to lay bacon out flat on a baking sheet and stick
it into the oven while the squash is roasting. I let it crisp up, then drain off
the excess fat and spread on paper towel. Chop it up when it's cool and set on
the side, to be served atop your soup! For an extra decadent treat, add a dollop
of sour cream, creme fraiche, or heavy cream right into your soup bowl and then
sprinkle on your bacon (Fig.~\ref{fig:butternut-squash-soup}). Yuuuuummm. Serve
with warm, crusty bread and enjoy!
%%
\begin{figure}
    \centering
    \includegraphics[width=0.7\textwidth]{figures/butternut-squash-final}
    \caption{Butternut squash soup!}
    \label{fig:butternut-squash-soup}
\end{figure}
\end{entry}

%%---------------------------------------------------------------------------%%
\begin{entry}{Potato Carrot Cheese Soup}
\index{soups!potato carrot cheese}
\index{Evans!Kate}

\begin{open}
  This recipe began from a recipe in the classic 80's cookbook, ``the Silver
  Palate cookbook'' by Rosso and Lukins, but has been altered so much over
  time that it's become one of Kate's staples she makes from some version of
  it in her memory. This works great on its own on a cold, rainy night or as a
  first course to a heavier meal. Feel free to add onions with the carrots
  instead of onion powder, they are missing since Kate is intolerant to them.
\end{open}
%%
\begin{ingredients}
    \SI{2}{\tblspoon} butter\\
    3-4 carrots, peeled and coarsely chopped\\
    \SI{1}{\teaspoon} onion powder\\
    5-6 medium-sized (2-3\SI{1/2}{\pound}) yukon gold potatoes, peeled and
    coarsely chopped \\
    \SI{5}{\cup} broth (chicken or vegetable)\\
    \SI{1/4}{\cup} white wine \\
    1 sprig fresh rosemary \\
    \SI{1/2}{\cup} fresh parlsey, chopped\\
    \SI{1}{\teaspoon} each salt and pepper\\
    \SI{1}{\cup} extra sharp cheddar cheese, grated
\end{ingredients}
In large dutch oven, melt butter over low heat. Then add carrots and onion
powder and saute in butter until soft, about 15-20 minutes. Then add potatoes,
broth, wine rosemary, parsley, salt and pepper and bring to a boil. Cover,
reduce heat, and simmer until potatoes are soft, about 30-45 minutes. remove
rosemary spring and discard. It's all right if some needles remain in the
pot. Blend the soup until smooth, either using an immersion hand blender or
placing the soup contents into a food processor in batches and blending. if
the latter, return blended soup to dutch oven. Slowly add cheddar and stir
until incorporated. Taste and add more salt, pepper, or broth as
desired. Serve immediately.
\end{entry}