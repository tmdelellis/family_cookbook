%%---------------------------------------------------------------------------%%
%% The Recipe Book
%%---------------------------------------------------------------------------%%
\documentclass[12pt]{article}
\usepackage{makeidx}
\usepackage{wrapfig}
\usepackage[margin=1in]{geometry}
\usepackage{xfrac}
\usepackage[detect-all, binary-units]{siunitx}
\usepackage{hyperref}
\usepackage{tocloft}
\usepackage{xspace}
\usepackage{color}
\usepackage[explicit]{titlesec}
\usepackage{fancyhdr}
\usepackage[breakable,many,minted]{tcolorbox}

%% Font stuff
\usepackage[T1]{fontenc}
\usepackage[utf8]{inputenc}

%% Using Bookman font
\usepackage{bookman}
\usepackage{CJKutf8}

%% Local packages
\usepackage{tmecolors}

%%---------------------------------------------------------------------------%%
%% >>> STYLE DEFINITIONS

%% Default colors for tcolorbox
\colorlet{TCBDefaultBack}{black!5!white}
\colorlet{TCBDefaultFrame}{black!75!white}

%% Index
\makeindex

%% Commands
\newcommand{\name}[1]{\textit{#1}}
\newcommand{\corp}[1]{\textsf{#1}}
\newcommand{\oven}[1]{#1$^{\circ}$}
\newcommand{\meas}[1]{\ensuremath{\mathbf{#1}}}
\newcommand{\fr}[2]{#1\,\footnotesize{#2}}

\newcommand{\saute}{saut\'{e}\xspace}
\newcommand{\sauteed}{saut\'{e}ed\xspace}
\newcommand{\Saute}{Saut\'{e}\xspace}

\newcommand{\secpart}[2]
{
    %%
    \clearpage
    \thispagestyle{plain}
    \vspace*{\fill}
    \section{#2}
    \vspace*{\fill}
}

%\newcommand{\protip}[1]
%{
%    \begin{center}
%    \fbox{%
%        \parbox{0.9\textwidth}{{\bf Pro Tip}: #1}
%    }
%    \end{center}
%}

\newcommand{\protip}[1]
{
    \begin{tcolorbox}%
        [title=Pro Tip,%
        center title,%
        width=0.8\textwidth,%
        beforeafter skip=1\baselineskip,%
        center]
        #1
    \end{tcolorbox}
}

%% Environments
\newenvironment{ingredients}
{
    \begin{list}{}{\setlength{\listparindent}{0pt}}
    \item\bf
}
{
    \end{list}
}

\newenvironment{open}
{
    \slshape
}
{
}

\newenvironment{entry}[2]
{
    %%
    \titleformat{\subsection}
    {\Large\bfseries}{}{0em}
    {%
        \setbox0=\hbox{\Large\bfseries\thesection}%
        \begin{tikzpicture}
            %\draw[orange, ultra thick](0,0) -- ++(0:\linewidth);
            \node[draw=black,
            rounded corners,
            fill=TCBDefaultBack,
            ultra thick,
            anchor=west,
            text width=\textwidth,
            ] {{\large\mdseries #2}\\\vspace{0.5\baselineskip}
            \thesubsection\hspace{.5em}\hangindent\wd0\strut{#1}\strut};
        \end{tikzpicture}%
    }
    %%
    \clearpage
    \subsection{#1}
}
{

}

% Minisections
\newcommand{\minisection}[1]{\vspace{0.5\baselineskip}%
    \noindent\underline{#1}%
    \vspace{0.5\baselineskip}\noindent}

% Units setup
\sisetup{range-phrase = \text{--},
  group-separator={\,},
  per-mode=reciprocal,
  group-minimum-digits=3,
  inter-unit-product = \ensuremath{{}\cdot{}},
  range-units=single,
  quotient-mode=fraction,
  product-units=single,
  fraction-function=\sfrac}
\DeclareSIUnit\tblspoon{Tbsp}
\DeclareSIUnit\teaspoon{tsp}
\DeclareSIUnit\cup{C}
\DeclareSIUnit\pound{lb}
\DeclareSIUnit\ounce{oz}
\DeclareSIUnit\fluidounce{fl oz}
\DeclareSIUnit\pint{pt}
\DeclareSIUnit\quart{qt}
\DeclareSIUnit\inch{$''$}
\DeclareSIUnit\degreeF{\degree F}
\DeclareSIUnit\mill{\ml}

\hypersetup{
    linkcolor=[rgb]{0.18,0.31,0.31},
    %linkcolor=[rgb]{0.28235294, 0.23921569, 0.54509804},
    %linkcolor=[rgb]{0.43921569, 0.50196078, 0.56470588},
    citecolor=[rgb]{0.780,0.647,0.258},
    urlcolor=[rgb]{0.325,0.494,0.658},
    colorlinks=true
}

%% TOC adjust
\cftsetindents{subsection}{1.5em}{3em}

%%---------------------------------------------------------------------------%%
\begin{document}

%%---------------------------------------------------------------------------%%
%% >>> FRONT MATTER
\titleformat{\section}{\Large\bfseries}{\thesection}{1em}{#1}[]

%% Title page
\begin{titlepage}
\vspace*{1.5in}
\begin{tcolorbox}%
    [title=The Extended Family Cookbook,%
    center title,%
    fonttitle=\Huge\bfseries,%
    fontupper=\LARGE\bfseries,%
    fontlower=\Large\bfseries,%
    center upper,%
    lower separated=false,%
    center lower,%
    boxsep=3mm,%
    top=2\baselineskip,%
    middle=2\baselineskip,%
    center]
    (Now with Soup!)
    \tcblower
    Second Edition
\end{tcolorbox}
\end{titlepage}

%% Copyright page
\vspace*{\fill}
%Copyright\,\copyright\, 1996, 2022 Tom and Kate Publishing People, Inc. 
This document was prepared using the \LaTeX\ typesetting language. Edited and
compiled by Tom and Kate Evans. First Edition printed on 12-20-96, with recipes
received between 9-95 and 12-96. Second Edition released 2-1-22, with recipes collected during the pandemic, summer 2020 to January 2022. May not be reprinted without permission (just kidding, go crazy! Just speak well of
us.).

\vspace{1\baselineskip}
We are not responsible for terrible recipes; your poison is another's delicacy, so keep it to yourself. That said, please send corrections to Tom \href{mailto:ktloo@tds.net}{tmdelellis@tds.net} or Kate
\href{mailto:ktloo@tds.net}{ktloo@tds.net}.
\pagebreak

%% TOC
\pagestyle{headings}
\tableofcontents

%% Forward
\clearpage
\thispagestyle{plain}
\phantomsection
\addcontentsline{toc}{section}{Forward}
\section*{Foreword}

%{\color{red}
\textbf{From the first edition: }
Finally, we know all of you have been holding your breath this last year
in anticipation of the release of the Family Cookbook.  Well Ma'
Flanders, you can rest at ease, because IT is here.  We hope everyone
enjoys each others' contributions.  We had a good time putting it
together -- well that isn't entirely true.  Let's say we enjoyed the
THOUGHT of doing it, and, of course, now that the deed is done we are
very happy with the results.

Illustrations in the cookbook are meant to spice things up. Both Kate and
Tom contributed artwork.  We hope you think that they are moderately
funny.  In the future any contributions in this area are welcome.  Some
of you may be wondering, `Why no pictures?'  Well, we felt that the realm
of photography was fraught with far too much peril.  What if we forgot a
picture of so-and-so or `That's a horrible picture of me to present to
the whole family!'  Frankly, we didn't want to deal with the possibility
of such horrors, so, artwork it was.


\textbf{The second edition:}
Wow the technology of creating a book have sure changed since we created the first edition. And thank goodness!
This edition is a `base' manuscript from which to work, but in theory we could include more submissions when you send them. In reality,  this utopia where many editions of the family cookbook could grace the bookshelves, cabinets, and, in some cases, trash cans of our once and future homes is limited by our time and motivation to enter them. Seriously though, this does mean that corrections can be added at any time and you can grab the latest pdf whenever you want to cook something. And of course, you are welcome to keep the recipes comin' and we'll eventually work them into the third edition, due out in April, 2034.

This time it was easy to add pictures as well as artwork, so that adds a nice touch, that to all who provided some visual interest along with the recipes! Well, that's all our comments for now as we are tired of doing this and
have no energy left to write, format, or look at this in general.  Please
try the recipes and let the authors know if you enjoyed them. Sending our love and wishes for some good home cooking!  
%}

\vspace{.5in}
\begin{flushright}
Tom and Kate Evans\\ Knoxville 2022
\end{flushright}

%%---------------------------------------------------------------------------%%
%% >>> MAIN DOCUMENT
%\titleformat{name=\section}%
%    {\Huge\bfseries\uppercase}%
%    {\thesection}%
%    {1em}%
%    {%
%        #1
%    }%
%    [\titlerule]
%%

\titleformat{\section}
  {\Huge\bfseries}{}{0em}
  {%
    \setbox0=\hbox{\Huge\bfseries\thesection}%
    \begin{tikzpicture}
        %\draw[orange, ultra thick](0,0) -- ++(0:\linewidth);
        \node[draw=black,
        rounded corners,
        fill=TCBDefaultBack,
        ultra thick,
        anchor=west,
        text width=\textwidth,
        ] {{\Large The Extended Family Cookbook}\\\vspace{0.5\baselineskip}
        \thesection\hspace{.5em}\hangindent\wd0\strut{#1}\strut};
    \end{tikzpicture}%
  }

%% SECTIONS
\chapter{Appetizers and Salads}

\section{Paul's Salad\index{appetizers and salads!Paul's salad}}
\index{Horne!Paul}

\begin{open}
    A salad recipe from Paul Horne that serves \numrange{2}{3}.
\end{open}
%%
\begin{ingredients}
    10 leaves of Romaine lettuce, cut into bite-sized pieces\\
    1 or 2 stalks of celery cut into cubes\\
    1 long cucumber, peeled, cut into cubes\\
    1 medium-sized tomato, cut into bite-sized pieces\\
    \SI{6}{\ounce} or so Gruyere cheese, cut into cubes\\
    \SI{5}{\ounce} tin of Albacore tuna in oil, in chunks
\end{ingredients}
Mix ingredients together and season with sea salt, freshly ground pepper and
sprinkle with a mix of herb, garlic, black pepper and sea salt

\minisection{Dressing}

\begin{ingredients}
    \SI{\sim 1/2}{\cup} extra virgin olive oil\\
    \SI{3}{\tblspoon} Balsamic vinegar\\
    \SI{1}{\tblspoon} Dijon mustard\\
    \SI{1}{\tblspoon} lemon juice\\
    A few drops of red pepper sauce (Cholula or Tabasco)\\
    Mayonnaise, \SIrange[range-phrase={ or }]{1}{2}{\tblspoon}\\
    (optional) a pinch of Sazon, a mixture of \SI{1}{\tblspoon} garlic powder
    \SI{1/2}{\teaspoon} black pepper\\
    \SI{1}{\tblspoon} onion powder\\
    \SI{2}{\tblspoon} fine sea salt\\
    \SI{1}{\tblspoon} ground cumin\\
    \SI{2}{\tblspoon} sweet paprika\\
    \SI{1}{\tblspoon} ground turmeric
\end{ingredients}
Accompany with buttered, toasted Mediterranean pita.

\section{Sausage Stuffed Mushrooms}
\index{appetizers and salads!sausage stuffed mushrooms}
\index{Lindquist!Dot}

\begin{open}
    Another great one from \url{myrunawaykitchen.com} by Dot Lindquist who writes: ``These are a Thanksgiving staple in my family, but I also bring them out as a heavy appetizer for parties, and sometimes as an extra special side for a cold November weeknight.  This recipe is for the ``Thanksgiving'' or ``Dinner Party'' yield.  If you are cooking for fewer than \numrange{15}{20} people, I suggest downsizing the quantities.  Don’t worry too much about measurement of ingredients: there is no wrong way to stuff a mushroom.  Use more of what you like, less of what you don't, and/or eliminate or substitute one cheese or vegetable for another.  It's all good, quite literally!''

    This one serves up to 20 people.  Prep time is \SI{40}{\minute} and cook time is \SI{30}{\minute}.
\end{open}
%%
\begin{ingredients}
    6 \SI{1}{\quart} packages white or ``stuffing'' mushrooms\\
    \SI{1}{\pound} Italian sausage (mild or spicy---you choose!)\\
    \SI{1/2}{\cup} butter\\
    5 cloves garlic, minced\\
    8 stalks celery, diced\\
    \SI{2}{\cup} breadcrumbs (seasoned)\\
    \SI{2}{\cup} grated/shredded cheese (parmesan, asiago, Romano blend is
    nice)\\
    \SI{1/2}{\cup} marscapone\\
    salt and pepper to taste
\end{ingredients}
Wiggle the stems out of the mushroom caps and line up the caps on a baking sheet
(or two). Use a damp paper towel to wipe mushrooms clean of debris. Cover the
caps with a damp paper towel while you are making the filling to keep them
moist. Pop your stems into the food processor and pulse until they are minced.
Set aside in a bowl.

Brown sausage in a deep pan on the stovetop over medium heat. Empty browned
sausage into food processor and pulse until it is the same texture as the
mushroom caps.

Melt a stick of butter (\SI{1/2}{\cup}) into the pan that you used to brown the
sausage. No need to wash the pan in between uses. Add the diced celery, the
minced mushroom caps and minced garlic and simmer until the mushrooms are soft
and have changed color (from white to brown). Add the marscapone and stir.
Remove from heat.

Pour mushroom mixture into minced sausage, and stir. Add bread crumbs and grated
cheese. Stir until combined. Using a teaspoon, add stuffing into mushroom caps
until filled. Bake at \SI{350}{\degreeF} until the mushrooms have softened,
change color, are slightly browned on top and release some fluid onto the baking
sheet. Cool until you're able to handle them with bare hands, serve and enjoy!
%%
\begin{figure}
    \centering
    \includegraphics[width=0.5\textwidth]{figures/sausage-stuffed-mushrooms}
    \caption*{Sausage Stuffed Mushrooms!}
\end{figure}
\secpart{Second Edition}{Soups}

%%---------------------------------------------------------------------------%%
\begin{entry}{Spicy Thai Chicken Soup}{Second Edition}
\index{Soups!spicy Thai chicken}
\index{Chicken!spicy Thai chicken soup}
\index{Lindquist!Jenn}

\begin{open}
    Here's a tempting soup recipe from Jenn Lindquist.
\end{open}
%%
\begin{ingredients}
    2 large cloves of garlic, chopped\\
    1 medium onion, chopped\\
    1 red bell pepper, chopped\\
    \SI{2}{\tblspoon} ginger paste or fresh ginger\\
    \SI{1}{\tblspoon} lemongrass paste\\
    \SI{2}{\tblspoon} red curry paste\\
    \SI{1}{\tblspoon} cilantro paste\\
    \SI{1}{\tblspoon} red chili paste (can also use 2 red chilies, chopped)\\
    \SI{2}{\tblspoon} coconut oil\\
    1 lime, zested and juiced\\
    \SI{1}{\tblspoon} fish sauce\\
    \SI{4}{\cup} chicken stock\\
    1 can (\SI{13.5}{\fluidounce}) coconut milk\\
    \SI{2}{\cup} shredded chicken
\end{ingredients}
For garnish:
\begin{ingredients}
    1 bunch of cilantro leaves chopped\\
    \numrange{4}{5} green onions, thinly sliced, both white and green parts\\
    1 green chili, sliced, seeds removed\\
    1 red chili, sliced, seeds removed
\end{ingredients}
To serve:
\begin{ingredients}
    roughly chopped coriander\\
    sliced red chili\\
    sliced green onion
\end{ingredients}
Combine garlic, onion, red pepper, ginger paste, lemongrass paste, red curry
paste, cilantro paste, red chili paste and coconut oil in a food processor, and
process until a paste forms. Add the paste to a medium pot and fry it for a
couple of minutes, just until it's fragrant. Add the chicken stock and coconut
milk and bring to a boil, reduce heat to a simmer and simmer for about
\SI{10}{\minute}. In the last couple of minutes, add your shredded chicken. Add
your fish sauce, lime juice and zest and rice noodles to the soup. Garnish with
chopped cilantro, green onion and green and red chilies---see
Fig.~\ref{fig:spicy-thai-soup}.
%%
\begin{figure}
    \centering
    \includegraphics[width=0.8\textwidth]{figures/spicy-thai-soup.pdf}
    \caption{Spicy Thai Chicken Soup and visuals of some of the ingredients}
    \label{fig:spicy-thai-soup}
\end{figure}
\end{entry}

%%---------------------------------------------------------------------------%%
\begin{entry}{Butternut Squash Soup}{Second Edition}
\index{Soups!butternut squash}
\index{Vegetarian!butternut squash}
\index{Lindquist!Dotty}

\begin{open}
  More from Dot Lindquist and \url{myrunawaykitchen.com}! According to Dot,
  ``This soup (with bacon and bread) eats like a meal and will warm your belly
  and keep the rest of your bread in your wallet\textellipsis have you noticed
  how super-cheap butternut squash is!?  They are practically giving it away.
  This soup freezes and reheats well and is a treat for lunch, too!'' (Note:
  we make no promises about the future price of butternut squash). Leave off the bacon and use veggie stock to make it vegetarian.

  This makes 15 servings, takes \SI{25}{\minute} of prep time and
  \SI{45}{\minute} of cooking time.
\end{open}
%%
\begin{ingredients}
    4  apples---any variety\\
    4  med sized yellow onions\\
    3  med sized butternut squash\\
    \SIrange{6}{8}{\cup} stock (chicken or vegetable)\\
    drizzle  extra virgin olive oil\\
    \SI{1/2}{\teaspoon}  nutmeg\\
    \SI{1/2}{\teaspoon}  sage\\
    \SI{1.75}{\pound} thick cut bacon (I like applewood smoked)
\end{ingredients}
Set your oven to \SI{375}{\degreeF} and roast your butternut squash. You can
do this simply by
\begin{description}
    \item[Option One] cutting long-ways, scooping out the insides, drizzle with
    olive oil, sprinkle with salt and pepper and place it cut-side down on a
    baking sheet
    \item[Option Two] you can do it the way shown above---halve
    your squash, scoop out the insides, peel and cube it, spread into a pan and
    drizzle with oil, sprinkle with salt and pepper, see Fig.~\ref{fig:butternut-squash-two}.
\end{description}
Once the squash is tender (if you can easily stick a fork into it, it's done),
you either have to scoop out the roasted squash from the outer shell, or you
have to put the work in up front and peel ahead of time. Your choice! Option
one is much quicker\textellipsis if you can afford to wait for the squash to
cool before you scoop it. Option two is quicker if you have lovely teenage
nieces who don't mind peeling for you!!
%%
\begin{figure}
    \centering
    \includegraphics[width=0.7\textwidth]{figures/butternut-squash.pdf}
    \caption{Option two for roasting butternut squash.}
    \label{fig:butternut-squash-two}
\end{figure}

Dice your onions, peel and dice your apples. Melt diced apples and onions
together with a drizzle of olive oil in the bottom of your soup pot, over
medium heat, until wilted and aromatic. Add sage, butternut squash, and
nutmeg, and stir. Cover mixture with stock, and bring to a boil. Then, reduce
heat. Blend with immersion blender. Season with salt and pepper to taste.

Don't forget the bacon! I like to lay bacon out flat on a baking sheet and stick
it into the oven while the squash is roasting. I let it crisp up, then drain off
the excess fat and spread on paper towel. Chop it up when it's cool and set on
the side, to be served atop your soup! For an extra decadent treat, add a dollop
of sour cream, creme fraiche, or heavy cream right into your soup bowl and then
sprinkle on your bacon (Fig.~\ref{fig:butternut-squash-soup}). Yuuuuummm. Serve
with warm, crusty bread and enjoy!
%%
\begin{figure}
    \centering
    \includegraphics[width=0.7\textwidth]{figures/butternut-squash-final.jpg}
    \caption{Butternut squash soup!}
    \label{fig:butternut-squash-soup}
\end{figure}
\end{entry}

%%---------------------------------------------------------------------------%%
\begin{entry}{Potato Carrot Cheese Soup}{Second Edition}
\index{Soups!potato carrot cheese}
\index{Vegetarian!potato carrot cheese soup}
\index{Evans!Kate}

\begin{open}
  This recipe began from a recipe in the classic 80's cookbook, ``the Silver
  Palate cookbook'' by Rosso and Lukins, but has been altered so much over
  time that it's become one of Kate's staples she makes from some version of
  it in her memory. This works great on its own on a cold, rainy night or as a
  first course to a heavier meal. Feel free to add onions with the carrots
  instead of onion powder, they are missing since Kate is intolerant to them. This as a vegetarian dish if made with veggie broth instead.
\end{open}
%%
\begin{ingredients}
    \SI{2}{\tblspoon} butter\\
    3-4 carrots, peeled and coarsely chopped\\
    \SI{1}{\teaspoon} onion powder\\
    5-6 medium-sized (2-3\SI{1/2}{\pound}) Yukon gold potatoes, peeled and
    coarsely chopped \\
    \SI{5}{\cup} broth (chicken or vegetable)\\
    \SI{1/4}{\cup} white wine \\
    1 sprig fresh rosemary \\
    \SI{1/2}{\cup} fresh parsley, chopped\\
    \SI{1}{\teaspoon} each salt and pepper\\
    \SI{1}{\cup} extra sharp cheddar cheese, grated
\end{ingredients}
In large dutch oven, melt butter over low heat. Then add carrots and onion
powder and \saute in butter until soft, about 15-20 minutes. Then add potatoes,
broth, wine rosemary, parsley, salt and pepper and bring to a boil. Cover,
reduce heat, and simmer until potatoes are soft, about 30-45 minutes. remove
rosemary spring and discard. It's all right if some needles remain in the
pot. Blend the soup until smooth, either using an immersion hand blender or
placing the soup contents into a food processor in batches and blending. if
the latter, return blended soup to dutch oven. Slowly add cheddar and stir
until incorporated. Taste and add more salt, pepper, or broth as
desired. Serve immediately.
\end{entry}

%%---------------------------------------------------------------------------%%
\begin{entry}{Braised Bok Choy in Broth}{Second Edition}
\index{Soups!bok choy}
\index{Seafood!bok choy}
\index{Liu!Geyi}

\begin{open}
    From Geyi, she writes:

    A classic Cantonese dish. It is basically braising bok choy in a soup base
    for \SIrange{2}{3}{\minute}. The art is in the soup base, which
    traditionally involves simmering together whole chicken, lean pork, and ham
    for a long time. These days in regular Chinese households, people routinely
    use a replacement ingredient: preserved duck egg, also translated as
    ``Century egg'', or ``thousand-year-old egg.'' (Fig.~\ref{fig:century-egg}).

    This is my attempt to recreate this dish from the common ingredients around,
    simulating the idea of a savory soup base with shrimp, ham and chicken
    stock.
\end{open}
%%
\begin{ingredients}
    \numrange{4}{5} cloves garlic\\
    sausage and/or ham\\
    \numrange{4}{5} chopped shrimp\\
    \SI{2}{\cup} chicken stock\\
    bok choy
\end{ingredients}
%%
\begin{figure}[b]
    \centering
    \includegraphics[width=.5\textwidth]{figures/century-egg.png}
    \caption{The ``Century Egg'' (\url{en.wikipedia.org/wiki/Century_egg}).}
    \label{fig:century-egg}
\end{figure}
%%
First, make the soup base.  In a large pan, add oil then stir fry the chopped
garlic, some sliced sausages/ham, the chopped shrimp, then add the chicken stock
and bring to a boil.

Add bok choy, this works well with broccoli too!.  Cook till vegetable softens
and serve (Fig.~\ref{fig:bok-choy}).
%%
\begin{figure}
    \centering
    \includegraphics[width=.8\textwidth]{figures/bok-choy.png}
    \caption{Braised Bok Choy soup is served!}
    \label{fig:bok-choy}
\end{figure}
\end{entry}
\chapter{Breads}

\section{Not Very French Crepes\index{breads!not very french crepes}}

\begin{open}
    Hey, anything that's not very French I love.  Also, are crepes (or
    cr\^{e}pes as I know some of you will insist, even though both spellings are
    valid), breads, deserts, or main dishes?  Who knows, who cares, we put them
    here.

    As Don and Gege say, ``The crepes are but a vesssel for whatever good stuff
    you put into the filling.''  We could agree more.

    This is a recipe derived from \url{allrecipes.com}; it is absolutely
    foolproof, at least according to Don and Gege
\end{open}
\begin{ingredients}
    \SI{1}{\cup} flour\\
    2 eggs\\
    \SI{1/2}{\cup} milk\\
    \SI{1/2}{\cup} water\\
    dash of salt\\
    \SI{2}{\tblspoon} melted butter\\
\end{ingredients}
Beat the eggs and mix thoroughly into the flour. Add milk and water and beat as
long as you can or until no lumps. Add salt and beat in melted butter. It's a
very thin batter---don't be alarmed. Drop small ladlefuls into a skillet and
rotate so the batter coats the bottom in a very thin layer; no oil needed after
the first one. Makes \numrange{8}{10} crepes.

\minisection{Filling}

\noindent No measurements here; use the Force and whatever you have handy
\begin{ingredients}
    Half pound to a pound ground meat or sausage\\
    A little soy sauce---dark is better than light\\
    Diced onion
    Chopped tomatoes---\numrange{1}{2} large or several small\\
\end{ingredients}
The above three ingredients are the base, add other things as convenient and to take the filling in different directions---mushrooms, spinach, chopped peppers, parmesan cheese are all good.

Brown the ground meat/sausage in oil, add a little soy sauce for flavor and
color. Add onion and cook until tender and transparent. If you have mushrooms,
peppers, etc. add them and saut\'{e}. Add tomatoes last and saut\'{e} until you
have a thick sauce or paste. If you are using spinach or cheese stir them in
right before serving.

Put a plate of crepes and a big bowl of sauce on the table and let people roll
their own. Crepes are tender and knife and fork are usually required.

\secpart{Second Edition}{Sides}

%%---------------------------------------------------------------------------%%
\begin{entry}{White Grapes and Sweetened ``Cream''}{Second Edition}
\index{Fruit!white grapes \& sweetened cream}
\index{Vegetarian!white grapes \& sweetened cream}
\index{Lindquist!Julie}

\begin{open}
  Here is a refreshing and simple summery side submitted by Julie Lindquist.
\end{open}
%%
\begin{ingredients}
    grapes\\
    plain yogurt or sour cream\\
    brown sugar
\end{ingredients}
Remove grapes from stems; rinse and dry. Place in a colorful serving dish. Top
with plain yogurt or sour cream (NB: sweet cream doesn't offer the contrast to
the sweet grape taste). Sprinkle some brown sugar on top. Refrigerate for at
least 4 hours; the attractive pattern of the dissolving brown sugar is a
bonus. Lasts a few days, if there’s any left.
\end{entry}

%%---------------------------------------------------------------------------%%
\begin{entry}{Turkish Rice}{Second Edition}
\index{Rice!turkish}
\index{Johnson!Martha}
\index{Johnson!Dodge}

\begin{open}
    From Martha and Dodge, a graduate school special that provides an infinitely
    variable way to prepare a pilaf with your favorite spices and ingredients.
\end{open}
%%
\begin{ingredients}
    chopped onion or shallot\\
    \SI{1}{\cup} long grain rice\\
    2\SI{1/3}{\cup} beef bouillon\\
    raisins, dried cranberries, nuts\\
    cumin, curry powder, garlic
\end{ingredients}
Cook a few tablespoons of chopped onion or shallot in some butter and oil. Add
uncooked rice and stir until light brown. Add the beef bouillon, plus a handful
of raisins, dried cranberries, nuts and seasonings (some mix of cumin, curry
powder, garlic, etc.) and simmer until done (\SI{20}{\minute}).
\end{entry}

%%---------------------------------------------------------------------------%%
\begin{entry}{Boursin Potatoes}{Second Edition}
\index{Potatoes!boursin potatoes}
\index{Vegetarian!boursin potatoes}
\index{Johnson!Martha}

\begin{open}
    From Martha: a favorite side from Barby Buckman at St.~Peter's in the Great
    Valley, PA. A great dish for a crowd (8), or a half recipe also works well
    for 4.
\end{open}
%%
\begin{ingredients}
    \SI{2}{\cup} whipping cream\\
    1 \SI{5}{\ounce} ounce package Boursin cheese (herbs)\\
    \SI{3}{\pound} red new potatoes, or Yukon Gold, unpeeled, scrubbed and
    thinly sliced\\
    salt and pepper\\
    \SIrange{1}{2}{\tblspoon} chopped fresh parsley
\end{ingredients}
Preheat oven to \SI{400}{\degreeF}. Stir whipping cream and Bousing cheese in a
heavy pan over medium heat until cheese melts and no lumps remain. Arrange half
the potatoes in overlapping rows in a buttered \SI{9x13}{\inch} baking dish.
Season with salt and pepper and pour half the cheese mixture over them. Arrange
the rest of the potatoes in a second layer with remaining cheese mix. Bake until
golden brown, about \SI{1}{\hour}. Sprinkle with parsley and serve.
\end{entry}

%%---------------------------------------------------------------------------%%
\begin{entry}{Immune-Boosting Chimichurri from Argentina}{Second Edition}
\index{Sauces!immune-boosting chimichurri}
\index{Horne Rona!Alison}

\begin{open}
    Chimichurri is basically Argentinian pesto.  You can use it on everything
    from meat dishes, vegetables, salads, and breads.  According to Alison Horne Rona, ``This will make any meat taste great!''
\end{open}
%%
\begin{ingredients}
    1 large bunch of flat leafed parsley chopped\\
    1 large bunch of fresh oregano leaves stripped from stems\\
    1 large bunch of fresh cilantro all chopped\\
    1 green jalape\~{n}o pepper chopped\\
    1 small yellow onion (or shallots) chopped\\
    1 lemon (I cut off most of the rind, cut it in half, take out the seeds, and
    throw both halves in the blender)\\
    6 large cloves of garlic chopped finely\\
    \SIrange{2}{3}{\tblspoon} of red wine vinegar to taste\\
    \SIrange{3}{4}{\tblspoon} olive oil\\
    sea salt and black pepper
\end{ingredients}
Throw everything in a blender and mix.
\end{entry}

%%---------------------------------------------------------------------------%%

\begin{entry}{Blanched lettuce with sauce}{Second Edition}
\index{Salads!blanched lettuce with sauce}
\index{Vegan!blanched lettuce with sauce}
\index{Liu!Geyi}

\begin{open}
  Submitted by Geyi, this is a dish often seen in Southern Chinese restaurants. It is often crisp and green in restaurants but soggy and dark at home. The secret is to boil the lettuce leaves very fast (5-10 seconds) in a large amount of water so that they stays green and crisp while keeping in all the nutrition 
\end{open}
%%
\begin{ingredients}

\end{ingredients}
Prepare sauce by mixing what you think is a good amount of oyster sauce, soy sauce, salt, sugar, (corn) starch, water together. Ask Geyi for tips on the right amount; she was busy moving when when we were entering this, but it sounded so good we included it anyway!

\begin{figure}[h]
    \centering
    \includegraphics[width=0.8\textwidth]{figures/lettuce1.png}
    \caption{Sauce Preparation}
    \label{fig:Lettuce1}
\end{figure}

Next cook the lettuce as follows: Take a large pot of water, add a tablespoon of vegetable oil, a teaspoon of salt and bring to a boil. Add lettuce, stir so that all areas get blanched, and take out after 5-10 seconds (refer to figure \ref{fig:lettuce2} to see how the lettuce is still quite green).
 
\begin{figure}
    \centering
    \includegraphics[width=0.8\textwidth]{figures/Lettuce2.png}
    \caption{Blanching the Lettuce}
    \label{fig:lettuce2}
\end{figure}
 
Then, heat oil in a pot, add chopped garlic and stir until it becomes yellow, and add the sauce mixture and stir until it thickens. 

Finally, pour the sauce over the lettuce and serve immediately.

\begin{figure}
    \centering
    \includegraphics[width=0.8\textwidth]{figures/Lettuce3.png}
    \caption{Pour the Sauce over the lettuce}
    \label{fig:lettuce3}
\end{figure}


\end{entry}

\begin{entry}{Roasted Brussell Sprouts}{Second Edition}
\index{Vegetarian!brussel sprouts}
\index{Pross!Katie}

\begin{open}
 This recipe looks absolutely amazing and super easy for our new cooks to try. Katie adds crumbled bacon to it, which sounds delicious!
\end{open}
%%
\begin{ingredients}
    \SI{1}{\pound} fresh brussell sprouts \\
    \SIrange{1}{2}{\tblspoon} olive oil\\
    \SI{1}{\tblspoon} \\
    salt and pepper\\
\end{ingredients}
Preheat oven to \SI{450}{\degreeF}. Trim ends of brussell sprouts and cut into \SI{1}/{2} or \SI{1}/{4}, depending on the size of each sprout. Drizzle sprouts with olive oil, salt and pepper, and toss to coat. 
Roast for \SIrange{10}{20}{\minute}, tossing a few times throughout until sprouts are browned.
\end{entry}
%% From the first edition

\chapter{Sides}

\section{Hash Brown Potatoes\index{sides!hash brown potatoes}}

\begin{open}
  Contributed by Joyce Evans. Try with the Swiss Meatloaf
  (Section~\ref{sec:swiss-meatloaf}), yum! Serves 4.
\end{open}
\begin{ingredients}
  3 large potatoes, boiled \\
  \SI{1/4}{\cup} milk \\
  \SI{3}{\tblspoon} all-purpose flour \\
  \SI{2}{\tblspoon} minced onion \\
  \SI{2}{\tblspoon} minced fresh parsley or chervil \\
  \SI{1/2}{\teaspoon} salt \\
  \SI{1/2}{\teaspoon} pepper \\
  \SI{1/4}{\teaspoon} dried oregano (opt.) \\
  Dash of Tabasco \\
  \SI{3}{\tblspoon} bacon drippings, rendered chicken fat, or vegetable oil
\end{ingredients}
Preheat in electric skillet to \SI{300}{\degree}. Peel and dice the boiled
potatoes and place into a medium bowl. You should have about \SI{3}{\cup}. Add
the rest of the ingredients except the cooking fat and blend.

Add the cooking fat to the skillet and heat. Pack the potato mixture in
firmly, spreading it out in an even layer. Cook \numrange{7}{9} minutes or
until the bottom side is richly brown. Turn the mixture over in segments and
smooth down again into a patty. Continue cooking until the other side is
brown, another \numrange{7}{9} minutes.  Cut into wedges and serve.

\section{Black Rice\index{sides!black rice}}

\begin{open}
  Contributed by Fermina Evans. Serves 4, or 2 healthy eaters. This is Tom and
  Katie's favorite peasant food.
\end{open}
\begin{ingredients}
  \SI{1}{\cup} dry black beans\\
  \SI{5}{\cup} chicken broth\\
  \SI{1/2}{\tblspoon} olive oil \\
  1 small onion, chopped \\
  4 cloves garlic, minced \\
  \SI{1}{\ounce} finely chopped Canadian or regular bacon \\
  \SI{1/2}{\cup} rice \\
  \SI{1/4}{\cup} white wine \\
  1 tomato coarsely chopped \\
  \SI{1/2}{\teaspoon} ground cumin \\
  pinch cayenne \\
  \SI{1/2}{\cup} finely chopped cilantro
\end{ingredients}
You may substitute 2 cans black beans for dry beans if you prefer. If using
dry beans, soak beans overnight in cold water, and simmer beans for
\numrange{120}{150} minutes in \SI{3}{\cup} broth until tender. Drain and
resolve liquid (\SI{1/2}[1]{\cup}). Otherwise, drain them under cold water and
use \SI{1/2}[1]{\cup} chicken broth or chicken bouillon stock for bean broth.

Heat oil in stockpot. Add onion, garlic, and bacon and stir fry for about 5
minutes.  Add rice and stir for 1 minute. Add wine and cook for 2 minutes. Add
tomatoes and cook for 2 more minutes.  Add bean broth \SI{1/2}{\cup} at a
time, stirring until liquid is absorbed before adding more broth. This will
take \numrange{20}{25} minutes to complete. Add the beans and remaining
broth. Season with cumin, cayenne, and cilantro and serve.

\section{Potato Gratin with Mustard and Cheese\index{sides!Potato Gratin}}

\begin{open}
  This is a great entertaining dish because its classy, very smooth and
  flavorful, yet can be prepared before guests arrive. Kate got it from
  \corp{Bon Appetit} magazine.
\end{open}
\begin{ingredients}
  \SI{1}{\tblspoon} butter\\
  \SI{1}{\cup} fresh breadcrumbs\\
  \SI{1}{\tblspoon} dried thyme\\
  \SI{2}{\teaspoon} salt\\
  \SI{1}{\teaspoon} ground pepper\\
  \SI{1}{\pound} sharp white cheddar cheese, grated\\
  \SI{1/4}{\cup} flour\\
  \SI{5}{\pound} russet potatoes, peeled and thinly sliced\\
  \SI{4}{\cup} canned low salt chicken broth\\
  \SI{1}{\cup} whipping cream\\
  \SI{6}{\tblspoon} Dijon mustard
\end{ingredients}
Melt butter in skillet and add breadcrumbs, stirring until golden brown (about
10 min.). Set aside. Preheat oven to \SI{400}{\degree}. Butter a
\SI{15x10x2}{\inch} baking dish. Mix thyme, salt, and pepper in small
bowl. Combine grated cheese and flour, tossing to coat the cheese. Arrange
\num{1/3} potato slices to cover the bottom of the baking dish. Sprinkle
\num{1/3} the thyme mixture, then \num{1/3} the cheese mixture. Repeat
layering 2 more times.  Next whisk chicken broth, cream, and mustard in a
separate bowl, and then pour it over the potato layers. Bake 30
minutes. Sprinkle buttered crumbs over, and bake until potatoes are tender and
top is golden brown, about 1 hour longer.  Enjoy!

\chapter{Entrees}

\section{Mushroom Cream Pasta\index{entrees!mushroom cream pasta}}

\textsl{Lightly adapted, mostly for convenience, from ``Chef John's Creamy
Mushroom Pasta'' on \url{allrecipes.com}, My family (who are NOT vegetarians)
are totally satisfied with this quick and easy dish as a main course. From: Don and Gege}.
\begin{ingredients}
    \SI{1/2}{\pound} linguine, fettuccine or other pasta\\
    \SI{1}{\pound} mushrooms. I like a mix of white and shiitake, but baby
    portobello also work well\\
    Olive oil, salt pepper\\
    Garlic\\
    \SI{1}{\tblspoon} sherry (For some reason, I more often have dry vermouth
    which seems to work fine.) \\
    Chicken stock (optional)\\
    \SI{1}{\cup} heavy whipping cream\\
    \SI{1/2}{\cup} grated parmesan cheese\\
    Fresh chopped thyme, chives, tarragon (n.b. I have never added these. I'm
    sure  they would make it better.)\\
\end{ingredients}
Saute sliced mushrooms in olive oil until they are tender and release their
liquid. Add several cloves of diced garlic. Add sherry followed by heavy cream,
and lick residual cream from cup measure. Add salt and pepper and simmer cream
until the mixture thickens a bit and foams---though it does not get very thick;
if it does add some chicken stock. When the mixture reaches reasonable
consistency, stir in the fresh spices, turn off the heat and mix in the parmesan
cheese. Stir the mushroom cream mixture into the pasta and serve.
\secpart{Second Edition}{Treats}

%%---------------------------------------------------------------------------%%
\begin{entry}{Mint Syrup}{Second Edition}
\index{syrup!mint}
\index{Johnson!Weedie}

\begin{open}
  This recipe comes from Louise Johnson of Spruce Head, ME, a.k.a. Weedie, the
  sister of the original Donald Dodge Johnson (Dodge and Julie’s father).
\end{open}
%%
\begin{ingredients}
    fresh spearmint or peppermint leaves\\
    granulated sugar\\
    1 or more lemons\\
    1 or more oranges
\end{ingredients}
Using a strainer that fits into a deep bowl, fill with finely cut fresh
spearmint or peppermint leaves. Boil for ten minutes equal measures of water
and granulated sugar, approximately 1\SI{1/2}{\cup} each. Pour directly over
the mint and allow to cool. Remove the strainer and add the juice of one or
more lemons and the juice of one or more oranges. Wonderful over ice cream, in
iced tea, or on a simple cake aching for a hint of mint.
\end{entry}

%%---------------------------------------------------------------------------%%
\begin{entry}{Southern Pecan Pie}{Second Edition}
\label{sec:pecanpie}
\index{pie!southern pecan}
\index{Johnson!Martha}

\begin{open}
  Hey y'all, from Martha! This is from an ancient, out-of-print Better Homes
  and Garden Cookbook One can double the sin by brushing the crust with a dark
  chocolate glaze before adding the filling, but it's pretty wonderful as is!
  And so easy! Either a pie crust mix or the real thing work equally well. The
  main thing is to have fresh pecans.
\end{open}
%%
\begin{ingredients}
    3 large eggs\\
    \SI{2/3}{\cup} sugar\\
    \SI{1}{\cup} dark corn syrup\\
    \SI{1/3}{\cup} melted butter\\
    \SI{1}{\cup} pecan halves\\
    1 \SI{9}{\inch} unbaked pastry shell
\end{ingredients}
Beat eggs thoroughly with sugar, a dash of salt, corn syrup, and melted butter.
Add pecans. (Brush shell with melted chocolate if desired). Bake in moderate
oven (\SI{350}{\degreeF}) \SI{50}{\minute} or until a knife inserted near center
comes out clean. Cool.
\end{entry}

%%---------------------------------------------------------------------------%%
\begin{entry}{Grapes \`{a} la Creme}{Second Edition}
\index{compote!grapes a la creme}
\index{Horne!Mildred}

\begin{open}
    This is from Paul Horne's mother, Mildred W. Horne, Alexandra, VA, 1959.
\end{open}
%%
\begin{ingredients}
    seedless grapes\\
    sour cream\\
    brown sugar
\end{ingredients}
Wash and de-stem 1\SIrange{1/2}{2}{\pound} of grapes. Drain and place in dessert
dishes, preferably long-stemmed ones.  Spread a tablespoon or so of sour cream
over each mound of grapes and leave in refrigerator for several hours.  An hour
or so before dinner, sprinkle liberally with brown sugar and replace in
refrigerator until dessert time.  Serve with coffee.
\end{entry}

%%---------------------------------------------------------------------------%%
\begin{entry}{Jen's Favorite Sugar Cookies}{Second Edition}
\index{cookies!sugar}
\index{Lindquist!Jen}

\begin{open}
  These are Jen Lindquist's favorite sugar cookies.  Judging by the pictures,
  creativity is a plus!
\end{open}
%%
\begin{ingredients}
    2 sticks butter\\
    \SI{8}{\ounce} cream cheese\\
    1\SI{1/2}{\cup} sugar\\
    3\SI{1/2}{\cup} flour\\
    \SI{1}{\teaspoon} vanilla\\
    \SI{1}{\teaspoon} almond\\
    \SI{1}{\teaspoon} baking powder\\
    1 egg
\end{ingredients}
Mix, roll, cut, and bake at \SI{350}{\degreeF} for \SI{8}{\minute}.  For some
amazing decorating ideas check out Fig.~\ref{fig:sugar-cookie-decorating}.
\begin{figure}
    \centering
    \includegraphics[width=0.7\textwidth]{figures/sugar-cookies.pdf}
    \caption{Sugar cookie decorating ideas.}
    \label{fig:sugar-cookie-decorating}
\end{figure}
%%
\begin{figure}[b]
    \centering
    \includegraphics[width=0.33\textwidth,clip]{figures/jen-making-cookies.jpg}
    \caption{Jen making sugar cookies!}
\end{figure}
\end{entry}

%%---------------------------------------------------------------------------%%
\begin{entry}{Gingerbread, King of Cakes}{Second Edition}
\index{cakes!gingerbread}
\index{Lindquist!Dot}

\begin{open}
    This is one of the many amazing recipes you can find at
    \url{myrunawaykitchen.com}, Dot Lindquist's cooking blog.  We agree that
    gingerbread is, if not the King, at least cake royalty.  This recipe makes
    \numrange{14}{16} slices, takes \SI{25}{\minute} of prep time and about
    \SI{1}{\hour} of cooking time.
\end{open}
%%
\begin{ingredients}
    \SI{1/2}{\cup}  granulated sugar\\
    \SI{1/2}{\cup}  unsalted butter\\
    1 large egg\\
    \SI{1}{\cup} molasses\\
    1 large orange, zested and juiced\\
    2\SI{1/2}{\cup}  all purpose flour\\
    1\SI{1/2}{\teaspoon}  baking soda\\
    \SI{2}{\teaspoon} cinnamon\\
    \SI{2}{\teaspoon} ground ginger\\
    \SI{3/4}{\teaspoon}   ground cloves\\
    \SI{1/2}{\teaspoon}  salt\\
    \SI{1/2}{\cup}  hot water
\end{ingredients}
%%
\begin{figure}
    \centering
    \includegraphics[width=0.8\textwidth]{figures/gingerbread.pdf}
    \caption{Gingerbread, indeed the King of Cakes!}
\end{figure}
%%
Preheat oven to \SI{350}{\degreeF}. Grease and lightly flour the inside of your
baking pan. I like a bundt pan for this recipe, but you could use a
\SI{9x13}{\inch} pan, \SI{9x9}{\inch}, or even make cupcakes. Whatever your
little heart desires. If you're using cupcake papers or a parchment lining in
your pan or baking dish, no need to grease and flour the pan.

If possible, use room-temperature butter. If your room is very cold, or you
forgot to leave a pound of butter on your countertop for baking, pop your butter
into the microwave for \SI{10}{\second} intervals until it's mushy but not
liquid. Cream butter and sugar together in the stand mixer with the paddle
attachment, or with the handheld mixer.

Add egg and molasses and mix, scraping down the sides and bottom of the bowl as
you go. In a separate bowl, sift together the dry ingredients: flour, baking
soda, cinnamon, ginger, cloves and salt. Add orange zest. Add dry ingredients to
the wet ingredients and mix until well blended. Add the \SI{1/2}{\cup} hot water
and the juice from your orange (should be about \SI{1/2}{\cup}, making the total
amount of liquid added in this step equal to one cup).

Bake in preheated oven for one hour or until a toothpick in the center comes out
clean and the cake springs back against your finger when you press into it.

Remove from pan and cool before frosting.
\end{entry}

%%---------------------------------------------------------------------------%%
\begin{entry}{OMG Avocado Chocolate Mousse}{Second Edition}
\index{OMG avocado chocolate mousse}
\index{Lindquist!Dot}

\begin{open}
  From \url{myrunawaykitchen.com}, Dottie says, ``You’re not going to believe
  how simple this is to make and how eye-poppingly great it tastes… and the
  silky texture: OMG.  All you need is a food processor for this one.  Unless
  you like to slather it with real whipped cream, in which case you’ll also
  want an electric mixer of some kind.  I have been known to whip by hand,
  yes, but only when no motorized option is available!''

  Also, as pointed out by Dot, this recipe only uses healthy fats, so enjoy
  almost guilt free; although frankly, we highly encourage the use of all fats
  in this cookbook!
\end{open}
%%
\begin{ingredients}
    4 ripe avocados, peeled and pitted\\
    \SI{8}{\ounce} semisweet chocolate, melted (baker’s chocolate or chips are fine---you can use a double boiler method if you want, but it's ok to melt this stuff, covered, in the microwave in \SI{30}{\second} increments, stirring until smooth)\\
    \SI{6}{\tblspoon} cocoa powder\\
    \SI{1/2}{\cup} milk of any variety\\
    \SI{2}{\teaspoon} vanilla\\
    \SI{1/4}{\teaspoon} salt\\
    \SI{3/4}{\cup} maple syrup
\end{ingredients}
Just combine all ingredients in a food processor and pulse until it’s as smooth
as a chocolate silk dream (Fig.~\ref{fig:mousse}).  (Add maple syrup to the
mixture last, to your desired sweetness level, pulsing until completely
incorporated.)
%%
\begin{figure}
    \centering
    \includegraphics[width=0.6\textwidth]{figures/avocado-choc-mousse.pdf}
    \caption{OMG Avocado Chocolate Mousse!}
    \label{fig:mousse}
\end{figure}
%%

\minisection{Whipped Cream}

\begin{ingredients}
    \SI{1}{\quart} ``heavy'' or ``whipping'' cream\\
    \SI{1/2}{\cup} confectioner's sugar\\
    \SI{2}{\teaspoon} vanilla
\end{ingredients}
Start your mixer on low or medium (so that the cream doesn't splatter all over
creation) and gradually increase the speed as you gradually add sugar and
vanilla.  Beat until soft peaks form.
\end{entry}

%%---------------------------------------------------------------------------%%
\begin{entry}{Rhubarb Raspberry Compote with Mint}{Second Edition}
\index{compote!rhubarb raspberry with mint}
\index{Rona!Alison}

\begin{open}
  A nice compote from Alison Rona that can be used with yogurt, on ice cream,
  or even as part of a fruit torte.
\end{open}
%%
\begin{ingredients}
    stalks of rhubarb\\
    1 banana\\
    1 apple\\
    1 blood orange\\
    fresh ginger\\
    cinnamon\\
    2 cloves\\
    orange peel (zested)\\
    nutmeg\\
    honey\\
    molasses\\
    brown sugar\\
    butter\\
    vanilla extract\\
    orange extract\\
    2 boxes of raspberry\\
    fresh mint leaves
\end{ingredients}
Chop the stalks of rhubarb, then simmer in a cup of water.  Cut a banana,
apple, and blood orange into equal sized chunks and add to the rhubarb.  Then
add minced fresh ginger, a spoonful of cinnamon, 2 cloves, a zested orange peel,
grated nutmeg, a spoonful of honey, molasses, brown sugar, butter, vanilla
extract, orange extract, and 2 boxes of raspberries.  You will need to play with
amounts for taste.  Add lots of fresh chopped mint near the end.

Pour it hot over vanilla ice cream or serve warm with plain goat milk yogurt.
Or, use the compote in a pie crust and add a few pieces of fruit on top!
\end{entry}

%%---------------------------------------------------------------------------%%
\begin{entry}{Chocolate cream cheese cupcakes}{Second Edition}
\index{cakes!chocolate cream cheese cupcakes}
\index{Evans!Fermina}

\begin{open}
  Kate has fond memories of visiting Tom's house when they were dating and
  Tom's mom, Fermina, would bring out these little harmless looking but
  completely addicting mini-cupcakes.
\end{open}
%%
\begin{ingredients}
    1\SI{1/2}{\cup} flour\\
    \SI{1}{\cup} sugar\\
    \SI{1/4}{\cup} cocoa\\
    \SI{1}{\teaspoon} baking soda\\
    \SI{1}{\cup} water\\
    \SI{1}{\teaspoon} vanilla\\
    \SI{1/3}{\cup} oil\\
    \SI{1}{\tblspoon} cider vinegar\\
    1 dozen cupcake liners
\end{ingredients}

\minisection{Topping}

\begin{ingredients}
    \SI{8}{\ounce} cream cheese, softened\\
    1 egg\\
    \SI{1/3}{\cup} sugar\\
    \SI{1/8}{\teaspoon} salt\\
    \SI{3/4}{package} miniature chocolate chips
\end{ingredients}
Preheat oven to (\SI{350}{\degreeF}). In a large bowl, combine the flour,
\SI{1}{\cup} sugar, cocoa, and baking soda. In a separate bowl, mix water,
vanilla, oil, and vinegar and then add to dry ingredients and mix well. Spoon
into liners in muffin tins until \SIrange{1/2}{3/4} full. Blend cream cheese,
egg, \SI{1/3}{\cup} sugar, and salt and mini chips. Place 1 spoonful on top of
chocolate mixture within each liner, and bake \SIrange{20}{23}{\minute}.
\end{entry}

%%---------------------------------------------------------------------------%%
\begin{entry}{Holiday chocolate cream pie}{Second Edition}
\index{pie!chocolate cream}
\index{Evans!Kate}
\index{Johnson!Martha}

\begin{open}
  This easy but delicious pie recipe came from a now out of print cookbook by
  Mable Hoffman titled ``Chocolate Cookery'' that Don got her for Christmas in
  1983. Kate wore it out from heavy use (thanks Don, you nailed it!) and
  recently procured a new copy from a secondhand seller. She makes this pie
  for most holiday events, and just this Christmas figured out how to avoid
  having the crust come out soggy! Regarding the pie shell, like Martha in the
  \ref{sec:pecanpie} recipe, the Betty Crocker mix works well here too.
\end{open}
%%
\begin{ingredients}
    \SI{1}{\cup} sugar\\
    \SI{1/4}{\cup} cornstarch\\
    \SI{1/4}{\teaspoon} salt\\
    1\SI{1/2}{\cup} cold water\\
    3 eggs, lightly beaten \\
    \SI{3}{\ounce} semisweet chocolate \\
    \SI{2}{\tblspoon} butter \\
    \SI{1}{\teaspoon} vanilla extract\\
    1 9-inch pie shell, baked \\
    \SI{1/2}{\cup} whipping cream
\end{ingredients}
In a medium saucepan, combine the sugar, cornstarch, and salt. Pour in water
and whisk until blended. Add eggs and chocolate and stir/wisk constantly until
thickened and smooth, about 10-20 minutes. Remove from heat and add vanilla
and butter. Cool a bit, then scoop and smooth into pie shell (NB: this bit
prevents the pudding from making the shell soggy). Refrigerate several hours
or until firm. Whip the cream and spread over chocolate and serve.
\end{entry}

%%---------------------------------------------------------------------------%%
\begin{entry}{Ricotta Cookies}{Second Edition}
\index{Cordova!Betsy}
\index{Cordova!Rich}
\index{cookies!ricotta}

\begin{open}
    From Betsy Cordova who writes, ``These are a favorite in the Cordova household. It was a recipe given to me by Nana McCarron (Richard's Nani) and I make them every holiday (and whenever anyone asks for them). I have two variations: traditional with icing and chocolate chip.''
\end{open}
%%
\begin{ingredients}
    3 eggs\\
    \SI{2}{\cup} sugar\\
    \SI{1/2}{\pound} butter\\
    \SI{1}{\pound} Ricotta\\
    \SI{2}{\teaspoon} vanilla\\
    mini chocolate chips (for chocolate chip version)\\
    Decorator's sugar (for chocolate chip version)
\end{ingredients}
%%
\minisection{Flour Mixture}
\begin{ingredients}
    \SI{4}{\cup} flour\\
    \SI{1}{\teaspoon} baking soda\\
    \SI{1}{\teaspoon} salt
\end{ingredients}
%%
Preheat the oven to \SI{350}{\degreeF}.  Cream butter and sugar together; mix in
vanilla and eggs. Add Ricotta and mix well. Add Flour Mixture one cup at a time
mixing well after each. If making the chocolate chip variation: add one bag
of mini chocolate chips.

Bake tablespoon-sized amounts on an ungreased cookie sheet for \SI{15}{\minute}.
For the chocolate chip variation, put Decorator's sugar on the tops prior to
baking. Cool on a baking rack (Fig.~\ref{fig:ricotta-cookies}).

For the iced version, the icing ingredients are:
\begin{ingredients}
    \SI{1}{\cup} 10$\boldsymbol\times$ sugar\\
    \SI{1/2}{\teaspoon} vanilla\\
    \SI{1/2}{\cup} milk
\end{ingredients}
Mix all ingredients together and ice cooled cookies---Betsy usually decorates with Nonpareils.
%%
\begin{figure}
    \centering
    \includegraphics[width=0.8\textwidth]{figures/ricotta-cookies.png}
    \caption{Ricotta Cookies!}
    \label{fig:ricotta-cookies}
\end{figure}

\end{entry}

%%---------------------------------------------------------------------------%%
\begin{entry}{Italian Cream}{Second Edition}
\index{Cordova!Betsy}
\index{DeLellis!Evelina}
\index{Evans!Fermina}
\index{Italian cream}

\begin{open}
    This is the Evelina DeLellis (Tom, Betsy, and Katie's Nana) Italian Cream
    recipe with modifications by Fermina presented by Betsy.  We know that
    sounds a little complicated, but the final result is great!  It can be used
    in parfait-style deserts (Fig.~\ref{fig:italian-cream}), as a filling for
    pastries, or just all by itself.
\end{open}
%%
\begin{ingredients}
    \SI{3/4}{\cup} cake flour\\
    1\SI{1/2}{\cup} whole milk\\
    \SI{1}{\cup} sugar\\
    \SI{4}{\teaspoon} butter\\
    6 egg yolks\\
    vanilla for flavoring\\
    \SI{1}{\quart} half \& half
\end{ingredients}
%%
Blend sugar and egg yolks together; add flour (I usually use a hand mixer). Put
in a heavy bottomed sauce pan. Add half \& half and milk (I usually use the hand
mixer in the saucepan to get them blended well). Heat on low while constantly
stirring until thickens. Once thick---add butter and vanilla

\protip{Give yourself plenty of time to make this; it usually takes around an
hour or more}
%%
\begin{figure}[b]
    \centering
    \includegraphics[scale=0.65]{figures/italian-creme.png}
    \caption{Italian cream!}
    \label{fig:italian-cream}
\end{figure}
\end{entry}

%%---------------------------------------------------------------------------%%
%% From the first edition

%%---------------------------------------------------------------------------%%
\begin{entry}{Fermina's Ginger Snaps}{First Edition}
\index{cookies!gingersnaps}
\index{Evans!Fermina}

\begin{open}
  This is a super-yummy cookie recipe sent in from Fermina.  She tells us that
  she always doubles this recipe when making a batch. Maybe the increased
  amounts of ingredients help the taste factor.  Either that or we're just
  gluttons.  Here's the recipe.
\end{open}
\begin{ingredients}
  \SI{3/4}{\cup} of shortening \\
  \SI{1}{\cup} sugar \\
  \SI{1/4}{\cup} light molasses \\
  1 slightly beaten egg \\
  \SI{2}{\cup} flour \\
  \SI{1/4}{\teaspoon}  salt \\
  \SI{1}{\teaspoon}  cinnamon \\
  \SI{2}{\teaspoon} soda \\
  \SI{1}{\teaspoon}  clove \\
  \SI{1/2}{\teaspoon} ginger
\end{ingredients}
Cream shortening and sugar, add molasses and egg.  Mix all dry ingredients.
Stir dry ingredients into creamed mixture.  Spoon into balls. Added step for
yumminess: \textit{Drop spoonfuls into sugar before putting on baking sheet}.
One spray of water before baking.  Bake \SI{350}{\degree} for \numrange{8}{10}
minutes.  The cookies should be split in the middle when finished.
\end{entry}

%%---------------------------------------------------------------------------%%
\begin{entry}{Betsy's Chocolate Chip Poundcake}{First Edition}
\index{cakes!chocolate chip pound cake}
\index{Evans!Betsy}

\begin{open}
  This is the famous Betsy Cordova's Chocolate Chip Cake. When she e-mailed this
  to use she sent a request for many treat recipes.  What we tell her we tell
  all.  Send us a recipe and you get a book.  This serves however many you feel
  like depending on your hunger.
\end{open}
\begin{ingredients}
  \SI{3}{\cup} sugar \\
  2 sticks butter \\
  6 eggs \\
  \SI{3}{\cup}s flour  \\
  1 carton heavy whipping cream (small size) \\
  \SI{2}{\tblspoon}  vanilla \\
  \num{1/2} bag mini chocolate chips
\end{ingredients}
Cream butter and sugar, add 2 eggs and \SI{1}{\cup} flour and beat.  Add 2
eggs and \SI{1}{\cup} flour and beat. Add 2 eggs and \SI{1}{\cup} flour and
beat.  Mix in vanilla and whipping cream and add chocolate chips.

Bake in greased and floured bundt pan at \SI{350}{\degree} for 60 to 75
minutes (depending on the temperature of your oven).
\begin{center}
\includegraphics[scale=.5,clip]{figures/pound.pdf}
\end{center}
\end{entry}

%%---------------------------------------------------------------------------%%
\begin{entry}{Amy's Cheesecake}{First Edition}
\index{cakes!cheesecake}
\index{DeLellis, Amelia}

\begin{open}
  This is a holiday favorite at the Evans/DeLellis households by Amelia
  DeLellis.
\end{open}
\begin{ingredients}
  1 box Graham Cracker Crumbs \\
  \SI{1/2}{\cup} sugar \\
  \SI{2}{\tblspoon}  flour \\
  \SI{1/4}{\teaspoon}  salt \\
  \SI{1}{\pound} cream cheese \\
  \SI{1}{\teaspoon}  vanilla extract \\
  4 eggs \\
  \SI{1}{\cup} heavy cream
\end{ingredients}
The topping ingredients are:
\begin{ingredients}
  \SI{2}{\cup}s sour cream \\
  \SI{3}{\tblspoon}  sugar \\
  \SI{1}{\teaspoon} vanilla
\end{ingredients}
Follow the directions on the Graham Cracker Box for the crust.  Use a
\SI{9}{\inch} spring form pan.  Press crumb mixture into the bottom and sides
of the pan.

Let cream cheese soften at room temperature (or use microwave).  Mix sugar,
flour, and salt.  Add dry ingredients to cream cheese.  Cream together with
low speed beater or by hand.  Separate eggs, save the whites in a clean bowl.
Add yolks to cream cheese mixture and beat until smooth.  Add vanilla.  Stir
in cream.  Beat egg whites until stiff.  Fold into cream cheese mixture.  Pour
on top of crumbs.  Bake at \SI{350}{\degree} for 1 hour.  Let cool.  Mix
topping ingredients.  Pour topping onto cheesecake and bake at
\SI{500}{\degree} for 10 minutes.  Serve with cherry, blueberry,
etc. etc. toppings.
\end{entry}

%%---------------------------------------------------------------------------%%
\begin{entry}{Pineapple Upside-down Cake}{First Edition}
\index{cakes!pineapple upside-down cake}
\index{Evans!Fermina}

\begin{open}
  Kate: On his/her birthday most kids I knew asked for chocolate cake, or
  ice-cream cake, or even cheesecake if sophisticated. But not Tom.  Tom always
  begged for this somewhat unusual birthday cake. Luckily, Fermina Evans has a
  great recipe for it!  Tom: I begged for it because it's delicious. Any kid
  would agree.
\end{open}
\begin{ingredients}
  \SI{1/2}{\cup} butter \\
  \SI{1/2}{\cup} packed brown sugar \\
  1 large can pineapple slices in syrup \\
  1 small jar maraschino cherries
  \SI{1/2}[1]{\cup} non packed flour (softasilk flour recommended) \\
  \SI{1}{\cup} sugar \\
  \SI{2}{\teaspoon} baking powder \\
  \SI{1/2}{\teaspoon} lt \\
  \SI{1/3}{\cup} soft shortening \\
  \SI{2/3}{\cup} milk \\
  \SI{1}{\teaspoon} vanilla \\
  1 large egg
\end{ingredients}
Melt butter with brown sugar in \SI{9}{\inch} baking pan (Fermina adds
\SI{1}{\tblspoon} Karo syrup here) Arrange pineapple slices on top of syrup
and place cherries in pineapple centers or wherever they look nice.

In mixing bowl, stir flour, sugar, baking powder and salt. Add shortening,
milk, and vanilla. Beat 2 minutes at medium speed with electric mixer. Add egg
and beat two more minutes. Pour batter over fruit. Bake at \SI{350}{\degree}
for \numrange{40}{50} minutes. Immediately turn upside down on serving dish
(if you don't, sugar will crystallize to pan and you will have a mess).
\end{entry}

%%---------------------------------------------------------------------------%%
\begin{entry}{Chocolate Mint Brownies}{First Edition}
\index{brownies!chocolate mint brownies}
\index{Evans!Fermina}

\begin{open}
  Kate: Blah blah blah chocolate blah. Tom: These are yummy!  My mom makes
  them.
\end{open}
\begin{ingredients}
  \SI{1}{\cup} sugar\\
  \SI{1/2}{\cup} butter or margarine\\
  4 eggs, beaten\\
  \SI{1}{\cup} flour\\
  \SI{1/2}{\teaspoon} salt\\
  1 can \corp{Hershey's Chocolate Syrup} (\SI{16}{\ounce})\\
  \SI{1}{\teaspoon} vanilla
\end{ingredients}
Mix together above ingredients and put in a greased \SI{9x13}{\inch} pan.
Bake at \SI{350}{\degree} for 30 minutes.

The middle layer ingredients are:
\begin{ingredients}
  \SI{2}{\cup} powdered sugar\\
  \SI{1/2}{\cup} butter or margarine\\
  \SI{2}{\tblspoon} \corp{Creme de Menthe} (preferably the green kind)
\end{ingredients}
Mix and spread over cooled cake.

The glaze ingredients are:
\begin{ingredients}
  \SI{1}{\cup} chocolate chips\\
  \SI{6}{\tblspoon} butter
\end{ingredients}
Let the cake cool slightly and spread over brownies.  Chill and cut into
squares.
\end{entry}

%%---------------------------------------------------------------------------%%
\begin{entry}{Gooey Butter Cake}{First Edition}
\index{cakes!gooey butter cake}
\index{Lefkowith, Pam}

\begin{open}
  A recipe from Fermi's friend Pam L.
\end{open}
\begin{ingredients}
  1 pkg. yellow cake mix\\
  \SI{1/2}{\cup} melted butter\\
  1 egg
\end{ingredients}
The topping ingredients are:
\begin{ingredients}
  \SI{8}{\ounce} cream cheese (1 pkg.)\\
  2 eggs\\
  1 box powdered sugar
\end{ingredients}
Preheat oven to \SI{350}{\degree}.  Mix together cake ingredients.  Pat into
\SI{9x13}{\inch} pan.  Beat topping ingredients together for three minutes.
Pour over cake mix.  Bake for 40 minutes.  Do not over-bake. Top should be
set, but not dry.
\end{entry}

%%---------------------------------------------------------------------------%%
\begin{entry}{Chocolate Surprise}{First Edition}
\index{cakes!chocolate surprise}
\index{Evans!Fermina}

\begin{open}
  Fermina's chocolate \& angel food cake.  Geo's birthday favorite.
\end{open}
\begin{ingredients}
  32 large marshmellows\\
  \SI{1/3}{\cup} water\\
  \SI{1/4}{\teaspoon} salt\\
  \SI{6}{\ounce} semi-sweet chocolate chips\\
  heavy whipping cream and sugar\\
  \SI{1/4}{\teaspoon} vanilla
\end{ingredients}
In sauce pan melt marshmallows, water, and salt.  Add chocolate bits.  Stir
until melted.  Let cool.

\begin{wrapfigure}{R}{.25\textwidth}
\centering\includegraphics[width=.22\textwidth,clip]{figures/kiss.pdf}
\end{wrapfigure}

Whip \SI{1}{\cup} whipping cream.  Pour chocolate over cream and fold
together.

Take a store brought or home make angel food cake.  Cut off entire top
\SI{1}{\inch} layer.  With spoon dig a tunnel in remaining cake.  Fill with
chocolate surprise.  Replace top layer.

Sprinkle cake with powdered sugar or frost with whipped cream (\SI{1}{\cup}
heavy cream beaten with \SI{1}{\tblspoon} sugar until stiff).
\end{entry}

%%---------------------------------------------------------------------------%%
\begin{entry}{Tiramisu}{First Edition}
\index{cakes!tiramisu}
\index{Horne!Mimi}

\begin{open}
  Mimi Horne brings us this delicious treat from an Italophile Brazilian Yale
  Art History Professor friend, Ester da Costa Meyer.  Those in less
  gastro-enlightened regions might need to replace the mascarpone with cream
  cheese and cream, and the Marsala with perhaps port or sherry.
\end{open}
\begin{ingredients}
  2-3 pkgs (about 24-30) lady fingers (boudoirs) \\
  \SI{3}{\cup} mascarpone (or part creme fraiche, part carre frais mushed together)\\
  3 egg yolks \\
  \SI{1/3}{\cup} sugar \\
  \SI{2}{\cup} strong coffee \\
  \SI{1/2}{\cup} or more Marsala wine \\
  \SI{1/2}{\cup} cocoa
\end{ingredients}
Prepare coffee and mix with Marsala. One at a time, dip lady fingers in
mixture briefly, then lay them in a row in an approx. \SI{10x18x2.5}{\inch}
deep serving dish.  Cut some to fit the remaining space in dish so that the
bottom is completely covered. Mix sugar, egg yolks and
mascarpone/cream. Spread about half the mixture over the first layer of
cookies to cover completely.  Dip more lady fingers in coffee/Marsala and lay
them over cream to form the next layer; cover remaining cream mixture. Dust
the top thoroughly with cocoa; chill overnight or for several hours before
serving. More cocoa may be added before serving. The texture can be made
lighter by beating the egg whites and folding them into the mascarpone
mixture, which also increases the amount.  Serves \numrange{6}{8}.
\end{entry}

%%---------------------------------------------------------------------------%%
\begin{entry}{Tortelettes}{First Edition}
\index{cookies!tortelettes}
\index{Niepold!Martha}

\begin{open}
  Another Nonnie/Grandpatty and Joy of Cooking original. Niepold kids remember
  Christmas at Lee St. when they eat Tortelettes and California dates stuffed
  with Georgia Pecans and rolled in confectioners' sugar.
\end{open}
\begin{ingredients}
  1 grated lemon rind\\
  \SI{1}{\cup} sugar\\
  \SI{3/4}{\cup} butter\\
  2 egg yolks \\
  \SI{1/2}{\cup} bread flour \\
  \SI{1}{\cup} blanched and shredded almonds or pecan pieces \\
  \SI{1/3}{\cup} sugar \\
  \SI{1}{\teaspoon} cinnamon\\
  \SI{1/4}{\teaspoon} nutmeg\\
  \SI{1/8}{\teaspoon} salt\\
  1 egg white\\
  \SI{1}{\tblspoon} water
\end{ingredients}
Preheat oven to \SI{375}{\degreeF}. Grate lemon into sugar. Cream sugar with
butter and beat in the egg yolks one at a time. Add flour gradually to make a
stiff dough. Pinch off about a teaspoonful of dough at a time. Roll it into a
ball and flatten on cookie sheet until very thin. Prepare nuts and combine
with next 4 ingredients (spices). Beat the egg white and water together
slightly.  Brush the cakes with the egg white mixture, then sprinkle nut/spice
mixture and bake until light brown.
\end{entry}

%%---------------------------------------------------------------------------%%
\begin{entry}{Lime Cream Pie}{First Edition}
\index{pie!lime cream pie}
\index{Evans!Kate}

\begin{open}
  Kate got this recipe from Edie, a receptionist at Bryn Mawr College with
  southern cooking blood.  It's very easy and delicious, especially after a
  rich meal. It's cool, refreshing, and slides right down.
\end{open}
\begin{ingredients}
  1 \corp{Graham Cracker Pie Crust}\\
  3 egg yolks\\
  \SI{2/3}[2]{\cup} sweetened condensed milk (2 cans)\\
  \SI{1}{\cup} plus \SI{2}{\tblspoon} lime juice (about 7 limes if you're
  squeezing)\\
  \SI{2}{\teaspoon} grated lime zest\\
  1 attractive lime
\end{ingredients}
Lightly whisk egg yolks in mixing bowl. Pour in condensed milk and whisk until
completely blended.  Add lime juice and zest and whisk to blend. Gently pour
filling into pie crust shell and smooth over top.  Refrigerate for at least 4
hours (don't skimp or it will be soup!). Slice the attractive lime paper thin
to garnish. I like to slice into half-circles and create a pinwheel pattern
around the center.
\end{entry}

%%---------------------------------------------------------------------------%%
%% >>> APPENDIX
\appendix
\titleformat{\section}{\Large\bfseries}{\thesection}{1em}{#1}
\newpage

\section{Family Information}

\subsection{List of Contributors}

In the last cookbook we added family trees at the end so that everyone could see how folks are connected across families, but we were afraid to do this time as its too easy to misspell, omit, etc. You would not believe the existential questions we pursued. For example, do we include the now passed on relatives in the tree, or only if we reference them in a recipe? Do we use the traditional maiden names often used in trees, or current names? Proper names or commmon? What are the chances we will swap them inadvertently (high).  Instead we provide you with a list of contributors of recipes, and their relationship to us, the authors and we used the name they gave us when sending the recipe. If made a mistake, we apologize!



\begin{enumerate}
 \itemsep0pt
 \parskip1pt
\item Betsy Cordova, Tom's eldest younger sister %\textit{411 Arrowhead Trail, Sinking Spring PA  19608}.
\item Evelina DeLellis, Tom's maternal grandmother
\item George, Sr. and Mickey Evans, Tom's paternal grandparents
\item Fermina Evans, Tom's mother % \textit{411 Arrowhead Trail, Sinking Spring PA, 19608.}
\item Martha Johnson, Kate's mother
\item Katie Evans Pross, Tom's youngest younger sister
\item Joyce Evans, Tom's Aunt on his father's side
\item Scott Evans, Tom's cousin, on his father's side
\item Tom and Kate Evans, yours truly %\textit{11112 Windward Dr, Knoxville, TN 37934}.
\item Amy DeLellis, Tom's aunt on his mother's side
\item Lauren Anton, Tom's cousin on his mother's side
\item Paul and Mimi Horne, Kate's uncle and aunt on her mother's side %\textit{28 ave. R Poincar\'{e}, 75116 Paris, France}.
\item Rich, Jenn, and Julian King, Kate's cousin on her mother's side, her husband and firstborn (the other Kings arrived after the 1996 edition when it came out)
\item Lillian ``Lil'' Johnson, Kate's paternal grandmother 
\item Don Johnson and Geyi Liu, Kate's brother and his wife
\item Liz Johnson, Kate's niece 
\item David and Dotty Lindquist, David Lindquist, Kate's cousin on her mother's side, and his wife
\item Louise ``Weedie'' Johnson, Kate's aunt on her father's side
\item Martha Hodgkins Niepold Lamb, Kate's maternal grandmother
\item Jen Tidey, mother of Eva and Axel Lindquist, Kate's niece and nephew
\item Horne Rona, Alison, Kate's cousin on her mother's side
\item Lindquist, Julie, Kate aunt on her father's side
\item Lindquist, Jenn, wife of Steven Lindquist, Kate's cousin on her father's side

\end{enumerate}

%\subsection{Family Trees}

%We realized while writing this recipe book that many of you don't know people from either Tom's or Kate's extended families.  To resolve this Tom proposed a biblical-style section, ie. Donald begot Dodge who was the Father of Kate...  Kate, on the other hand hated that format, and, for once, Tom agreed.  Therefore, we settled on family trees because they remind Tom of Feynman Diagrams and because Kate thinks they look like something having to do with weather.  Anyway, these are meant for people from one side of the hill to see who is related to who on the other side. They are not conclusive and only reflect the situation as it is now.
%}

%\begin{figure}
%    \centering
%    \includegraphics[width=.9\textwidth,clip]{figures/Johnson-Tree.pdf}
%    \caption{Johnson Family Tree.}
%    \label{fig:johnson-tree}
%\end{figure}

%\begin{figure}
%    \centering
%    \includegraphics[width=.9\textwidth,clip]{figures/DeLellis-Tree.pdf}
%    \caption{DeLellis Family Tree.}
%    \label{fig:delellis-tree}
%\end{figure}

%%---------------------------------------------------------------------------%%
%% >>> BACKMATTER
\printindex

%%---------------------------------------------------------------------------%%
\end{document}