%%-The Recipe Book
\documentclass[12pt]{article}
\usepackage{makeidx}
\usepackage{wrapfig}
\usepackage[margin=1in]{geometry}
\usepackage{xfrac}
\usepackage[detect-all, binary-units]{siunitx}
\usepackage{hyperref}
\usepackage{tocloft}
\usepackage{xspace}
\usepackage{titlesec}
\usepackage{color}

%% >>> STYLE DEFINITIONS

\makeindex

\newcommand{\name}[1]{\textit{#1}}
\newcommand{\corp}[1]{\textsf{#1}}
\newcommand{\oven}[1]{#1$^{\circ}$}
\newcommand{\meas}[1]{\ensuremath{\mathbf{#1}}}
\newcommand{\fr}[2]{#1\,\footnotesize{#2}}

\newcommand{\saute}{saut\'{e}\xspace}
\newcommand{\sauteed}{saut\'{e}ed\xspace}
\newcommand{\Saute}{Saut\'{e}\xspace}

%% Sectioning
%\titleformat{\section}{\bfseries\uppercase}{\thesection}{1em}{}
%\titleformat{\section}{\Large\bfseries\uppercase}{\thesection}{1em}{}

%% Environments
\newenvironment{ingredients}
{
    \begin{list}{}{\setlength{\listparindent}{0pt}}
    \item\bf
}
{
    \end{list}
}

\newenvironment{open}
{
    \slshape
}
{
}

\newenvironment{entry}[1]
{
    \clearpage
    \subsection{#1}
}
{

}

\newcommand{\secpart}[2]
{
    \clearpage
    \phantom{x}
    \vfill
    {\noindent\Large#1
    \section{#2}}
    \vfill
}

\newcommand{\protip}[1]
{
    \begin{center}
    \fbox{%
        \parbox{0.9\textwidth}{{\bf Pro Tip}: #1}
    }
    \end{center}
}

% Minisections
\newcommand{\minisection}[1]{\vspace{0.5\baselineskip}%
    \noindent\underline{#1}%
    \vspace{0.5\baselineskip}\noindent}

\sisetup{range-phrase = \text{--},
  group-separator={\,},
  per-mode=reciprocal,
  group-minimum-digits=3,
  inter-unit-product = \ensuremath{{}\cdot{}},
  range-units=single,
  quotient-mode=fraction,
  product-units=single,
  fraction-function=\sfrac}
\DeclareSIUnit\tblspoon{Tbsp}
\DeclareSIUnit\teaspoon{tsp}
\DeclareSIUnit\cup{C}
\DeclareSIUnit\pound{lb}
\DeclareSIUnit\ounce{oz}
\DeclareSIUnit\fluidounce{fl oz}
\DeclareSIUnit\pint{pt}
\DeclareSIUnit\quart{qt}
\DeclareSIUnit\inch{$''$}
\DeclareSIUnit\degreeF{\degree F}

\hypersetup{
    linkcolor=[rgb]{0.18,0.31,0.31},
    %linkcolor=[rgb]{0.28235294, 0.23921569, 0.54509804},
    %linkcolor=[rgb]{0.43921569, 0.50196078, 0.56470588},
    citecolor=[rgb]{0.780,0.647,0.258},
    urlcolor=[rgb]{0.325,0.494,0.658},
    colorlinks=true
}

%% >>> TOC settings

%\cftsetindents{section}{1.5em}{3em}

%% >>> DOCUMENT

\begin{document}

%% >>> FRONT MATTER

%\frontmatter
\titleformat{\section}{\Large\bfseries}{\thesection}{1em}{}[]

%% Title page
\begin{titlepage}
\vspace*{1.5in}
\begin{center}
\begin{Huge}
\vspace{2\baselineskip}
{\bf The Extended Family Cookbook}\\
\end{Huge}
\begin{huge}
\vspace{\baselineskip}
({\bf Now with Soup!})\\
\end{huge}
\begin{LARGE}
\vspace{3\baselineskip}
Second Edition!
\end{LARGE}
\end{center}
\end{titlepage}

%% Copyright page
\vspace*{\fill}
Copyright\,\copyright\, 1996, 2020 Tom and Kate Publishing People, Inc.
This document was prepared using the \LaTeX\ typesetting language.
Edited and compiled by Tom and Kate Evans.
%Tom and Kate Publishing People, Inc.
First Edition printed on 12-20-96, with recipes received between 9-95 and
12-96.
Second Edition released Dec 31, 2020.
May not be reprinted without permission (just kidding, go crazy! Just speak well of us.).

\vspace{1\baselineskip}
We are not responsible for any bad recipes. Please send corrections to Tom and Kate, 11112 Windward Dr, Knoxville, TN 37934 or email \href{mailto:ktloo@tds.net}{ktloo@tds.net}.
\pagebreak

%% TOC
\pagestyle{headings}
\tableofcontents

%% Forward
\clearpage
\section*{Foreword}

{\color{red}
Finally, we know all of you have been holding your breath this last year
in anticipation of the release of the Family Cookbook.  Well Ma'
Flanders, you can rest at ease, because IT is here.  We hope everyone
enjoys each others contributions.  Kate and Tom had a good time putting it
together, well that isn't entirely true.  Let's say we enjoyed the
THOUGHT of doing it, and, of course, now that the deed is done we are
very happy with the results.

Now that we have a `base' manuscript from which to work, inclusions will
be welcome.  We foresee that many editions of the family cookbook could
grace the bookshelves, cabinets, and, in some cases, trash cans of our
once and future homes.  Keep the recipes and corrections comin' and we'll
eventually work them in two the second edition due out in April, 2034.

Illustrations in the cookbook are meant to spice things up. Both Kate and
Tom contributed artwork.  We hope you think that they are moderately
funny.  In the future any contributions in this area are welcome.  Some
of you may be wondering, `Why no pictures?'  Well, we felt that the realm
of photography was fraught with far too much peril.  What if we forgot a
picture of so-and-so or `That's a horrible picture of me to present to
the whole family!'  Frankly, we didn't want to deal with the possibility
of such horrors, so, artwork it was.

Well, that's all our comments for now as we are tired of doing this and
have no energy left to write, format, or look at this in general.  Please
try the recipes and let us and others know which are keepers (hopefully
all) and which need modification.  May all enjoy and we look forward to
hearing from all of you  in the near future!  Happy Holidays!
}

\vspace{.5in}
\begin{flushright}
Tom and Kate Evans\\ Knoxville 2020
\end{flushright}

%% >>> MAIN DOCUMENT

%\mainmatter
\titleformat{\section}{\Huge\bfseries\uppercase}{\thesection}{1em}{}[\titlerule]

%\part{Recipes from the Second Edition}
\secpart{Second Edition}{Appetizers and Salads}

%%---------------------------------------------------------------------------%%
\begin{entry}{Paul's Salad}{Second Edition}
\index{salads!Paul's salad}
\index{Horne!Paul}

\begin{open}
    A salad recipe from Paul Horne that serves \numrange{2}{3}.
\end{open}
%%
\begin{ingredients}
    10 leaves of Romaine lettuce, cut into bite-sized pieces\\
    1 or 2 stalks of celery cut into cubes\\
    1 long cucumber, peeled, cut into cubes\\
    1 medium-sized tomato, cut into bite-sized pieces\\
    \SI{6}{\ounce} or so Gruyere cheese, cut into cubes\\
    \SI{5}{\ounce} tin of Albacore tuna in oil, in chunks
\end{ingredients}
Mix ingredients together and season with sea salt, freshly ground pepper and
sprinkle with a mix of herb, garlic, black pepper and sea salt

\minisection{Dressing}

\begin{ingredients}
    \SI{\sim 1/2}{\cup} extra virgin olive oil\\
    \SI{3}{\tblspoon} Balsamic vinegar\\
    \SI{1}{\tblspoon} Dijon mustard\\
    \SI{1}{\tblspoon} lemon juice\\
    A few drops of red pepper sauce (Cholula or Tabasco)\\
    Mayonnaise, \SIrange[range-phrase={ or }]{1}{2}{\tblspoon}\\
    (optional) a pinch of Sazon, a mixture of \SI{1}{\tblspoon} garlic powder
    \SI{1/2}{\teaspoon} black pepper\\
    \SI{1}{\tblspoon} onion powder\\
    \SI{2}{\tblspoon} fine sea salt\\
    \SI{1}{\tblspoon} ground cumin\\
    \SI{2}{\tblspoon} sweet paprika\\
    \SI{1}{\tblspoon} ground turmeric
\end{ingredients}
Accompany with buttered, toasted Mediterranean pita.
\end{entry}

%%---------------------------------------------------------------------------%%
\begin{entry}{Sausage Stuffed Mushrooms}{Second Edition}
\index{appetizers!sausage stuffed mushrooms}
\index{Lindquist!Dot}

\begin{open}
  Here is a great one by Dot Lindquist from her treasure trove food blog,
  \url{myrunawaykitchen.com}, where she writes: ``These are a Thanksgiving
  staple in my family, but I also bring them out as a heavy appetizer for
  parties, and sometimes as an extra special side for a cold November
  weeknight.  This recipe is for the ``Thanksgiving'' or ``Dinner Party''
  yield.  If you are cooking for fewer than \numrange{15}{20} people, I
  suggest downsizing the quantities.  Don’t worry too much about measurement
  of ingredients: there is no wrong way to stuff a mushroom.  Use more of what
  you like, less of what you don't, and/or eliminate or substitute one cheese
  or vegetable for another.  It's all good, quite literally!''

  This one serves up to 20 people.  Prep time is \SI{40}{\minute} and cook
  time is \SI{30}{\minute}.
\end{open}
%%
\begin{ingredients}
    6 \SI{1}{\quart} packages white or ``stuffing'' mushrooms\\
    \SI{1}{\pound} Italian sausage (mild or spicy---you choose!)\\
    \SI{1/2}{\cup} butter\\
    5 cloves garlic, minced\\
    8 stalks celery, diced\\
    \SI{2}{\cup} breadcrumbs (seasoned)\\
    \SI{2}{\cup} grated/shredded cheese (parmesan, asiago, Romano blend is
    nice)\\
    \SI{1/2}{\cup} marscapone\\
    salt and pepper to taste
\end{ingredients}
Wiggle the stems out of the mushroom caps and line up the caps on a baking sheet
(or two). Use a damp paper towel to wipe mushrooms clean of debris. Cover the
caps with a damp paper towel while you are making the filling to keep them
moist. Pop your stems into the food processor and pulse until they are minced.
Set aside in a bowl.

Brown sausage in a deep pan on the stovetop over medium heat. Empty browned
sausage into food processor and pulse until it is the same texture as the
mushroom caps.

Melt a stick of butter (\SI{1/2}{\cup}) into the pan that you used to brown the
sausage. No need to wash the pan in between uses. Add the diced celery, the
minced mushroom caps and minced garlic and simmer until the mushrooms are soft
and have changed color (from white to brown). Add the marscapone and stir.
Remove from heat.

Pour mushroom mixture into minced sausage, and stir. Add bread crumbs and grated
cheese. Stir until combined. Using a teaspoon, add stuffing into mushroom caps
until filled. Bake at \SI{350}{\degreeF} until the mushrooms have softened,
change color, are slightly browned on top and release some fluid onto the baking
sheet. Cool until you're able to handle them with bare hands, serve and enjoy (Fig.~\ref{fig:sausage-stuffed-mushrooms})!
%%
\begin{figure}
    \centering
    \includegraphics[width=0.6\textwidth]{figures/sausage-stuffed-mushrooms.jpg}
    \caption{Sausage Stuffed Mushrooms!}
    \label{fig:sausage-stuffed-mushrooms}
\end{figure}
\end{entry}

%%---------------------------------------------------------------------------%%
\begin{entry}{Taboul\'{e} from Provence}{Second Edition}
\index{salads!taboule from provence}
\index{Rona!Alison}

\begin{open}
  From Alison Rona, I've always loved Taboul\'{e} so I'm really excited to try
  this version that goes with couscous.
\end{open}
%%
\begin{ingredients}
    chopped mint\\
    chopped flat-leaf parsley\\
    chopped scallions\\
    6 cloves of minced garlic\\
    chopped ripe tomatoes (cherry, campari, heirloom, beefsteak)\\
    1 chopped whole cucumber\\
    juice from 2 lemons\\
    \SI{1}{\tblspoon} red wine vinegar\\
    sea salt\\
    black pepper\\
    olive oil
\end{ingredients}
Cook couscous with chicken broth then toss together with vegetables in a
mixing bowl and serve.
\end{entry}

%%---------------------------------------------------------------------------%%
\begin{entry}{Baked Ham and Cheese Party Sandwiches}{Second Edition}
\index{sandwiches!baked ham and cheese}
\index{Evans!Fermina}

\begin{open}
  Fermina is a pro at hosting baby showers, and this dish is one reason. She
  adapted this from \url{allrecipes.com}. Each sandwich is small and delicious
  so they are not a big commitment for a guest to grab one or two, and they
  satisfy even picky eaters.
\end{open}
%%
\begin{ingredients}
    \SI{3/4}{\cup} melted butter\\
    1 \SI{1/2}{\tblspoon} Dijon mustard\\
    1 \SI{1/2}{\teaspoon} Worchestershire sauce\\
    1 \SI{1/2}{\tblspoon} poppy seeds\\
    \SI{1}{\tblspoon} dried minced onion\\
    24 mini sandwich rolls\\
    1 pound thinly sliced cooked deli ham\\
    1 pound thinly sliced Swiss cheese
\end{ingredients}
Preheat oven to \SI{350}{\degreeF}. Grease a \SI{9x13}{\inch} baking pan. in a
bowl, mix together butter, mustard, Worchestershire sauce, poppy seeds, and
onion. Separate the tops and bottoms of the rolls, and place the bottom pieces
into the prepared baking dish. Layer about half the ham onto the
rolls. Arrange all the cheese over the ham, and then layer the rest of the ham
on top of the cheese. Place the tops of the rolls onto the sandwiches. Pour
the mustard mixture evenly over the rolls and cover with foil. Bake in
preheated oven until rolls are lightly browned and cheese is melted, about 30
minutes. Remove foil and continue baking for 2 more minutes.
\end{entry}

%%---------------------------------------------------------------------------%%
\begin{entry}{Reading Hoagie dip in a bread bowl}{Second Edition}
\index{appetizers!hoagie dip}
\index{Evans!Fermina}

\begin{open}
  Here is a another crowd pleaser from Fermina Evans, and it is sure to make
  you start cheering the Philadelphia Eagles!
\end{open}

\begin{ingredients}
    1 medium onion\\
    2 pickled pepperoncini peppers\\
    \SI{1/2}{head} iceberg lettuce\\
    2 large tomato, halved and and seeded\\
    \SI{1/4}{\pound} sliced genoa salami\\
    \SI{1/4}{\pound} sliced ham\\
    \SI{1/4}{\pound} sliced prosciutto\\
    \SI{1/4}{\pound} sliced roast turkey\\
    \SI{1/4}{\pound} sliced provolone cheese\\
    \SI{1/2}{\cup} mayonnaise\\
    \SI{1}{\tblspoon} extra virgin olive oil\\
    \SI{1}{\teaspoon} dried oregano\\
    1 \SI{1/2}{\teaspoon} dried basil\\
    \SI{1/4}{\teaspoon} red pepper flakes\\
    1 round loaf Italian bread\\
    8 hoagie rolls, cut into pieces for dipping
\end{ingredients}
%
\protip{Pepperoncini peppers look similar to banana peppers, but they are
  rounder and more yellow and even the mildest has a little heat. (Editor's
  note: pepperoncini peppers are likely what you have been eating on hoagies
  forever, thinking that they were banana peppers)}
%
Chop the onion, peppers, lettuce, tomato, deli meats and cheeses into bite
sized pieces. Combine into a large bowl and add mayonnaise, oil, oregano,
basil, and flakes. Stir until everything is evenly mixed. Refrigerate until
ready to serve. When serving, carve out the center of the bread loaf and safe
the scraps to make more bite sized pieces for dipping. Place hoagie dip in
bread bowl and serve surrounded by bread pieces for dipping.
\end{entry}

%%---------------------------------------------------------------------------%%
%% From the first edition!

%%---------------------------------------------------------------------------%%
\begin{entry}{Wedding Brunch Gazpacho}{First Edition}
\index{Soups!gazpacho}
\index{Vegetarian!gazpacho}
\index{Tidey, Jen}

\begin{open}
  This gazpacho recipe comes by way of Jen Tidey.  We were first
  introduced to it during the ``day after'' wedding brunch.  The amounts of
  ingredients are variable, so add to your personal taste preferences.
\end{open}
\begin{ingredients}
  celery \\
  peeled and seeded cucumbers \\
  scallions and/or onions \\
  red and green peppers \\
  minced garlic \\
  \num{\sim 6} tomatoes (canned are O.K.) \\
  \SI{\sim 1/2}{\cup} of red wine vinegar \\
  \SI{\sim 1/2}{\cup} of olive oil \\
  \numrange{1}{1}\SI{1/2}{\cup} tomato juice or \corp{V8} \\
  lemon juice \\
  black pepper \\
  Tabasco or cayenne pepper \\
  fresh cilantro
\end{ingredients}
Chop the celery, cucumbers onion, peppers, tomatoes, and garlic to a
consistency that you find happy.  A food processor may be used for ``soupy''
consistency.  Mix the red wine vinegar and olive oil (these should be in equal
amounts).  To this add the tomato juice (\corp{V8}).  Add lemon juice, black
pepper, Tabasco, and fresh cilantro.  Combine veggies and liquid ingredients
in a glass pitcher or other aesthetically pleasing container.  Chill for
several hours or overnight.  Serve with crunchy bread and a lightish red wine.
\end{entry}

%%---------------------------------------------------------------------------%%
\begin{entry}{Frog Mustard Salad Dressing}{First Edition}
\index{Dressings!frog mustard salad dressing}
\index{Vegetarian!frog mustard salad dressing}
\index{Vegan!frog mustard salad dressing}
\index{Evans!Kate}

\begin{open}
  This is a quick preparation dressing that can be prepared in about 10
  minutes.  It makes \numrange{8}{10} servings.
\end{open}
\begin{ingredients}
  \SI{1/2}{\cup} dijon mustard \\
  \SI{2}{\tblspoon} red wine vinegar \\
  \SI{1/4}{\teaspoon} salt \\
  \SI{3/4}{\teaspoon} pepper \\
  \SI{1}{\cup} corn oil (or olive)
\end{ingredients}
The crouton ingredients are:
\begin{ingredients}
  4 slices fine textured white bread \\
  \SI{4}{\tblspoon}  butter \\
  \SI{1}{\teaspoon} dried thyme \\
  \SI{1/8}{\teaspoon} salt \\
  dash of pepper \\
  \SI{2}{\teaspoon} minced parsley
\end{ingredients}
\begin{wrapfigure}{R}{.45\textwidth}
\centering\includegraphics[width=.42\textwidth,clip]{figures/frog.pdf}
\end{wrapfigure}
To make the salad dressing whisk the mustard, vinegar, salt and pepper in a
small bowl.  Gradually add oil.

To make the croutons preheat the oven to \SI{350}{\degree}.  Trim crusts (if
desired) and cut slices into \SI{1/2}{\inch} cubes.  Spread single layer on
tray, bake for \numrange{10}{15} minutes until dry and lightly brown.  Heat
butter in a skillet.  Add croutons and remaining ingredients and toss well.
Saut\'{e} \numrange{1}{2} minutes over medium heat and cool.  Kate says that
good salad stuffs are greenleaf or romaine lettuce with spinach.  Add tomatoes,
cucumber, and other salad stuff as your heart desires.  Another yummy thing is
to add pieces of chicken, ham, or turkey and other meats, which makes it taste
almost like a bite-sized sandwich.
\end{entry}

%%---------------------------------------------------------------------------%%
\begin{entry}{Easiest Tomato Aspic}{First Edition}
\index{Appetizers!tomato aspic}
\index{Johnson!Lil}

\begin{open}
  We received this recipe from Grammie (Lil Johnson) and we must confess
  we had no idea what an ``aspic'' was!  For those not in the know, its a tasty
  treat.
\end{open}
\begin{ingredients}
  1 small pkg. lemon \corp{Jello}\\
  \SI{1}{\cup} boiling water \\
  1 \SI{8}{\ounce} can \corp{Hunt's} Tomato Sauce \\
  \SI{1}{\teaspoon} horseradish \\
  (\SI{1}{\teaspoon} lemon juice or vinegar optional)
\end{ingredients}
Add the following if you desire.
\begin{ingredients}
  cooked shrimp \\
  crabmeat \\
  chopped celery
  hard cooked egg \\
  asparagus \\
  artichoke hearts \\
  avocado
\end{ingredients}
Pour 1 cup boiling water over \corp{Jello} and mix until smooth. Add tomato
sauce, horseradish, and lemon juice or vinegar.  To this you can add any of
the optional ingredients. Grandaddy's (Don) favorites are seafood and chopped
celery. Place aspic into fridge and jell about 3 hours.
\end{entry}

%%---------------------------------------------------------------------------%%
\begin{entry}{Meaty Cheese dip, aka ``Queso''}{First Edition}
\index{Appetizers!meaty cheese dip}
\index{Beef!meaty cheese dip}
\index{Evans!Kate}

\begin{open}
  Just so you know, this dish isn't good for you. But it's so delicious, we
  don't care. Kate originally got this recipe from \corp{Southern Living}, but
  has since improved it using input from our dear friend Dave Court\index{Court,
  Dave}.  For you food snobs, don't be put off by the \corp{Velveeta}.  It
  provides a smoothness in this dish.
\end{open}
\begin{ingredients}
  \SI{1}{\pound} ground turkey or beef\\
  \SI{1/2}{\pound} hot bulk pork sausage\\
  1 \SI{8}{\ounce} jar salsa or 1 can Rotel diced tomatoes and green chiles, any heat level you like\\
  1 \SI{2}{\pound} loaf \corp{Velveeta} with jalapenos, cut into cubes\\
  1 \SI{10.5}{\ounce} can Campbell's cream of green chile (or celery, if you are not in New Mexico) condensed soup
\end{ingredients}
Brown ground meat and sausage in large skillet, stirring so it crumbles. Add
salsa, cheese, and soup and cook over low heat until the cheese melts, about 2
hours in a slow cooker. Serve warm with nacho or corn chips. Yum!
\end{entry}

%%---------------------------------------------------------------------------%%
\begin{entry}{Hot crab dip}{First Edition}
\index{Appetizers!hot crab dip}
\index{Johnson!Dodge}
\index{Johnson!Martha}

\begin{open}
  This is sooo good! Martha and Dodge submitted this. This is a dip they
  should make more often. But ha ha, now we have the recipe!
\end{open}
\begin{ingredients}
  \SI{8}{\ounce} cream cheese, softened\\
  \SI{1/2}{\pound} seafood flakes (fake crab legs)\\
  \SI{2}{\tblspoon} chopped onion\\
  Worcestershire sauce
\end{ingredients}
Preheat oven to \SI{350}{\degreeF}. Mix cream cheese. Slice and add ``crab.''
Next add onion and Worcestershire sauce. Bake for \numrange{15}{20} minutes,
or when bubbly.
\end{entry}

%%---------------------------------------------------------------------------%%
\begin{entry}{Blue Cheese-Pecan Spread}{First Edition}
\index{Appetizers!blue cheese-pecan spread}
\index{Vegetarian!blue cheese-pecan spread}
\index{Johnson!Dodge}
\index{Johnson!Martha}

\begin{open}
  This is an easy and yummy appetizer submitted by Martha and Dodge.  We
  usually spoil our appetite for dinner eating it with all kinds of crackers!
\end{open}
\begin{ingredients}
  \SI{1/2}{\cup} pecan pieces\\
  \SIrange{4}{5}{\ounce} cream cheese\\
  At least \SI{2}{\tblspoon} blue cheese, Gorgonzola, or Roquefort\\
  \SIrange{1}{2}{\tblspoon} butter \\
  Worcestershire sauce and/or hot pepper flakes, optional
\end{ingredients}
In food processor, process pecans until fine. Add cream cheese and blue cheese
in small chunks. Add more blue cheese if it doesn't taste like enough.
``Smooth'' out flavors with the butter, if necessary. Add hot stuff if
desired.  Spoon into crock or pretty bowl and refrigerate until ready to
serve.
\end{entry}
\chapter{Soups}

\section{Spicy Thai Chicken Soup\index{soups!spicy Thai chicken soup}}

\begin{open}
    Here's a nice soup recipe from Jen Lindquist.
\end{open}
%%
\begin{ingredients}
    2 large cloves of garlic, chopped\\
    1 medium onion, chopped\\
    1 red bell pepper, chopped\\
    \SI{2}{\tblspoon} ginger paste or fresh ginger\\
    \SI{1}{\tblspoon} lemongrass paste\\
    \SI{2}{\tblspoon} red curry paste\\
    \SI{1}{\tblspoon} cilantro paste\\
    \SI{1}{\tblspoon} red chili paste (can also use 2 red chilies, chopped)\\
    \SI{2}{\tblspoon} coconut oil\\
    1 lime, zested and juiced\\
    \SI{1}{\tblspoon} fish sauce\\
    \SI{4}{\cup} chicken stock\\
    1 can (\SI{13.5}{\fluidounce}) coconut milk\\
    \SI{2}{\cup} shredded chicken
\end{ingredients}
For garnish:
\begin{ingredients}
    1 bunch of cilantro leaves chopped\\
    \numrange{4}{5} green onions, thinly sliced, both white and green parts\\
    1 green chili, sliced, seeds removed\\
    1 red chili, sliced, seeds removed
\end{ingredients}
To serve:
\begin{ingredients}
    roughly chopped coriander\\
    sliced red chili\\
    sliced green onion
\end{ingredients}
Combine garlic, onion, red pepper, ginger paste, lemongrass paste, red curry
paste, cilantro paste, red chili paste and coconut oil in a food processor, and
process until a paste forms. Add the paste to a medium pot and fry it for a
couple of minutes, just until it's fragrant. Add the chicken stock and coconut
milk and bring to a boil, reduce heat to a simmer and simmer for about
\SI{10}{\minute}. In the last couple of minutes, add your shredded chicken. Add
your fish sauce, lime juice and zest and rice noodles to the soup. Garnish with
chopped cilantro, green onion and green and red chilies.
%%
\begin{figure}
    \centering
    \begin{subfigure}{0.5\textwidth}
        \centering
        \includegraphics[width=\textwidth]{figures/spicy-thai-chick-soup}
    \end{subfigure}
    \begin{subfigure}{0.37\textwidth}
        \centering
        \includegraphics[width=\textwidth,angle=270]{figures/thai-spices}
    \end{subfigure}
    \caption*{Spicy Thai Chicken Soup!}
\end{figure}
\chapter{Breads}

\section{Not Very French Crepes\index{breads!not very french crepes}}
\index{Johnson!Gege}
\index{Johnson!Don}

\begin{open}
    Hey, anything that's not very French I love.  Also, are crepes (or
    cr\^{e}pes as I know some of you will insist, even though both spellings are
    valid), breads, deserts, or main dishes?  Who knows, who cares, we put them
    here.

    As Don and Gege say, ``The crepes are but a vesssel for whatever good stuff
    you put into the filling.''  We could not agree more.

    This is a recipe derived from \url{allrecipes.com}; it is absolutely
    foolproof, at least according to Don and Gege
\end{open}
\begin{ingredients}
    \SI{1}{\cup} flour\\
    2 eggs\\
    \SI{1/2}{\cup} milk\\
    \SI{1/2}{\cup} water\\
    dash of salt\\
    \SI{2}{\tblspoon} melted butter
\end{ingredients}
Beat the eggs and mix thoroughly into the flour. Add milk and water and beat as
long as you can or until no lumps. Add salt and beat in melted butter. It's a
very thin batter---don't be alarmed. Drop small ladlefuls into a skillet and
rotate so the batter coats the bottom in a very thin layer; no oil needed after
the first one. Makes \numrange{8}{10} crepes.

\minisection{Filling}

\noindent No measurements here; use the Force and whatever you have handy
\begin{ingredients}
    Half pound to a pound ground meat or sausage\\
    A little soy sauce---dark is better than light\\
    Diced onion\\
    Chopped tomatoes---\numrange{1}{2} large or several small
\end{ingredients}
The above three ingredients are the base, add other things as convenient and to take the filling in different directions---mushrooms, spinach, chopped peppers, parmesan cheese are all good.

Brown the ground meat/sausage in oil, add a little soy sauce for flavor and
color. Add onion and cook until tender and transparent. If you have mushrooms,
peppers, etc. add them and saut\'{e}. Add tomatoes last and saut\'{e} until you
have a thick sauce or paste. If you are using spinach or cheese stir them in
right before serving.

Put a plate of crepes and a big bowl of sauce on the table and let people roll
their own. Crepes are tender and knife and fork are usually required.

\section{Alison's Favorite Maltese Dish from Gozo}
\index{breads!Alison's Maltese Dish}
\index{Rona!Alison}

\begin{open}
    For those of you geographically challenged, or at the least, not familiar with the geography of Malta, Gozo is a small island just north of the main island of Malta.  This spread comes from Alison Rona.
\end{open}
%%
\begin{ingredients}
    very ripe red tomatoes\\
    capers\\
    black olives with herbs\\
    garlic\\
    basil\\
    mint\\
    olive oil\\
    salt\\
    pepper
\end{ingredients}
Marinate the tomatoes for several days with capers, olives and herbs, lots of chopped garlic, lots of chopped basil and mint, lots of olive oil, and salt and pepper.  Once the mixture is tasty, spread thickly on fresh crusty thick bread.

\section{Cr\`eme Br\^ul\'ee French Toast}
\index{breakfast!cr\`eme br\^ul\'ee french toast}
\index{Evans!Fermina}

\begin{open}

\end{open}
%%
\begin{ingredients}
    \SI{1/2}{\cup} (1 stick) unsalted butter \\
    \SI{1}{\cup} packed brown sugar \\
    \SI{2}{\tblspoon} corn syrup \\
    Bread (see comments below)\\
    5 large eggs\\
    1 \SI{1/2}{\cup} half-and-half cream \\
    \SI{1}{\teaspoon} vanilla extract \\
    \SI{1}{\teaspoon} Grand Marnier \\
    \SI{1/4}{\teaspoon} salt \\
\end{ingredients}
Regarding the bread, the recipe calls for 8 to 9 inch round country loaf with crusts removed, but challah is Fermina's favorite, with brioche a close second. Really you can use any type, even baguettes with the crusts still on. And day-old bread is even better, as it soaks up more of the sauce, so whatever you have around to use up is the reason to make this dish!

In a small heavy saucepan, melt butter with brown sugar and corn syrup over medium heat, stirring constantly until smooth and pour into a \SI{9x13}{\inch} baking pan. Cut 6 1-inch thick slices from center portion of bread, reserving ends for another use, and trim crusts if you choose. Arrange bread slices in one layer in baking dish over butter/sugar mixture, squeezing them slightly to fit. In a bowl, wisk together eggs, cream, vanilla, Grand Marnier, and salt until well combined and pour evenly over bread. Chill bread mixture, covered, for at least 8 hours and up to 1 day. 

When ready to bake, bring bread to room temperature and preheat oven to \SI{350}{\degreeF}. Bake bread mixture, uncovered, in the middle of the oven until puffed and edges are pale golden, 35-40 minutes. Serve immediately. 

\section{World's Best Granola Ever}
\index{breakfast!World's best}
\index{Evans!Kate}

\begin{open}
Kate has no strong feelings about granola, but recently read an article about how there exists a granola so good, people cannot stop talking about it. I mean, the maple sugar and olive oil is divine! So she made it herself with several adaptations to suit her tastes and WOW. She now makes it monthly so that there is always some on hand. Here is the link to the article, \url{ https://food52.com/recipes/15831-nekisia-davis-olive-oil-maple-granola} so you can go see the original version or check out the comments section for more adaptation ideas. 
\end{open}
%%
\begin{ingredients}
    \SI{3}{\cup} rolled oats (e.g. Bob's Red Mill whole grain)\\
    \SI{1}{\cup} hulled raw pumpkin seeds\\
    \SI{1}{\cup} dried cranberries\\
    \SI{1}{\cup} unsweetened coconut chips\\
    1 \SI{1/4}{\cup} raw pecans, chopped\\
    \SI{3/4}{\cup} pure maple syrup\\
    \SI{3/4}{\cup} extra-virgin olive oil\\
    1 pinch coarse salt, to taste\\
\end{ingredients}
Preheat oven to \SI{300}{\degreeF}. Place oats, pumpkin seeds, cranberries, coconut, pecans, syrup, olive oil, sugar, and salt in a large bowl and mix until well combined. Spread granola mixture in an even layer on a rimmed baking sheet. Transfer to oven and bake, stirring about every 15 minutes until granola is toasted, about 45 minutes. Remove granola from oven and let cool completely before serving. Store in an airtight container for up to 1 month.
\secpart{Second Edition}{Sides}

%%---------------------------------------------------------------------------%%
\begin{entry}{White Grapes and Sweetened ``Cream''}{Second Edition}
\index{Fruit!white grapes \& sweetened cream}
\index{Vegetarian!white grapes \& sweetened cream}
\index{Lindquist!Julie}

\begin{open}
  Here is a refreshing and simple summery side submitted by Julie Lindquist.
\end{open}
%%
\begin{ingredients}
    grapes\\
    plain yogurt or sour cream\\
    brown sugar
\end{ingredients}
Remove grapes from stems; rinse and dry. Place in a colorful serving dish. Top
with plain yogurt or sour cream (NB: sweet cream doesn't offer the contrast to
the sweet grape taste). Sprinkle some brown sugar on top. Refrigerate for at
least 4 hours; the attractive pattern of the dissolving brown sugar is a
bonus. Lasts a few days, if there’s any left.
\end{entry}

%%---------------------------------------------------------------------------%%
\begin{entry}{Turkish Rice}{Second Edition}
\index{Rice!turkish}
\index{Johnson!Martha}
\index{Johnson!Dodge}

\begin{open}
    From Martha and Dodge, a graduate school special that provides an infinitely
    variable way to prepare a pilaf with your favorite spices and ingredients.
\end{open}
%%
\begin{ingredients}
    chopped onion or shallot\\
    \SI{1}{\cup} long grain rice\\
    2\SI{1/3}{\cup} beef bouillon\\
    raisins, dried cranberries, nuts\\
    cumin, curry powder, garlic
\end{ingredients}
Cook a few tablespoons of chopped onion or shallot in some butter and oil. Add
uncooked rice and stir until light brown. Add the beef bouillon, plus a handful
of raisins, dried cranberries, nuts and seasonings (some mix of cumin, curry
powder, garlic, etc.) and simmer until done (\SI{20}{\minute}).
\end{entry}

%%---------------------------------------------------------------------------%%
\begin{entry}{Boursin Potatoes}{Second Edition}
\index{Potatoes!boursin potatoes}
\index{Vegetarian!boursin potatoes}
\index{Johnson!Martha}

\begin{open}
    From Martha: a favorite side from Barby Buckman at St.~Peter's in the Great
    Valley, PA. A great dish for a crowd (8), or a half recipe also works well
    for 4.
\end{open}
%%
\begin{ingredients}
    \SI{2}{\cup} whipping cream\\
    1 \SI{5}{\ounce} ounce package Boursin cheese (herbs)\\
    \SI{3}{\pound} red new potatoes, or Yukon Gold, unpeeled, scrubbed and
    thinly sliced\\
    salt and pepper\\
    \SIrange{1}{2}{\tblspoon} chopped fresh parsley
\end{ingredients}
Preheat oven to \SI{400}{\degreeF}. Stir whipping cream and Bousing cheese in a
heavy pan over medium heat until cheese melts and no lumps remain. Arrange half
the potatoes in overlapping rows in a buttered \SI{9x13}{\inch} baking dish.
Season with salt and pepper and pour half the cheese mixture over them. Arrange
the rest of the potatoes in a second layer with remaining cheese mix. Bake until
golden brown, about \SI{1}{\hour}. Sprinkle with parsley and serve.
\end{entry}

%%---------------------------------------------------------------------------%%
\begin{entry}{Immune-Boosting Chimichurri from Argentina}{Second Edition}
\index{Sauces!immune-boosting chimichurri}
\index{Horne Rona!Alison}

\begin{open}
    Chimichurri is basically Argentinian pesto.  You can use it on everything
    from meat dishes, vegetables, salads, and breads.  According to Alison Horne Rona, ``This will make any meat taste great!''
\end{open}
%%
\begin{ingredients}
    1 large bunch of flat leafed parsley chopped\\
    1 large bunch of fresh oregano leaves stripped from stems\\
    1 large bunch of fresh cilantro all chopped\\
    1 green jalape\~{n}o pepper chopped\\
    1 small yellow onion (or shallots) chopped\\
    1 lemon (I cut off most of the rind, cut it in half, take out the seeds, and
    throw both halves in the blender)\\
    6 large cloves of garlic chopped finely\\
    \SIrange{2}{3}{\tblspoon} of red wine vinegar to taste\\
    \SIrange{3}{4}{\tblspoon} olive oil\\
    sea salt and black pepper
\end{ingredients}
Throw everything in a blender and mix.
\end{entry}

%%---------------------------------------------------------------------------%%

\begin{entry}{Blanched lettuce with sauce}{Second Edition}
\index{Salads!blanched lettuce with sauce}
\index{Vegan!blanched lettuce with sauce}
\index{Liu!Geyi}

\begin{open}
  Submitted by Geyi, this is a dish often seen in Southern Chinese restaurants. It is often crisp and green in restaurants but soggy and dark at home. The secret is to boil the lettuce leaves very fast (5-10 seconds) in a large amount of water so that they stays green and crisp while keeping in all the nutrition.
\end{open}

%\begin{ingredients}
%
%\end{ingredients}

\vspace{1\baselineskip}
Prepare sauce by mixing what you think is a good amount of oyster sauce, soy sauce, salt, sugar, (corn) starch, water together. Ask Geyi for tips on the right amount; she was busy moving when when we were entering this, but it sounded so good we included it anyway!

\begin{figure}[h]
    \centering
    \includegraphics[width=0.8\textwidth]{figures/lettuce1.png}
    \caption{Sauce Preparation}
    \label{fig:Lettuce1}
\end{figure}

Next cook the lettuce as follows: Take a large pot of water, add a tablespoon of vegetable oil, a teaspoon of salt and bring to a boil. Add lettuce, stir so that all areas get blanched, and take out after 5-10 seconds (refer to figure \ref{fig:lettuce2} to see how the lettuce is still quite green).

\begin{figure}
    \centering
    \includegraphics[width=0.8\textwidth]{figures/Lettuce2.png}
    \caption{Blanching the Lettuce}
    \label{fig:lettuce2}
\end{figure}

Then, heat oil in a pot, add chopped garlic and stir until it becomes yellow, and add the sauce mixture and stir until it thickens.

Finally, pour the sauce over the lettuce and serve immediately.

\begin{figure}
    \centering
    \includegraphics[width=0.8\textwidth]{figures/Lettuce3.png}
    \caption{Pour the Sauce over the lettuce}
    \label{fig:lettuce3}
\end{figure}

\end{entry}

%%---------------------------------------------------------------------------%%
\begin{entry}{Roasted Brussels Sprouts}{Second Edition}
\index{Vegetarian!brussel sprouts}
\index{Pross!Katie}

\begin{open}
 This recipe looks absolutely amazing and super easy for our new cooks to try. Katie adds crumbled bacon to it, which sounds delicious!
\end{open}
%%
\begin{ingredients}
    \SI{1}{\pound} fresh Brussels sprouts \\
    \SIrange{1}{2}{\tblspoon} olive oil\\
    \SI{1}{\tblspoon} \\
    salt and pepper\\
\end{ingredients}
Preheat oven to \SI{450}{\degreeF}. Trim ends of Brussels sprouts and cut into
halves or quarters, depending on the size of each sprout. Drizzle sprouts with
olive oil, salt and pepper, and toss to coat. Roast for
\SIrange{10}{20}{\minute}, tossing a few times throughout until sprouts are
browned.
\end{entry}

%%---------------------------------------------------------------------------%%
%% From the first edition

%%---------------------------------------------------------------------------%%
\begin{entry}{Hash Brown Potatoes}{First Edition}
\index{Potatoes!hash brown potatoes}
\index{Evans!Joyce}

\begin{open}
  Contributed by Joyce Evans. Try with the Swiss Meatloaf
  (Section~\ref{sec:swiss-meatloaf}), yum! Serves 4.
\end{open}
\begin{ingredients}
  3 large potatoes, boiled \\
  \SI{1/4}{\cup} milk \\
  \SI{3}{\tblspoon} all-purpose flour \\
  \SI{2}{\tblspoon} minced onion \\
  \SI{2}{\tblspoon} minced fresh parsley or chervil \\
  \SI{1/2}{\teaspoon} salt \\
  \SI{1/2}{\teaspoon} pepper \\
  \SI{1/4}{\teaspoon} dried oregano (opt.) \\
  Dash of Tabasco \\
  \SI{3}{\tblspoon} bacon drippings, rendered chicken fat, or vegetable oil
\end{ingredients}
Preheat in electric skillet to \SI{300}{\degree}. Peel and dice the boiled
potatoes and place into a medium bowl. You should have about \SI{3}{\cup}. Add
the rest of the ingredients except the cooking fat and blend.

Add the cooking fat to the skillet and heat. Pack the potato mixture in
firmly, spreading it out in an even layer. Cook \numrange{7}{9} minutes or
until the bottom side is richly brown. Turn the mixture over in segments and
smooth down again into a patty. Continue cooking until the other side is
brown, another \numrange{7}{9} minutes.  Cut into wedges and serve.
\end{entry}

%%---------------------------------------------------------------------------%%
\begin{entry}{Black Rice}{First Edition}
\index{Rice!black}
\index{Evans!Fermina}

\begin{open}
  Contributed by Fermina Evans. Serves 4, or 2 healthy eaters. This is Tom and
  Katie's favorite peasant food.
\end{open}
%%
\begin{ingredients}
  \SI{1}{\cup} dry black beans\\
  \SI{5}{\cup} chicken broth\\
  \SI{1/2}{\tblspoon} olive oil \\
  1 small onion, chopped \\
  4 cloves garlic, minced \\
  \SI{1}{\ounce} finely chopped Canadian or regular bacon \\
  \SI{1/2}{\cup} rice \\
  \SI{1/4}{\cup} white wine \\
  1 tomato coarsely chopped \\
  \SI{1/2}{\teaspoon} ground cumin \\
  pinch cayenne \\
  \SI{1/2}{\cup} finely chopped cilantro
\end{ingredients}
You may substitute 2 cans black beans for dry beans if you prefer. If using
dry beans, soak beans overnight in cold water, and simmer beans for
\numrange{120}{150} minutes in \SI{3}{\cup} broth until tender. Drain and
resolve liquid (\SI{1/2}[1]{\cup}). Otherwise, drain them under cold water and
use \SI{1/2}[1]{\cup} chicken broth or chicken bouillon stock for bean broth.

Heat oil in stockpot. Add onion, garlic, and bacon and stir fry for about 5
minutes.  Add rice and stir for 1 minute. Add wine and cook for 2 minutes. Add
tomatoes and cook for 2 more minutes.  Add bean broth \SI{1/2}{\cup} at a
time, stirring until liquid is absorbed before adding more broth. This will
take \numrange{20}{25} minutes to complete. Add the beans and remaining
broth. Season with cumin, cayenne, and cilantro and serve.
\end{entry}

%%---------------------------------------------------------------------------%%
\begin{entry}{Potato Gratin with Mustard and Cheese}{First Edition}
\index{Potatoes!potato gratin}
\index{Vegetarian!potato gratin}
\index{Evans!Kate}

\begin{open}
  This is a great entertaining dish because its classy, very smooth and
  flavorful, yet can be prepared before guests arrive. Kate got it from
  \corp{Bon Appetit} magazine.
\end{open}
\begin{ingredients}
  \SI{1}{\tblspoon} butter\\
  \SI{1}{\cup} fresh breadcrumbs\\
  \SI{1}{\tblspoon} dried thyme\\
  \SI{2}{\teaspoon} salt\\
  \SI{1}{\teaspoon} ground pepper\\
  \SI{1}{\pound} sharp white cheddar cheese, grated\\
  \SI{1/4}{\cup} flour\\
  \SI{5}{\pound} russet potatoes, peeled and thinly sliced\\
  \SI{4}{\cup} canned low salt chicken broth (veggie broth can be substituted) \\
  \SI{1}{\cup} whipping cream\\
  \SI{6}{\tblspoon} Dijon mustard
\end{ingredients}
Melt butter in skillet and add breadcrumbs, stirring until golden brown (about
10 min.). Set aside. Preheat oven to \SI{400}{\degree}. Butter a
\SI{15x10x2}{\inch} baking dish. Mix thyme, salt, and pepper in small
bowl. Combine grated cheese and flour, tossing to coat the cheese. Arrange
\num{1/3} potato slices to cover the bottom of the baking dish. Sprinkle
\num{1/3} the thyme mixture, then \num{1/3} the cheese mixture. Repeat
layering 2 more times.  Next whisk chicken broth, cream, and mustard in a
separate bowl, and then pour it over the potato layers. Bake 30
minutes. Sprinkle buttered crumbs over, and bake until potatoes are tender and
top is golden brown, about 1 hour longer.  Enjoy!
\end{entry}
\secpart{Second Edition}{Entrees}

%%---------------------------------------------------------------------------%%
\begin{entry}{Mushroom Cream Pasta}{Second Edition}
\index{Pasta!mushroom cream}
\index{Vegetarian!mushroom cream pasta}
\index{Liu!Geyi}
\index{Johnson!Don}

\begin{open}
    Lightly adapted, mostly for convenience, from ``Chef John's Creamy Mushroom
    Pasta'' on \url{allrecipes.com}, My family (who are NOT vegetarians) are
    totally satisfied with this quick and easy dish as a main course. From: Don
    and Geyi.
\end{open}
\begin{ingredients}
    \SI{1/2}{\pound} linguine, fettuccine or other pasta\\
    \SI{1}{\pound} mushrooms. I like a mix of white and shiitake, but baby
    portobello also work well\\
    Olive oil, salt pepper\\
    Garlic\\
    \SI{1}{\tblspoon} sherry (For some reason, I more often have dry vermouth
    which seems to work fine.) \\
    Chicken stock (optional)\\
    \SI{1}{\cup} heavy whipping cream\\
    \SI{1/2}{\cup} grated parmesan cheese\\
    Fresh chopped thyme, chives, tarragon (n.b. I have never added these. I'm
    sure  they would make it better.)
\end{ingredients}
\Saute sliced mushrooms in olive oil until they are tender and release their
liquid. Add several cloves of diced garlic. Add sherry followed by heavy cream,
and lick residual cream from cup measure. Add salt and pepper and simmer cream
until the mixture thickens a bit and foams---though it does not get very thick;
if it does add some chicken stock. When the mixture reaches reasonable
consistency, stir in the fresh spices, turn off the heat and mix in the parmesan
cheese. Stir the mushroom cream mixture into the pasta and serve.
\end{entry}

%%---------------------------------------------------------------------------%%
\begin{entry}{Broccolini and Chourico Portuguese Sausage}{Second Edition}
\index{Sausage!broccolini \& Chourico sausage}
\index{Pork!broccolini \& Chourico sausage}
\index{Lindquist!Julie}

\begin{open}
    This is one of Julie's Simple/Quick/Tasty entries.  Chourico is similar to
    Spanish Chorizo, and I use only Gaspar's (and have found the meat department
    variety of chourico isn't as flavorful and spicy)---see
    Fig.~\ref{fig:chourico}. This is my go-to dinner!

    As a historical side-note, broccolini was developed in Japan and is a hybrid
    between broccoli and Chinese broccoli (Chinese kale).
\end{open}
%%
\begin{figure}
  \centering
  \includegraphics[width=0.6\textwidth]{figures/broccolini-chourizo.pdf}
  \caption{Gaspar's chourico and broccolini.}
  \label{fig:chourico}
\end{figure}
%%
\begin{ingredients}
    Chourico (Gaspar's brand)\\
    broccolini
\end{ingredients}
Cut up raw broccolini (stems especially are so much sweeter than broccoli) and
place in wide soup bowl; microwave Chourico at 8 power for about
\SI{1.5}{\minute} to heat the meat through then cut into bite-size pieces.
Place on top of the broccolini and enjoy with a hearty red (I find a Malbec the
best accompaniment); some warmed sourdough baguette on the side or afterwards
with an excellent creamy French cheese doesn't hurt.
\end{entry}

%%---------------------------------------------------------------------------%%
\begin{entry}{Portobello Caps Stuffed with Crab}{Second Edition}
\index{Seafood!crab stuffed portobello caps}
\index{Lindquist!Julie}

\begin{open}
  From Julie Lindquist; this dish harmonizes well with a medium red or a
  hearty semi-sweet white. Tasty as an hors d'oeuvre but also great for a main
  dish.
\end{open}
%%
\begin{ingredients}
    Portobello mushrooms\\
    fresh crabmeat\\
    red pepper\\
    parsley
\end{ingredients}
Using large or medium Portobellos, remove stems. Stuff caps with fresh
crabmeat (preferably not previously frozen; canned just doesn't do it!). Place
under broiler for a few minutes (5 or so) until lightly browned. Top with a
few slices of red pepper and a little curly parsley for color. Season with
salt and pepper at the table as desired.  (One could also add a little
Hellman's or homemade mayonnaise and/or finely chopped celery to the crab
before stuffing.)
\end{entry}

%%---------------------------------------------------------------------------%%
\begin{entry}{Nothing but Crab Cakes}{Second Edition}
\index{Seafood!nothing but crab cakes}
\index{Johnson!Martha}

\begin{open}
  From Martha, this recipe is adapted from the Paoli Auxiliary cookbook
  ``Quilted Cuisine.'' Being able to get real Chesapeake Bay crab occasionally
  means I make these probably way too often. But so easy! Serves 4, so often
  just make half a recipe for the two of us. (NB: Kate decided we should do
  this cookbook crab mostly so we could get our hands on this recipe. Thanks,
  Mom!)
\end{open}
%%
\begin{figure}
    \centering
    \includegraphics[width=4in]{figures/crab_cakes.jpg}
    \caption{Nothing but crab cakes!}
    \label{fig:crab-cakes}
\end{figure}
%%
\begin{ingredients}
    \SI{1}{\pound} crab meat (lump works best)\\
    1 egg\\
    \SI{1}{\tblspoon} chopped parsley\\
    \SI{1}{\tblspoon} mayonnaise\\
    \SI{2}{\teaspoon} Worcestershire sauce\\
    \SI{1}{\tblspoon} melted butter\\
    \SI{1}{\teaspoon} dry mustard \\
    pinch salt\\
    ground pepper\\
    big slurp of Tabasco\\
    onion powder
\end{ingredients}
Combine all these ingredients and mix well. Probably be quite moist. Shape
into patties (I like small ones for ease of turning). Coat with Panko or
regular bread crumbs mixed with a little Parmesan. Chill until ready to
cook. \Saute in medium hot olive oil until brown on both sides.  Serve on plate with lemon and parsley leaves (Fig.~\ref{fig:crab-cakes}.)
\end{entry}

%%---------------------------------------------------------------------------%%
\begin{entry}{``Manhattan'' Pancakes}{Second Edition}
\index{Breakfast!manhattan pancakes}
\index{Johnson!Liz}
\index{Johnson!Don}

\begin{open}
  Since this was sent to us from Liz Johnson and she didn't give us a
  title, we decided to call them ``Manhattan'' pancakes. She says, ``we use
  this recipe for breakfast almost every Saturday morning, it is always very
  good. Dad (Don Johnson) found it on \url{allrecipes.com}.''
\end{open}
%%
\begin{ingredients}
    \SI{1/2}{\cup} of flour\\
    \SI{1}{\tblspoon} of sugar\\
    1\SI{1/2}{\teaspoon} of baking powder\\
    \SI{1}{\teaspoon} of salt\\
    \SI{1/4}{\cup} of milk\\
    1 egg\\
    \SI{3}{\tblspoon} of butter
\end{ingredients}
Put all the ingredients in a bowl, the dry ones first, and mix until no more
lumps. Heat a skillet to medium. ladle some batter into the pan, and flip it
when the surface bubbles. Keep doing that until you run out of batter, then
eat.
\end{entry}

%%---------------------------------------------------------------------------%%
\begin{entry}{Lettuce-Wrapped Fish}{Second Edition}
\index{Seafood!lettuce-wrapped fish}
\index{Horne!Mimi}

\begin{open}
  From Mimi Horne, this recipe yields 4 servings and takes about 30 minutes to prepare. It looks like a good recipes for the young-ins to try. as there are not too many ingredients!
\end{open}
%%
\begin{ingredients}
    Salt and freshly ground black pepper\\
    Several big leaves of romaine lettuce, Bibb lettuce or white cabbage\\
    1\SI{1/2}{\pound} thick white fish fillet (rockfish, cod, hake, snapper), in
    pieces about \SIrange[range-phrase={ to }]{3/4}{1}{\inch} thick,
    \SI{1}{\inch} wide, and \SI{2}{\inch} or less across\\
    \SI{1}{\cup} white wine\\
    \SIrange[range-phrase={ to }]{2}{3}{\tblspoon} butter
\end{ingredients}
Bring a large pot of water to a boil and salt it. Take as many big, intact
leaves of lettuce or cabbage as you have pieces of fish. With large outer
leaves, cut out center veins 2 to 3 inches up from bottom of leaves, to the
point where the leaf is more pliable; with inner leaves this may not be
necessary. One or 2 at a time, blanch leaves in boiling water until they are
tender and flexible, 30 seconds to a minute. Remove and drain on paper towels.

Put a piece of fish on each leaf and sprinkle with salt and pepper; fold or roll
fish in leaf so edges overlap. It is not important to make a tight seal, but it
is nice if package covers all the fish. When done, you can cover and refrigerate
packages until ready to serve, or continue.

In a large, broad skillet or casserole with a cover, bring wine to a boil with
butter. Reduce heat to a simmer and add fish packages. Cover and simmer until a
thin-bladed knife easily penetrates fish, \SIrange{5}{10}{\minute}. Remove fish
to a warm platter.

Over high heat, quickly reduce liquid in skillet; it is likely there will be
more than there was when you started. When it is thickened a bit, pour over fish
and serve.
\end{entry}

%%---------------------------------------------------------------------------%%
\begin{entry}{Bachelor Days Stir Fry}{Second Edition}
\index{Stir-Fry!bachelor days stir fry}
\index{Beef!bachelor days stir fry}
\index{Seafood!bachelor days stir fry}
\index{Chicken!bachelor days stir fry}
\index{Pork!bachelor days stir fry}
\index{Vegetarian!bachelor days stir fry}
\index{Lindquist!David}

\begin{open}
  Here's a recipe from Dave Lindquist, it sounds like it's from days in the
  Trauma Ward. Any vegetables can be omitted or substituted. Leftovers get
  combined in Tupperware or big yogurt container for tomorrow's lunch!
\end{open}
%%
\begin{ingredients}
    Rice (enough for leftovers for lunch)\\
    1 onion\\
    \numrange{2}{5} garlic cloves, depending on preference\\
    ginger root\\
    \numrange{1}{2} carrots\\
    1 handful of snow peas\\
    1 small/medium zucchini or yellow squash\\
    1 red pepper\\
    Protein: shrimp, chicken cutlets, beef strips, or tofu\\
    Spices as desired (salt, pepper, soy sauce, cumin, curry, etc.)\\
    Cooking oil
\end{ingredients}
Heat water for rice. Start low heat to cast iron skillet or wok. Mince garlic
coarsely. Begin chopping vegetables to bite-sized pieces, exact shape is up to
the chef. (If you chop onions last, your crying will interfere less with the
food prep.)

By now water for rice should be boiling. Add rice and simmer as per directions
on container. Usually \SIrange{15}{20}{\minute}. Turn up heat on skillet or
wok.

Finish chopping vegetables and protein. Lightly oil skillet/wok. It should be
smoking gently. \Saute onions, then peppers, then carrots, then squash, then
snow peas. Avoid overloading skillet/wok; it will stay hotter via cooking
smaller batches. Gently season each batch of vegetables as they cook. Combine
each batch of \sauteed vegetables to a large mixing bowl.

\Saute protein. Just before protein is ready, lower the heat, add the garlic,
and grate ginger into the skillet/wok. Add back in all the previously cooked
vegetables. Complete seasoning to taste. A dash of hot pepper will add some
kick.

By now your rice should be ready. Add salt, butter to taste. Fluff.

To serve, I prefer a bed of rice with the veggies and protein on top, drizzled
with soy sauce.
\end{entry}

%%---------------------------------------------------------------------------%%
\begin{entry}{Fish \`{a} la Dave}{Second Edition}
\index{Seafood!fish \`{a} la Dave}
\index{Lindquist!David}

\begin{open}
    Another quick and easy dish featuring soy sauce from Dave Lindquist.
\end{open}
%%
\begin{ingredients}
    Rice (enough for leftovers)\\
    \SIrange{0.5}{2}{\pound} cod, haddock, or other whitefish, depending on
    number of servings desired\\
    1 red pepper\\
    1 yellow pepper\\
    1 onion\\
    \numrange{3}{6} garlic cloves\\
    ginger root (sensing a theme here?)\\
    1 zucchini and/or yellow squash\\
    soy sauce\\
    olive oil
\end{ingredients}
Heat water for rice. Preheat oven to \SIrange{300}{325}{\degreeF} (depends on
how hot your oven runs). Chop all vegetables into bite-sized pieces. Exact shape
is at chef's discretion. Coarsely chop garlic Lay fish in baking pan---use a
large baking dish with cover (or aluminum foil). Sprinkle garlic liberally on
fish Loosely place vegetables around fish; it's ok if they overflow onto the
fish. Drizzle olive oil all over fish and vegetables. Sprinkle with soy sauce.
Sprinkle with freshly ground pepper. Place fish in oven with cover or foil.

By now water for rice should be boiling. Add rice and simmer as per packaging
directions (usually \SIrange{15}{20}{\minute}). Fish should cook for
approximately \SI{20}{\minute}, or until it looks firm and flaky. Remove cover
to speed cooking, if needed.

Serve fish and vegetables over bed of rice. Add soy sauce if desired.
\end{entry}

%%---------------------------------------------------------------------------%%
\begin{entry}{Pasta with Anchovies}{Second Edition}
\index{Pasta!pasta w/ anchovies}
\index{Seafood!pasta w/ anchovies}
\index{Horne Rona!Alison}

\begin{open}
  From Alison Horne Rona, she likes to use two tins of anchovies with enough short
  fusilli pasta for one lunch and one dinner.
\end{open}
%%
\begin{ingredients}
    anchovy filets in olive oil\\
    \numrange{2}{5} garlic cloves (to taste), chopped\\
    chili pepper flakes\\
    olives, chopped\\
    anchovy paste\\
    flat-leaf parsley, chopped
\end{ingredients}
Finely chop the anchovy filets and briefly \saute with the garlic, chili
pepper, olives, and anchovy paste while cooking the pasta.  Drain the pasta
and toss in the anchovy mixture with lots of parsley.
\end{entry}

%%---------------------------------------------------------------------------%%
\begin{entry}{Swedish insanity meatballs}{Second Edition}
\index{Turkey!swedish meatballs}
\index{Pork!swedish meatballs}
\index{Evans!Kate}

\begin{open}
  Kate normally avoids meatballs to avoid eating too much beef (and onions),
  but turkey meatballs just don't have the same zing (a.k.a. fat). So she
  altered a recipe from fine cooking to arrive at these rich---yet
  non-beef---meatballs. Serve with german egg noodles and a sweet side veggie
  like honey carrots or green beans for a complete meal! As with all of Kate's
  recipes, if you see onion powder in the ingredients, feel free to replace
  with \SI{1/2}cup minced onion.
\end{open}
%%
\begin{ingredients}
    3 slides whole wheat bread (approx.)\\
    \SI{1/4}{\cup} milk (whole or 2\%) \\
    \SI{12}{\ounce} ground turkey, 93 or 99\% fat free\\
    \SI{12}{\ounce} pork sausage\\
    \SI{1}{\teaspoon} onion powder\\
    1 large egg, lightly beaten\\
    \SI{1/4}{\teaspoon} ground allspice\\
    \SI{1/4}{\teaspoon} ground nutmeg\\
    \SI{1/2}{\teaspoon} each of Kosher salt and pepper\\
    \SI{2}{\tblspoon} butter\\
    \SI{2}{\tblspoon} olive oil\\
    \SI{1}{\tblspoon} flour\\
    \SI{1}{\cup} lower-salt chicken broth\\
    \SI{1/4}{\cup} heavy cream
\end{ingredients}
In a stand mixer fitted with a paddle attachment (or in a bowl and use your
hands---works great!), soak the bread in the milk until softened, about 5
minutes. Mix on low speed until uniform, about 30 seconds. Add the turkey,
pork, onion powder, egg, allspice, nutmeg, salt and pepper and mix on low
speed until evenly combined. Using your hands (a bowl of water nearby helps
keep them from getting too gummed up), form about 25 \SI{1.5}{\inch} sized
balls. Heat a large skillet over medium-high heat and add \SI{1}{\tblspoon}
each of the butter and oil. As soon as butter is melted, add half the
meatballs, turning very several minutes on several sides until browned, then
remove to a plate on the side. Repeat with the rest of the meatballs. Then
turn skillet heat to medium and add the other \SI{1}{\tblspoon} each of the
butter and oil. Add the flour and whisk it in with the fat until smooth. Whisk
in the chicken broth, then the cream, in small batches. The roux will first
become firmer as liquid is added but then become a thin sauce. Return all the
meatballs to the skillet and reduce the heat slightly, cover, and cook until
meatballs are cooked through and sauce thickens, about
\SIrange{8}{10}{\minute}. Season the sauce as needed with salt and pepper.
\end{entry}

%%---------------------------------------------------------------------------%%
\begin{entry}{Six Cheese Lasagne}{Second Edition}
\index{Pasta!six cheese lasagne}
\index{Beef!six cheese lasagne}
\index{Turkey!six cheese lasagne}
\index{Evans!Kate}

\begin{open}
  This is not a typo. Repeat. This is not a typo. It turns out that adding 6
  different cheeses to a traditional lasagne recipe makes it awesome. This
  recipe was taken from a 1992 Southern living cookbook Kate received as a
  wedding shower gift. I am looking at you, Fermina lunch buddies!
\end{open}

\begin{ingredients}
\numrange{9}{12} lasagne noodles, cooked according to package directions\\
\SI{1/2}{\cup} sharp Cheddar cheese, shredded\\
\SI{1/2}{\cup} Romano cheese, grated\\
\SI{1/2}{\cup} Parmesan cheese, grated \\
\SI{8}{\ounce} Mozzarella cheese, sliced
\end{ingredients}

\minisection{Tomato sauce}
\begin{ingredients}
    \SI{1}{\tblspoon} olive oil\\
    1 clove garlic, minced\\
    \SI{1}{\ounce} ground beef or turkey \\
    \SI{1}{\teaspoon} onion powder\\
    1 \SI{12}{\ounce} can tomato paste\\
    \SI{8}{\ounce} sour cream\\
    1 \SI{16}{\ounce} can tomato sauce\\
    1\SI{1/2}{\cup} water\\
    \SI{1}{\tblspoon} dried basil\\
    \SI{1}{\teaspoon} salt\\
    \SI{1/2}{\teaspoon} dried rosemary\\
    2 bay leaves
\end{ingredients}

\minisection{Cheese layer}
\begin{ingredients}
    2 large eggs, lightly beaten\\
    \SI{2}{\cup} ricotta cheese\\
    \SI{8}{\ounce} sour cream \\
    \SI{1/4}{\cup} chopped parsley, or \SI{2}{\tblspoon} dried\\
    \SI{1/2}{\teaspoon} salt\\
    \SI{1/4}{\teaspoon} pepper
\end{ingredients}
%
\protip{An acceptable weeknight cheat is to use \SI{1}{\cup} of the green
  cylinder grated combo Parmesan and Romano cheese instead of grating your own
  separately.}
%
Heat oil over medium heat in large skillet or saucepan. Add garlic and
stir-fry briefly until browned, then add ground meat and \saute until
cooked. Add tomato paste, tomato sauce, water, basil, salt, rosemary, and bay
leaves and stir gently. Bring to a boil, then reduce heat and cover and cook
until its time to use in the lasagna, anytime more than 15 minutes or so. In a
separate and medium sized bowl, add the cheese layer ingredients and stir to
combine. Preheat oven to \SI{375}{\degreeF}. Next, arrange 3-4 lasagna noodles
(depending if you have 9 or 12) onto bottom of greased \SI{9x13}{\inch} glass
baking pan. Layer with \SI{1/3} of the tomato sauce, \SI{1/3} or the cheese
layer, then \SI{1/3} each of the Cheddar, Romano, and Parmesan cheeses. Repeat
two more times. Place on middle shelf of oven and bake until bubbly, about
\SIrange{30}{35}{minutes}. Then take out and arrange the Mozzarella slices
along the top and return to oven until Mozzarella is melted and starting to
bubble and brown, about \SIrange{5}{10}{minutes}.

\begin{figure}
  \centering
  \includegraphics[height=0.9\textheight]{figures/PastaCortaMNHleCookbook.jpg}
  \caption{Although these pastas are not used in the lasagne they inspire us to cook italian. Credit: Mimi Horne, and first appeared in ``Le Cookbook.''}
  \label{fig:mimi_pasta}
\end{figure}

\end{entry}

%%---------------------------------------------------------------------------%%
\begin{entry}{Whole beef tenderloin}{Second Edition}
\index{Beef!whole tenderloin}
\index{Evans!Fermina}

\begin{open}
  Every home cook should have this dish in their arsenal for when you have
  company and want to look like you are a chef. Its deceptively easy (but plan
  ahead time wise!) and makes a great presentation piece.
\end{open}

\begin{ingredients}
    \SI{1/2}{\cup} fresh parsley or mint, chopped\\
    \SI{1}{\cup} red wine\\
    \SI{1/2}{\cup} soy sauce\\
    \SI{1/4}{\cup} Worcestershire sauce\\
    \SI{2}{\tblspoon} fresh rosemary or \SI{2}{\teaspoon} dry\\
    4 cloves garlic, chopped\\
    black pepper
\end{ingredients}
%
\protip{Fermina notes that it's cheaper per pound to get a whole one and trim
  it, but if you are new to working with tenderloin, its perhaps safer to get
  one that is already trimmed.}
%
The day before, combine all ingredients but beef in a small bowl and add with
beef into a plastic bag to marinate the beef overnight in the
refrigerator. Remove from fridge, place beef on roasting dish, discard
marinade, and keep out at room temp at least two hours before cooking. Preheat
oven to \SI{450}{\degreeF}. Roast for 15 minutes at \SI{450}{\degreeF}, then
reduce temperature to 350 and roast for another 15 minutes. Remove from oven
and double wrap in heavy foil before serving.
%
\protip{For a rare center, the rule of thumb is that the meat thermometer in
  the center registers \SI{450}{\degreeF}.}
\end{entry}

%%---------------------------------------------------------------------------%%
\begin{entry}{Braised Pork in Soy Sauce, ``Pork Candy''}{Second Edition}
\index{Liu!Geyi}
\index{Pork!braised pork in soy sauce}

\begin{open}
  From Geyi, who writes:

  This is a common household dish in China, it is in every child's memory as one of their favorites of their mom's. Every family has its own way of making it. In general, there are regional differences in the ingredients: families in the south like to use carameled soy sauces for a dark color, Northern families do
  not think much of the color, and only use rock crystal sugar for the caramel effect.

  I remember my mom’s, she poaches the pork first and does not put other things
  with her meat. My aunt’s family in Beijing put tofu in the pot, which soaked
  up the meat juice and flavors and tended to be even more sought-after than the
  meat. Our ayi (nanny) in Shanghai added boiled eggs to it to the same effect.

  I cook it sometimes this way, and sometimes that way. They all come through well. Some say it tastes better the day after, but we never had the chance to test the theory---it is always gone the first day. My current standard is a variation from ``Miss Vegetable's gourmet cooking diary'' a WeChat channel that uses black tea to cut through the grease. ``Miss Vegetable'' started cooking for herself and friends, then got so popular that she started her own company with staff and everything.

  \begin{CJK*}{UTF8}{gbsn}红烧肉\end{CJK*} is the name of the dish. Literally translated it means ``red roast pork;'' a more common translation is Braised Pork in Soy Sauce---neither captures the image, experience or culture. I cooked this dish for Martha's birthday 2019, and it is she who was able to give this dish a proper name---Pork Candy!
\end{open}
%%
\begin{ingredients}
  \SI{2}{\pound} pork belly with skin (Fig.~\ref{fig:pork-candy-ingredients})\\
  Rock Crystal sugar\\
  \numrange{3}{5} pieces of sliced ginger\\
  \SI{3}{\tblspoon} Chinese cooking wine (or sherry)\\
  \SI{3}{\tblspoon} soy sauce (\begin{CJK*}{UTF8}{gbsn}生抽\end{CJK*})\\
  \SI{2}{\tblspoon} dark soy sauce (regular soy sauce with an additional
  caramelization step. Used often for cooking meat to give it a darker color and
  maybe more complex flavor.)\\
  black tea
\end{ingredients}
%%
Before starting cut the pork belly into \SI{1}{\inch} cubes and make a large pot
of hot tea.  The cooking instructions are:
\begin{enumerate}
    \item  Heat a pan with lid on low. Add the meat without adding oil. Cook the
    meat slowly on low until the oil seeps out, then turn the pieces one by one
    to cook the other side (Fig.~\ref{fig:pork-candy-prep}a).
    \item When the surfaces are all slightly browned and the oil has leaked out,
    add rock sugar and stir fry until it is completely dissolved
    (Fig.~\ref{fig:pork-candy-prep}b).
    \item Add ginger slices, bay leaves, dried chili, then pour rice wine, dark
    soy sauce and light soy sauce, and stir fry evenly
    (Fig.~\ref{fig:pork-candy-prep}c).
    \item Pour the hot tea into the pot, make sure there is enough to cover the
    meat. Bring the liquid to a boil, then simmer for one hour.
    \item Uncover, turn heat to high, stir till the liquid caramelizes, coats
    the meat cubes and shines (Fig.~\ref{fig:pork-candy-prep}d).
\end{enumerate}
%%
\begin{figure}
  \centering
  \includegraphics[width=0.5\textwidth]{figures/pork-candy-ingredients.png}
  \caption{Pork candy ingredients.}
  \label{fig:pork-candy-ingredients}
\end{figure}
%%
\begin{figure}
  \centering
  \includegraphics[height=0.9\textheight]{figures/pork-candy.pdf}
  \caption{Preparing Pork Candy!}
  \label{fig:pork-candy-prep}
\end{figure}

\end{entry}

%%---------------------------------------------------------------------------%%
\begin{entry}{Herb Roasted Pork Tenderloin}{Second Edition}
\index{Cordova!Betsy}
\index{Cordova!Rich}
\index{Pork!herb roasted pork tenderloin}

\begin{open}
  From Betsy, who says ``Fun Fact:  This is the first meal I made when Rich and I had company as a married couple!!!''
\end{open}
%%
\begin{ingredients}
  1 package pork tenderloin (about 1\SI{1/2}{\pound} per tenderloin)
\end{ingredients}
\minisection{Marinade}
\begin{ingredients}
  \SI{1/2}{\cup} soy sauce\\
  \SI{1/4}{\cup} vegetable oil\\
  \SI{1/4}{\cup} Worcestershire sauce\\
  \SI{1}{\teaspoon} rubbed sage\\
  \SI{1}{\teaspoon} onion powder\\
  \SI{1}{\teaspoon} salt\\
  \SI{1}{\teaspoon} dried marjoram\\
  \SI{1}{\teaspoon} pepper\\
  \SI{1}{\teaspoon} garlic powder\\
  \SI{1}{\teaspoon} dried ginger\\
  \SI{1}{\teaspoon} dried thyme
\end{ingredients}
%%
Mix together all the ingredients (I use a Pampered Chef salad dressing mixer) to
make the marinade. Pierce both pieces of pork with a fork and put in shallow
dish or heavy duty zip lock bag. Add marinade making sure to coat the pork.

Cover dish or close bag and allow pork to stand at room temperature for
\SI{30}{\minute}. Cook on baking dish or rack for \SI{30}{\minute} at
\SI{350}{\degreeF} or until pork registers \SI{160}{\degreeF}.

\end{entry}

\begin{entry}{Oven Tacos}{Second Edition}
\index{Pross!Katie}
\index{Tacos!oven baked}
\index{Beef!oven baked tacos}

\begin{open}
These are finished in the oven so they are less messy and easier to eat than if assembled at the end. Make sure to drain the beef well so it doesn't make the shells soggy.
\end{open}
%%
\begin{ingredients}
  \SI{2}{\pound} ground beef/turkey/chicken/whatever\\
  \SI{1} small onion, diced\\
  \SI{1} small can diced green chiles\\
  \SI{8}{\ounce} can low sodium tomato sauce \\
  \SI{1}{\pound} can fat free refried beans \\
  \SI{2}{\cup} shredded reduced fat Colby-Jack cheese\\
  \numrange{18}{20} hard taco shells\\
  optional condiments, including shredded lettuce, tomato, salsa, guacamole, sour cream, olives, etc.
\end{ingredients}
\minisection{Taco Seasoning}
\begin{ingredients}
  \SI{1}{\tblspoon} chili powder\\
  \SI{1/4}{\teaspoon} onion powder\\
  \SI{1/4}{\teaspoon} garlic powder\\
  \SI{1/4}{\teaspoon} crushed red pepper flakes\\
  \SI{1/4}{\teaspoon} dried oregano\\
  \SI{1/2}{\teaspoon} paprika\\
  \SIrange{1}{1/2}{\teaspoon} ground cumin\\
  \SI{1}{\teaspoon} salt\\
\SI{1}{\teaspoon} pepper\\
\end{ingredients}

%%
Preheat oven to \SI{400}{\degreeF}. In a large skillet, brown ground beef and onion over medium high heat. Drain off any excess liquid. Return to pan, add chiles, refried beans, and tomato sauce. Combine taco seasoning ingredients and add to beef mixture. Mix well and cook for a few minutes if mixture seems runny.

Spoon the meat mixture into the taco shells and place into a \num{9x13} inch baking dish, standing up. Sprinkle cheese over the top of the meat mixture in each shell. Place into the oven and bake at \SI{400}{\degreeF} for \numrange{10}{12} minutes or until the cheese has melted and the tacos are cooked through. Remove from the oven and top with any optional condiments for serving.

\end{entry}

\begin{entry}{Ratatouille}{Second Edition}
\index{Horne!Mimi}
\index{Vegetarian!Ratatouille}
\label{sec:lecookbook}

\begin{open}
Given that Ratatouille is a classic French dish, we tell the story here from Paul Horne about Mimi's efforts on ``Le Cookbook,'' from where she first published this recipe. She illustrated the cover, a drawing of her and Paul's Paris apartment kitchen shown in figure \ref{fig:lecookbook}, as well as many of the drawings throughout the book.

\begin{figure}
  \centering
  \includegraphics[height=0.9\textheight]{figures/LeCookbookCover23nov20.jpg}
  \caption{Le Cookbook cover, credit: Mimi Horne}
  \label{fig:lecookbook}
\end{figure}

Paul sent a wonderful summary of Mimi's volunteer work that involved this cookbook, so we include some of his notes here:

Mimi was a volunteer at the American Hospital of Paris from the early 1980s until we moved to London in 1998. The volunteers
provide services in what is one of Europe's best and chic-est hospitals. Founded in 1906, AHP played a crucial role in the world wars and is today a high-tech healthcare center in France. One of the volunteers' most successful projects was "Le Cookbook", a bilingual volume with recipes in English and American measures on the right hand page and in French with French measures on the left page. First published in 1976, Le Cookbook was a great success, contributing financially to AHP. By the early 1990s all copies had been sold so Mimi volunteered to lead a group of volunteers to do a second edition. This proved to be an intensive job of culling, updating and adding recipes and ideas (editor's note: don't we know it!!). But the second edition of "Le Cookbook" was published in 1998, not long after we got to London, and has been a success although sales are now limited to the hospital itself since the French did not want it to "compete" with French cookbooks !

With that, here is the showcase of the recipes we received from ``Le cookbook.'' This dish is best prepared ahead so the flavors have time to blend.

\end{open}
%%
\begin{ingredients}
  5 tomatoes (or a large can), cut in slices\\
  2 eggplants, sliced, not peeled \\
  2 zucchini, sliced \\
  1 red pepper, cut in strips \\
  1 green pepper, cut in strips \\
  4 small onions, chopped \\
  2 cloves garlic, crushed \\
  Tabasco sauce \\
  \SI{1}{\tblspoon} \emph{herbes de provence} \\
  5 bay leaves \\
  salt and pepper \\
  \SI{4}{\tblspoon} olive oil \\

\end{ingredients}

%%
Salt eggplant slices and place a weight on top to make them disgorge their liquid. After an hour, rinse, dry and fry them in olive oil a few at a time, leaving them to dry on paper towels. In a small amount of oil, cook the oinions, garlic, peppers, and zucchini, putting aside each vegetable when it is softened. In a large oven dish make layers of all ingredients, adding to each layer tomato slices, salt, pepper, dash of Tabasco, bay leaf, and \emph{herbes de provence}. Bake uncovered at \SI{350}{\degreeF} for 45 minutes. May be served hot or tepid.

\begin{figure}
  \centering
  \includegraphics[width=0.6\textwidth]{figures/GrilledEggPlantMNHleCookbook.jpg}
  \caption{}
  \label{fig:mimi_eggplant}
\end{figure}

\begin{figure}
  \centering
  \includegraphics[width=0.6\textwidth]{figures/GarlicPressMNHleCookbook.jpg}
  \caption{Illustration by Mimi Horne of grilling the eggplant for Ratatouille.}
  \label{fig:mimi_garlic}
\end{figure}

\end{entry}

\begin{entry}{Beef with Snow Peas}{Second Edition}
\index{Pross!Katie}
\index{Beef!beef with snow peas}
\index{Stir-Fry!beef with snow peas}

\begin{open}
From Katie Pross, here is a delicious quick-and-easy Asian-inspired dinner ready in 20 minutes. Serves 8.
\end{open}
%%
\begin{ingredients}
  \SIrange{1}{1/2}{\pound} flank steak, trimmed of fat and sliced very thin against the grain\\
  \SI{1/2} cup low sodium soy sauce\\
  \SI{3}{\tblspoon} cooking or drinking sherry\\
  \SI{2}{\tblspoon} brown sugar \\
  \SI{2}{\tblspoon} cornstarch \\
  \SI{2}{\teaspoon} minced fresh ginger\\
  \SI{8}{\ounce} snow peas, ends trimmed \\
   5 whole scallions, cut into half-inch pieces on the diagonal \\
  salt as needed \\
    \SI{3}{\tblspoon} peanut or olive oil \\
  crushed red pepper for sprinkling \\

\end{ingredients}

%%
In a bowl, mix together the soy sauce, sherry, brown sugar, cornstarch, and ginger. Pour half the liquid over the sliced meat in a bowl and toss. Reserve the other half of the liquid. Set aside.

Heat oil in a heavy skillet (recommend iron) or wok over high heat. Add snow peas and stir for 45 seconds. Remove to a separate plate and set aside.

Allow pan to get very hot again. With tongs, add half the meat mixture, leaving most of the marinade still in the bowl. Add half the scallions. Spread out meat as you add it to the pan but do not stir it for a good minute. Turn the meat to the other side and cook for another 30 seconds and remove to a clean plate. Repeat with the other half of the meat after pan is hot again. Then add first half of meat back to the pan with the second, the rest of the marinade, and the snow peas. Check to see if it needs more salt. Mixture will thicken as it sits.

Serve immediately with cooked jasmine or other long-grained rice. Sprinkle with crushed flakes as desired.

\end{entry}
%%---------------------------------------------------------------------------%%
%% From the first edition

%%---------------------------------------------------------------------------%%
\begin{entry}{Scott's Killer Chili}{First Edition}
\index{Beef!chili}
\index{Evans!Scott}

\begin{open}
  I hope you're prepared for this.  This recipe from Scott Evans makes a {\em
  thick} and {\em spicy} chili.  In the word's of the author ``It is pretty
  spicy.''  This is one of those recipes that should include a Disney-style
  warning label, ``\textellipsis those with heart conditions or over the age of
  sixty-five...etc. etc.''  This is this recipe's first time in print so some
  experimentation may be required.  Supposedly this is a campout recipe, however
  I see no way that anyone could possibly carry all these ingredients. The
  recipe makes about 16 servings or less for REALLY big people.  Good luck and
  here it goes.
\end{open}
\begin{ingredients}
  \SI{3}{\pound} of hot Italian sausage (ie. Hot Cincinnati Brand, that homer) \\
  \SI{3}{\pound} bacon \\
  3 large onions \\
  3 bell peppers (2 green, 1 red) \\
  \numrange{4}{5} cloves of garlic \\
  \numrange{4}{5} hot peppers (a cornucopia of jalape\~{n}os, habaneros, and
  others) \\
  3 cans Italian pear tomatoes \\
  \SI{1}{\tblspoon} olive oil \\
  \SI{1}{\tblspoon} mustard powder \\
  \SI{1}{\tblspoon} celery seed \\
  \SI{1}{\tblspoon} chili powder \\
  \SI{1}{\tblspoon} bay leaves \\
  \SI{1}{\tblspoon} Worcester sauce \\
  \SI{1}{\tblspoon} vinegar \\
  red wine \\
  water \\
  salt and pepper
\end{ingredients}
Start with a large Dutch oven and a campfire right after breakfast.  Fry the
sausage and set aside.  Fry the bacon and set aside.  Leave a bit of grease in
the pot and add the minced garlic followed by the roughly chopped onions and
bell peppers (no bell pepper seeds).  Chop the hot peppers and add to the pot,
remember that the seeds make the dish VERY spicy.  Add olive oil as needed.
%%
\begin{wrapfigure}{R}{.2\textwidth}
\centering
\includegraphics[width=.15\textwidth, trim=.5in .25in .5in .25in, clip]{figures/chilli.pdf}
\end{wrapfigure}
%%
Add about one Tbsp full each of: mustard powder, celery seed, Worcester sauce,
vinegar, and chili powder.  This may require some experimentation to alter to
your taste.  Stir and cook until onions become clear and peppers begin to
soften.  Add up to one cup of red wine.  Next add tomatoes and juices.  Stir
and chop tomatoes.  Add sausage, bacon, and two bay leaves.  Season with salt
and pepper.  Now everything should look a bit like chunky soup, but don't
worry.

Let the chili simmer over low heat for a minimum of three hours, but try for
eight (trust Scott on this one). Check periodically and stir.  If mixture
thickens too much add some water.  Taste and adjust to preference.

Serve with shredded cheddar cheese and garlic bread.  This recipe freezes well
in personalized zip-lock bags (In case you're not hungry enough to eat six
pounds of meat in one sitting).
\end{entry}

%%---------------------------------------------------------------------------%%
\begin{entry}{Chicken Breasts with Orange Sauce}{First Edition}
\index{Chicken!chicken with orange sauce}
\index{Johnson!Lil}

\begin{open}
  This is a recipe from Lil and Don Johnson Sr.  Grammie (Lil) passed it to Martha, who passed it to Kate, and so on, and so on, and so on\textellipsis
  It's a good recipe for new cooks.
\end{open}
\begin{ingredients}
  4 halved chicken breasts \\
  1 small can undiluted O.J. concentrate \\
  1 package Lipton's Onion Soup Mix \\
  paprika
\end{ingredients}
In a long baking pan arrange the 8 pieces of chicken.  Pour the O.J.
concentrate (at room temperature) over the chicken.  Sprinkle the soup mix
over the chicken.  Add a little paprika for seasoning.

Cover pan with foil.  Bake at \SI{350}{\degree} for 40 minutes.  Remove foil and
baste chicken.  Back for an additional 20 minutes uncovered.
\end{entry}

%%---------------------------------------------------------------------------%%
\begin{entry}{Porcupine Meatballs}{First Edition}
\index{Beef!meatballs}
\index{Evans!Mickey}
\index{Evans!George Sr.}

\begin{open}
  This is a recipe from Mickey and George, Sr. Evans.
\end{open}
\begin{ingredients}
  \SI{1/2}[1]{\pound} hamburger \\
  \SI{3/4}{\cup} uncooked rice \\
  \SI{1}{\teaspoon} salt \\
  1 egg \\
  \SI{1/2}{\teaspoon} pepper \\
  \SI{1/4}{\cup} chopped onion \\
  \SI{1/2}[2]{\cup} stewed tomatoes \\
  \SI{1}{\teaspoon} chili \\
  \SI{1}{\teaspoon} sugar
\end{ingredients}
Combine hamburger, rice, salt, egg, pepper, and onion.  Shape into
\SI{1/2}[1]{\inch} balls.

Heat sauce and chili to boiling in a kettle.  Drop balls in sauce.  Simmer for
1\num{1/2} hour covered.  Strips of bacon may be wrapped around meatballs
and secured with toothpicks before cooking in sauce.
\end{entry}

%%---------------------------------------------------------------------------%%
\begin{entry}{Lemon-Herb Chicken}{First Edition}
\index{Chicken!lemon-herb chicken}
\index{Evans!Fermina}

\begin{open}
  This is Fermina's favorite.
\end{open}
\begin{ingredients}
  1 chicken (cut) or \SI{1/2}[3]{\pound} of chicken parts      \\
  \SI{1/2}{\cup} olive oil                                     \\
  \SI{1/4}{\cup} lemon juice                                   \\
  2 garlic cloves minced                                       \\
  \SI{3}{\tblspoon} chopped fresh oregano or \SI{1}{\tblspoon} dry \\
  \SI{1/2}{\teaspoon} salt                                     \\
  \SI{1/8}{\teaspoon} pepper                                   \\
  \SI{1}{\tblspoon}chopped fresh rosemary or \SI{1}{\teaspoon} dry \\
  \SI{2}{\tblspoon}chopped fresh parsley
\end{ingredients}
Place chicken meaty side down in a \SI{13x9x2}{\inch} baking pan.  Combine
next 8 ingredients, mix well.  Pour mixture over chicken.  Marinate in
refrigerator for two hours.  Bake uncovered at \SI{350}{\degree} for 40 minutes.
Turn chicken.  Broil 6~inches from heat for \numrange{5}{10} minutes or
until crisp and lightly browned.
\end{entry}

%%---------------------------------------------------------------------------%%
\begin{entry}{Stuffed Flank Steak Teriyaki}{First Edition}
\index{Beef!flank steak teriyaki}
\index{Evans!Tom}
\index{Evans!Fermina}

\begin{open}
  This belongs because it is MY favorite and since I'm (Tom) in charge,
  well there you go.  Makes \numrange{4}{5} servings.
\end{open}
\begin{ingredients}
  1 medium to large beef flank steak (1\SIrange{1/4}{2}{\pound}) \\
  \SI{1/2}{\cup} soy sauce                                    \\
  \SI{1/4}{\cup} cooking oil                                  \\
  \SI{2}{\tblspoon} molasses                                   \\
  \SI{2}{\teaspoon} dry mustard                               \\
  \SI{1}{\teaspoon} ginger root or \SI{1/2}{\teaspoon} dry ginger \\
  1 clove garlic minced                                       \\
  \SI{1}{\cup} water                                          \\
  \SI{1/2}{\cup} long-grain rice                              \\
  \SI{1/2}{\cup} of shredded carrots                          \\
  \SI{1/2}{\cup} sliced water chestnuts (optional)            \\
  \SI{1/4}{\cup} sliced green onions
\end{ingredients}
\begin{wrapfigure}{L}{.3\textwidth}
\centering\includegraphics[width=.25\textwidth,clip]{figures/flank.pdf}
\end{wrapfigure}
Cut a large pocket in flank steak or have your butcher do it.  Combine soy
sauce, oil, molasses, mustard, ginger, and garlic.  Place meat in shallow pan
or plastic bag.  Pour marinade into pocket and over meat. Let stand at room
temp (\SI{300}{\kelvin}) for 30 minutes or in refrigerator for
\numrange{2}{3} hours.

In saucepan combine water, rice, carrots, water chestnuts, and green onion.
Bring to a boil; reduce heat and simmer while covered for 8 minutes.  Remove
from heat and set aside.

Drain meat reserving marinade.  Add \SI{1/4}{\cup} of reserved marinade to the
rice mixture.  Spoon rice stuffing into pocket of meat.  Secure end with
wooden toothpicks.  Place meat in shallow roasting pan and cover with foil.
Bake at \SI{350}{\degree} for 1 hour until meat is done.

Fermina allows an extra \numrange{10}{15} minutes without foil to brown meat.
Brush with marinade while browning and check often.  Slice meat diagonally
across grain to serve.
\end{entry}

%%---------------------------------------------------------------------------%%
\begin{entry}{Better-than-vegetarian Pasta Sauce}{First Edition}
\index{Pasta!better-than-veggie pasta sauce}
\index{Vegetarian!better-than-veggie pasta sauce}
\index{Johnson!Don}

\begin{open}
  This from Kate's brother Don, and he got it from his friend Dawn Ollila. The honey/brown sugar and cinnamon addition is what makes it taste so special.
\end{open}
\begin{ingredients}
  \SI{1}{\tblspoon}olive oil \\
  1 onion \\
  1 green pepper \\
  2 cloves of minced garlic \\
  1 tomato\\
  1 large or two small carrots OR \SI{1/4}{\cup} cup lentil beans \\
  1 or 2 cans of tomato sauce \\
  thyme \\
  basil \\
  oregano \\
  rosemary \\
  \SI{1}{\tblspoon} honey or brown sugar \\
  cinnamon \\
  salt and pepper
\end{ingredients}
\Saute the 4 vegetables in olive oil, adding them as ordered above. When the
onion is clear and the tomato is soft, add the tomato sauce.  Bring to a
simmer. The sauce is now tasty, but to thin to stick to the pasta.  Choose
either the carrots or lentils to give it body. Lentils add a great dark,
almost meaty, flavor but you will need to boil them in 4 times their
measurement in boiling water for \numrange{45}{90} minutes first (no
presoaking required). Do not add them to the sauce until they are bean-like
mush. or, you can grate the carrot as finely as you have the technology to do
and add to the sauce. The taste is minimal, but the texture is great. Add the
rest of the ingredients and adjust to taste. Add just enough cinnamon to make
your guests look at you funny and say, ``What did you put in this?'' The
flavor actually works quite well.
\end{entry}

%%---------------------------------------------------------------------------%%
\begin{entry}{Peachy Chicken}{First Edition}
\index{Chicken!peachy chicken}
\index{Evans!Fermina}
\index{Evans!Betsy}
\index{Evans!Tom}

\begin{open}
  From Fermi, this is Betsy's favorite.  Tom: Actually I really don't
  like this recipe and the only reason she ``likes'' this one is because she
  hates my favorite recipe (stuffed flank steak).  Note: the ``peachy'' in
  ``peachy chicken'' is not a southern thing.  Kate: When Tom and I were first
  married (Tom interjects: forty years ago) I made a chicken dish with fruit.
  He honestly thought I was trying to annoy him (how did he know?).
\end{open}
\begin{ingredients}
  chicken parts for \numrange{4}{6} people\\
  1 large can peach halves (drained, reserve syrup)\\
  \SI{2}{\tblspoon} soy sauce\\
  \SI{2}{\tblspoon} lemon juice\\
  \SI{1/2}{\tblspoon} ginger\\
  2 cloves minced garlic
\end{ingredients}
Preheat oven to \SI{375}{\degree}.  Mix last four ingredients with reserved peach
syrup.  Pour marinade over chicken parts in roasting or baking pan. Bake in
oven for \numrange{45}{60} minutes. Turn once.  Add peach halves last
15 minutes.  Brown under broiler for a few minutes if further browning
is needed.
\begin{center}
    \includegraphics[scale=.5,clip]{figures/meatloaf.pdf}
\end{center}
\end{entry}

%%---------------------------------------------------------------------------%%
\begin{entry}{Swiss Meatloaf}{First Edition}
\index{Beef!swiss meatloaf}
\label{sec:swiss-meatloaf}
\index{Evans!Joyce}

\begin{open}
  This was contributed by Joyce Evans, and is definitely ``comfort
  food''! Serves 6.
\end{open}
\begin{ingredients}
  1 egg \\
  \SI{1/2}{\cup} evaporated milk \\
  \SI{1}{\teaspoon} rubbed sage \\
  \SI{1}{\teaspoon} salt \\
  \SI{1/2}{\teaspoon} black pepper \\
  \SI{1/2}[1]{\pound} lean ground beef \\
  \SI{1}{\cup} cracker crumbs (round buttery type, approx. 24) \\
  \SI{3/4}{\cup} grated Swiss cheese \\
  \SI{1/4}{\cup} finely chopped onion \\
  \numrange{2}{3} strips bacon, cut into \SI{1}{\inch} pieces
\end{ingredients}
Preheat oven to \SI{350}{\degree}. Beat the egg in a large bowl. Add evaporated milk,
sage, salt, and pepper.  Mix together. Add beef, crumbs, \SI{1/2}{\cup} of the
cheese and the onion. Blend. Form into a loaf and place in a \SI{2}{\quart}
rectangular dish.  Arrange bacon pieces on top of the loaf, and bake for
40 minutes. Sprinkle remaining \SI{1/4}{\cup} cheese on top and bake
40 minutes longer.
\end{entry}

%%---------------------------------------------------------------------------%%
\begin{entry}{Devilled Crabs}{First Edition}
\index{Seafood!devilled crabs}
\index{Johnson!Martha}
\index{Johnson!Dodge}
\index{Lamb!Martha}

\begin{open}
  This is from Dodge and Martha Johnson in memory of a great southern cook, Martha Hodgkins Niepold Lamb (Kate's Grandmother)and her mother, Mrs. Henry Bell Hodgkins. Funny story, this was actually submitted for this cookbook as well by Martha's sister Mimi, so you know its a family favorite! Spice amounts can be adjusted to taste.
\end{open}
\begin{ingredients}
  1 stick butter\\
  2 eggs, beaten\\
  \SI{2}{\tblspoon} flour\\
  \SI{1/2}{\teaspoon} Worcestershire sauce\\
  \SI{1}{\teaspoon} dry mustard\\
  \SI{1}{\tblspoon} vinegar\\
  pinch of sugar, salt, pepper, and MSG\\
  \SI{1}{\pound} crab\\
  bread crumbs \\
  paprika (optional) \\ 
\end{ingredients}
Preheat oven to \SI{350}{\degree}.  Melt butter. Mix eggs with flour and a little water and add
to butter. Continue mixing and add everything but paprika. Divide and stuff in scallop shells or
small casseroles and dot with butter and bread crumbs and if desired, sprinkle with paprika. Bake for 1 hour.
\begin{center}
    \includegraphics[width=.3\textwidth,clip]{figures/crab.pdf}
\end{center}
\end{entry}

%%---------------------------------------------------------------------------%%
\begin{entry}{Orange Game Hens}{First Edition}
\index{Chicken!orange game hens}
%\index{Nicholsons}
\index{Johnson!Martha}
\index{Johnson!Dodge}

\begin{open}
  This is a recipe courtesy of Martha and Dodge Johnson's friends, the Nicolsons. It's a great dish for company-especially at their house because they are such nice people and good cooks!
\end{open}
\begin{ingredients}
  2 or more Cornish game hens, whole or halved\\
  Joyce Chen's orange Szechuan sauce (or sub in soy sauce with orange
  concentrate)\\
  \SIrange{2}{3}{\tblspoon} of orange concentrate\\
  several \si{\tblspoon} white wine\\
  garlic powder\\
  ground ginger (fresh or frozen root is best)
\end{ingredients}
Preheat oven to \SI{350}{\degreeF}. Pour sauce, concentrate, and wine over
hens. Sprinkle with garlic powder and ginger. Cover and bake for
1 hour, and uncover the last 10 minutes. Good over a bed of rice.
\end{entry}

%%---------------------------------------------------------------------------%%
\begin{entry}{Chapel Hill Chicken Pie}{First Edition}
\index{Chicken!Chapel Hill chicken pie}
\index{Beef!Chapel Hill chicken pie}
\index{Evans!Kate}
\index{Johnson!Martha}
\index{Johnson!Dodge}

\begin{open}
  This is from Martha and Dodge, and Kate. Kate's addition is only the
  rosemary and measured amounts, for convenience (and my subtractions are
  those nasty onions) No one cares what or how much you put in, as long as you
  are happy. This is one of those recipes that you put some in, then you take
  some out (Nana, does this sound familiar?) This is also the kind of recipe
  where you vary it based on what you like or what's sitting in the fridge!
  It's best when you have leftover gravy along with meat from a past meal.
\end{open}
\begin{ingredients}
  \SI{2}{\cup} chopped meat (roast lamb, beef, or chicken)\\
  \numrange{3}{4} cubed and peeled potatoes\\
  \SIrange{1.5}{2}{\cup} gravy or combination of stock and wine\\
  \SI{2}{\tblspoon} flour\\
  \SI{1}{\teaspoon} salt\\
  \SI{2}{\teaspoon} pepper\\
  \SI{1}{\tblspoon} dried parsley\\
  \SI{1}{\teaspoon} garlic powder or 1 garlic clove\\
  \SI{1}{\teaspoon} thyme (if using chicken or beef)\\
  \SI{1}{\tblspoon} fresh rosemary \\
  \SI{1}{\teaspoon} marjoram (if using lamb)\\
  \SI{1}{\teaspoon} tarragon (if using chicken)\\
  Pie Crust (\corp{Betty Crocker's} mix is good, sorry
  \corp{Duncan Hines}, you don't make one!)
\end{ingredients}
\SIrange{1}{2}{\cup} each of your favorite vegetables, such as
\begin{ingredients}
  chopped carrots\\
  green beans\\
  celery\\
  onion\\
  mushrooms\\
  peas
\end{ingredients}
Preheat oven to \SI{450}{\degreeF}. Boil potatoes until somewhat cooked through, about
15 minutes. In a flat-ish casserole (\SIrange{1.5}{2}{\quart}), layer meat,
vegetables, and potatoes. sprinkle spices over, and then flour. Add gravy
mixture; adjust so the liquid comes up about half the height of the
ingredients.  Top with crust, seal edges, and add fork holes or vents. Brown
for 15 minutes, then then lower temperature to \SI{350}{\degree} and bake for
45 minutes longer. I find that I have to cover it for the last
15 minutes or so to keep the crust from getting too brown.
\end{entry}

%%---------------------------------------------------------------------------%%
\begin{entry}{Pasta with Prosciutto}{First Edition}
\index{Pasta!pasta w/ prosciutto}
\index{Johnson!Martha}
\index{Johnson!Dodge}

\begin{open}
  Martha and Dodge originally got this from the New York Times, but it has
  evolved. It's rather quick and satisfying. They say the order of tasks is a
  little tricky for non-Italian cooks. Luckily, half the family need not worry.
\end{open}
\begin{ingredients}
  \SI{3}{\cup} chopped plum tomatoes\\
  \numrange{2}{3} thinly sliced small zucchini\\
  \SIrange{1/8}{1/4}{\pound} prosciutto, cut into strips\\
  \SI{1}{\teaspoon} salt\\
  \SI{2}{\teaspoon} pepper\\
  \SI{1/2}{\teaspoon} red pepper flakes (optional)\\
  \SI{1}{\cup} whipping cream
  \SI{1/2}{\cup} chopped fresh basil\\
  \SI{1/4}{\cup} grated parmesam (use the real thing not the cylinder, people)\\
  1+ cloves garlic, chopped\\
  \SI{1}{\tblspoon} olive oil\\
  about \SI{3/4}{\pound} pasta
\end{ingredients}
Cook pasta. Save \SI{1/3}{\cup} cooking water. In frying pan, sear garlic, add
zucchini, prosciutto, salt and pepper, red pepper flakes, then tomatoes.  Stir
for \numrange{2}{3} minutes.  Add saved water, cream and simmer briefly. Add
pasta, basil, and Parmesan, and toss. Transfer to serving dish and eat
immediately (not difficult to do!). Serves \numrange{2}{3}.
\end{entry}

%%---------------------------------------------------------------------------%%
\begin{entry}{Pasta al Cavolfiore (with Cauliflower)}{First Edition}
\index{Pasta!pasta al cavolfiore}
\index{Vegetarian!pasta al cavolfiore}
\index{Johnson!Don}

\begin{open}
  Don sends this yummy looking dish from the Moosewood cookbook. Don
  says it's good and adds ``so enough of sending pasta recipes to the
  Italians.''
\end{open}
\begin{ingredients}
  1 onion (optional if you're Kate)\\
  1 cauliflower head, chopped into bite sizes\\
  1 tomato\\
  garlic to taste\\
  \SI{2}{\cup} grated cheese (see below)\\
  \SI{1/4}{\cup} olive oil\\
  1 can tomato sauce\\
  \SI{3/4}{\pound} pasta\\
  basil, dried and some fresh too if possible\\
  \SI{1}{\teaspoon} salt\\
  \SI{1}{\teaspoon} pepper
\end{ingredients}
Chop the onion and garlic and \saute them in \SI{1}{\teaspoon} oil with the
basil. When onion is clear, add cauliflower and cook until tender. (Don tip: add
a handful of water, and cover to speed this along.) Add chopped tomato, tomato
sauce, salt and pepper, and simmer for about twenty minutes. During this time,
cook and drain pasta. Add the remaining olive oil to pasta along with fresh
basil and half the cheese. Don recommends the cheese be a mixture of Parmesan,
Romano, mozzarella, and cheddar. Spread this on a big platter and top with the
cauliflower mixture. Top with remaining cheese. Don recommends a California
Gewurztraminer ``to go with.''  Serves \numrange{2}{3}.
\end{entry}

%%---------------------------------------------------------------------------%%
\begin{entry}{Sweet and Sour Pork}{First Edition}
\index{Pork!sweet and sour pork}
\index{Evans!Kate}
\index{Evans!Tom}

\begin{open}
  Tom and Kate eat this a lot; its a ``regular''. It's word-for-word from a
  Southern Living year-end cookbook I love (1992, if curious). Its quite
  delicious, and its even somewhat healthy.
\end{open}
\begin{ingredients}
  \SI{1}{\tblspoon} sherry\\
  \SI{1}{\tblspoon} soy sauce\\
  \SI{1}{\tblspoon} cornstarch\\
  \SI{1}{\pound} boneless pork, cut into cubes\\
  \SI{1/4}{\cup} vegetable oil, divided\\
  1 clove garlic, minced\\
  1 small onion (optional)\\
  2 green peppers, cut into \SI{1}{\inch} pieces\\
  \SI{1/3}{\cup} sugar\\
  \SI{1/4}{\cup} ketchup\\\
  \SI{1}{\tblspoon} sherry\\
  \SI{2}{\tblspoon} soy sauce\\
  \SI{2}{\tblspoon} white vinegar\\
  \SI{1}{\tblspoon} cornstarch\\
  \SI{1/3}{\cup} water\\
  1 \SI{8}{\ounce} can pineapple slices in juice, each cut into about 8 pieces
\end{ingredients}
%
\protip{Using fresh pineapple makes this dish so much better!}
%
Combine first 3 ingredients, add pork, and let marinate 20 minutes (or however
long it takes to prepare everything else). Heat \SI{2}{\tblspoon} oil in big
frying pan.  Stir fry onion garlic, and green pepper over med-high heat until
crisp tender.  Remove from skillet. Add rest of oil and cook pork until cooked
through.  Stir in cooked vegetables. Combine sugar and next 6 ingredients,
stirring until cornstarch dissolves. Add to pork mixture and cook until it
comes to a boil. Add pineapple and and boil for about 1 minute. Serve over hot
cooked rice. Serves \numrange{2}{3} hungry people.
\end{entry}
\chapter{Treats}

\section{Mint Syrup\index{treats!mint syrup}}

\begin{open}
    This recipe comes from Louise Johnson of Spruce Head, ME, aka Weedie, the sister of the original Donald Dodge Johnson (Dodge and Julie’s father).
\end{open}
%%
\begin{ingredients}
    fresh spearmint or peppermint leaves\\
    granulated sugar\\
    1 or more lemons\\
    1 or more oranges\\
\end{ingredients}
Using a strainer that fits into a deep bowl, fill with finely cut fresh spearmint or peppermint leaves. Boil for ten minutes equal measures of water and granulated sugar, approximately 1\SI{1/2}{\cup} each. Pour directly over the mint and allow to cool. Remove the strainer and add the juice of one or more lemons and the juice of one or more oranges. Wonderful over ice cream, in iced tea, or on a simple cake aching for a hint of mint.

%\part{Recipes from the First Edition}
%%%---------------------------------------------------------------------------%%
\begin{entry}{Wedding Brunch Gazpacho}{First Edition}
\index{Soups!gazpacho}
\index{Vegetarian!gazpacho}
\index{Tidey, Jen}

\begin{open}
  This gazpacho recipe comes by way of Jen Tidey.  We were first
  introduced to it during the ``day after'' wedding brunch.  The amounts of
  ingredients are variable, so add to your personal taste preferences.
\end{open}
\begin{ingredients}
  celery \\
  peeled and seeded cucumbers \\
  scallions and/or onions \\
  red and green peppers \\
  minced garlic \\
  \num{\sim 6} tomatoes (canned are O.K.) \\
  \SI{\sim 1/2}{\cup} of red wine vinegar \\
  \SI{\sim 1/2}{\cup} of olive oil \\
  \numrange{1}{1}\SI{1/2}{\cup} tomato juice or \corp{V8} \\
  lemon juice \\
  black pepper \\
  Tabasco or cayenne pepper \\
  fresh cilantro
\end{ingredients}
Chop the celery, cucumbers onion, peppers, tomatoes, and garlic to a
consistency that you find happy.  A food processor may be used for ``soupy''
consistency.  Mix the red wine vinegar and olive oil (these should be in equal
amounts).  To this add the tomato juice (\corp{V8}).  Add lemon juice, black
pepper, Tabasco, and fresh cilantro.  Combine veggies and liquid ingredients
in a glass pitcher or other aesthetically pleasing container.  Chill for
several hours or overnight.  Serve with crunchy bread and a lightish red wine.
\end{entry}

%%---------------------------------------------------------------------------%%
\begin{entry}{Frog Mustard Salad Dressing}{First Edition}
\index{Dressings!frog mustard salad dressing}
\index{Vegetarian!frog mustard salad dressing}
\index{Vegan!frog mustard salad dressing}
\index{Evans!Kate}

\begin{open}
  This is a quick preparation dressing that can be prepared in about 10
  minutes.  It makes \numrange{8}{10} servings.
\end{open}
\begin{ingredients}
  \SI{1/2}{\cup} dijon mustard \\
  \SI{2}{\tblspoon} red wine vinegar \\
  \SI{1/4}{\teaspoon} salt \\
  \SI{3/4}{\teaspoon} pepper \\
  \SI{1}{\cup} corn oil (or olive)
\end{ingredients}
The crouton ingredients are:
\begin{ingredients}
  4 slices fine textured white bread \\
  \SI{4}{\tblspoon}  butter \\
  \SI{1}{\teaspoon} dried thyme \\
  \SI{1/8}{\teaspoon} salt \\
  dash of pepper \\
  \SI{2}{\teaspoon} minced parsley
\end{ingredients}
\begin{wrapfigure}{R}{.45\textwidth}
\centering\includegraphics[width=.42\textwidth,clip]{figures/frog.pdf}
\end{wrapfigure}
To make the salad dressing whisk the mustard, vinegar, salt and pepper in a
small bowl.  Gradually add oil.

To make the croutons preheat the oven to \SI{350}{\degree}.  Trim crusts (if
desired) and cut slices into \SI{1/2}{\inch} cubes.  Spread single layer on
tray, bake for \numrange{10}{15} minutes until dry and lightly brown.  Heat
butter in a skillet.  Add croutons and remaining ingredients and toss well.
Saut\'{e} \numrange{1}{2} minutes over medium heat and cool.  Kate says that
good salad stuffs are greenleaf or romaine lettuce with spinach.  Add tomatoes,
cucumber, and other salad stuff as your heart desires.  Another yummy thing is
to add pieces of chicken, ham, or turkey and other meats, which makes it taste
almost like a bite-sized sandwich.
\end{entry}

%%---------------------------------------------------------------------------%%
\begin{entry}{Easiest Tomato Aspic}{First Edition}
\index{Appetizers!tomato aspic}
\index{Johnson!Lil}

\begin{open}
  We received this recipe from Grammie (Lil Johnson) and we must confess
  we had no idea what an ``aspic'' was!  For those not in the know, its a tasty
  treat.
\end{open}
\begin{ingredients}
  1 small pkg. lemon \corp{Jello}\\
  \SI{1}{\cup} boiling water \\
  1 \SI{8}{\ounce} can \corp{Hunt's} Tomato Sauce \\
  \SI{1}{\teaspoon} horseradish \\
  (\SI{1}{\teaspoon} lemon juice or vinegar optional)
\end{ingredients}
Add the following if you desire.
\begin{ingredients}
  cooked shrimp \\
  crabmeat \\
  chopped celery
  hard cooked egg \\
  asparagus \\
  artichoke hearts \\
  avocado
\end{ingredients}
Pour 1 cup boiling water over \corp{Jello} and mix until smooth. Add tomato
sauce, horseradish, and lemon juice or vinegar.  To this you can add any of
the optional ingredients. Grandaddy's (Don) favorites are seafood and chopped
celery. Place aspic into fridge and jell about 3 hours.
\end{entry}

%%---------------------------------------------------------------------------%%
\begin{entry}{Meaty Cheese dip, aka ``Queso''}{First Edition}
\index{Appetizers!meaty cheese dip}
\index{Beef!meaty cheese dip}
\index{Evans!Kate}

\begin{open}
  Just so you know, this dish isn't good for you. But it's so delicious, we
  don't care. Kate originally got this recipe from \corp{Southern Living}, but
  has since improved it using input from our dear friend Dave Court\index{Court,
  Dave}.  For you food snobs, don't be put off by the \corp{Velveeta}.  It
  provides a smoothness in this dish.
\end{open}
\begin{ingredients}
  \SI{1}{\pound} ground turkey or beef\\
  \SI{1/2}{\pound} hot bulk pork sausage\\
  1 \SI{8}{\ounce} jar salsa or 1 can Rotel diced tomatoes and green chiles, any heat level you like\\
  1 \SI{2}{\pound} loaf \corp{Velveeta} with jalapenos, cut into cubes\\
  1 \SI{10.5}{\ounce} can Campbell's cream of green chile (or celery, if you are not in New Mexico) condensed soup
\end{ingredients}
Brown ground meat and sausage in large skillet, stirring so it crumbles. Add
salsa, cheese, and soup and cook over low heat until the cheese melts, about 2
hours in a slow cooker. Serve warm with nacho or corn chips. Yum!
\end{entry}

%%---------------------------------------------------------------------------%%
\begin{entry}{Hot crab dip}{First Edition}
\index{Appetizers!hot crab dip}
\index{Johnson!Dodge}
\index{Johnson!Martha}

\begin{open}
  This is sooo good! Martha and Dodge submitted this. This is a dip they
  should make more often. But ha ha, now we have the recipe!
\end{open}
\begin{ingredients}
  \SI{8}{\ounce} cream cheese, softened\\
  \SI{1/2}{\pound} seafood flakes (fake crab legs)\\
  \SI{2}{\tblspoon} chopped onion\\
  Worcestershire sauce
\end{ingredients}
Preheat oven to \SI{350}{\degreeF}. Mix cream cheese. Slice and add ``crab.''
Next add onion and Worcestershire sauce. Bake for \numrange{15}{20} minutes,
or when bubbly.
\end{entry}

%%---------------------------------------------------------------------------%%
\begin{entry}{Blue Cheese-Pecan Spread}{First Edition}
\index{Appetizers!blue cheese-pecan spread}
\index{Vegetarian!blue cheese-pecan spread}
\index{Johnson!Dodge}
\index{Johnson!Martha}

\begin{open}
  This is an easy and yummy appetizer submitted by Martha and Dodge.  We
  usually spoil our appetite for dinner eating it with all kinds of crackers!
\end{open}
\begin{ingredients}
  \SI{1/2}{\cup} pecan pieces\\
  \SIrange{4}{5}{\ounce} cream cheese\\
  At least \SI{2}{\tblspoon} blue cheese, Gorgonzola, or Roquefort\\
  \SIrange{1}{2}{\tblspoon} butter \\
  Worcestershire sauce and/or hot pepper flakes, optional
\end{ingredients}
In food processor, process pecans until fine. Add cream cheese and blue cheese
in small chunks. Add more blue cheese if it doesn't taste like enough.
``Smooth'' out flavors with the butter, if necessary. Add hot stuff if
desired.  Spoon into crock or pretty bowl and refrigerate until ready to
serve.
\end{entry}
%%%---------------------------------------------------------------------------%%
\begin{entry}{Three Kings Bread (and St. Nick)}{First Edition}
\index{Breads!Three Kings Bread}
\index{Vegetarian!Three Kings Bread}
\index{King!Jenn}
\index{King!Rich}
\index{King!Julian}

\begin{center}
    \includegraphics[width=.4\textwidth,clip]{figures/kings.pdf}
\end{center}

\begin{open}
  Is this the real name or is it because it comes from three Kings?
  This is a recipe supplied to us from Rich, Jenn and Julian King (Nicholas arrived after the project, began-hence the name). (editor's note from the second edition: yet more children arrived and have come of age in the King family)
  This recipe makes one loaf.
\end{open}
\begin{ingredients}
  \SI{1/4}{\cup} plus one \si{\tblspoon} sour cream \\
  \SI{1}{\teaspoon} baking soda \\
  \SI{1/2}{\cup} butter at room temperature \\
  \SI{1}{\cup} sugar \\
  2 eggs lightly beaten \\
  1 ripe mashed banana and 1 medium apple (2 bananas an option) \\
  \SI{1/2}{\teaspoon} baking powder \\
  \SI{1}{\cup} chopped nuts \\
  \SI{1/2}{\teaspoon} cardamom \\
  1 zest of lemon \\
  \SI{1}{\teaspoon} vanilla extract
\end{ingredients}
Preheat the oven to \SI{350}{\degree}.  Grease a \SI{9x5}{\inch} loaf pan.
Combine sour cream and baking soda in small bowl.  Set aside (it will foam).
Cream butter and sugar in a small bowl.  Beat in eggs, fruit, and sour cream
mixture.  Slowly mix in all dry ingredients.  Bake until a toothpick inserted
into center comes out sort of clean and loaf is golden brown.  This should be
about 1 hour.

Cool 10 minutes in pan. Turn loaf out onto rack and cool completely.  Eat
thinking of your most favorite Kings$\ldots$
\end{entry}

%%---------------------------------------------------------------------------%%
\begin{entry}{Kate's Standard Bagels}{First Edition}
\index{Breads!bagels}
\index{Vegetarian!bagels}
\index{Evans!Kate}

\begin{open}
  This is a recipe by Kate that she picked up in Washington, D.C. during
  a 91'--92' winter internship at NIST from Dr. Lucatorto.  We enjoy this one
  a lot (especially in the south where getting good bagels is not always
  easy). Note: you need a food processor for this recipe. It makes 16 bagels.
\end{open}
\begin{ingredients}
  2 packets yeast \\
  2 scant \si{\tblspoon} sugar\\
  \SI{1/2}[3]{\cup} warm water with salt \\
  Lots (several pounds) of flour
\end{ingredients}
Combine yeast, sugar, and water.  Add \SI{\sim 2}{\cup} flour to food
processor. Pour in water mixture until a dough is formed (stop pouring when
processor begins to ``growl.'' Listen you'll hear it).  Remove dough to a
casserole dish with lid. Repeat until all mixture is used.  Microwave dish at
\SI{30}{\percent} for 3 minutes.  Let rise for 30~minutes.  Punch down roll
into loaf, cut in 16 pieces and make into bagels. Kate uses a doughnut
stamper. Let rise 30~minutes. In large frying pan, set \SI{2}{\inch} water to
boil.  Boil bagels for 10~seconds each side.  Place on greased cookie sheet
and bake 30~minutes or until lightly brown at \SI{350}{\degree}. If you want
to add extras such as cinnamon or raisins, add when processor starts to
growl. If you would like toppings such as sesame seeds, brush bagel with egg
white and sprinkle on top just before baking.
\end{entry}

%%---------------------------------------------------------------------------%%
\begin{entry}{Kate's Super Stromboli Dough}{First Edition}
\index{Evans!Kate}
\index{Breads!stromboli dough}
\index{Vegetarian!stromboli dough}
\label{sec:stromboli}

\begin{open}
  This is a recipe by Kate.  While this recipe is intended for Stromboli
  or calzone it also makes a fine pizza dough.  The
  recipe serves 6.
\end{open}
\begin{ingredients}
  4 scant \si{\cup} flour \\
  1 package \corp{Quick Rise} yeast \\
  \SI{1}{\teaspoon} salt \\
  \SI{1}{\tblspoon} sugar \\
  \SI{1/3}[1]{\cup} warm water \\
  \SI{1/4}{\cup} oil
\end{ingredients}
In mixing bowl combine flour, yeast, sugar, and salt.  Add water and oil and
form a soft dough. Add flour or water as necessary.  Let rise 30 minutes, then
punch down (you can freeze at this point to thaw later in microwave).  Roll
into \numrange{4}{6} circles (depending on crowd hunger). Add your favorite
toppings, including sauce if desired, and of course cheese. Possible fillings
are broccoli (a Kate favorite), spinach, pepperoni, ham, onions, mushrooms,
and almost anything edible you can think of. Bake at \SI{400}{\degree} for
\numrange{12}{15} minutes until golden brown.
\end{entry}

%%---------------------------------------------------------------------------%%
\begin{entry}{Kuchen}{First Edition}
\index{Breads!kuchen}
\index{Vegetarian!kuchen}
\index{Evans!Fermina}

\begin{open}
  Provided by Fermina Evans, this German ``bread'' is a Christmas morning tradition that she has carried on from Geo's parents and passed on to the Evans's kids. She always makes a double batch and it's still barely enough!
  Don't be fooled by its location in the ``breads'' section; its definitely a
  treat!
\end{open}
\begin{ingredients}
  \SI{1/4}{\cup} shortening (vegetable makes it a vegetarian dish)\\
  \SI{1}{\cup} sugar \\
  1 egg \\
  \SI{1/2}{\cup} milk \\
  \SI{1/2}[1]{\cup} flour \\
  \SI{2}{\teaspoon} baking powder \\
  salt to taste
\end{ingredients}
The topping:
\begin{ingredients}
  \SI{1/2}{\cup} brown sugar \\
  \SI{1/3}{\cup} flour \\
  \SI{1}{\teaspoon} cinnamon \\
  dash salt\\
  \SI{1/4}{\cup} butter
\end{ingredients}
Preheat oven to \SI{350}{\degreeF}. Cream shortening and sugar. Add egg and mix
well.  Add baking powder, flour and milk. Pour into greased and floured
\SI{9}{\inch} round or \SI{8}{\inch} square baking pan. Combine the rest of
the topping ingredients except butter. Then, cut butter into topping mixture
and sprinkle on top of batter. Bake for \numrange{30}{40} minutes (test with
toothpick for done-ness).
\end{entry}

%%---------------------------------------------------------------------------%%
\begin{entry}{Weedie's Blueberry Muffins}{First Edition}
\index{Breakfast!blueberry muffins}
\index{Johnson!Weedie}

\begin{open}
  No joke, Don and Kate (and surely David and Steve) used to beg Weedie to make these muffins every time we visited. And she always did. Martha and Dodge say, ``Oh Heaven, these and some Lobster salad!'' They are perfection,
  we promise you. Just make the effort to get quality blueberries. We
  recommend going to Maine to get them.
\end{open}
\begin{ingredients}
  2 scant \si{\cup} flour\\
  \SI{3}{\teaspoon} baking powder\\
  \SI{1/2}{\cup} sugar\\
  \SI{1/2}{\teaspoon} salt\\
  \num{1/2} stick butter, melted (\SI{1/4}{\cup})\\
  2 eggs\\
  \SI{1}{\cup} milk\\
  \SI{1}{\cup} blueberries (little wild ones are best)\\
  sugar\\
  cinnamon
\end{ingredients}
Preheat oven to \SI{400}{\degreeF}. Weedie says, ``Usually I wash the berries
awhile before using them, so they can dry off before being added.'' In mixing
bowl combine flour, baking powder, sugar, and salt. Mix butter eggs and milk
in a separate bowl. Pour this over the flour mixture and stir until
smooth. Add blueberries, stir gently, and spoon into buttered muffin
tins. Sprinkle with sugar and cinnamon and bake for \numrange{20}{25} minutes.
\end{entry}
%%%---------------------------------------------------------------------------%%
\begin{entry}{Hash Brown Potatoes}{First Edition}
\index{Potatoes!hash brown potatoes}
\index{Evans!Joyce}

\begin{open}
  Contributed by Joyce Evans. Try with the Swiss Meatloaf
  (Section~\ref{sec:swiss-meatloaf}), yum! Serves 4.
\end{open}
\begin{ingredients}
  3 large potatoes, boiled \\
  \SI{1/4}{\cup} milk \\
  \SI{3}{\tblspoon} all-purpose flour \\
  \SI{2}{\tblspoon} minced onion \\
  \SI{2}{\tblspoon} minced fresh parsley or chervil \\
  \SI{1/2}{\teaspoon} salt \\
  \SI{1/2}{\teaspoon} pepper \\
  \SI{1/4}{\teaspoon} dried oregano (opt.) \\
  Dash of Tabasco \\
  \SI{3}{\tblspoon} bacon drippings, rendered chicken fat, or vegetable oil
\end{ingredients}
Preheat in electric skillet to \SI{300}{\degree}. Peel and dice the boiled
potatoes and place into a medium bowl. You should have about \SI{3}{\cup}. Add
the rest of the ingredients except the cooking fat and blend.

Add the cooking fat to the skillet and heat. Pack the potato mixture in
firmly, spreading it out in an even layer. Cook \numrange{7}{9} minutes or
until the bottom side is richly brown. Turn the mixture over in segments and
smooth down again into a patty. Continue cooking until the other side is
brown, another \numrange{7}{9} minutes.  Cut into wedges and serve.
\end{entry}

%%---------------------------------------------------------------------------%%
\begin{entry}{Black Rice}{First Edition}
\index{Rice!black}
\index{Evans!Fermina}

\begin{open}
  Contributed by Fermina Evans. Serves 4, or 2 healthy eaters. This is Tom and
  Katie's favorite peasant food.
\end{open}
%%
\begin{ingredients}
  \SI{1}{\cup} dry black beans\\
  \SI{5}{\cup} chicken broth\\
  \SI{1/2}{\tblspoon} olive oil \\
  1 small onion, chopped \\
  4 cloves garlic, minced \\
  \SI{1}{\ounce} finely chopped Canadian or regular bacon \\
  \SI{1/2}{\cup} rice \\
  \SI{1/4}{\cup} white wine \\
  1 tomato coarsely chopped \\
  \SI{1/2}{\teaspoon} ground cumin \\
  pinch cayenne \\
  \SI{1/2}{\cup} finely chopped cilantro
\end{ingredients}
You may substitute 2 cans black beans for dry beans if you prefer. If using
dry beans, soak beans overnight in cold water, and simmer beans for
\numrange{120}{150} minutes in \SI{3}{\cup} broth until tender. Drain and
resolve liquid (\SI{1/2}[1]{\cup}). Otherwise, drain them under cold water and
use \SI{1/2}[1]{\cup} chicken broth or chicken bouillon stock for bean broth.

Heat oil in stockpot. Add onion, garlic, and bacon and stir fry for about 5
minutes.  Add rice and stir for 1 minute. Add wine and cook for 2 minutes. Add
tomatoes and cook for 2 more minutes.  Add bean broth \SI{1/2}{\cup} at a
time, stirring until liquid is absorbed before adding more broth. This will
take \numrange{20}{25} minutes to complete. Add the beans and remaining
broth. Season with cumin, cayenne, and cilantro and serve.
\end{entry}

%%---------------------------------------------------------------------------%%
\begin{entry}{Potato Gratin with Mustard and Cheese}{First Edition}
\index{Potatoes!potato gratin}
\index{Vegetarian!potato gratin}
\index{Evans!Kate}

\begin{open}
  This is a great entertaining dish because its classy, very smooth and
  flavorful, yet can be prepared before guests arrive. Kate got it from
  \corp{Bon Appetit} magazine.
\end{open}
\begin{ingredients}
  \SI{1}{\tblspoon} butter\\
  \SI{1}{\cup} fresh breadcrumbs\\
  \SI{1}{\tblspoon} dried thyme\\
  \SI{2}{\teaspoon} salt\\
  \SI{1}{\teaspoon} ground pepper\\
  \SI{1}{\pound} sharp white cheddar cheese, grated\\
  \SI{1/4}{\cup} flour\\
  \SI{5}{\pound} russet potatoes, peeled and thinly sliced\\
  \SI{4}{\cup} canned low salt chicken broth (veggie broth can be substituted) \\
  \SI{1}{\cup} whipping cream\\
  \SI{6}{\tblspoon} Dijon mustard
\end{ingredients}
Melt butter in skillet and add breadcrumbs, stirring until golden brown (about
10 min.). Set aside. Preheat oven to \SI{400}{\degree}. Butter a
\SI{15x10x2}{\inch} baking dish. Mix thyme, salt, and pepper in small
bowl. Combine grated cheese and flour, tossing to coat the cheese. Arrange
\num{1/3} potato slices to cover the bottom of the baking dish. Sprinkle
\num{1/3} the thyme mixture, then \num{1/3} the cheese mixture. Repeat
layering 2 more times.  Next whisk chicken broth, cream, and mustard in a
separate bowl, and then pour it over the potato layers. Bake 30
minutes. Sprinkle buttered crumbs over, and bake until potatoes are tender and
top is golden brown, about 1 hour longer.  Enjoy!
\end{entry}
%%%---------------------------------------------------------------------------%%
\begin{entry}{Scott's Killer Chili}{First Edition}
\index{Beef!chili}
\index{Evans!Scott}

\begin{open}
  I hope you're prepared for this.  This recipe from Scott Evans makes a {\em
  thick} and {\em spicy} chili.  In the word's of the author ``It is pretty
  spicy.''  This is one of those recipes that should include a Disney-style
  warning label, ``\textellipsis those with heart conditions or over the age of
  sixty-five...etc. etc.''  This is this recipe's first time in print so some
  experimentation may be required.  Supposedly this is a campout recipe, however
  I see no way that anyone could possibly carry all these ingredients. The
  recipe makes about 16 servings or less for REALLY big people.  Good luck and
  here it goes.
\end{open}
\begin{ingredients}
  \SI{3}{\pound} of hot Italian sausage (ie. Hot Cincinnati Brand, that homer) \\
  \SI{3}{\pound} bacon \\
  3 large onions \\
  3 bell peppers (2 green, 1 red) \\
  \numrange{4}{5} cloves of garlic \\
  \numrange{4}{5} hot peppers (a cornucopia of jalape\~{n}os, habaneros, and
  others) \\
  3 cans Italian pear tomatoes \\
  \SI{1}{\tblspoon} olive oil \\
  \SI{1}{\tblspoon} mustard powder \\
  \SI{1}{\tblspoon} celery seed \\
  \SI{1}{\tblspoon} chili powder \\
  \SI{1}{\tblspoon} bay leaves \\
  \SI{1}{\tblspoon} Worcester sauce \\
  \SI{1}{\tblspoon} vinegar \\
  red wine \\
  water \\
  salt and pepper
\end{ingredients}
Start with a large Dutch oven and a campfire right after breakfast.  Fry the
sausage and set aside.  Fry the bacon and set aside.  Leave a bit of grease in
the pot and add the minced garlic followed by the roughly chopped onions and
bell peppers (no bell pepper seeds).  Chop the hot peppers and add to the pot,
remember that the seeds make the dish VERY spicy.  Add olive oil as needed.
%%
\begin{wrapfigure}{R}{.2\textwidth}
\centering
\includegraphics[width=.15\textwidth, trim=.5in .25in .5in .25in, clip]{figures/chilli.pdf}
\end{wrapfigure}
%%
Add about one Tbsp full each of: mustard powder, celery seed, Worcester sauce,
vinegar, and chili powder.  This may require some experimentation to alter to
your taste.  Stir and cook until onions become clear and peppers begin to
soften.  Add up to one cup of red wine.  Next add tomatoes and juices.  Stir
and chop tomatoes.  Add sausage, bacon, and two bay leaves.  Season with salt
and pepper.  Now everything should look a bit like chunky soup, but don't
worry.

Let the chili simmer over low heat for a minimum of three hours, but try for
eight (trust Scott on this one). Check periodically and stir.  If mixture
thickens too much add some water.  Taste and adjust to preference.

Serve with shredded cheddar cheese and garlic bread.  This recipe freezes well
in personalized zip-lock bags (In case you're not hungry enough to eat six
pounds of meat in one sitting).
\end{entry}

%%---------------------------------------------------------------------------%%
\begin{entry}{Chicken Breasts with Orange Sauce}{First Edition}
\index{Chicken!chicken with orange sauce}
\index{Johnson!Lil}

\begin{open}
  This is a recipe from Lil and Don Johnson Sr.  Grammie (Lil) passed it to Martha, who passed it to Kate, and so on, and so on, and so on\textellipsis
  It's a good recipe for new cooks.
\end{open}
\begin{ingredients}
  4 halved chicken breasts \\
  1 small can undiluted O.J. concentrate \\
  1 package Lipton's Onion Soup Mix \\
  paprika
\end{ingredients}
In a long baking pan arrange the 8 pieces of chicken.  Pour the O.J.
concentrate (at room temperature) over the chicken.  Sprinkle the soup mix
over the chicken.  Add a little paprika for seasoning.

Cover pan with foil.  Bake at \SI{350}{\degree} for 40 minutes.  Remove foil and
baste chicken.  Back for an additional 20 minutes uncovered.
\end{entry}

%%---------------------------------------------------------------------------%%
\begin{entry}{Porcupine Meatballs}{First Edition}
\index{Beef!meatballs}
\index{Evans!Mickey}
\index{Evans!George Sr.}

\begin{open}
  This is a recipe from Mickey and George, Sr. Evans.
\end{open}
\begin{ingredients}
  \SI{1/2}[1]{\pound} hamburger \\
  \SI{3/4}{\cup} uncooked rice \\
  \SI{1}{\teaspoon} salt \\
  1 egg \\
  \SI{1/2}{\teaspoon} pepper \\
  \SI{1/4}{\cup} chopped onion \\
  \SI{1/2}[2]{\cup} stewed tomatoes \\
  \SI{1}{\teaspoon} chili \\
  \SI{1}{\teaspoon} sugar
\end{ingredients}
Combine hamburger, rice, salt, egg, pepper, and onion.  Shape into
\SI{1/2}[1]{\inch} balls.

Heat sauce and chili to boiling in a kettle.  Drop balls in sauce.  Simmer for
1\num{1/2} hour covered.  Strips of bacon may be wrapped around meatballs
and secured with toothpicks before cooking in sauce.
\end{entry}

%%---------------------------------------------------------------------------%%
\begin{entry}{Lemon-Herb Chicken}{First Edition}
\index{Chicken!lemon-herb chicken}
\index{Evans!Fermina}

\begin{open}
  This is Fermina's favorite.
\end{open}
\begin{ingredients}
  1 chicken (cut) or \SI{1/2}[3]{\pound} of chicken parts      \\
  \SI{1/2}{\cup} olive oil                                     \\
  \SI{1/4}{\cup} lemon juice                                   \\
  2 garlic cloves minced                                       \\
  \SI{3}{\tblspoon} chopped fresh oregano or \SI{1}{\tblspoon} dry \\
  \SI{1/2}{\teaspoon} salt                                     \\
  \SI{1/8}{\teaspoon} pepper                                   \\
  \SI{1}{\tblspoon}chopped fresh rosemary or \SI{1}{\teaspoon} dry \\
  \SI{2}{\tblspoon}chopped fresh parsley
\end{ingredients}
Place chicken meaty side down in a \SI{13x9x2}{\inch} baking pan.  Combine
next 8 ingredients, mix well.  Pour mixture over chicken.  Marinate in
refrigerator for two hours.  Bake uncovered at \SI{350}{\degree} for 40 minutes.
Turn chicken.  Broil 6~inches from heat for \numrange{5}{10} minutes or
until crisp and lightly browned.
\end{entry}

%%---------------------------------------------------------------------------%%
\begin{entry}{Stuffed Flank Steak Teriyaki}{First Edition}
\index{Beef!flank steak teriyaki}
\index{Evans!Tom}
\index{Evans!Fermina}

\begin{open}
  This belongs because it is MY favorite and since I'm (Tom) in charge,
  well there you go.  Makes \numrange{4}{5} servings.
\end{open}
\begin{ingredients}
  1 medium to large beef flank steak (1\SIrange{1/4}{2}{\pound}) \\
  \SI{1/2}{\cup} soy sauce                                    \\
  \SI{1/4}{\cup} cooking oil                                  \\
  \SI{2}{\tblspoon} molasses                                   \\
  \SI{2}{\teaspoon} dry mustard                               \\
  \SI{1}{\teaspoon} ginger root or \SI{1/2}{\teaspoon} dry ginger \\
  1 clove garlic minced                                       \\
  \SI{1}{\cup} water                                          \\
  \SI{1/2}{\cup} long-grain rice                              \\
  \SI{1/2}{\cup} of shredded carrots                          \\
  \SI{1/2}{\cup} sliced water chestnuts (optional)            \\
  \SI{1/4}{\cup} sliced green onions
\end{ingredients}
\begin{wrapfigure}{L}{.3\textwidth}
\centering\includegraphics[width=.25\textwidth,clip]{figures/flank.pdf}
\end{wrapfigure}
Cut a large pocket in flank steak or have your butcher do it.  Combine soy
sauce, oil, molasses, mustard, ginger, and garlic.  Place meat in shallow pan
or plastic bag.  Pour marinade into pocket and over meat. Let stand at room
temp (\SI{300}{\kelvin}) for 30 minutes or in refrigerator for
\numrange{2}{3} hours.

In saucepan combine water, rice, carrots, water chestnuts, and green onion.
Bring to a boil; reduce heat and simmer while covered for 8 minutes.  Remove
from heat and set aside.

Drain meat reserving marinade.  Add \SI{1/4}{\cup} of reserved marinade to the
rice mixture.  Spoon rice stuffing into pocket of meat.  Secure end with
wooden toothpicks.  Place meat in shallow roasting pan and cover with foil.
Bake at \SI{350}{\degree} for 1 hour until meat is done.

Fermina allows an extra \numrange{10}{15} minutes without foil to brown meat.
Brush with marinade while browning and check often.  Slice meat diagonally
across grain to serve.
\end{entry}

%%---------------------------------------------------------------------------%%
\begin{entry}{Better-than-vegetarian Pasta Sauce}{First Edition}
\index{Pasta!better-than-veggie pasta sauce}
\index{Vegetarian!better-than-veggie pasta sauce}
\index{Johnson!Don}

\begin{open}
  This from Kate's brother Don, and he got it from his friend Dawn Ollila. The honey/brown sugar and cinnamon addition is what makes it taste so special.
\end{open}
\begin{ingredients}
  \SI{1}{\tblspoon}olive oil \\
  1 onion \\
  1 green pepper \\
  2 cloves of minced garlic \\
  1 tomato\\
  1 large or two small carrots OR \SI{1/4}{\cup} cup lentil beans \\
  1 or 2 cans of tomato sauce \\
  thyme \\
  basil \\
  oregano \\
  rosemary \\
  \SI{1}{\tblspoon} honey or brown sugar \\
  cinnamon \\
  salt and pepper
\end{ingredients}
\Saute the 4 vegetables in olive oil, adding them as ordered above. When the
onion is clear and the tomato is soft, add the tomato sauce.  Bring to a
simmer. The sauce is now tasty, but to thin to stick to the pasta.  Choose
either the carrots or lentils to give it body. Lentils add a great dark,
almost meaty, flavor but you will need to boil them in 4 times their
measurement in boiling water for \numrange{45}{90} minutes first (no
presoaking required). Do not add them to the sauce until they are bean-like
mush. or, you can grate the carrot as finely as you have the technology to do
and add to the sauce. The taste is minimal, but the texture is great. Add the
rest of the ingredients and adjust to taste. Add just enough cinnamon to make
your guests look at you funny and say, ``What did you put in this?'' The
flavor actually works quite well.
\end{entry}

%%---------------------------------------------------------------------------%%
\begin{entry}{Peachy Chicken}{First Edition}
\index{Chicken!peachy chicken}
\index{Evans!Fermina}
\index{Evans!Betsy}
\index{Evans!Tom}

\begin{open}
  From Fermi, this is Betsy's favorite.  Tom: Actually I really don't
  like this recipe and the only reason she ``likes'' this one is because she
  hates my favorite recipe (stuffed flank steak).  Note: the ``peachy'' in
  ``peachy chicken'' is not a southern thing.  Kate: When Tom and I were first
  married (Tom interjects: forty years ago) I made a chicken dish with fruit.
  He honestly thought I was trying to annoy him (how did he know?).
\end{open}
\begin{ingredients}
  chicken parts for \numrange{4}{6} people\\
  1 large can peach halves (drained, reserve syrup)\\
  \SI{2}{\tblspoon} soy sauce\\
  \SI{2}{\tblspoon} lemon juice\\
  \SI{1/2}{\tblspoon} ginger\\
  2 cloves minced garlic
\end{ingredients}
Preheat oven to \SI{375}{\degree}.  Mix last four ingredients with reserved peach
syrup.  Pour marinade over chicken parts in roasting or baking pan. Bake in
oven for \numrange{45}{60} minutes. Turn once.  Add peach halves last
15 minutes.  Brown under broiler for a few minutes if further browning
is needed.
\begin{center}
    \includegraphics[scale=.5,clip]{figures/meatloaf.pdf}
\end{center}
\end{entry}

%%---------------------------------------------------------------------------%%
\begin{entry}{Swiss Meatloaf}{First Edition}
\index{Beef!swiss meatloaf}
\label{sec:swiss-meatloaf}
\index{Evans!Joyce}

\begin{open}
  This was contributed by Joyce Evans, and is definitely ``comfort
  food''! Serves 6.
\end{open}
\begin{ingredients}
  1 egg \\
  \SI{1/2}{\cup} evaporated milk \\
  \SI{1}{\teaspoon} rubbed sage \\
  \SI{1}{\teaspoon} salt \\
  \SI{1/2}{\teaspoon} black pepper \\
  \SI{1/2}[1]{\pound} lean ground beef \\
  \SI{1}{\cup} cracker crumbs (round buttery type, approx. 24) \\
  \SI{3/4}{\cup} grated Swiss cheese \\
  \SI{1/4}{\cup} finely chopped onion \\
  \numrange{2}{3} strips bacon, cut into \SI{1}{\inch} pieces
\end{ingredients}
Preheat oven to \SI{350}{\degree}. Beat the egg in a large bowl. Add evaporated milk,
sage, salt, and pepper.  Mix together. Add beef, crumbs, \SI{1/2}{\cup} of the
cheese and the onion. Blend. Form into a loaf and place in a \SI{2}{\quart}
rectangular dish.  Arrange bacon pieces on top of the loaf, and bake for
40 minutes. Sprinkle remaining \SI{1/4}{\cup} cheese on top and bake
40 minutes longer.
\end{entry}

%%---------------------------------------------------------------------------%%
\begin{entry}{Devilled Crabs}{First Edition}
\index{Seafood!devilled crabs}
\index{Johnson!Martha}
\index{Johnson!Dodge}
\index{Lamb!Martha}

\begin{open}
  This is from Dodge and Martha Johnson in memory of a great southern cook, Martha Hodgkins Niepold Lamb (Kate's Grandmother)and her mother, Mrs. Henry Bell Hodgkins. Funny story, this was actually submitted for this cookbook as well by Martha's sister Mimi, so you know its a family favorite! Spice amounts can be adjusted to taste.
\end{open}
\begin{ingredients}
  1 stick butter\\
  2 eggs, beaten\\
  \SI{2}{\tblspoon} flour\\
  \SI{1/2}{\teaspoon} Worcestershire sauce\\
  \SI{1}{\teaspoon} dry mustard\\
  \SI{1}{\tblspoon} vinegar\\
  pinch of sugar, salt, pepper, and MSG\\
  \SI{1}{\pound} crab\\
  bread crumbs \\
  paprika (optional) \\ 
\end{ingredients}
Preheat oven to \SI{350}{\degree}.  Melt butter. Mix eggs with flour and a little water and add
to butter. Continue mixing and add everything but paprika. Divide and stuff in scallop shells or
small casseroles and dot with butter and bread crumbs and if desired, sprinkle with paprika. Bake for 1 hour.
\begin{center}
    \includegraphics[width=.3\textwidth,clip]{figures/crab.pdf}
\end{center}
\end{entry}

%%---------------------------------------------------------------------------%%
\begin{entry}{Orange Game Hens}{First Edition}
\index{Chicken!orange game hens}
%\index{Nicholsons}
\index{Johnson!Martha}
\index{Johnson!Dodge}

\begin{open}
  This is a recipe courtesy of Martha and Dodge Johnson's friends, the Nicolsons. It's a great dish for company-especially at their house because they are such nice people and good cooks!
\end{open}
\begin{ingredients}
  2 or more Cornish game hens, whole or halved\\
  Joyce Chen's orange Szechuan sauce (or sub in soy sauce with orange
  concentrate)\\
  \SIrange{2}{3}{\tblspoon} of orange concentrate\\
  several \si{\tblspoon} white wine\\
  garlic powder\\
  ground ginger (fresh or frozen root is best)
\end{ingredients}
Preheat oven to \SI{350}{\degreeF}. Pour sauce, concentrate, and wine over
hens. Sprinkle with garlic powder and ginger. Cover and bake for
1 hour, and uncover the last 10 minutes. Good over a bed of rice.
\end{entry}

%%---------------------------------------------------------------------------%%
\begin{entry}{Chapel Hill Chicken Pie}{First Edition}
\index{Chicken!Chapel Hill chicken pie}
\index{Beef!Chapel Hill chicken pie}
\index{Evans!Kate}
\index{Johnson!Martha}
\index{Johnson!Dodge}

\begin{open}
  This is from Martha and Dodge, and Kate. Kate's addition is only the
  rosemary and measured amounts, for convenience (and my subtractions are
  those nasty onions) No one cares what or how much you put in, as long as you
  are happy. This is one of those recipes that you put some in, then you take
  some out (Nana, does this sound familiar?) This is also the kind of recipe
  where you vary it based on what you like or what's sitting in the fridge!
  It's best when you have leftover gravy along with meat from a past meal.
\end{open}
\begin{ingredients}
  \SI{2}{\cup} chopped meat (roast lamb, beef, or chicken)\\
  \numrange{3}{4} cubed and peeled potatoes\\
  \SIrange{1.5}{2}{\cup} gravy or combination of stock and wine\\
  \SI{2}{\tblspoon} flour\\
  \SI{1}{\teaspoon} salt\\
  \SI{2}{\teaspoon} pepper\\
  \SI{1}{\tblspoon} dried parsley\\
  \SI{1}{\teaspoon} garlic powder or 1 garlic clove\\
  \SI{1}{\teaspoon} thyme (if using chicken or beef)\\
  \SI{1}{\tblspoon} fresh rosemary \\
  \SI{1}{\teaspoon} marjoram (if using lamb)\\
  \SI{1}{\teaspoon} tarragon (if using chicken)\\
  Pie Crust (\corp{Betty Crocker's} mix is good, sorry
  \corp{Duncan Hines}, you don't make one!)
\end{ingredients}
\SIrange{1}{2}{\cup} each of your favorite vegetables, such as
\begin{ingredients}
  chopped carrots\\
  green beans\\
  celery\\
  onion\\
  mushrooms\\
  peas
\end{ingredients}
Preheat oven to \SI{450}{\degreeF}. Boil potatoes until somewhat cooked through, about
15 minutes. In a flat-ish casserole (\SIrange{1.5}{2}{\quart}), layer meat,
vegetables, and potatoes. sprinkle spices over, and then flour. Add gravy
mixture; adjust so the liquid comes up about half the height of the
ingredients.  Top with crust, seal edges, and add fork holes or vents. Brown
for 15 minutes, then then lower temperature to \SI{350}{\degree} and bake for
45 minutes longer. I find that I have to cover it for the last
15 minutes or so to keep the crust from getting too brown.
\end{entry}

%%---------------------------------------------------------------------------%%
\begin{entry}{Pasta with Prosciutto}{First Edition}
\index{Pasta!pasta w/ prosciutto}
\index{Johnson!Martha}
\index{Johnson!Dodge}

\begin{open}
  Martha and Dodge originally got this from the New York Times, but it has
  evolved. It's rather quick and satisfying. They say the order of tasks is a
  little tricky for non-Italian cooks. Luckily, half the family need not worry.
\end{open}
\begin{ingredients}
  \SI{3}{\cup} chopped plum tomatoes\\
  \numrange{2}{3} thinly sliced small zucchini\\
  \SIrange{1/8}{1/4}{\pound} prosciutto, cut into strips\\
  \SI{1}{\teaspoon} salt\\
  \SI{2}{\teaspoon} pepper\\
  \SI{1/2}{\teaspoon} red pepper flakes (optional)\\
  \SI{1}{\cup} whipping cream
  \SI{1/2}{\cup} chopped fresh basil\\
  \SI{1/4}{\cup} grated parmesam (use the real thing not the cylinder, people)\\
  1+ cloves garlic, chopped\\
  \SI{1}{\tblspoon} olive oil\\
  about \SI{3/4}{\pound} pasta
\end{ingredients}
Cook pasta. Save \SI{1/3}{\cup} cooking water. In frying pan, sear garlic, add
zucchini, prosciutto, salt and pepper, red pepper flakes, then tomatoes.  Stir
for \numrange{2}{3} minutes.  Add saved water, cream and simmer briefly. Add
pasta, basil, and Parmesan, and toss. Transfer to serving dish and eat
immediately (not difficult to do!). Serves \numrange{2}{3}.
\end{entry}

%%---------------------------------------------------------------------------%%
\begin{entry}{Pasta al Cavolfiore (with Cauliflower)}{First Edition}
\index{Pasta!pasta al cavolfiore}
\index{Vegetarian!pasta al cavolfiore}
\index{Johnson!Don}

\begin{open}
  Don sends this yummy looking dish from the Moosewood cookbook. Don
  says it's good and adds ``so enough of sending pasta recipes to the
  Italians.''
\end{open}
\begin{ingredients}
  1 onion (optional if you're Kate)\\
  1 cauliflower head, chopped into bite sizes\\
  1 tomato\\
  garlic to taste\\
  \SI{2}{\cup} grated cheese (see below)\\
  \SI{1/4}{\cup} olive oil\\
  1 can tomato sauce\\
  \SI{3/4}{\pound} pasta\\
  basil, dried and some fresh too if possible\\
  \SI{1}{\teaspoon} salt\\
  \SI{1}{\teaspoon} pepper
\end{ingredients}
Chop the onion and garlic and \saute them in \SI{1}{\teaspoon} oil with the
basil. When onion is clear, add cauliflower and cook until tender. (Don tip: add
a handful of water, and cover to speed this along.) Add chopped tomato, tomato
sauce, salt and pepper, and simmer for about twenty minutes. During this time,
cook and drain pasta. Add the remaining olive oil to pasta along with fresh
basil and half the cheese. Don recommends the cheese be a mixture of Parmesan,
Romano, mozzarella, and cheddar. Spread this on a big platter and top with the
cauliflower mixture. Top with remaining cheese. Don recommends a California
Gewurztraminer ``to go with.''  Serves \numrange{2}{3}.
\end{entry}

%%---------------------------------------------------------------------------%%
\begin{entry}{Sweet and Sour Pork}{First Edition}
\index{Pork!sweet and sour pork}
\index{Evans!Kate}
\index{Evans!Tom}

\begin{open}
  Tom and Kate eat this a lot; its a ``regular''. It's word-for-word from a
  Southern Living year-end cookbook I love (1992, if curious). Its quite
  delicious, and its even somewhat healthy.
\end{open}
\begin{ingredients}
  \SI{1}{\tblspoon} sherry\\
  \SI{1}{\tblspoon} soy sauce\\
  \SI{1}{\tblspoon} cornstarch\\
  \SI{1}{\pound} boneless pork, cut into cubes\\
  \SI{1/4}{\cup} vegetable oil, divided\\
  1 clove garlic, minced\\
  1 small onion (optional)\\
  2 green peppers, cut into \SI{1}{\inch} pieces\\
  \SI{1/3}{\cup} sugar\\
  \SI{1/4}{\cup} ketchup\\\
  \SI{1}{\tblspoon} sherry\\
  \SI{2}{\tblspoon} soy sauce\\
  \SI{2}{\tblspoon} white vinegar\\
  \SI{1}{\tblspoon} cornstarch\\
  \SI{1/3}{\cup} water\\
  1 \SI{8}{\ounce} can pineapple slices in juice, each cut into about 8 pieces
\end{ingredients}
%
\protip{Using fresh pineapple makes this dish so much better!}
%
Combine first 3 ingredients, add pork, and let marinate 20 minutes (or however
long it takes to prepare everything else). Heat \SI{2}{\tblspoon} oil in big
frying pan.  Stir fry onion garlic, and green pepper over med-high heat until
crisp tender.  Remove from skillet. Add rest of oil and cook pork until cooked
through.  Stir in cooked vegetables. Combine sugar and next 6 ingredients,
stirring until cornstarch dissolves. Add to pork mixture and cook until it
comes to a boil. Add pineapple and and boil for about 1 minute. Serve over hot
cooked rice. Serves \numrange{2}{3} hungry people.
\end{entry}
%\chapter{Treats}

\section{Fermina's Ginger Snaps}
\index{cookies!gingersnaps}
\index{Evans!Fermina}

\begin{open}
  This is a super-yummy cookie recipe sent in from Fermina.  She tells us that
  she always doubles this recipe when making a batch. Maybe the increased
  amounts of ingredients help the taste factor.  Either that or we're just
  gluttons.  Here's the recipe.
\end{open}
\begin{ingredients}
  \SI{3/4}{\cup} of shortening \\
  \SI{1}{\cup} sugar \\
  \SI{1/4}{\cup} light molasses \\
  1 slightly beaten egg \\
  \SI{2}{\cup} flour \\
  \SI{1/4}{\teaspoon}  salt \\
  \SI{1}{\teaspoon}  cinnamon \\
  \SI{2}{\teaspoon} soda \\
  \SI{1}{\teaspoon}  clove \\
  \SI{1/2}{\teaspoon} ginger
\end{ingredients}
Cream shortening and sugar, add molasses and egg.  Mix all dry ingredients.
Stir dry ingredients into creamed mixture.  Spoon into balls. Added step for
yumminess: \textit{Drop spoonfuls into sugar before putting on baking sheet}.
One spray of water before baking.  Bake \SI{350}{\degree} for \numrange{8}{10}
minutes.  The cookies should be split in the middle when finished.

\section{Betsy's Chocolate Chip Poundcake}
\index{cakes!chocolate chip pound cake}
\index{Evans!Betsy}

\begin{open}
  This is the famous Betsy Cordova's Chocolate Chip Cake. When she e-mailed this
  to use she sent a request for many treat recipes.  What we tell her we tell
  all.  Send us a recipe and you get a book.  This serves however many you feel
  like depending on your hunger.
\end{open}
\begin{ingredients}
  \SI{3}{\cup} sugar \\
  2 sticks butter \\
  6 eggs \\
  \SI{3}{\cup}s flour  \\
  1 carton heavy whipping cream (small size) \\
  \SI{2}{\tblspoon}  vanilla \\
  \num{1/2} bag mini chocolate chips
\end{ingredients}
Cream butter and sugar, add 2 eggs and \SI{1}{\cup} flour and beat.  Add 2
eggs and \SI{1}{\cup} flour and beat. Add 2 eggs and \SI{1}{\cup} flour and
beat.  Mix in vanilla and whipping cream and add chocolate chips.

Bake in greased and floured bundt pan at \SI{350}{\degree} for 60 to 75
minutes (depending on the temperature of your oven).
\begin{center}
\includegraphics[scale=.5,clip]{figures/pound.pdf}
\end{center}

\section{Amy's Cheesecake}
\index{cakes!cheesecake}
\index{DeLellis, Amelia}

\begin{open}
  This is a holiday favorite at the Evans/DeLellis households by Amelia
  DeLellis.
\end{open}
\begin{ingredients}
  1 box Graham Cracker Crumbs \\
  \SI{1/2}{\cup} sugar \\
  \SI{2}{\tblspoon}  flour \\
  \SI{1/4}{\teaspoon}  salt \\
  \SI{1}{\pound} cream cheese \\
  \SI{1}{\teaspoon}  vanilla extract \\
  4 eggs \\
  \SI{1}{\cup} heavy cream
\end{ingredients}
The topping ingredients are:
\begin{ingredients}
  \SI{2}{\cup}s sour cream \\
  \SI{3}{\tblspoon}  sugar \\
  \SI{1}{\teaspoon} vanilla
\end{ingredients}
Follow the directions on the Graham Cracker Box for the crust.  Use a
\SI{9}{\inch} spring form pan.  Press crumb mixture into the bottom and sides
of the pan.

Let cream cheese soften at room temperature (or use microwave).  Mix sugar,
flour, and salt.  Add dry ingredients to cream cheese.  Cream together with
low speed beater or by hand.  Separate eggs, save the whites in a clean bowl.
Add yolks to cream cheese mixture and beat until smooth.  Add vanilla.  Stir
in cream.  Beat egg whites until stiff.  Fold into cream cheese mixture.  Pour
on top of crumbs.  Bake at \SI{350}{\degree} for 1 hour.  Let cool.  Mix
topping ingredients.  Pour topping onto cheesecake and bake at
\SI{500}{\degree} for 10 minutes.  Serve with cherry, blueberry,
etc. etc. toppings.

\section{Pineapple Upside-down Cake}
\index{cakes!pineapple upside-down cake}
\index{Evans!Fermina}

\begin{open}
  Kate: On his/her birthday most kids I knew asked for chocolate cake, or
  ice-cream cake, or even cheesecake if sophisticated. But not Tom.  Tom always
  begged for this somewhat unusual birthday cake. Luckily, Fermina Evans has a
  great recipe for it!  Tom: I begged for it because it's delicious. Any kid
  would agree.
\end{open}
\begin{ingredients}
  \SI{1/2}{\cup} butter \\
  \SI{1/2}{\cup} packed brown sugar \\
  1 large can pineapple slices in syrup \\
  1 small jar maraschino cherries
  \SI{1/2}[1]{\cup} non packed flour (softasilk flour recommended) \\
  \SI{1}{\cup} sugar \\
  \SI{2}{\teaspoon} baking powder \\
  \SI{1/2}{\teaspoon} lt \\
  \SI{1/3}{\cup} soft shortening \\
  \SI{2/3}{\cup} milk \\
  \SI{1}{\teaspoon} vanilla \\
  1 large egg
\end{ingredients}
Melt butter with brown sugar in \SI{9}{\inch} baking pan (Fermina adds
\SI{1}{\tblspoon} Karo syrup here) Arrange pineapple slices on top of syrup
and place cherries in pineapple centers or wherever they look nice.

In mixing bowl, stir flour, sugar, baking powder and salt. Add shortening,
milk, and vanilla. Beat 2 minutes at medium speed with electric mixer. Add egg
and beat two more minutes. Pour batter over fruit. Bake at \SI{350}{\degree}
for \numrange{40}{50} minutes. Immediately turn upside down on serving dish
(if you don't, sugar will crystallize to pan and you will have a mess).

\section{Chocolate Mint Brownies}
\index{brownies!chocolate mint brownies}
\index{Evans!Fermina}

\begin{open}
  Kate: Blah blah blah chocolate blah. Tom: These are yummy!  My mom makes
  them.
\end{open}
\begin{ingredients}
  \SI{1}{\cup} sugar\\
  \SI{1/2}{\cup} butter or margarine\\
  4 eggs, beaten\\
  \SI{1}{\cup} flour\\
  \SI{1/2}{\teaspoon} salt\\
  1 can \corp{Hershey's Chocolate Syrup} (\SI{16}{\ounce})\\
  \SI{1}{\teaspoon} vanilla
\end{ingredients}
Mix together above ingredients and put in a greased \SI{9x13}{\inch} pan.
Bake at \SI{350}{\degree} for 30 minutes.

The middle layer ingredients are:
\begin{ingredients}
  \SI{2}{\cup} powdered sugar\\
  \SI{1/2}{\cup} butter or margarine\\
  \SI{2}{\tblspoon} \corp{Creme de Menthe} (preferably the green kind)
\end{ingredients}
Mix and spread over cooled cake.

The glaze ingredients are:
\begin{ingredients}
  \SI{1}{\cup} chocolate chips\\
  \SI{6}{\tblspoon} butter
\end{ingredients}
Let the cake cool slightly and spread over brownies.  Chill and cut into
squares.

\section{Gooey Butter Cake}
\index{cakes!gooey butter cake}
\index{Lefkowith, Pam}

\begin{open}
  A recipe from Fermi's friend Pam L.
\end{open}
\begin{ingredients}
  1 pkg. yellow cake mix\\
  \SI{1/2}{\cup} melted butter\\
  1 egg
\end{ingredients}
The topping ingredients are:
\begin{ingredients}
  \SI{8}{\ounce} cream cheese (1 pkg.)\\
  2 eggs\\
  1 box powdered sugar
\end{ingredients}
Preheat oven to \SI{350}{\degree}.  Mix together cake ingredients.  Pat into
\SI{9x13}{\inch} pan.  Beat topping ingredients together for three minutes.
Pour over cake mix.  Bake for 40 minutes.  Do not over-bake. Top should be
set, but not dry.

\section{Chocolate Surprise}
\index{cakes!chocolate surprise}
\index{Evans!Fermina}

\begin{open}
  Fermina's chocolate \& angel food cake.  Geo's birthday favorite.
\end{open}
\begin{ingredients}
  32 large marshmellows\\
  \SI{1/3}{\cup} water\\
  \SI{1/4}{\teaspoon} salt\\
  \SI{6}{\ounce} semi-sweet chocolate chips\\
  heavy whipping cream and sugar\\
  \SI{1/4}{\teaspoon} vanilla
\end{ingredients}
In sauce pan melt marshmallows, water, and salt.  Add chocolate bits.  Stir
until melted.  Let cool.

\begin{wrapfigure}{R}{.25\textwidth}
\centering\includegraphics[width=.22\textwidth,clip]{figures/kiss.pdf}
\end{wrapfigure}

Whip \SI{1}{\cup} whipping cream.  Pour chocolate over cream and fold
together.

Take a store brought or home make angel food cake.  Cut off entire top
\SI{1}{\inch} layer.  With spoon dig a tunnel in remaining cake.  Fill with
chocolate surprise.  Replace top layer.

Sprinkle cake with powdered sugar or frost with whipped cream (\SI{1}{\cup}
heavy cream beaten with \SI{1}{\tblspoon} sugar until stiff).

\section{Tiramisu}
\index{cakes!tiramisu}
\index{Horne!Mimi}

\begin{open}
  Mimi Horne brings us this delicious treat from an Italophile Brazilian Yale Art History Professor friend, Ester da Costa Meyer.  Those in less gastro-enlightened regions might need to replace the mascarpone with cream cheese and cream, and the Marsala with perhaps port or sherry.
\end{open}
\begin{ingredients}
  2-3 pkgs (about 24-30) lady fingers (boudoirs) \\
  \SI{3}{\cup} mascarpone (or part creme fraiche, part carre frais mushed together)\\
  3 egg yolks \\
  \SI{1/3}{\cup} sugar \\
  \SI{2}{\cup} strong coffee \\
  \SI{1/2}{\cup} or more Marsala wine \\
  \SI{1/2}{\cup} cocoa
\end{ingredients}
Prepare coffee and mix with Marsala. One at a time, dip lady fingers in
mixture briefly, then lay them in a row in an approx. \SI{10x18x2.5}{\inch}
deep serving dish.  Cut some to fit the remaining space in dish so that the
bottom is completely covered. Mix sugar, egg yolks and
mascarpone/cream. Spread about half the mixture over the first layer of
cookies to cover completely.  Dip more lady fingers in coffee/Marsala and lay
them over cream to form the next layer; cover remaining cream mixture. Dust
the top thoroughly with cocoa; chill overnight or for several hours before
serving. More cocoa may be added before serving. The texture can be made
lighter by beating the egg whites and folding them into the mascarpone
mixture, which also increases the amount.  Serves \numrange{6}{8}.

\section{Tortelettes}
\index{cookies!tortelettes}
\index{Niepold!Martha}

\begin{open}
  Another Nonnie/Grandpatty and Joy of Cooking original. Niepold kids remember Christmas at Lee St. when they eat Tortelettes and California dates stuffed with Georgia Pecans and rolled in confectioners' sugar.
\end{open}
\begin{ingredients}
  1 grated lemon rind\\
  \SI{1}{\cup} sugar\\
  \SI{3/4}{\cup} butter\\
  2 egg yolks \\
  \SI{1/2}{\cup} bread flour \\
  \SI{1}{\cup} blanched and shredded almonds or pecan pieces \\
  \SI{1/3}{\cup} sugar \\
  \SI{1}{\teaspoon} cinnamon\\
  \SI{1/4}{\teaspoon} nutmeg\\
  \SI{1/8}{\teaspoon} salt\\
  1 egg white\\
  \SI{1}{\tblspoon} water
\end{ingredients}
Preheat oven to \SI{375}{\degreeF}. Grate lemon into sugar. Cream sugar with butter and beat in the egg yolks one at a time. Add flour gradually to make a stiff dough. Pinch off about a teaspoonful of dough at a time. Roll it into a
ball and flatten on cookie sheet until very thin. Prepare nuts and combine with next 4 ingredients (spices). Beat the egg white and water together slightly.  Brush the cakes with the egg white mixture, then sprinkle nut/spice mixture and bake until light brown.

\section{Lime Cream Pie}
\index{pie!lime cream pie}
\index{Evans!Kate}

\begin{open}
  Kate got this recipe from Edie, a receptionist at Bryn Mawr College with southern cooking blood.  It's very easy and delicious, especially after a rich meal. It's cool, refreshing, and slides right down.
\end{open}
\begin{ingredients}
  1 \corp{Graham Cracker Pie Crust}\\
  3 egg yolks\\
  \SI{2/3}[2]{\cup} sweetened condensed milk (2 cans)\\
  \SI{1}{\cup} plus \SI{2}{\tblspoon} lime juice (about 7 limes if you're
  squeezing)\\
  \SI{2}{\teaspoon} grated lime zest\\
  1 attractive lime
\end{ingredients}
Lightly whisk egg yolks in mixing bowl. Pour in condensed milk and whisk until
completely blended.  Add lime juice and zest and whisk to blend. Gently pour
filling into pie crust shell and smooth over top.  Refrigerate for at least 4
hours (don't skimp or it will be soup!). Slice the attractive lime paper thin
to garnish. I like to slice into half-circles and create a pinwheel pattern
around the center.


\appendix

\titleformat{\section}{\Large\bfseries}{\thesection}{1em}{}

\newpage
\section{Family Information}

\subsection{List of Contributors}

\begin{enumerate}
 \color{red}
 \itemsep0pt
 \parskip1pt
\item Betsy and Richard Cordova %\textit{411 Arrowhead Trail, Sinking Spring PA  19608}.
\item Angelo and Evelina DeLellis
\item George, Sr. and Mickey Evans
\item Fermina and George Evans % \textit{411 Arrowhead Trail, Sinking Spring PA, 19608.}
\item Katie Evans
\item Steve and Joyce Evans
\item Scott Evans
\item Tom and Kate Evans %\textit{11112 Windward Dr, Knoxville, TN 37934}.
\item Amelia DeLellis
\item Lauren Anton
\item Paul and Mimi Horne %\textit{28 ave. R Poincar\'{e}, 75116 Paris, France}.
\item Rich and Jenn King.
\item Don and Lil Johnson
\item Donald and Geyi Johnson
\item David and Dottie Lindquist
\end{enumerate}

\subsection{Family Trees}

{\color{red}
We realized while writing this recipe book that many of you don't know people from either Tom's or Kate's extended families.  To resolve this Tom proposed a biblical-style section, ie. Donald begot Dodge who was the Father of Kate...  Kate, on the other hand hated that format, and, for once, Tom agreed.  Therefore, we settled on family trees because they remind Tom of Feynman Diagrams and because Kate thinks they look like something having to do with weather.  Anyway, these are meant for people from one side of the hill to see who is related to who on the other side.
They are not conclusive and only reflect the situation as it is now.
}

\begin{figure}
    \centering
    \includegraphics[width=.9\textwidth,clip]{figures/Johnson-Tree}
    \caption{Johnson Family Tree.}
    \label{fig:johnson-tree}
\end{figure}

\begin{figure}
    \centering
    \includegraphics[width=.9\textwidth,clip]{figures/DeLellis-Tree}
    \caption{DeLellis Family Tree.}
    \label{fig:delellis-tree}
\end{figure}

%% >>> BACKMATTER

%\backmatter
\printindex

\end{document}
