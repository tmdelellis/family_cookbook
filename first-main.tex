\chapter{Entrees}

\section{Scott's Killer Chili\index{entrees!chili}}

\begin{open}
  I hope you're prepared for this.  This recipe from Scott Evans makes a {\em
  thick} and {\em spicy} chili.  In the word's of the author ``It is pretty
  spicy.''  This is one of those recipes that should include a Disney-style
  warning label, ``\textellipsis those with heart conditions or over the age of
  sixty-five...etc. etc.''  This is this recipe's first time in print so some
  experimentation may be required.  Supposedly this is a campout recipe, however
  I see no way that anyone could possibly carry all these ingredients. The
  recipe makes about 16 servings or less for REALLY big people.  Good luck and
  here it goes.
\end{open}
\begin{ingredients}
  \SI{3}{\pound} of hot Italian sausage (ie. Hot Cincinnati Brand, that homer) \\
  \SI{3}{\pound} bacon \\
  3 large onions \\
  3 bell peppers (2 green, 1 red) \\
  \numrange{4}{5} cloves of garlic \\
  \numrange{4}{5} hot peppers (a cornucopia of jalapenos, habeneros, and others) \\
  3 cans Italian pear tomatoes \\
  \SI{1}{\tblspoon} olive oil \\
  \SI{1}{\tblspoon} mustard powder \\
  \SI{1}{\tblspoon} celery seed \\
  \SI{1}{\tblspoon} chili powder \\
  \SI{1}{\tblspoon} bay leaves \\
  \SI{1}{\tblspoon} Worcester sauce \\
  \SI{1}{\tblspoon} vinegar \\
  red wine \\
  water \\
  salt and pepper
\end{ingredients}
Start with a large Dutch oven and a campfire right after breakfast.  Fry the
sausage and set aside.  Fry the bacon and set aside.  Leave a bit of grease in
the pot and add the minced garlic followed by the roughly chopped onions and
bell peppers (no bell pepper seeds).  Chop the hot peppers and add to the pot,
remember that the seeds make the dish VERY spicy.  Add olive oil as needed.
%%
\begin{wrapfigure}{R}{.2\textwidth}
\centering
\includegraphics[width=.15\textwidth, trim=.5in .25in .5in .25in, clip]{figures/chilli}
\end{wrapfigure}
%%
Add about one Tbsp full each of: mustard powder, celery seed, Worcester sauce,
vinegar, and chili powder.  This may require some experimentation to alter to
your taste.  Stir and cook until onions become clear and peppers begin to
soften.  Add up to one cup of red wine.  Next add tomatoes and juices.  Stir
and chop tomatoes.  Add sausage, bacon, and two bay leaves.  Season with salt
and pepper.  Now everything should look a bit like chunky soup, but don't
worry.

Let the chili simmer over low heat for a minimum of three hours, but try for
eight (trust Scott on this one). Check periodically and stir.  If mixture
thickens too much add some water.  Taste and adjust to preference.

Serve with shredded cheddar cheese and garlic bread.  This recipe freezes well
in personalized zip-lock bags (In case you're not hungry enough to eat six
pounds of meat in one sitting).

\section{Chicken Breasts with Orange Sauce\index{entrees!chicken,
orange sauce}}

\begin{open}
  This is a recipe from Lil and Don Johnson.  Grammie (Lil) passed it to Martha,
  who passed it to Kate, and so on, and so on, and so on\textellipsis It's good.
\end{open}
\begin{ingredients}
  4 halved chicken breasts \\
  1 small can undiluted O.J. concentrate \\
  1 package Lipton's Onion Soup Mix \\
  paprika
\end{ingredients}
In a long baking pan arrange the 8 pieces of chicken.  Pour the O.J.
concentrate (at room temperature) over the chicken.  Sprinkle the soup mix
over the chicken.  Add a little paprika for seasoning.

Cover pan with foil.  Bake at \SI{350}{\degree} for 40 minutes.  Remove foil and
baste chicken.  Back for an additional 20 minutes uncovered.

\section{Porcupine Meatballs\index{entrees!meatballs}}

\begin{open}
  This is a recipe from Mickey and George, Sr. Evans.
\end{open}
\begin{ingredients}
  \SI{1/2}[1]{\pound} hamburger \\
  \SI{3/4}{\cup} uncooked rice \\
  \SI{1}{\teaspoon} salt \\
  1 egg \\
  \SI{1/2}{\teaspoon} pepper \\
  \SI{1/4}{\cup} chopped onion \\
  \SI{1/2}[2]{\cup} stewed tomatoes \\
  \SI{1}{\teaspoon} chili \\
  \SI{1}{\teaspoon} sugar
\end{ingredients}
Combine hamburger, rice, salt, egg, pepper, and onion.  Shape into
\SI{1/2}[1]{\inch} balls.

Heat sauce and chili to boiling in a kettle.  Drop balls in sauce.  Simmer for
1\num{1/2} hour covered.  Strips of bacon may be wrapped around meatballs
and secured with toothpicks before cooking in sauce.

\section{Lemon-Herb Chicken\index{entrees!lemon-herb chicken}}

\begin{open}
  This is Fermina's favorite.
\end{open}
\begin{ingredients}
  1 chicken (cut) or \SI{1/2}[3]{\pound} of chicken parts      \\
  \SI{1/2}{\cup} olive oil                                     \\
  \SI{1/4}{\cup} lemon juice                                   \\
  2 garlic cloves minced                                       \\
  \SI{3}{\tblspoon} chopped fresh oregano or \SI{1}{\tblspoon} dry \\
  \SI{1/2}{\teaspoon} salt                                     \\
  \SI{1/8}{\teaspoon} pepper                                   \\
  \SI{1}{\tblspoon}chopped fresh rosemary or \SI{1}{\teaspoon} dry \\
  \SI{2}{\tblspoon}chopped fresh parsley
\end{ingredients}
Place chicken meaty side down in a \SI{13x9x2}{\inch} baking pan.  Combine
next 8 ingredients, mix well.  Pour mixture over chicken.  Marinate in
refrigerator for two hours.  Bake uncovered at \SI{350}{\degree} for 40 minutes.
Turn chicken.  Broil 6~inches from heat for \numrange{5}{10} minutes or
until crisp and lightly browned.

\section{Stuffed Flank Steak Teriyaki\index{entrees!flank steak
teriyaki}}

\begin{open}
  This belongs because it is MY favorite and since I'm (Tom) in charge,
  well there you go.  Makes \numrange{4}{5} servings.
\end{open}
\begin{ingredients}
  1 medium to large beef flank steak (1\SIrange{1/4}{2}{\pound}) \\
  \SI{1/2}{\cup} soy sauce                                    \\
  \SI{1/4}{\cup} cooking oil                                  \\
  \SI{2}{\tblspoon} molasses                                   \\
  \SI{2}{\teaspoon} dry mustard                               \\
  \SI{1}{\teaspoon} ginger root or \SI{1/2}{\teaspoon} dry ginger \\
  1 clove garlic minced                                       \\
  \SI{1}{\cup} water                                          \\
  \SI{1/2}{\cup} long-grain rice                              \\
  \SI{1/2}{\cup} of shredded carrots                          \\
  \SI{1/2}{\cup} sliced water chestnuts (optional)            \\
  \SI{1/4}{\cup} sliced green onions
\end{ingredients}
\begin{wrapfigure}{L}{.3\textwidth}
\centering\includegraphics[width=.25\textwidth,clip]{figures/flank}
\end{wrapfigure}
Cut a large pocket in flank steak or have your butcher do it.  Combine soy
sauce, oil, molasses, mustard, ginger, and garlic.  Place meat in shallow pan
or plastic bag.  Pour marinade into pocket and over meat. Let stand at room
temp (\SI{300}{\kelvin}) for 30 minutes or in refrigerator for
\numrange{2}{3} hours.

In saucepan combine water, rice, carrots, water chestnuts, and green onion.
Bring to a boil; reduce heat and simmer while covered for 8 minutes.  Remove
from heat and set aside.

Drain meat reserving marinade.  Add \SI{1/4}{\cup} of reserved marinade to the
rice mixture.  Spoon rice stuffing into pocket of meat.  Secure end with
wooden toothpicks.  Place meat in shallow roasting pan and cover with foil.
Bake at \SI{350}{\degree} for 1 hour until meat is done.

Fermina allows an extra \numrange{10}{15} minutes without foil to brown meat.
Brush with marinade while browning and check often.  Slice meat diagonally
across grain to serve.

\section{Better-than-vegetarian Pasta Sauce\index{entrees!better-than-veggie
pasta sauce}}

\begin{open}
  This from Kate's brother Don, and he got it from his friend Dawn
  Ollila. The honey/brown sugar and cinnamon addition is what makes it taste
  so special.
\end{open}
\begin{ingredients}
  \SI{1}{\tblspoon}olive oil \\
  1 onion \\
  1 green pepper \\
  2 cloves of minced garlic \\
  1 tomato\\
  1 large or two small carrots OR \SI{1/4}{\cup} cup lentil beans \\
  1 or 2 cans of tomato sauce \\
  thyme \\
  basil \\
  oregano \\
  rosemary \\
  \SI{1}{\tblspoon}honey or brown sugar \\
  cinnamon \\
  salt and pepper \\
\end{ingredients}
Saute the 4 vegetables in olive oil, adding them as ordered above. When the
onion is clear and the tomato is soft, add the tomato sauce.  Bring to a
simmer. The sauce is now tasty, but to thin to stick to the pasta.  Choose
either the carrots or lentils to give it body. Lentils add a great dark,
almost meaty, flavor but you will need to boil them in 4 times their
measurement in boiling water for \numrange{45}{90} minutes first
(no presoaking required). Do not add them to the sauce until they are
bean-like mush. or, you can grate the carrot as finely as you have the
technology to do and add to the sauce. The taste is minimal, but the texture
is great. Add the rest of the ingredients and adjust to taste. Add just enough
cinnamon to make your guests look at you funny and say, ``What did you put in
this?'' The flavor actually works quite well.

\section{Peachy Chicken\index{entrees!peachy chicken}}

\begin{open}
  From Fermi, this is Betsy's favorite.  Tom: Actually I really don't
  like this recipe and the only reason she ``likes'' this one is because she
  hates my favorite recipe (stuffed flank steak).  Note: the ``peachy'' in
  ``peachy chicken'' is not a southern thing.  Kate: When Tom and I were first
  married (Tom interjects: forty years ago) I made a chicken dish with fruit.
  He honestly thought I was trying to annoy him (how did he know?).
\end{open}
\begin{ingredients}
  chicken parts for \numrange{4}{6} people\\
  1 large can peach halves (drained, reserve syrup)\\
  \SI{2}{\tblspoon} soy sauce\\
  \SI{2}{\tblspoon} lemon juice\\
  \SI{1/2}{\tblspoon} ginger\\
  2 cloves minced garlic
\end{ingredients}
Preheat oven to \SI{375}{\degree}.  Mix last four ingredients with reserved peach
syrup.  Pour marinade over chicken parts in roasting or baking pan. Bake in
oven for \numrange{45}{60} minutes. Turn once.  Add peach halves last
15 minutes.  Brown under broiler for a few minutes if further browning
is needed.
\begin{center}
    \includegraphics[scale=.5,clip]{figures/meatloaf}
\end{center}

\section{Swiss Meatloaf\index{entrees!swiss meatloaf}}
\label{sec:swiss-meatloaf}

\begin{open}
  This was contributed by Joyce Evans, and is definitely ``comfort
  food''! Serves 6.
\end{open}
\begin{ingredients}
  1 egg \\
  \SI{1/2}{\cup} evaporated milk \\
  \SI{1}{\teaspoon} rubbed sage \\
  \SI{1}{\teaspoon} salt \\
  \SI{1/2}{\teaspoon} black pepper \\
  \SI{1/2}[1]{\pound} lean ground beef \\
  \SI{1}{\cup} cracker crumbs (round buttery type, approx. 24) \\
  \SI{3/4}{\cup} grated Swiss cheese \\
  \SI{1/4}{\cup} finely chopped onion \\
  \numrange{2}{3} strips bacon, cut into 1 in. pieces \\
\end{ingredients}
Preheat oven to \SI{350}{\degree}. Beat the egg in a large bowl. Add evaporated milk,
sage, salt, and pepper.  Mix together. Add beef, crumbs, \SI{1/2}{\cup} of the
cheese and the onion. Blend. Form into a loaf and place in a \SI{2}{\quart}
rectangular dish.  Arrange bacon pieces on top of the loaf, and bake for
40 minutes. Sprinkle remaining \SI{1/4}{\cup} cheese on top and bake
40 minutes longer.

\section{Devilled Crabs\index{entrees!devilled crabs}}

\begin{open}
  This is from Dodge and Martha Johnson in memory of a great southern
  cook, Martha Hodgkins Niepold Lamb (Kate's Grandmother).  Don't be scared by
  the lack of ingredient quantities. Its easy and delicious.  Just adjust
  until it tastes good.
\end{open}
\begin{ingredients}
  1 stick butter\\
  2 eggs\\
  a little flour and water\\
  Worcestershire sauce\\
  dry mustard\\
  vinegar\\
  pinch of sugar, salt, pepper, and MSG\\
  \SI{1}{\pound} crab
\end{ingredients}
Preheat oven to \SI{350}{\degree}.  Melt butter. Mix eggs with flour and water and add
to butter. Continue mixing and add everything else. stuff in scallop shells or
small casseroles and dot with butter and bread crumbs. Bake for 1 hour.
\begin{center}
    \includegraphics[width=.3\textwidth,clip]{figures/crab}
\end{center}

\section{Orange Game Hens\index{entrees!Orange Game Hens}}

\begin{open}
  This is a recipe courtesy of Martha and Dodge Johnson's friends, the
  Nicolsons. It's a great dish for company-especially at their house because
  they are such nice people and good cooks!
\end{open}
\begin{ingredients}
  2 or more Cornish game hens, whole or halved\\
  Joyce Chen's orange Szechuan sauce (or sub in soy sauce with orange
  concentrate)\\
  \SIrange{2}{3}{\tblspoon} of orange concentrate\\
  several \si{\tblspoon} white wine\\
  garlic powder\\
  ground ginger (fresh or frozen root is best)
\end{ingredients}
Preheat oven to \SI{350}{\degree}. Pour sauce, concentrate, and wine over
hens. Sprinkle with garlic powder and ginger. Cover and bake for
1 hour, and uncover the last 10 minutes. Good over a bed of rice.

\section{Chapel Hill Chicken Pie\index{entrees!Chapel Hill Chicken Pie}}

\begin{open}
  This is from Martha and Dodge, and Kate. Kate's addition is only the rosemary
  and measured amounts, for convenience (and my subtractions are those nasty
  onions) No one cares what or how much you put in, as long as you are happy.
  This is one of those recipes that you put some in, then you take some out
  (Nana, does this sound familiar?) This is also the kind of recipe where you
  vary it based on what you like or what's sitting in the fridge! It's best when
  you have leftover gravy along with meat from a past meal.
\end{open}
\begin{ingredients}
  \SI{2}{\cup} chopped meat (roast lamb, beef, or chicken)\\
  \numrange{3}{4} cubed and peeled potatoes\\
  \SIrange{1.5}{2}{\cup} gravy or combination of stock and wine\\
  \SI{2}{\tblspoon} flour\\
  \SI{1}{\teaspoon} salt\\
  \SI{2}{\teaspoon} pepper\\
  \SI{1}{\tblspoon} dried parsley\\
  \SI{1}{\teaspoon} garlic powder or 1 garlic clove\\
  \SI{1}{\teaspoon} thyme (if using chicken or beef)\\
  \SI{1}{\tblspoon} fresh rosemary \\
  \SI{1}{\teaspoon} marjoram (if using lamb)\\
  \SI{1}{\teaspoon} tarragon (if using chicken)\\
  Pie Crust (\corp{Betty Crocker's} mix is good, sorry
  \corp{Duncan Hines}, you don't make one!)
\end{ingredients}
\SIrange{1}{2}{\cup} each of your favorite vegetables, such as
\begin{ingredients}
  chopped carrots\\
  green beans\\
  celery\\
  onion\\
  mushrooms\\
  peas\\
\end{ingredients}
Preheat oven to \SI{450}{\degree}. Boil potatoes until somewhat cooked through, about
15 minutes. In a flat-ish casserole (\SIrange{1.5}{2}{\quart}), layer meat,
vegetables, and potatoes. sprinkle spices over, and then flour. Add gravy
mixture; adjust so the liquid comes up about half the height of the
ingredients.  Top with crust, seal edges, and add fork holes or vents. Brown
for 15 minutes, then then lower temperature to \SI{350}{\degree} and bake for
45 minutes longer. I find that I have to cover it for the last
15 minutes or so to keep the crust from getting too brown.

\section{Pasta with Prosciutto\index{entrees!Pasta with Prosciutto}}

\begin{open}
  Martha and Dodge originally got this from the New York Times, but it has
  evolved. It's rather quick and satisfying. They say the order of tasks is a
  little tricky for non-Italian cooks. Luckily, half the family need not worry.
\end{open}
\begin{ingredients}
  \SI{3}{\cup} chopped plum tomatoes\\
  \numrange{2}{3} thinly sliced small zucchini\\
  \SIrange{1/8}{1/4}{\pound} prosciutto, cut into strips\\
  \SI{1}{\teaspoon} salt\\
  \SI{2}{\teaspoon} pepper\\
  \SI{1/2}{\teaspoon} red pepper flakes (optional)\\
  \SI{1}{\cup} whipping cream
  \SI{1/2}{\cup} chopped fresh basil\\
  \SI{1/4}{\cup} grated parmesam (use the real thing not the cylinder, people)\\
  1+ cloves garlic, chopped\\
  \SI{1}{\tblspoon} olive oil\\
  about \SI{3/4}{\pound} pasta
\end{ingredients}
Cook pasta. Save \SI{1/3}{\cup} cooking water. In frying pan, sear garlic, add
zucchini, prosciutto, salt and pepper, red pepper flakes, then tomatoes.  Stir
for \numrange{2}{3} minutes.  Add saved water, cream and simmer briefly. Add
pasta, basil, and Parmesan, and toss. Transfer to serving dish and eat
immediately (not difficult to do!). Serves \numrange{2}{3}.

\section{Pasta al Cavalfiore (with Cauliflower)
\index{entrees!Pasta al Cavalfiore}}

\begin{open}
  Don sends this yummy looking dish from the Moosewood cookbook. Don
  says it's good and adds ``so enough of sending pasta recipes to the
  Italians.''
\end{open}
\begin{ingredients}
  1 onion (optional if you're Kate)\\
  1 cauliflower head, chopped into bite sizes\\
  1 tomato\\
  garlic to taste\\
  \SI{2}{\cup} grated cheese (see below)\\
  \SI{1/4}{\cup} olive oil\\
  1 can tomato sauce\\
  \SI{3/4}{\pound} pasta\\
  basil, dried and some fresh too if possible\\
  \SI{1}{\teaspoon} salt\\
  \SI{1}{\teaspoon} pepper
\end{ingredients}
Chop the onion and garlic and saute them in \SI{1}{\teaspoon} oil with the
basil. When onion is clear, add cauliflower and cook until tender. (Don tip:
add a handful of water, and cover to speed this along.) Add chopped tomato,
tomato sauce, salt and pepper, and simmer for about twenty minutes. During
this time, cook and drain pasta. Add the remaining olive oil to pasta along
with fresh basil and half the cheese. Don recommends the cheese be a mixture
of Parmesan, Romano, mozzarella, and cheddar. Spread this on a big platter and
top with the cauliflower mixture. Top with remaining cheese. Don recommends a
California Gewurztraminer ``to go with.''  Serves \numrange{2}{3}.

\section{Sweet and Sour Pork\index{entrees!Sweet and Sour Pork}}

\begin{open}
  Tom and Kate eat this a lot; its a ``regular''. It's word-for-word from a
  Southern Living year-end cookbook I love (1992, if curious). Its quite
  delicious, and its even somewhat healthy.
\end{open}
\begin{ingredients}
  \SI{1}{\tblspoon} sherry\\
  \SI{1}{\tblspoon} soy sauce\\
  \SI{1}{\tblspoon} cornstarch\\
  \SI{1}{\pound} boneless pork, cut into cubes\\
  \SI{1/4}{\cup} vegetable oil, divided\\
  1 clove garlic, minced\\
  1 small onion (optional)\\
  2 green peppers, cut into \SI{1}{\inch} pieces\\
  \SI{1/3}{\cup} sugar\\
  \SI{1/4}{\cup} ketchup\\\
  \SI{1}{\tblspoon} sherry\\
  \SI{2}{\tblspoon} soy sauce\\
  \SI{2}{\tblspoon} white vinegar\\
  \SI{1}{\tblspoon} cornstarch\\
  \SI{1/3}{\cup} water\\
  1 \SI{8}{\ounce} can pineapple slices in juice, each cut into about 8 pieces
\end{ingredients}
Combine first 3 ingredients, add pork, and let marinate 20 minutes (or
however long it takes to prepare everything else). Heat \SI{2}{\tblspoon} oil
in big frying pan.  Stir fry onion garlic, and green pepper over med-high heat
until crisp tender.  Remove from skillet. Add rest of oil and cook pork until
cooked through.  Stir in cooked vegetables. Combine sugar and next 6
ingredients, stirring until cornstarch dissolves. Add to pork mixture and cook
until it comes to a boil. Add pineapple and and boil for about
1 minute. Serve over hot cooked rice. Serves \numrange{2}{3} hungry
people.